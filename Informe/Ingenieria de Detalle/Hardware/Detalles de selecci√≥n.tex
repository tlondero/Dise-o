\Subsubsubsection{C\'alculo potencias}
Para comenzar los calculos se asumen las siguientes variables sacadas de estadísticas propias de la zona donde se realiza el estudio.
\begin{enumerate}
\item Duración noche = 8:30 Horas
\item Duración día = 15:30 Horas
\item Horas de sol  óptimo = 8 Horas
\end{enumerate}
\note{esto seguro conviene estructurarlo distinto pero lo dejo asi por ahora}
\begin{itemize}
\item Raspberry Pi:
$P_{min} = 2W$
$P_{est} = 2.7W$
Si se multiplica por un  día se obtiene la energía necesaria poro día
$E_{min} = 172.8KJ$
$E_{est} = 233.28KJ$
\end{itemize}
Los valores de potencia fueron sacados de \note{referencia a de donde sacamos el dato me imagino ene l final de la tesis}

\Subsubsubsection{C\'alculo resistencias}
Tanto para el Bus de $I^2C$ como para el DHT22 se calcularon las resistencias de pull-up de la siguiente manera:
\begin{equation}
R_p \ = \  \frac{V_{dd}}{I_{R_p}} \ = \ \frac{3.3 \ V}{1 \ mA} \ = \ 3.3K\Omega  
\end{equation}
\note{Acá van los calculos de paneles, bateria, de la boost si e sque la hacemos o si usamos un integrado}