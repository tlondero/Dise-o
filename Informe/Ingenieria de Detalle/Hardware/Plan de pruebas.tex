
Test \TBC se utiliza el banco de pruebas \#1:
\begin{itemize}
\item se debe conectar la Raspberry Pi a la fuente de tensión mediante el cable de alimentación mini USB
\item Se debe utilizar la computadora para conectarse para conectarse a la misma red WIFI a la que esté conectada la Raspberry Pi.
\item Desde la terminal de la computadora se deben insertar las siguientes lineas:
\begin{lstlisting}[language=Python]
ssh pi@[IP Raspberry Pi] 
cd simfs/Python_Scripts/
python3 getTime.py
\end{lstlisting}
\item Verificar que la hora impresa en la terminal coincida con la hora correspondiente al lugar de instalación del producto.
\end{itemize}

Test \TBC se utiliza el banco de pruebas \#1:
\begin{itemize}
\item se debe conectar la Raspberry Pi a la fuente de tensión mediante el cable de alimentación mini USB
\item Se debe utilizar la computadora para conectarse para conectarse a la misma red WIFI a la que esté conectada la Raspberry Pi.
\item Desde la terminal de la computadora se deben insertar las siguientes lineas:
\begin{lstlisting}[language=Python]
ssh pi@[IP Raspberry Pi] 
cd simfs/Python_Scripts/Clases
python3 print_measures.py
\end{lstlisting}
\item Verificar que la diferencia horaria impresa en la terminal correspondiente a las mediciones sea la especificada.
\end{itemize}


Test \TBC se utiliza el banco de pruebas \#2:
\begin{itemize}
\item se debe conectar la Raspberry Pi a la fuente de tensión mediante el cable de alimentación mini USB
\item Se debe utilizar la computadora para conectarse para conectarse a la misma red WIFI a la que esté conectada la Raspberry Pi.
\item Abrir un navegador de internet y conectarse a la página "192.168.0.1:1880"
\item Prender el sensor de temperatura calibrado y colocarlo próximo al sensor del producto.
\item Corroborar que la lectura en el sensor calibrada coincida con el provisto por la página con una desviación de como máximo  $ \pm 0.5^\circ C $
\end{itemize}

Test \TBC se utiliza el banco de pruebas \#2:
\begin{itemize}
\item se debe conectar la Raspberry Pi a la fuente de tensión mediante el cable de alimentación mini USB
\item Se debe utilizar la computadora para conectarse para conectarse a la misma red WIFI a la que esté conectada la Raspberry Pi.
\item Abrir un navegador de internet y conectarse a la página "192.168.0.1:1880"
\item Prender el sensor de humedad calibrado y colocarlo próximo al sensor del producto.
\item Corroborar que la lectura en el sensor calibrada coincida con el provisto por la página con una desviación de como máximo  $ \pm  2\%	 $
\end{itemize}

Test \TBC se utiliza el banco de pruebas \#2:
\begin{itemize}
\item se debe conectar la Raspberry Pi a la fuente de tensión mediante el cable de alimentación mini USB
\item Se debe utilizar la computadora para conectarse para conectarse a la misma red WIFI a la que esté conectada la Raspberry Pi.
\item Abrir un navegador de internet y conectarse a la página "192.168.0.1:1880"
\item Prender el sensor de luminosidad calibrado y colocarlo próximo al sensor del producto.
\item Corroborar que la lectura en el sensor calibrada coincida con el provisto por la página con una desviación de como máximo  $ \pm 0.5 lux	 $
\end{itemize}


Test \TBC se utiliza el banco de pruebas \#4:
\begin{itemize}
\item se debe conectar la Raspberry Pi a la fuente de tensión mediante el cable de alimentación mini USB
\item Se debe utilizar la computadora para conectarse para conectarse a la misma red WiFi a la que esté conectada la Raspberry Pi.
\item Abrir un navegador de internet y conectarse a la página "192.168.0.1:1880"
\item Energizar el módulo BLE previamente configurado
\item Colocar el módulo a una distancia menor a 60cm del sensor de presencia.
\item Corroborar que el estado de presencia mostrado en la página web coincida la presencia del módulo.
\end{itemize}

Test \TBC se utiliza el banco de pruebas \#2: 	
\begin{itemize}
\item se debe conectar la Raspberry Pi a la fuente de tensión mediante el cable de alimentación mini USB
\item Se debe utilizar la computadora para conectarse para conectarse a la misma red WIFI a la que esté conectada la Raspberry Pi.
\item Abrir un navegador de internet y conectarse a la página "192.168.0.1:1880"
\item Seleccionar un rango de fechas que abarquen todos los datos y descargarlos.
\item Desde la terminal de la computadora se deben insertar las siguientes lineas:
\begin{lstlisting}[language=Python]
ssh pi@[IP Raspberry Pi] 
cd simfs/Python_Scripts/Clases
python3 print_measures.py
\end{lstlisting}

\item Corroborar que la lectura de la terminal coincida con los archivos descargados.
\end{itemize}


Test \TBC se utiliza el banco de pruebas \#3:

\begin{itemize}
\item Se conectan todos los módulos del producto, y se pone en funcionamiento.
\TBC
\end{itemize}
Test \TBC se utiliza el banco de pruebas \#5:
\begin{itemize}
\item Se toma le producto junto a la cinta métrica
\item Se miden las dimensiones del producto, todos sus componentes y se anotan por separado.
\item Se coloca las partes del producto en una balanza, para pesarlos por separado.
\item Se cotejan los resultados con las especificaciones 
\end{itemize}

Test \TBC se utiliza el banco de pruebas \#1:
\begin{itemize}
\item se debe conectar la Raspberry Pi a la fuente de tensión mediante el cable de alimentación mini USB
\item Se debe utilizar la computadora para conectarse para conectarse a la misma red WIFI a la que esté conectada la Raspberry Pi.
\item Desde la terminal de la computadora se deben insertar las siguientes lineas:
\begin{lstlisting}[language=Python]
ssh pi@[IP Raspberry Pi] 
cd simfs/Python_Scripts/
python3 getTime.py
\end{lstlisting}
\item Anotar el valor que se encuentra en la terminal
\item Desde la terminal de la computadora se deben insertar las siguientes lineas:
\begin{lstlisting}[language=Python]
sudo shutdown -h now
\end{lstlisting}
\item Esperar 1 minuto
\item Renergizar la Rapsberry Pi
\item Desde la terminal de la computadora se deben insertar las siguientes lineas:
\begin{lstlisting}[language=Python]
ssh pi@[IP Raspberry Pi] 
cd simfs/Python_Scripts/
python3 getTime.py
\end{lstlisting}
\item Verificar que la hora impresa en la terminal coincida con la hora correspondiente al lugar de instalación del producto.
\end{itemize}

