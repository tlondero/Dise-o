\Subsubsubsection{Potencia}

Para la unidad de potencia se utilizarán paneles solares y una batería de gel de ciclo profundo conectados a la placa DFR0580, la cual se ocupa de cargar la batería. De esta se obtienen 4 salidas de tensión:
\begin{itemize}
	\item Dos salidas de 5 V y 2.5 A [USB].
	\item Una salida de 5 V y 5 A.
	\item Una salida de 12 V y 8 A.
\end{itemize}

Con estas salidas se alimentarán todos los módulos, a excepción del oscilador de potencia, ya que esta necesita una etapa DC-DC para así la cual eleva la tensión. Para ello se emplea una fuente switching de topología Boost.

\Subsubsubsection{Cargador}

El bloque del cargador de la UBM consiste en un receptor y un transmisor de potencia. La transmisión inalámbrica consta, por el lado del transmisor, de un oscilador HM-TRPW-RS232 de $915 \ MHz$ el cual está comandado por la R-Pi y se comunica mediante UART; como así también de un amplificador de potencia de $3 \ W$ máximos alimentado por una etapa DC-DC Xl6009 de $12 \ V$ a $15 \ V$.

Por el lado del receptor, se encuentra el integrado P1110B, el cual almacena energía temporalmente en un capacitor para realizar posteriormente la carga de la UBM.

\begin{figure}[H]
	\centering	
	\includegraphics[width=0.9\textwidth, page=8]{ImagenesIngenieria de Detalle/FlowChart.pdf}
	\caption{Diagrama en bloques cargador.}
	\label{fig:diagrama_hardware_antenas}
\end{figure}

Para la transmisión de la señal de $915 \ MHz$ se utilizará cable del tipo \textit{RG-213} el cual posee bajas pérdidas de $23.054 \ \nicefrac{dB}{100m}$.\\

\textbf{Oscilador 915 MHz}

\begin{figure}[H]
	\centering	
	\includegraphics[width=0.5\textwidth, page=8]{ImagenesIngenieria de Detalle/hmtrpwrs232}
	\caption{Módulo utilizado como oscilador HM-TRPW-RS232.}
	\label{fig:oscilador}
\end{figure}

Debido a la falta de stock mundial de componentes electrónicos a causa de la pandemia, se utilizó un módulo transciever FSK que opera en la banda de 915MHz. Este módulo HM-TRPW-RS232 puede generar la señal carrier de hasta $20 \ dBm$ y posee interfaz RS-232 para la comunicación UART.

\begin{itemize}
	\item Potencia máxima de $20 \ dBm$.
	\item Alimentación $100 \ mA@20 \ dBm$.
	\item Corriente suspendido $1 \ \mu A$.
	\item Velocidad de comunicación $1.2 \ kbps - 115.2 \ kbps$.
	\item Dimensiones $44.1 \times 30  \times 1.2 \ mm$.
\end{itemize}

\textbf{Amplificador de Potencia}

Para el amplificador de potencia de RF, como las unidades a producir serán muy pocas, no se justifican las horas necesarias para diseñar el circuito y la placa impresa de un amplificador de RF. Por esta razón, se utilizará uno comercial. 

\textbf{DC-DC}



\textbf{P1110B}


\Subsubsubsection{Sensado}

En esta sección, ademas del conexionado de los sensores, se hace detalle en los pines de la R-Pi que serán utilizados.

\begin{figure}[H]
	\centering
	\includegraphics[width=0.9\linewidth]{ImagenesIngenieria de Detalle/Conexionado_rpi}		
	\caption{Conexionado Raspberry Pi.}
	\label{fig:conexionado_Rpi}
\end{figure}

El sensor de humedad y temperatura (DHT-22) se comunica de manera serial a través de un pin de GPIO. Para que esto sea posible, el sensor necesita alimentación de $3.3 \ V$, tierra y el pin de GPIO por donde se realiza la comunicación, teniendo en cuenta que es necesario una resistencia de pull-up entre la linea de datos y $3.3 \ V$.

El oscilador necesita de una señal de alimentación, tierra y su comunicación utilizá el protocolo UART. Por otro lado, la cámara simplemente se conecta al zócalo destinado para este propósito en la placa. Para el conexionado con el sensor de luminosidad y RTC se utilizará el protocolo de comunicación $I^2C$ en modo multi-slave.

Finalmente, se cuenta con un pin de GPIO el cual indicará el momento de prendido del cargador.

\begin{figure}[H]
	\centering
	\includegraphics[width=0.7\linewidth]{ImagenesIngenieria de Detalle/I2C_conexionado}	
	\caption{Conexionado $I^2C$ Multi-slave.}
	\label{fig:conexionado_i2c}
\end{figure}