\note{Mi idea para esta parte era explicar que tecnologías íbamos a usar y eso}
El centro de control, compuesto por una Raspberry-Pi, utiliza una distribución de Linux Server. Esto se debe a que no es de interés contar con una interfaz gráfica y es importante reservar espacio para almacenar los datos obtenidos desde los sensores y la cámara. 
Prescindir de la interfaz gráfica también permite ahorrar potencia dado que se apaga el puerto HDMI.


La configuración de la RPI se realiza de forma remota mediante SSH (secure shell).
Dado que el proyecto apunta a ser escalable, tener más de un nido operativo, se busco una forma de configurar la RPI manera automática y replicable. Para ello utilizaremos la herramienta  de automatización llamada Ansible. Esta nos permite de volcar sobre archivos de texto las configuraciones que necesita la RPI.

Se desarrollo la lógica de control utilizando el lenguaje de programación Python. Su rico ecosistema de paquetes nos permiten utilizar los diversos sensores y manejar las bases de datos.

Sensado

Para realizar la medición de las magnitudes de interés 











