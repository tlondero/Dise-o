La configuración de la RPI se realiza de forma remota mediante SSH (secure shell). Esto permite ejecutar comandos sobre la computadora remota sin la necesidad de un entorno gráfico. 
El proyecto apunta será escalado a más de un nido. Se busco una forma de configurar la RPI de manera automática y replicable para agilizar el proceso de desarrollo. Para ello utilizamos la herramienta  de automatización llamada Ansible. Esta nos permite empaquetar diferentes configuraciones que deben ser aplicadas como por ejemplo: configuraciones de ahorro de energía, conectivad, entre otras. 

% TODO: \usepackage{graphicx} required
\begin{figure}[H]
	\centering
	\includegraphics[width=0.2\linewidth]{"../Ingenieria de Detalle/ImagenesIngenieria de Detalle/Ansible_logo"}
	\caption{Tecnología de automatización Ansible}
	\label{fig:ansiblelogo}
\end{figure}














