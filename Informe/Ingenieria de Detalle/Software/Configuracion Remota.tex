La configuración de la \rspi se realiza de forma remota mediante SSH (\textit{secure shell}). Dado que el proyecto apunta a ser escalable y tener más de un nido operativo, se buscó una forma de configurar la \rpi manera automática y replicable. Para ello se utiliza una herramienta de automatización llamada Ansible. Esta nos permite guardar sobre archivos de texto las configuraciones que necesita la \rspi.

% TODO: \usepackage{graphicx} required
\begin{figure}[H]
	\centering
	\includegraphics[width=0.2\linewidth]{"../Ingenieria de Detalle/ImagenesIngenieria de Detalle/Ansible_logo"}
	\caption{Tecnología de automatización Ansible.}
	\label{fig:ansiblelogo}
\end{figure}














