La configuración de la RPI se realiza de forma remota mediante SSH (secure shell).
Dado que el proyecto apunta a ser escalable, tener más de un nido operativo, se busco una forma de configurar la RPI manera automática y replicable. Para ello utilizaremos la herramienta  de automatización llamada Ansible. Esta nos permite de volcar sobre archivos de texto las configuraciones que necesita la RPI.

% TODO: \usepackage{graphicx} required
\begin{figure}[H]
	\centering
	\includegraphics[width=0.2\linewidth]{"../Ingenieria de Detalle/ImagenesIngenieria de Detalle/Ansible_logo"}
	\caption{Tecnología de automatización Ansible.}
	\label{fig:ansiblelogo}
\end{figure}


Se desarrollo la lógica de control utilizando el lenguaje de programación Python. Su rico ecosistema de paquetes nos permiten utilizar los diversos sensores y manejar las bases de datos.













