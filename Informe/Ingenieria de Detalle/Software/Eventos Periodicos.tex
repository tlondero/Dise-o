Durante la duración del proyecto la base de control deberá ejecutar una serie de acciones que se repetirán en el tiempo a intervalos de regulares. Para poder orquestrar estas actividades se hace uso de la utilidad nativa de Linux, los cronjobs. 
Utilizar esta actividad es ventajosa por sobre el método clásico de generar un único script que se encargue de todo. De esta forma evitamos un punto de falla que pondría en riesgo la ejecución de las demás acciones. 

% TODO: \usepackage{graphicx} required
\begin{figure}[H]
	\centering
	\includegraphics[width=0.7\linewidth]{"../Ingenieria de Detalle/ImagenesIngenieria de Detalle/cron"}
	\caption{Estructura de un cronjob.}
	\label{fig:cron}
\end{figure}

Por lo tanto, según los intervalos pre-establecidos se ejecutaran diferentes tareas.

A continuación se detallan las acciones que se ejecutaran de forma periódica  
\begin{itemize}
	\item Encendido de la antena BLE durante el día.
	\item Reinicio periódico cada 6 horas.
	\item Apagado y encendido del server. 
\end{itemize}










