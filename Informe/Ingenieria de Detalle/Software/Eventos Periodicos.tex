Durante la duración del proyecto la base de control deberá ejecutar una serie de acciones que se repetirán en el tiempo a intervalos de regulares. Para poder orquestrar estas actividades se hace uso de la utilidad nativa de Linux, los cronjobs. 
Utilizar esta actividad es ventajosa por sobre el método clásico de generar un unico script que se encargue de todo. De esta forma evitamos un punto de falla que pondría en riesgo la ejecuciones de las demás acciones. 

% TODO: \usepackage{graphicx} required
\begin{figure}[H]
	\centering
	\includegraphics[width=0.7\linewidth]{"../Ingenieria de Detalle/ImagenesIngenieria de Detalle/cron"}
	\caption{Estructura de un cronjob}
	\label{fig:cron}
\end{figure}

Por lo tanto, según los intervalos pre-establecidos se ejecutaran diferentes tareas.

Además se utilizara esta metodología para activar y desactivar los diferentes modos de ahorro de energía como así el borrado periódico de archivos y reinicios programados.

A continuación se detallan las acciones que se ejecutaran de forma periódica  
\begin{itemize}
	\item Medición de Temperatura y Humedad
	\item Medición del nivel de luminosidad
	\item Toma de fotografías
	\item Borrado de datos ante posibilidad de disco lleno
	\item Apagado de node-red durante la noche
	\item Apagado de la antena wifi durante la noche
	\item Apagado de la antena BLE durante la noche
	\item Encendido de node-red durante el día
	\item Encendido de la antena wifi durante el día
	\item Encendido de la antena BLE durante el día
	\item Reinicio periódico cada 6 horas.
\end{itemize}










