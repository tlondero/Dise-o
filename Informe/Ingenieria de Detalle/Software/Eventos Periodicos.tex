Para realizar la medición de las magnitudes de interés se opto por utilizar la utilidad nativa de Linux, los cronjobs. Los cronjobs son tareas programables que el propio OS se encarga de ejecutar según un cronograma pre-establecido en la llamada crontab. Esta una opción mucho más segura ante la alternativa de realizar un polleo continuo mediante un cronograma de sensado programado por nosotros.
% TODO: \usepackage{graphicx} required
\begin{figure}[H]
	\centering
	\includegraphics[width=0.7\linewidth]{"../Ingenieria de Detalle/ImagenesIngenieria de Detalle/cron"}
	\caption{Estructura de un cronjob}
	\label{fig:cron}
\end{figure}

Por lo tanto, según los intervalos pre-establecidos se ejecutaran diferentes tareas.

Además se utilizara esta metodología para activar y desactivar los diferentes modos de ahorro de energía como así el borrado periódico de archivos.

A continuación se detallan las acciones que se ejecutaran de forma periódica  
\begin{itemize}
	\item Medición de Temperatura y Humedad
	\item Medición del nivel de luminosidad
	\item Toma de fotografías
	\item Borrado de datos ante posibilidad de disco lleno
	\item Apagado de node-red durante la noche
	\item Apagado de la antena wifi durante la noche
	\item Apagado de la antena BLE durante la noche
	\item Encendido de node-red durante el día
	\item Encendido de la antena wifi durante el día
	\item Encendido de la antena BLE durante el día
\end{itemize}










