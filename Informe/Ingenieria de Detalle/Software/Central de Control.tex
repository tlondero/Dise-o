El centro de control, compuesto por una Raspberry-Pi, utiliza una distribución de Linux Server. Esto se debe a que no es de interés contar con una interfaz gráfica y es importante reservar espacio para almacenar los datos obtenidos desde los sensores y la cámara. 
Prescindir de la interfaz gráfica también permite ahorrar potencia dado que se apaga el puerto HDMI.

% TODO: \usepackage{graphicx} required
\begin{figure}[H]
	\centering
	\includegraphics[width=0.7\linewidth]{"../Ingenieria de Detalle/ImagenesIngenieria de Detalle/rpi_with_board"}
	\caption{}
	\label{fig:rpiwithboard}
\end{figure}



Se desarrollo la lógica de control utilizando el lenguaje de programación Python. Su rico ecosistema de paquetes nos permiten utilizar los diversos sensores y manejar las bases de datos.

% TODO: \usepackage{graphicx} required
\begin{figure}[H]
	\centering
	\includegraphics[width=0.7\linewidth]{"../Ingenieria de Detalle/ImagenesIngenieria de Detalle/python-logo@2x"}
	\caption{}
	\label{fig:python-logo2x}
\end{figure}










