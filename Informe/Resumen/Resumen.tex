%\documentclass[a4paper]{article}
\usepackage[utf8]{inputenc}
\usepackage[spanish, es-tabla, es-noshorthands]{babel}
\usepackage[table,xcdraw]{xcolor}
\usepackage[a4paper, footnotesep=1.25cm, headheight=1.25cm, top=2.54cm, left=2.54cm, bottom=2.54cm, right=2.54cm]{geometry}
%\geometry{showframe}
%VERIFICAR EL HEAD Y EL FOOT EN
%https://ctan.dcc.uchile.cl/macros/latex/contrib/geometry/geometry.pdf

%\usepackage{lipsum}			%LOREM IPSUM

%\usepackage{wrapfig}		%Wrap figure in text
\usepackage[export]{adjustbox}	%Move images
\usepackage{changepage}			%Move tables

%Font
\usepackage{anyfontsize}	%Font size
% #1 = size, #2 = text
\newcommand{\setparagraphsize}[2]{{\fontsize{#1}{6}\selectfont#2 \par}}		%Cambia el size de todo el parrafo
\newcommand{\setlinesize}[2]{{\fontsize{#1}{6}\selectfont#2}}				%Cambia el font de una oración


%FONTS (IMPORTANTE): Compilar en XeLaTex o LuaLaTeX
\usepackage{fontspec}
%Si sigue sin andar comentar \usepackage[utf8]{inputenc}
%https://ctan.dcc.uchile.cl/macros/unicodetex/latex/fontspec/fontspec.pdf
%https://www.overleaf.com/learn/latex/XeLaTeX

\usepackage{tikz}
\usepackage{amsmath}
\usepackage{amsfonts}
\usepackage{amssymb}
\usepackage{float}
\usepackage{graphicx}
\usepackage{caption}
\usepackage{subcaption}
\usepackage{multicol}
\usepackage{multirow}
\setlength{\doublerulesep}{\arrayrulewidth}
\usepackage{booktabs}

\usepackage{hyperref}
\hypersetup{
    colorlinks=true,
    linkcolor=black,
    filecolor=magenta,      
    urlcolor=blue,
    citecolor=blue,    
}

\newcommand{\captionsection}{\setcounter{figure}{0} \renewcommand{\thetable}{\arabic{section}.\arabic{table}} \renewcommand{\thefigure}{\arabic{section}.\arabic{figure}}}

\newcommand{\captionsubsection}{\setcounter{figure}{0} \renewcommand{\thetable}{\arabic{section}.\arabic{subsection}.\arabic{table}} \renewcommand{\thefigure}{\arabic{section}.\arabic{subsection}.\arabic{figure}}}

\newcommand{\captionsubsubsection}{\setcounter{figure}{0} \renewcommand{\thetable}{\arabic{section}.\arabic{subsection}.\arabic{subsubsection}.\arabic{table}} \renewcommand{\thefigure}{\arabic{section}.\arabic{subsection}.\arabic{subsubsection}.\arabic{figure}}}

%PICTURES AND TABLE INDEX:
\newcommand{\Section}[1]{ \section{#1}\captionsection }
\newcommand{\Subsection}[1]{ \subsection{#1}\captionsubsection }
\newcommand{\Subsubsection}[1]{ \subsubsection{#1}\captionsubsubsection }

%NOTAS GRANDES
\newcommand{\note}[1]{
	\begin{center}
		\huge{ \textcolor{red}{#1} }
	\end{center}
}
%Notas pequeñas
\newcommand{\lnote}[1]{{\fontsize{20}{6}\selectfont\textcolor{green}{#1}}}

\newcommand{\quotes}[1]{``#1''}
\usepackage{array}
\newcolumntype{C}[1]{>{\centering\let\newline\\\arraybackslash\hspace{0pt}}m{#1}}
\usepackage[american]{circuitikz}
\usetikzlibrary{calc}
\usepackage{fancyhdr}
\usepackage{units} 

\graphicspath{{../Prefacio/}{../Acronimos/}{../Introduccion/}{../Objetivos/}{../Definicion/}{../Plan de validacion/}{../Resumen/}}

%COLORES
\definecolor{AzulFoot}{rgb}{0.682,0.809,0.926}	%RGB	%{174,206,235}
\definecolor{AzulInfo}{rgb}{0.180,0.455,0.710}	%RGB	%{46,116,181}
\definecolor{AzulTable}{rgb}{0.302,0.507,0.871}	%RGB	%{68,114,196}
\definecolor{PName}{rgb}{0.353,0.353,0.353}	%RGB	%{90,90,90}

\usepackage{xcolor}
\usepackage{sectsty}
\chapterfont{\color{AzulInfo}}  % sets colour of chapters
\sectionfont{\color{AzulInfo}}  % sets colour of sections
\subsectionfont{\color{AzulInfo}}
\subsubsectionfont{\color{AzulInfo}}

%Header y footer
\usepackage{etoolbox}

\pagestyle{fancy}
\fancyhf{}
\rfoot{\thepage}
\renewcommand{\footrulewidth}{4pt}
\renewcommand{\headrulewidth}{0pt}
\patchcmd{\footrule}{\hrule}{\color{AzulFoot}\hrule}{}{}

%\begin{document}

%En este informe se introduce brevemente al estado del arte de la adquisición de datos en la naturaleza y al sistema propuesto que opera en el hábitat particular de aves pequeñas, en este caso diseñado (pero no limitado) a la especie \textit{Campephilus Magellanicus}. 

%Se detalla el diseño de una plataforma de recolección de información autónoma, que permitirá conocer con profundidad el comportamiento y hábitat de las aves. Esta debe tener la capacidad de almacenar diversos datos del interior del nido.% por la duración de una semana.

%El nivel de autonomía que se busca está ligado no solamente a la recolección de datos, sino también a la alimentación del producto, debido a las condiciones del entrono en el cual habita el ave. 

%El sistema debe además ser capaz de transmitir estos datos de manera inalámbrica para no perturbar el comportamiento de la especie estudiada.

%Se analizan los requerimientos y especificaciones de producto considerando a los clientes involucrados, entre ellos el equipo de biólogos que realizarán las observaciones, los entes reguladores de vida silvestre, el estado y los fabricantes de circuitos impresos, entre otros.

%Se presentan los procedimientos tomados para las pruebas, los criterios de aceptación, las precondiciones, postcondiciones y el banco de pruebas.

%Este proyecto es el complemento a otro donde se diseña una mochila la cual también recolecta datos y se monta sobre el ave. La mayor dificultad de esta solución es la de recargar inalámbricamente las baterías que alimentan a la mochila, por lo que se realiza una investigación sobre diversas maneras de transmitir potencia inalámbricamente.

%Además, se comparan diversas tecnologías de sensado y comunicación de datos, eligiendo la más efectiva.

%Finalmente, se valida la solución con la construcción de un prototipo, realizando también un análisis económico y legal del proyecto.

%Además, se realiza un análisis modal de fallos y efectos, identificando posibles modos de falla, sus consecuencias y maneras de mitigar sean sus efectos, disminuir probabilidad de ocurrencia o aumentar la detectabilidad de los problemas.

En este proyecto final de ingeniería se detalla el diseño del S.I.M.F.S., una plataforma de adquisición y almacenamiento de datos que reside dentro del nido de un ave durante la etapa de anidamiento. Además, este trabajo incluye desarrollo en investigación de tecnologías de radiación electromagnética para carga inalámbrica. Esta plataforma es utilizada en un estudio de investigación de la especie \textit{Campephilus Magellanicus}, mejor conocido como Carpintero Gigante, en conjunto a un dispositivo ajeno a este proyecto que se coloca sobre el mismo ave.

El S.I.M.F.S. es una red que abarca un amplio margen de tecnologías. Emplea protocolo de comunicación Bluetooth con otros dispositivos; se encarga de la obtención y almacenamiento de datos de periféricos; emplea sistemas de alimentación para áreas remotas; corre un servidor web con acceso restringido donde se alcanza al usuario el acceso a estos datos, entre otras funcionalidades.

%Mediante Bluetooth se reciben los datos almacenados en el dispositivo que se encuentra situado en el ave. También permite la descarga de todos los datos obtenidos y almacenados en el sistema mediante el protocolo de Wi-Fi. 

Entrando en detalle en lo que respecta a la elección de las tecnologías se puede especificar que:
\begin{itemize}
	\item Para la alimentación del sistema en ausencia de una red eléctrica cercana, se utilizan paneles solares.
	\item Para el auxilio de los paneles solares al caer la noche, se utiliza una batería de gel de ciclo profundo.
	\item Para la recolección de datos, distintos tipos de sensores.
	\item Para el almacenamiento, una tarjeta SD de grado industrial.
	\item Para la unidad de procesamiento, una \rspi 3B.
\end{itemize}

Además, se incurre en una investigación en el área de transmisión de potencia para analizar la factibilidad tecnológica de la recarga inalámbrica de las baterías de aquel dispositivo cuyo desarrollo es ajeno al proyecto. Se estudia el tipo de acople electromagnético, la banda de frecuencia, se realiza un análisis de distinto tipo de antenas y un estudio de reglamentaciones legales.

A lo largo del trabajo se concluye que la mejor elección para la transmisión inalámbrica de potencia dadas las circunstancias del proyecto es el uso de radiofrecuencia de campo lejano en la banda de frecuencia centrada en $915 \ MHz$ con antenas del tipo \textit{patch}. Esto se debe a que para los clientes es de suma importancia la determinación de la factibilidad del desarrollo de una tecnología que cumpla con las características mencionadas. Sin embargo, el estudio de factibilidad dicta que no es posible dentro de las reglamentaciones legales y la eficiencia conseguida obtener la potencia requerida dentro de las restricciones impuestas por la naturaleza del proyecto.

En consecuencia, y para satisfacción de los requerimientos provistos por los solicitantes del proyecto, se propone una alternativa que, si bien no realiza la carga inalámbrica mencionada, garantiza el correcto funcionamiento de los elementos que requieren del uso de dicha carga.
Así también aumentando la propuesta de valor provista por la propuesta que incluía el cargador inalámbrico.

Finalmente se detalla sobre la construcción de un prototipo. Con este se aplica el plan de validación, se realiza un estudio económico y legal del proyecto, y se construye continuamente un análisis modal de fallos y efectos.

%\end{document}