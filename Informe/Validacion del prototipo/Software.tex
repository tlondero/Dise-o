De esta forma, se corroboró mediante observación que la imagen mostrada al usuario

\begin{figure}[H]
\centering
    	\begin{subfigure}{0.49\textwidth}
        	\centering
        	\includegraphics[width=\linewidth]{ImagenesValidacion del prototipo/t-int-fun-04-12-1}		
			\caption{Vista de la transmisión en vivo del nido.}
        \end{subfigure}\hfill
        \begin{subfigure}{0.49\textwidth}
        	\centering
        	\includegraphics[width=0.5\linewidth]{ImagenesValidacion del prototipo/t-int-fun-04-12-2}
        	\caption{Imagen del exterior del nido en el momento de la trasmisión.}
        \end{subfigure}
	\caption{Validación de funcionalidad \textit{T-INT-FUN-04} y \textit{T-INT-FUN-12}.}
\end{figure}


Se dispuso del banco de pruebas \#1 para analizar la aplicabilidad de \textit{T-RAM-SEG-01}. El acceso tanto a la interfaz gráfica del usuario como la del administrador se encuentran restringidas para cualquiera que no posea las credenciales adecuadas.

\begin{figure}[H]
\centering
    	\begin{subfigure}{0.49\textwidth}
        	\centering
        	\includegraphics[width=\linewidth]{ImagenesValidacion del prototipo/t-ram-seg-01-1}		
			\caption{Autenticación de usuario para la interfaz gráfica.}
        \end{subfigure}\hfill
        \begin{subfigure}{0.49\textwidth}
        	\centering
        	\includegraphics[width=\linewidth]{ImagenesValidacion del prototipo/t-ram-seg-01-2}
        	\caption{Autenticación de administrador para la interfaz gráfica.}
        \end{subfigure}
	\caption{Validación de seguridad \textit{T-RAM-SEG-01}.}
\end{figure}

Se obtuvo un resultado mejor del esperado ya que el hecho de emplear una red \wifi para conectar la unidad del nido con el usuario también requiere del uso de una contraseña. Esto agrega una capa de seguridad extra al sistema.

