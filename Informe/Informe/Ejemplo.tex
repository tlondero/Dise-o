\documentclass[a4paper]{article}
\usepackage[utf8]{inputenc}
\usepackage[spanish, es-tabla, es-noshorthands]{babel}
\usepackage[table,xcdraw]{xcolor}
\usepackage[a4paper, footnotesep=1.25cm, headheight=1.25cm, top=2.54cm, left=2.54cm, bottom=2.54cm, right=2.54cm]{geometry}
%\geometry{showframe}
%VERIFICAR EL HEAD Y EL FOOT EN
%https://ctan.dcc.uchile.cl/macros/latex/contrib/geometry/geometry.pdf

%\usepackage{lipsum}			%LOREM IPSUM

%\usepackage{wrapfig}		%Wrap figure in text
\usepackage[export]{adjustbox}	%Move images
\usepackage{changepage}			%Move tables

%Font
\usepackage{anyfontsize}	%Font size
% #1 = size, #2 = text
\newcommand{\setparagraphsize}[2]{{\fontsize{#1}{6}\selectfont#2 \par}}		%Cambia el size de todo el parrafo
\newcommand{\setlinesize}[2]{{\fontsize{#1}{6}\selectfont#2}}				%Cambia el font de una oración


%FONTS (IMPORTANTE): Compilar en XeLaTex o LuaLaTeX
\usepackage{fontspec}
%Si sigue sin andar comentar \usepackage[utf8]{inputenc}
%https://ctan.dcc.uchile.cl/macros/unicodetex/latex/fontspec/fontspec.pdf
%https://www.overleaf.com/learn/latex/XeLaTeX

\usepackage{tikz}
\usepackage{amsmath}
\usepackage{amsfonts}
\usepackage{amssymb}
\usepackage{float}
\usepackage{graphicx}
\usepackage{caption}
\usepackage{subcaption}
\usepackage{multicol}
\usepackage{multirow}
\setlength{\doublerulesep}{\arrayrulewidth}
\usepackage{booktabs}

\usepackage{hyperref}
\hypersetup{
    colorlinks=true,
    linkcolor=black,
    filecolor=magenta,      
    urlcolor=blue,
    citecolor=blue,    
}

\newcommand{\captionsection}{\setcounter{figure}{0} \renewcommand{\thetable}{\arabic{section}.\arabic{table}} \renewcommand{\thefigure}{\arabic{section}.\arabic{figure}}}

\newcommand{\captionsubsection}{\setcounter{figure}{0} \renewcommand{\thetable}{\arabic{section}.\arabic{subsection}.\arabic{table}} \renewcommand{\thefigure}{\arabic{section}.\arabic{subsection}.\arabic{figure}}}

\newcommand{\captionsubsubsection}{\setcounter{figure}{0} \renewcommand{\thetable}{\arabic{section}.\arabic{subsection}.\arabic{subsubsection}.\arabic{table}} \renewcommand{\thefigure}{\arabic{section}.\arabic{subsection}.\arabic{subsubsection}.\arabic{figure}}}

%PICTURES AND TABLE INDEX:
\newcommand{\Section}[1]{ \section{#1}\captionsection }
\newcommand{\Subsection}[1]{ \subsection{#1}\captionsubsection }
\newcommand{\Subsubsection}[1]{ \subsubsection{#1}\captionsubsubsection }

%NOTAS GRANDES
\newcommand{\note}[1]{
	\begin{center}
		\huge{ \textcolor{red}{#1} }
	\end{center}
}
%Notas pequeñas
\newcommand{\lnote}[1]{{\fontsize{20}{6}\selectfont\textcolor{green}{#1}}}

\newcommand{\quotes}[1]{``#1''}
\usepackage{array}
\newcolumntype{C}[1]{>{\centering\let\newline\\\arraybackslash\hspace{0pt}}m{#1}}
\usepackage[american]{circuitikz}
\usetikzlibrary{calc}
\usepackage{fancyhdr}
\usepackage{units} 

\graphicspath{{../Prefacio/}{../Acronimos/}{../Introduccion/}{../Objetivos/}{../Definicion/}{../Plan de validacion/}{../Resumen/}}

%COLORES
\definecolor{AzulFoot}{rgb}{0.682,0.809,0.926}	%RGB	%{174,206,235}
\definecolor{AzulInfo}{rgb}{0.180,0.455,0.710}	%RGB	%{46,116,181}
\definecolor{AzulTable}{rgb}{0.302,0.507,0.871}	%RGB	%{68,114,196}
\definecolor{PName}{rgb}{0.353,0.353,0.353}	%RGB	%{90,90,90}

\usepackage{xcolor}
\usepackage{sectsty}
\chapterfont{\color{AzulInfo}}  % sets colour of chapters
\sectionfont{\color{AzulInfo}}  % sets colour of sections
\subsectionfont{\color{AzulInfo}}
\subsubsectionfont{\color{AzulInfo}}

%Header y footer
\usepackage{etoolbox}

\pagestyle{fancy}
\fancyhf{}
\rfoot{\thepage}
\renewcommand{\footrulewidth}{4pt}
\renewcommand{\headrulewidth}{0pt}
\patchcmd{\footrule}{\hrule}{\color{AzulFoot}\hrule}{}{}

\usepackage{lipsum}			%LOREM IPSUM

\begin{document}

\def\PrimerObs{1}
\def\SegundaObs{0}

%%%%%%%%%%%%%%%%%%%%%%%%%
%		Caratula		%
%%%%%%%%%%%%%%%%%%%%%%%%%

\setmainfont{Avenir LT Std 55 Roman}
\begin{titlepage}

\begin{tikzpicture}[remember picture, overlay, black, line width = 0.5pt]
	\coordinate (a) at (-2cm,2cm);
	\coordinate (b) at (17cm,-25.5cm);
	
	\coordinate (ap) at (-2.1cm,2.1cm);
	\coordinate (bp) at (17.1cm,-25.6cm);
	
	\draw[] (a) -| (b);
	\draw[] (a) |- (b);
	
	\draw[] (ap) -| (bp);
	\draw[] (ap) |- (bp);
	
	%footnotesep=1.25cm, headheight=1.25cm, top=2.54cm, left=2.54cm, bottom=2.54cm, right=2.54cm

\end{tikzpicture}

\begin{figure}[H]
	\includegraphics[width=0.3\linewidth, right]{./Utils/ITBA_1}
\end{figure}

\vspace*{1.5cm}

\noindent \textbf{\setlinesize{11}{INSTITUTO TECNOLÓGICO DE BUENOS AIRES - ITBA}}

\noindent \textbf{\setlinesize{11}{ESCUELA DE INGENIERÍA Y TECNOLOGÍA}}

\vspace*{4cm}

\center{
\setlinesize{24}{ \textbf{SISTEMA INTEGRAL DE MONITOREO DE FAUNA SILVESTRE} }

\vspace*{1.5cm}

%\setlinesize{24}{ \textbf{Subtítulo del trabajo (cuando corresponda)} }
}

\vspace*{1.5cm}

\begin{figure}[H]
\begin{adjustwidth}{-1cm}{}
\begin{tabular}{llr} 
	\textbf{AUTORES:} & \textbf{Mechoulam, Alan}  &  \textbf{(Leg. N}$\mathbf{^o}$ \textbf{58438)}\\
	 & \textbf{Lambertucci, Guido Enrique} & \textbf{(Leg. N}$\mathbf{^o}$ \textbf{58009)} \\
	 & \textbf{Rodriguez Turco, Martín Sebastian} & \textbf{(Leg. N}$\mathbf{^o}$ \textbf{56629)} \\
	 & \textbf{Londero Bonaparte, Tomás Guillermo} & \textbf{(Leg. N}$\mathbf{^o}$ \textbf{58150)} \\
 &  & \\
 &  & \\
	\textbf{DOCENTES}: & \textbf{Orchessi, Walter} & \\
	 & \textbf{Pingitore, Ricardo} & \\
	 & \textbf{Ugarte, Alejandro} & \\
\end{tabular}
\end{adjustwidth}
\end{figure}

\vspace*{0.5cm}
{\noindent \textbf{TRABAJO FINAL PRESENTADO PARA LA OBTENCIÓN DEL TÍTULO DE INGENIERO ELECTRÓNICO}}
\vspace*{1.5cm}

\center{\textbf{BUENOS AIRES}}

\end{titlepage}


\begin{titlepage}

\begin{tikzpicture}[remember picture, overlay, black, line width = 0.5pt]
	\coordinate (a) at (-2cm,2cm);
	\coordinate (b) at (17cm,-25.5cm);
	
	\coordinate (ap) at (-2.1cm,2.1cm);
	\coordinate (bp) at (17.1cm,-25.6cm);
	
	\draw[] (a) -| (b);
	\draw[] (a) |- (b);
	
	\draw[] (ap) -| (bp);
	\draw[] (ap) |- (bp);
	
	%footnotesep=1.25cm, headheight=1.25cm, top=2.54cm, left=2.54cm, bottom=2.54cm, right=2.54cm

\end{tikzpicture}

\vspace*{4cm}

\textbf{TRABAJO FINAL PRESENTADO PARA LA OBTENCIÓN DEL TÍTULO DE INGENIERO ELECTRÓNICO}

\vspace*{4cm}

\center

\textbf{BUENOS AIRES}

\vspace*{1cm}

\textbf{PRIMER / SEGUNDO CUATRIMESTRE, 202X}


\end{titlepage}


\setmainfont{Calibri}
\begin{titlepage}
\begin{figure}[H]
	\centering
	\includegraphics[width=0.5\linewidth]{./Utils/ITBA_2}
\end{figure}

\vspace*{1.5cm}

\center
{\Huge Proyecto final de Ingeniería Electrónica }

\vspace*{1cm}

{\LARGE \textcolor{PName}{Sistema Integral de Monitoreo De Fauna Silvestre} }

\vspace*{3cm}

\begin{tabular}{llr} 	
\textbf{Autores}: & Mechoulam, Alan  &  (58438)\\
 & Lambertucci, Guido Enrique  & (58009) \\
 & Rodriguez Turco, Martín Sebastian  & (56629) \\
 & Londero Bonaparte, Tomás Guillermo  & (58150) \\
 &  & \\
 &  & \\
 &  & \\
\textbf{Tutores}: & Orchessi, Walter & \hspace*{4cm} \\
 & Pingitore, Ricardo & \hspace*{4cm} \\
 & Ugarte, Alejandro & \hspace*{4cm} \\
 &  & \\
 &  & \\
 &  & \\
\textbf{Fecha}: & 28/04/2021 & \hspace*{4.5cm}\\
\end{tabular}

\end{titlepage}

%%%%%%%%%%%%%%%%%%%%%%%%%%%%%%%%%
%			Indice				%
%%%%%%%%%%%%%%%%%%%%%%%%%%%%%%%%%

\MyIndex

\hypersetup{linkcolor=AzulInfo}

\newpage

%%%%%%%%%%%%%%%%%%%%%%%%%%%%%%%%%
%			Informe				%
%%%%%%%%%%%%%%%%%%%%%%%%%%%%%%%%%

\note{Este archivo sirve como guía/tutorial:}

Se crearon varios comandos útiles para desarrollar el informe, todos se encuentran en el archivos \quotes{New-Definitions.tex}. En este mismo archivo se muestran ejemplos de como funcionan la mayoría.

Se sigue una estructura donde cada sección es un archivo distinto. Viendo este documento, el \quotes{Header.tex} y las demás carpetas se puede replicar.

\note{PARA PODER COMPILAR:}
Usar XeLaTex y en la carpeta de Utils instalar el archivo \quotes{AvenirLTStd-Roman.otf}. Actualizar todos los paquetes desde Miktex (si no lo tenés descargalo) y habilitar la instalación automática de los mismos.

\note{Parte del formato:}

USAR \textbackslash Section, \textbackslash Subsection y \textbackslash Subsubsection (existen también las \textbackslash Subsubsubsection) en lugar de \textbackslash section, \textbackslash subsection y \textbackslash subsubsection. De esta forma antes del indice, cada imagen y tabla tendrá el indice de la sección, subsección, subsubsección y subsubsubsección, para identificarlo mejor. Esto se entiende mejor viendo los index de las figuras más abajo.

El paquete de lipsum (linea 13 del header) se puede sacar, es solo para rellenar con texto y probar el formato (borrar también las instancias en este tex).

\observacion{\PrimerObs}{Esto sí se puede ver.}

\observacion{\SegundaObs}{Esto no se puede ver.}

\Section{Acrónimos y definiciones}
\documentclass[a4paper]{article}
\usepackage[utf8]{inputenc}
\usepackage[spanish, es-tabla, es-noshorthands]{babel}
\usepackage[table,xcdraw]{xcolor}
\usepackage[a4paper, footnotesep=1.25cm, headheight=1.25cm, top=2.54cm, left=2.54cm, bottom=2.54cm, right=2.54cm]{geometry}
%\geometry{showframe}
%VERIFICAR EL HEAD Y EL FOOT EN
%https://ctan.dcc.uchile.cl/macros/latex/contrib/geometry/geometry.pdf

%\usepackage{lipsum}			%LOREM IPSUM

%\usepackage{wrapfig}		%Wrap figure in text
\usepackage[export]{adjustbox}	%Move images
\usepackage{changepage}			%Move tables

%Font
\usepackage{anyfontsize}	%Font size
% #1 = size, #2 = text
\newcommand{\setparagraphsize}[2]{{\fontsize{#1}{6}\selectfont#2 \par}}		%Cambia el size de todo el parrafo
\newcommand{\setlinesize}[2]{{\fontsize{#1}{6}\selectfont#2}}				%Cambia el font de una oración


%FONTS (IMPORTANTE): Compilar en XeLaTex o LuaLaTeX
\usepackage{fontspec}
%Si sigue sin andar comentar \usepackage[utf8]{inputenc}
%https://ctan.dcc.uchile.cl/macros/unicodetex/latex/fontspec/fontspec.pdf
%https://www.overleaf.com/learn/latex/XeLaTeX

\usepackage{tikz}
\usepackage{amsmath}
\usepackage{amsfonts}
\usepackage{amssymb}
\usepackage{float}
\usepackage{graphicx}
\usepackage{caption}
\usepackage{subcaption}
\usepackage{multicol}
\usepackage{multirow}
\setlength{\doublerulesep}{\arrayrulewidth}
\usepackage{booktabs}

\usepackage{hyperref}
\hypersetup{
    colorlinks=true,
    linkcolor=black,
    filecolor=magenta,      
    urlcolor=blue,
    citecolor=blue,    
}

\newcommand{\captionsection}{\setcounter{figure}{0} \renewcommand{\thetable}{\arabic{section}.\arabic{table}} \renewcommand{\thefigure}{\arabic{section}.\arabic{figure}}}

\newcommand{\captionsubsection}{\setcounter{figure}{0} \renewcommand{\thetable}{\arabic{section}.\arabic{subsection}.\arabic{table}} \renewcommand{\thefigure}{\arabic{section}.\arabic{subsection}.\arabic{figure}}}

\newcommand{\captionsubsubsection}{\setcounter{figure}{0} \renewcommand{\thetable}{\arabic{section}.\arabic{subsection}.\arabic{subsubsection}.\arabic{table}} \renewcommand{\thefigure}{\arabic{section}.\arabic{subsection}.\arabic{subsubsection}.\arabic{figure}}}

%PICTURES AND TABLE INDEX:
\newcommand{\Section}[1]{ \section{#1}\captionsection }
\newcommand{\Subsection}[1]{ \subsection{#1}\captionsubsection }
\newcommand{\Subsubsection}[1]{ \subsubsection{#1}\captionsubsubsection }

%NOTAS GRANDES
\newcommand{\note}[1]{
	\begin{center}
		\huge{ \textcolor{red}{#1} }
	\end{center}
}
%Notas pequeñas
\newcommand{\lnote}[1]{{\fontsize{20}{6}\selectfont\textcolor{green}{#1}}}

\newcommand{\quotes}[1]{``#1''}
\usepackage{array}
\newcolumntype{C}[1]{>{\centering\let\newline\\\arraybackslash\hspace{0pt}}m{#1}}
\usepackage[american]{circuitikz}
\usetikzlibrary{calc}
\usepackage{fancyhdr}
\usepackage{units} 

\graphicspath{{../Prefacio/}{../Acronimos/}{../Introduccion/}{../Objetivos/}{../Definicion/}{../Plan de validacion/}{../Resumen/}}

%COLORES
\definecolor{AzulFoot}{rgb}{0.682,0.809,0.926}	%RGB	%{174,206,235}
\definecolor{AzulInfo}{rgb}{0.180,0.455,0.710}	%RGB	%{46,116,181}
\definecolor{AzulTable}{rgb}{0.302,0.507,0.871}	%RGB	%{68,114,196}
\definecolor{PName}{rgb}{0.353,0.353,0.353}	%RGB	%{90,90,90}

\usepackage{xcolor}
\usepackage{sectsty}
\chapterfont{\color{AzulInfo}}  % sets colour of chapters
\sectionfont{\color{AzulInfo}}  % sets colour of sections
\subsectionfont{\color{AzulInfo}}
\subsubsectionfont{\color{AzulInfo}}

%Header y footer
\usepackage{etoolbox}

\pagestyle{fancy}
\fancyhf{}
\rfoot{\thepage}
\renewcommand{\footrulewidth}{4pt}
\renewcommand{\headrulewidth}{0pt}
\patchcmd{\footrule}{\hrule}{\color{AzulFoot}\hrule}{}{}

\begin{document}

\begin{table}[H]
\centering
\begin{tabular}{!{\color{AzulTable}\vrule}ll!{\color{AzulTable}\vrule}}
\arrayrulecolor{AzulTable}
\hline
\rowcolor{AzulTable}
\multicolumn{1}{|c}{\textcolor{white}{Acrónimo}} & \multicolumn{1}{c|}{\textcolor{white}{Descripción}} \\ \hline
\textbf{AMB}                         & Ambiente                         \\
\textbf{COM}                         & Comunicación                         \\
\textbf{EGM}                         & Electromecánica                         \\
\textbf{MEC}                         & Mecánico                         \\
\textbf{SER}                         & Servicio Técnico                         \\
\textbf{USD}                         & Dólares                         \\	\hline
\end{tabular}
\end{table}

\begin{table}[H]
\centering
\begin{tabular}{!{\color{AzulTable}\vrule}ll!{\color{AzulTable}\vrule}}
\arrayrulecolor{AzulTable}
\hline
\rowcolor{AzulTable}
\multicolumn{1}{|c}{\textcolor{white}{Término}} & \multicolumn{1}{c|}{\textcolor{white}{Descripción}} \\ \hline
\textbf{1}                         & 2                         \\
\textbf{3}                         & 4                         \\ \hline
\end{tabular}
\end{table}

\end{document}

\Section{Lorem Ipsum}

\lipsum[1-2]
																			
\setparagraphsize{16}{\textcolor{red}{\lipsum[3-4]}}

También se puede \setlinesize{6}{cambiar el tamaño} en el medio de una oración. Pequeñas \lnote{notas verdes} para poner una nota (y que se vea) en el texto \cite{ref:cita1}.

Para la versión del informe que se entregue se puede usar \tbd y \tbc: El producto costará \TBD, ocupará un volumen de \tbd, pesará \TBC y durará \tbc.

\note{Nota más importante para no olvidar.}

\Section{Ejercicico N}
\documentclass[a4paper]{article}
\usepackage[utf8]{inputenc}
\usepackage[spanish, es-tabla, es-noshorthands]{babel}
\usepackage[table,xcdraw]{xcolor}
\usepackage[a4paper, footnotesep=1.25cm, headheight=1.25cm, top=2.54cm, left=2.54cm, bottom=2.54cm, right=2.54cm]{geometry}
%\geometry{showframe}
%VERIFICAR EL HEAD Y EL FOOT EN
%https://ctan.dcc.uchile.cl/macros/latex/contrib/geometry/geometry.pdf

%\usepackage{lipsum}			%LOREM IPSUM

%\usepackage{wrapfig}		%Wrap figure in text
\usepackage[export]{adjustbox}	%Move images
\usepackage{changepage}			%Move tables

%Font
\usepackage{anyfontsize}	%Font size
% #1 = size, #2 = text
\newcommand{\setparagraphsize}[2]{{\fontsize{#1}{6}\selectfont#2 \par}}		%Cambia el size de todo el parrafo
\newcommand{\setlinesize}[2]{{\fontsize{#1}{6}\selectfont#2}}				%Cambia el font de una oración


%FONTS (IMPORTANTE): Compilar en XeLaTex o LuaLaTeX
\usepackage{fontspec}
%Si sigue sin andar comentar \usepackage[utf8]{inputenc}
%https://ctan.dcc.uchile.cl/macros/unicodetex/latex/fontspec/fontspec.pdf
%https://www.overleaf.com/learn/latex/XeLaTeX

\usepackage{tikz}
\usepackage{amsmath}
\usepackage{amsfonts}
\usepackage{amssymb}
\usepackage{float}
\usepackage{graphicx}
\usepackage{caption}
\usepackage{subcaption}
\usepackage{multicol}
\usepackage{multirow}
\setlength{\doublerulesep}{\arrayrulewidth}
\usepackage{booktabs}

\usepackage{hyperref}
\hypersetup{
    colorlinks=true,
    linkcolor=black,
    filecolor=magenta,      
    urlcolor=blue,
    citecolor=blue,    
}

\newcommand{\captionsection}{\setcounter{figure}{0} \renewcommand{\thetable}{\arabic{section}.\arabic{table}} \renewcommand{\thefigure}{\arabic{section}.\arabic{figure}}}

\newcommand{\captionsubsection}{\setcounter{figure}{0} \renewcommand{\thetable}{\arabic{section}.\arabic{subsection}.\arabic{table}} \renewcommand{\thefigure}{\arabic{section}.\arabic{subsection}.\arabic{figure}}}

\newcommand{\captionsubsubsection}{\setcounter{figure}{0} \renewcommand{\thetable}{\arabic{section}.\arabic{subsection}.\arabic{subsubsection}.\arabic{table}} \renewcommand{\thefigure}{\arabic{section}.\arabic{subsection}.\arabic{subsubsection}.\arabic{figure}}}

%PICTURES AND TABLE INDEX:
\newcommand{\Section}[1]{ \section{#1}\captionsection }
\newcommand{\Subsection}[1]{ \subsection{#1}\captionsubsection }
\newcommand{\Subsubsection}[1]{ \subsubsection{#1}\captionsubsubsection }

%NOTAS GRANDES
\newcommand{\note}[1]{
	\begin{center}
		\huge{ \textcolor{red}{#1} }
	\end{center}
}
%Notas pequeñas
\newcommand{\lnote}[1]{{\fontsize{20}{6}\selectfont\textcolor{green}{#1}}}

\newcommand{\quotes}[1]{``#1''}
\usepackage{array}
\newcolumntype{C}[1]{>{\centering\let\newline\\\arraybackslash\hspace{0pt}}m{#1}}
\usepackage[american]{circuitikz}
\usetikzlibrary{calc}
\usepackage{fancyhdr}
\usepackage{units} 

\graphicspath{{../Prefacio/}{../Acronimos/}{../Introduccion/}{../Objetivos/}{../Definicion/}{../Plan de validacion/}{../Resumen/}}

%COLORES
\definecolor{AzulFoot}{rgb}{0.682,0.809,0.926}	%RGB	%{174,206,235}
\definecolor{AzulInfo}{rgb}{0.180,0.455,0.710}	%RGB	%{46,116,181}
\definecolor{AzulTable}{rgb}{0.302,0.507,0.871}	%RGB	%{68,114,196}
\definecolor{PName}{rgb}{0.353,0.353,0.353}	%RGB	%{90,90,90}

\usepackage{xcolor}
\usepackage{sectsty}
\chapterfont{\color{AzulInfo}}  % sets colour of chapters
\sectionfont{\color{AzulInfo}}  % sets colour of sections
\subsectionfont{\color{AzulInfo}}
\subsubsectionfont{\color{AzulInfo}}

%Header y footer
\usepackage{etoolbox}

\pagestyle{fancy}
\fancyhf{}
\rfoot{\thepage}
\renewcommand{\footrulewidth}{4pt}
\renewcommand{\headrulewidth}{0pt}
\patchcmd{\footrule}{\hrule}{\color{AzulFoot}\hrule}{}{}

\begin{document}

\TBD

\end{document}

\newpage
\begin{flushleft}
\begin{thebibliography}{9}
\bibitem{ref:cita1}
T. Bangalter, G. M. de Homem-Christo. \textit{Random Acces Memory}. Vol. 8, 2013.

\end{thebibliography}
\end{flushleft}


\end{document}
