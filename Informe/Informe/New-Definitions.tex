%COLORES:
\definecolor{AzulFoot}{rgb}{0.682,0.809,0.926}	%RGB	%{174,206,235}
\definecolor{AzulInfo}{rgb}{0.180,0.455,0.710}	%RGB	%{46,116,181}
\definecolor{AzulTable}{rgb}{0.302,0.507,0.871}	%RGB	%{68,114,196}
\definecolor{PName}{rgb}{0.353,0.353,0.353}		%RGB	%{90,90,90}
\definecolor{mygreen}{rgb}{28,172,0} % color values Red, Green, Blue
\definecolor{mylilas}{rgb}{170,55,241}

%Change Font Size

% #1 = size, #2 = text
\newcommand{\setparagraphsize}[2]{{\fontsize{#1}{6}\selectfont#2 \par}}		%Cambia el size de todo el parrafo
\newcommand{\setlinesize}[2]{{\fontsize{#1}{6}\selectfont#2}}				%Cambia el font de una oración

%IMAGE IN TABLE:			%Ejemplo: \includeintable{.3}{ImagenesFactibilidad/pend}
\renewcommand\fbox{\fcolorbox{white}{white}}
\setlength{\fboxrule}{0pt}	%padding thickness
\setlength{\fboxsep}{4pt}	%border thickness
\newcommand{\includeintable}[2]{	
	\fbox{
		\begin{minipage}{#1\textwidth}
        	\includegraphics[width=\linewidth]{#2}
    	\end{minipage}
	}
}

%LINK IN REF
\newcommand{\reflink}[1]{		%LINK
	\href{#1}{#1}
}

%NOTAS:
\newcommand{\note}[1]{		%RED BIG NOTE (TODO)
	\begin{center}
		\huge{ \textcolor{red}{#1} }
	\end{center}
}

\newcommand{\lnote}[1]{{\fontsize{14}{6}\selectfont\textcolor{green}{#1}}}	%Notas pequeñas

\newcommand{\observacion}[2]{  \ifnumequal{1}{#1}{ { \todo[inline,backgroundcolor=red!25,bordercolor=red!100]{\textbf{Observación: #2}} } }{  }  }

\newcommand{\red}[1]{\textcolor{red}{#1}}

\newcommand{\TBD}{\textcolor{red}{(TBD)\xspace}}
\newcommand{\tbd}{\textcolor{red}{(TBD)\xspace}}

\newcommand{\TBC}{\textcolor{red}{(TBC)\xspace}}
\newcommand{\tbc}{\textcolor{red}{(TBC)\xspace}}

\newcommand{\quotes}[1]{``#1''}
\newcommand{\q}[1]{``#1''}

\newcommand{\ip}{192.168.0.10:1880\xspace}
\newcommand{\ipadmin}{192.168.0.10:1880/admin\xspace}
\newcommand{\nodered}{\textit{Node Red}\xspace}
\newcommand{\rspi}{Raspberry Pi\xspace}
\newcommand{\rpi}{R-Pi\xspace}
\def\precio{1000\xspace} %precio del producto 
\def\unidadespostfin{quince\xspace} %cuantos productos se podrian hacer luego de finalizar el proyecto


% Comandos para agregar elementos en tablas de acronimos y definiciones
\newcommand{\addacronym}[2]{\textbf{#1} & \begin{tabular}[l]{@{}l@{}}#2\end{tabular} \\ \hline}

% tabItem
\newcommand{\tabitem}{~~\llap{\textbullet}~~}
