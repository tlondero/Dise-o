%% IMPORTANTE:
% Verificar que esté \usepackage[dvipsnames]{xcolor}

%\usepackage{listingsutf8}
\usepackage{listings}

\renewcommand{\lstlistingname}{Código}

%LSTSET: Pone un recuadro y contador de linea en el codigo
\newcommand{\boxstyle}{
	\lstset{
		basicstyle=\sffamily\color{black},
		frame=single,
		numbers=left,
		numbersep=5pt,
		numberstyle=\color{gray},
		showspaces=false,
		showstringspaces=false
	}
}

\newcommand{\defaultstyle}{
	\lstset{
		basicstyle=\sffamily\color{white},
		frame=none,
		numbers=none,
		showspaces=true,
		showstringspaces=true
	}
}

\lstdefinelanguage{Python}{
  captionpos=b,
  comment=[l]{//},
  commentstyle={\color{gray}\ttfamily},
  emph={filter, first, firstOrNull, forEach, lazy, map, mapNotNull, println},
  emphstyle={\color{OrangeRed}},
  identifierstyle=\color{black},
  keywords={False, None, True, and, as, assert, async, await, break, class, continue, def, del, elif, else, except, finally, for, from, global, if, import, in, is, lambda, nonlocal, not, or, pass, raise, return, try, while, with, yield},
  keywordstyle={\color{NavyBlue}\bfseries},
  morecomment=[s]{/*}{*/},
  morestring=[b]",
  morestring=[s]{"""*}{*"""},
  ndkeywords={@Deprecated, @JvmField, @JvmName, @JvmOverloads, @JvmStatic, @JvmSynthetic, Array, Byte, Double, Float, Int, Integer, Iterable, Long, Runnable, Short, String, Any, Unit, Nothing},
  ndkeywordstyle={\color{BurntOrange}\bfseries},
  sensitive=true,
  stringstyle={\color{ForestGreen}\ttfamily},
}


%Como usarlo:

%\begin{lstlisting}[caption={Simple code listing.}, label={lst:example1}, language=Kotlin]
%// this is a simple code listing:
%println("hello kotlin from latex")
%\end{lstlisting}

%Si se corta en 2 páginas distintas:

%\vspace{1mm}
%\noindent{\begin{minipage}{\linewidth}
%\begin{lstlisting}[...]
%...
%\end{lstlisting}
%\end{minipage}}


