% \iffalse meta-comment
%
% File: transparent.dtx
% Version: 2022-10-27 v1.5
% Info: Transparency via pdfTeX's color stack
%
% Copyright (C)
%    2007 Heiko Oberdiek
%    2016-2022 Oberdiek Package Support Group 2018
%    https://github.com/ho-tex/transparent/issues
%
% This work may be distributed and/or modified under the
% conditions of the LaTeX Project Public License, either
% version 1.3c of this license or (at your option) any later
% version. This version of this license is in
%    https://www.latex-project.org/lppl/lppl-1-3c.txt
% and the latest version of this license is in
%    https://www.latex-project.org/lppl.txt
% and version 1.3 or later is part of all distributions of
% LaTeX version 2005/12/01 or later.
%
% This work has the LPPL maintenance status "maintained".
%
% The Current Maintainers of this work are
% Heiko Oberdiek and the Oberdiek Package Support Group
% https://github.com/ho-tex/transparent/issues
%
% This work consists of the main source file transparent.dtx
% and the derived files
%    transparent.sty, transparent-nometadata.sty, transparent.pdf, 
%    transparent.ins, transparent.drv, transparent-example.tex.
%
% Distribution:
%    CTAN:macros/latex/contrib/transparent/transparent.dtx
%    CTAN:macros/latex/contrib/transparent/transparent.pdf
%
% Unpacking:
%    (a) If transparent.ins is present:
%           tex transparent.ins
%    (b) Without transparent.ins:
%           tex transparent.dtx
%    (c) If you insist on using LaTeX
%           latex \let\install=y% \iffalse meta-comment
%
% File: transparent.dtx
% Version: 2022-10-27 v1.5
% Info: Transparency via pdfTeX's color stack
%
% Copyright (C)
%    2007 Heiko Oberdiek
%    2016-2022 Oberdiek Package Support Group 2018
%    https://github.com/ho-tex/transparent/issues
%
% This work may be distributed and/or modified under the
% conditions of the LaTeX Project Public License, either
% version 1.3c of this license or (at your option) any later
% version. This version of this license is in
%    https://www.latex-project.org/lppl/lppl-1-3c.txt
% and the latest version of this license is in
%    https://www.latex-project.org/lppl.txt
% and version 1.3 or later is part of all distributions of
% LaTeX version 2005/12/01 or later.
%
% This work has the LPPL maintenance status "maintained".
%
% The Current Maintainers of this work are
% Heiko Oberdiek and the Oberdiek Package Support Group
% https://github.com/ho-tex/transparent/issues
%
% This work consists of the main source file transparent.dtx
% and the derived files
%    transparent.sty, transparent-nometadata.sty, transparent.pdf, 
%    transparent.ins, transparent.drv, transparent-example.tex.
%
% Distribution:
%    CTAN:macros/latex/contrib/transparent/transparent.dtx
%    CTAN:macros/latex/contrib/transparent/transparent.pdf
%
% Unpacking:
%    (a) If transparent.ins is present:
%           tex transparent.ins
%    (b) Without transparent.ins:
%           tex transparent.dtx
%    (c) If you insist on using LaTeX
%           latex \let\install=y% \iffalse meta-comment
%
% File: transparent.dtx
% Version: 2022-10-27 v1.5
% Info: Transparency via pdfTeX's color stack
%
% Copyright (C)
%    2007 Heiko Oberdiek
%    2016-2022 Oberdiek Package Support Group 2018
%    https://github.com/ho-tex/transparent/issues
%
% This work may be distributed and/or modified under the
% conditions of the LaTeX Project Public License, either
% version 1.3c of this license or (at your option) any later
% version. This version of this license is in
%    https://www.latex-project.org/lppl/lppl-1-3c.txt
% and the latest version of this license is in
%    https://www.latex-project.org/lppl.txt
% and version 1.3 or later is part of all distributions of
% LaTeX version 2005/12/01 or later.
%
% This work has the LPPL maintenance status "maintained".
%
% The Current Maintainers of this work are
% Heiko Oberdiek and the Oberdiek Package Support Group
% https://github.com/ho-tex/transparent/issues
%
% This work consists of the main source file transparent.dtx
% and the derived files
%    transparent.sty, transparent-nometadata.sty, transparent.pdf, 
%    transparent.ins, transparent.drv, transparent-example.tex.
%
% Distribution:
%    CTAN:macros/latex/contrib/transparent/transparent.dtx
%    CTAN:macros/latex/contrib/transparent/transparent.pdf
%
% Unpacking:
%    (a) If transparent.ins is present:
%           tex transparent.ins
%    (b) Without transparent.ins:
%           tex transparent.dtx
%    (c) If you insist on using LaTeX
%           latex \let\install=y% \iffalse meta-comment
%
% File: transparent.dtx
% Version: 2022-10-27 v1.5
% Info: Transparency via pdfTeX's color stack
%
% Copyright (C)
%    2007 Heiko Oberdiek
%    2016-2022 Oberdiek Package Support Group 2018
%    https://github.com/ho-tex/transparent/issues
%
% This work may be distributed and/or modified under the
% conditions of the LaTeX Project Public License, either
% version 1.3c of this license or (at your option) any later
% version. This version of this license is in
%    https://www.latex-project.org/lppl/lppl-1-3c.txt
% and the latest version of this license is in
%    https://www.latex-project.org/lppl.txt
% and version 1.3 or later is part of all distributions of
% LaTeX version 2005/12/01 or later.
%
% This work has the LPPL maintenance status "maintained".
%
% The Current Maintainers of this work are
% Heiko Oberdiek and the Oberdiek Package Support Group
% https://github.com/ho-tex/transparent/issues
%
% This work consists of the main source file transparent.dtx
% and the derived files
%    transparent.sty, transparent-nometadata.sty, transparent.pdf, 
%    transparent.ins, transparent.drv, transparent-example.tex.
%
% Distribution:
%    CTAN:macros/latex/contrib/transparent/transparent.dtx
%    CTAN:macros/latex/contrib/transparent/transparent.pdf
%
% Unpacking:
%    (a) If transparent.ins is present:
%           tex transparent.ins
%    (b) Without transparent.ins:
%           tex transparent.dtx
%    (c) If you insist on using LaTeX
%           latex \let\install=y\input{transparent.dtx}
%        (quote the arguments according to the demands of your shell)
%
% Documentation:
%    (a) If transparent.drv is present:
%           latex transparent.drv
%    (b) Without transparent.drv:
%           latex transparent.dtx; ...
%    The class ltxdoc loads the configuration file ltxdoc.cfg
%    if available. Here you can specify further options, e.g.
%    use A4 as paper format:
%       \PassOptionsToClass{a4paper}{article}
%
%    Programm calls to get the documentation (example):
%       pdflatex transparent.dtx
%       makeindex -s gind.ist transparent.idx
%       pdflatex transparent.dtx
%       makeindex -s gind.ist transparent.idx
%       pdflatex transparent.dtx
%
% Installation:
%    TDS:tex/latex/transparent/transparent.sty
%    TDS:tex/latex/transparent/transparent-nometadata.sty
%    TDS:doc/latex/transparent/transparent.pdf
%    TDS:source/latex/transparent/transparent.dtx
%
%<*ignore>
\begingroup
  \catcode123=1 %
  \catcode125=2 %
  \def\x{LaTeX2e}%
\expandafter\endgroup
\ifcase 0\ifx\install y1\fi\expandafter
         \ifx\csname processbatchFile\endcsname\relax\else1\fi
         \ifx\fmtname\x\else 1\fi\relax
\else\csname fi\endcsname
%</ignore>
%<*install>
\input docstrip.tex
\Msg{************************************************************************}
\Msg{* Installation}
\Msg{* Package: transparent 2022-10-27 v1.5 Transparency via pdfTeX's color stack (HO)}
\Msg{************************************************************************}

\keepsilent
\askforoverwritefalse

\let\MetaPrefix\relax
\preamble

This is a generated file.

Project: transparent
Version: 2022-10-27 v1.5

Copyright (C)
   2007 Heiko Oberdiek
   2016-2022 Oberdiek Package Support Group

This work may be distributed and/or modified under the
conditions of the LaTeX Project Public License, either
version 1.3c of this license or (at your option) any later
version. This version of this license is in
   https://www.latex-project.org/lppl/lppl-1-3c.txt
and the latest version of this license is in
   https://www.latex-project.org/lppl.txt
and version 1.3 or later is part of all distributions of
LaTeX version 2005/12/01 or later.

This work has the LPPL maintenance status "maintained".

The Current Maintainers of this work are
Heiko Oberdiek and the Oberdiek Package Support Group
https://github.com/ho-tex/transparent/issues


This work consists of the main source file transparent.dtx
and the derived files
   transparent.sty, transparent-nometadata.sty, transparent.pdf,
   transparent.ins, transparent.drv, transparent-example.tex.

\endpreamble
\let\MetaPrefix\DoubleperCent

\generate{%
  \file{transparent.ins}{\from{transparent.dtx}{install}}%
  \file{transparent.drv}{\from{transparent.dtx}{driver}}%
  \usedir{tex/latex/transparent}%
  \file{transparent-nometadata.sty}{\from{transparent.dtx}{package}}%
  \file{transparent.sty}{\from{transparent.dtx}{package-new}}%
  \usedir{doc/latex/transparent}%
  \file{transparent-example.tex}{\from{transparent.dtx}{example}}%
}

\catcode32=13\relax% active space
\let =\space%
\Msg{************************************************************************}
\Msg{*}
\Msg{* To finish the installation you have to move the following}
\Msg{* file into a directory searched by TeX:}
\Msg{*}
\Msg{*     transparent.sty}
\Msg{*     transparent-nometadata.sty}
\Msg{*}
\Msg{* To produce the documentation run the file `transparent.drv'}
\Msg{* through LaTeX.}
\Msg{*}
\Msg{* Happy TeXing!}
\Msg{*}
\Msg{************************************************************************}

\endbatchfile
%</install>
%<*ignore>
\fi
%</ignore>
%<*driver>
\NeedsTeXFormat{LaTeX2e}
\ProvidesFile{transparent.drv}%
  [2022-10-27 v1.5 Transparency via pdfTeX's color stack (HO)]%
\documentclass{ltxdoc}
\usepackage{holtxdoc}[2011/11/22]
\begin{document}
  \DocInput{transparent.dtx}%
\end{document}
%</driver>
% \fi
%
%
%
% \GetFileInfo{transparent.drv}
%
% \title{The \xpackage{transparent} package}
% \date{2022-10-27 v1.5}
% \author{Heiko Oberdiek\thanks
% {Please report any issues at \url{https://github.com/ho-tex/transparent/issues}}}
%
% \maketitle
%
% \begin{abstract}
% \pdfTeX\ and lua\TeX{} supports several color stacks. This
% package shows, how a separate color stack can be used for transparency,
% a property besides color, that also works across page break.
% \end{abstract}
%
% \tableofcontents
%
% \section{User interface}
%
% The package \xpackage{transparent} defines \cs{transparent} and
% \cs{texttransparent}. They are used like \cs{color} and \cs{textcolor}.
% The first argument is the transparency value between 0 and 1 where 0
% is fully transparent and 1 is opaque.
% 
% \section{Engine support}
% If the PDF management is loaded by using \verb+\DocumentMetadata{}+ at the beginning
% of the document the package makes use of \texttt{l3opacity} and
% can be used with all engines supported by \texttt{l3backend}.
% But only with \pdfTeX\ and lua\TeX{} the transparency will also work across page break.
% 
% 
% Without the PDF management, the package works only with \pdfTeX\ and lua\TeX{}.
% It then uses the original code by Heiko Oberdiek. It then uses the
% primitive \cs{pdfpageresources} and can clash with other packages that also use
% \cs{pdfpageresources}.
% 
%
% \section{Example for usage}
%    \begin{macrocode}
%<*example>
\documentclass[12pt]{article}

\usepackage{color}
\usepackage{transparent}

\begin{document}
\colorbox{yellow}{%
  \bfseries
  \color{blue}%
  Blue and %
  \transparent{0.6}%
  transparent blue%
}

\bigskip
Hello World
\texttransparent{0.5}{Hello\newpage World}
Hello World
\end{document}
%</example>
%    \end{macrocode}
%
% \StopEventually{
% }
%
% \section{Implementation}
% \subsection{New version using the methods of the PDF management} 
% Identification
%    \begin{macrocode}
%<*package-new>
\NeedsTeXFormat{LaTeX2e}[2020/10/01]
\ProvidesExplPackage{transparent}{2022-10-27}{1.5}
  {Transparency with color stacks}%
%    \end{macrocode}
% Testing if the pdfmanagement is used
%    \begin{macrocode}
\@ifl@t@r\fmtversion{2022-06-01}
  {
    \IfDocumentMetadataTF {}{\RequirePackage{transparent-nometadata}}
    \IfDocumentMetadataTF {}{\endinput}
  }
  {\RequirePackage{transparent-nometadata}}  
%    \end{macrocode}
%    \begin{macrocode}
\RequirePackage{l3opacity}

\NewDocumentCommand{\transparent} { m }
      {
       \opacity_select:n{\fp_eval:n{ min(max(0,#1),1) } }
      }

\NewDocumentCommand{\texttransparent}{m m}
  {
   \mode_leave_vertical:
   \group_begin:
     \transparent{#1}
     #2
   \group_end:
  }
%</package-new>
%    \end{macrocode}
% \subsection{Old version without the PDF management}
%    \begin{macrocode}
%<*package>
\NeedsTeXFormat{LaTeX2e}
\ProvidesPackage{transparent-nometadata}%
  [2022-10-27 v1.5 Transparency via pdfTeX's color stack (HO)]%
%    \end{macrocode}
%
% \subsection{Initial checks}
%
% \subsubsection{Check for \pdfTeX\ in PDF mode}
%    \begin{macrocode}
\RequirePackage{iftex}
\ifpdf
\else
  \PackageWarningNoLine{transparent}{%
    Loading aborted, because pdfTeX is not running in PDF mode%
  }%
  \expandafter\endinput
\fi
%    \end{macrocode}
%
% \subsubsection{Check \pdfTeX\ or Lua\TeX\ version}
%    \begin{macrocode}
\ifx\pdfextension\@undefined
  \let\TRP@pdfcolorstackinit\pdfcolorstackinit
  \let\TRP@pdfpageresources\pdfpageresources
  \let\TRP@pdfcolorstack\pdfcolorstack
\else
  \def\TRP@pdfcolorstackinit           {\pdffeedback colorstackinit}
  \protected\edef\TRP@pdfpageresources {\pdfvariable pageresources}
  \protected\def\TRP@pdfcolorstack     {\pdfextension colorstack}
\fi
%    \end{macrocode}
%
%    \begin{macrocode}
\ifcsname TRP@pdfcolorstackinit\endcsname\else
  \PackageWarningNoLine{transparent}{%
    Your pdfTeX version does not support color stacks%
  }%
  \expandafter\endinput
\fi
%    \end{macrocode}
%
% \subsection{Compatibility with pgf}
%  https://github.com/ho-tex/transparent/issues/1
%    \begin{macrocode}
\AtBeginDocument
 {%
   \ifcsname pgfutil@addpdfresource@extgs\endcsname
    \let\TRP@addresource\relax
    \pgfutil@addpdfresource@extgs{\TRP@list}%
   \fi
 }
%    \end{macrocode}
% \subsection{Transparency}
%
%    The setting for the different transparency values must
%    be added to the page resources. In the first run the values
%    are recorded in the \xfile{.aux} file. In the second run
%    the values are set and transparency is available.
%    \begin{macrocode}
\RequirePackage{auxhook}
\AddLineBeginAux{%
  \string\providecommand{\string\transparent@use}[1]{}%
}
\gdef\TRP@list{/TRP1<</ca 1/CA 1>>}
\def\transparent@use#1{%
  \@ifundefined{TRP#1}{%
    \g@addto@macro\TRP@list{%
      /TRP#1<</ca #1/CA #1>>%
    }%
    \expandafter\gdef\csname TRP#1\endcsname{/TRP#1 gs}%
  }{%
    % #1 is already known, nothing to do
  }%
}
\AtBeginDocument{%
  \TRP@addresource
  \let\transparent@use\@gobble
}
%    \end{macrocode}
%    Unhappily the interface setting page resources is very
%    poor, only a token register \cs{pdfpageresources}.
%    Thus this package tries to be cooperative in the way that
%    it embeds the previous contents of \cs{pdfpageresources}.
%    However it does not solve the problem, if several packages
%    want to set |/ExtGState|.
%    \begin{macrocode}
\def\TRP@addresource{%
  \begingroup
    \edef\x{\endgroup
      \TRP@pdfpageresources{%
        \the\TRP@pdfpageresources
        /ExtGState<<\TRP@list>>%
      }%
    }%
  \x
}
\newif\ifTRP@rerun
\xdef\TRP@colorstack{%
  \TRP@pdfcolorstackinit page direct{/TRP1 gs}%
}
%    \end{macrocode}
%    \begin{macro}{\transparent}
%    \begin{macrocode}
\newcommand*{\transparent}[1]{%
  \begingroup
    \dimen@=#1\p@\relax
    \ifdim\dimen@>\p@
      \dimen@=\p@
    \fi
    \ifdim\dimen@<\z@
      \dimen@=\z@
    \fi
    \ifdim\dimen@=\p@
      \def\x{1}%
    \else
      \ifdim\dimen@=\z@
        \def\x{0}%
      \else
        \edef\x{\strip@pt\dimen@}%
        \edef\x{\expandafter\@gobble\x}%
      \fi
    \fi
    \if@filesw
      \immediate\write\@auxout{%
        \string\transparent@use{\x}%
      }%
    \fi
    \edef\x{\endgroup
      \def\noexpand\transparent@current{\x}%
    }%
  \x
  \transparent@set
}
%    \end{macrocode}
%    \end{macro}
%    \begin{macrocode}
\AtEndDocument{%
  \ifTRP@rerun
    \PackageWarningNoLine{transparent}{%
      Rerun to get transparencies right%
    }%
  \fi
}
\def\transparent@current{/TRP1 gs}
\def\transparent@set{%
  \@ifundefined{TRP\transparent@current}{%
    \global\TRP@reruntrue
  }{%
    \TRP@pdfcolorstack\TRP@colorstack push{%
      \csname TRP\transparent@current\endcsname
    }%
    \aftergroup\transparent@reset
  }%
}
\def\transparent@reset{%
  \TRP@pdfcolorstack\TRP@colorstack pop\relax
}
%    \end{macrocode}
%    \begin{macro}{\texttransparent}
%    \begin{macrocode}
\newcommand*{\texttransparent}[2]{%
  \protect\leavevmode
  \begingroup
    \transparent{#1}%
    #2%
  \endgroup
}
%    \end{macrocode}
%    \end{macro}
%
%    \begin{macrocode}
%</package>
%    \end{macrocode}
%
% \section{Installation}
%
% \subsection{Download}
%
% \paragraph{Package.} This package is available on
% CTAN\footnote{\CTANpkg{transparent}}:
% \begin{description}
% \item[\CTAN{macros/latex/contrib/transparent/transparent.dtx}] The source file.
% \item[\CTAN{macros/latex/contrib/transparent/transparent.pdf}] Documentation.
% \end{description}
%
%
%
% \subsection{Package installation}
%
% \paragraph{Unpacking.} The \xfile{.dtx} file is a self-extracting
% \docstrip\ archive. The files are extracted by running the
% \xfile{.dtx} through \plainTeX:
% \begin{quote}
%   \verb|tex transparent.dtx|
% \end{quote}
%
% \paragraph{TDS.} Now the different files must be moved into
% the different directories in your installation TDS tree
% (also known as \xfile{texmf} tree):
% \begin{quote}
% \def\t{^^A
% \begin{tabular}{@{}>{\ttfamily}l@{ $\rightarrow$ }>{\ttfamily}l@{}}
%   transparent.sty & tex/latex/transparent/transparent.sty\\
%   transparent-nometadata.sty & tex/latex/transparent/transparent-nometadata.sty\\   
%   transparent.pdf & doc/latex/transparent/transparent.pdf\\
%   transparent-example.tex & doc/latex/transparent/transparent-example.tex\\
%   transparent.dtx & source/latex/transparent/transparent.dtx\\
% \end{tabular}^^A
% }^^A
% \sbox0{\t}^^A
% \ifdim\wd0>\linewidth
%   \begingroup
%     \advance\linewidth by\leftmargin
%     \advance\linewidth by\rightmargin
%   \edef\x{\endgroup
%     \def\noexpand\lw{\the\linewidth}^^A
%   }\x
%   \def\lwbox{^^A
%     \leavevmode
%     \hbox to \linewidth{^^A
%       \kern-\leftmargin\relax
%       \hss
%       \usebox0
%       \hss
%       \kern-\rightmargin\relax
%     }^^A
%   }^^A
%   \ifdim\wd0>\lw
%     \sbox0{\small\t}^^A
%     \ifdim\wd0>\linewidth
%       \ifdim\wd0>\lw
%         \sbox0{\footnotesize\t}^^A
%         \ifdim\wd0>\linewidth
%           \ifdim\wd0>\lw
%             \sbox0{\scriptsize\t}^^A
%             \ifdim\wd0>\linewidth
%               \ifdim\wd0>\lw
%                 \sbox0{\tiny\t}^^A
%                 \ifdim\wd0>\linewidth
%                   \lwbox
%                 \else
%                   \usebox0
%                 \fi
%               \else
%                 \lwbox
%               \fi
%             \else
%               \usebox0
%             \fi
%           \else
%             \lwbox
%           \fi
%         \else
%           \usebox0
%         \fi
%       \else
%         \lwbox
%       \fi
%     \else
%       \usebox0
%     \fi
%   \else
%     \lwbox
%   \fi
% \else
%   \usebox0
% \fi
% \end{quote}
% If you have a \xfile{docstrip.cfg} that configures and enables \docstrip's
% TDS installing feature, then some files can already be in the right
% place, see the documentation of \docstrip.
%
% \subsection{Refresh file name databases}
%
% If your \TeX~distribution
% (\TeX\,Live, \mikTeX, \dots) relies on file name databases, you must refresh
% these. For example, \TeX\,Live\ users run \verb|texhash| or
% \verb|mktexlsr|.
%
% \subsection{Some details for the interested}
%
% \paragraph{Unpacking with \LaTeX.}
% The \xfile{.dtx} chooses its action depending on the format:
% \begin{description}
% \item[\plainTeX:] Run \docstrip\ and extract the files.
% \item[\LaTeX:] Generate the documentation.
% \end{description}
% If you insist on using \LaTeX\ for \docstrip\ (really,
% \docstrip\ does not need \LaTeX), then inform the autodetect routine
% about your intention:
% \begin{quote}
%   \verb|latex \let\install=y\input{transparent.dtx}|
% \end{quote}
% Do not forget to quote the argument according to the demands
% of your shell.
%
% \paragraph{Generating the documentation.}
% You can use both the \xfile{.dtx} or the \xfile{.drv} to generate
% the documentation. The process can be configured by the
% configuration file \xfile{ltxdoc.cfg}. For instance, put this
% line into this file, if you want to have A4 as paper format:
% \begin{quote}
%   \verb|\PassOptionsToClass{a4paper}{article}|
% \end{quote}
% An example follows how to generate the
% documentation with pdf\LaTeX:
% \begin{quote}
%\begin{verbatim}
%pdflatex transparent.dtx
%makeindex -s gind.ist transparent.idx
%pdflatex transparent.dtx
%makeindex -s gind.ist transparent.idx
%pdflatex transparent.dtx
%\end{verbatim}
% \end{quote}
%
% \begin{History}
%   \begin{Version}{2007/01/08 v1.0}
%   \item
%     First version.
%   \end{Version}
%   \begin{Version}{2016/05/16 v1.1}
%   \item
%     Documentation updates.
%   \end{Version}
%   \begin{Version}{2018/09/10 v1.2}
%   \item
%     Update for LuaTeX, remove dependency on Pdf\TeX\ command names.
%   \end{Version}
%   \begin{Version}{2018/11/18 v1.3}
%   \item
%     Added code for pgf compatibility, see https://github.com/ho-tex/transparent/issues/1
%   \end{Version}
%   \begin{Version}{2019/11/29 v1.4}
%   \item
%     Documentation updates.
% \item
% Use \xpackage{iftex} package.
%   \end{Version}
%   \begin{Version}{2022-10-27 v1.5}
%   \item Made the package compatible with the PDF management.
%   \end{Version}
% \end{History}
%
% \PrintIndex
%
% \Finale
\endinput

%        (quote the arguments according to the demands of your shell)
%
% Documentation:
%    (a) If transparent.drv is present:
%           latex transparent.drv
%    (b) Without transparent.drv:
%           latex transparent.dtx; ...
%    The class ltxdoc loads the configuration file ltxdoc.cfg
%    if available. Here you can specify further options, e.g.
%    use A4 as paper format:
%       \PassOptionsToClass{a4paper}{article}
%
%    Programm calls to get the documentation (example):
%       pdflatex transparent.dtx
%       makeindex -s gind.ist transparent.idx
%       pdflatex transparent.dtx
%       makeindex -s gind.ist transparent.idx
%       pdflatex transparent.dtx
%
% Installation:
%    TDS:tex/latex/transparent/transparent.sty
%    TDS:tex/latex/transparent/transparent-nometadata.sty
%    TDS:doc/latex/transparent/transparent.pdf
%    TDS:source/latex/transparent/transparent.dtx
%
%<*ignore>
\begingroup
  \catcode123=1 %
  \catcode125=2 %
  \def\x{LaTeX2e}%
\expandafter\endgroup
\ifcase 0\ifx\install y1\fi\expandafter
         \ifx\csname processbatchFile\endcsname\relax\else1\fi
         \ifx\fmtname\x\else 1\fi\relax
\else\csname fi\endcsname
%</ignore>
%<*install>
\input docstrip.tex
\Msg{************************************************************************}
\Msg{* Installation}
\Msg{* Package: transparent 2022-10-27 v1.5 Transparency via pdfTeX's color stack (HO)}
\Msg{************************************************************************}

\keepsilent
\askforoverwritefalse

\let\MetaPrefix\relax
\preamble

This is a generated file.

Project: transparent
Version: 2022-10-27 v1.5

Copyright (C)
   2007 Heiko Oberdiek
   2016-2022 Oberdiek Package Support Group

This work may be distributed and/or modified under the
conditions of the LaTeX Project Public License, either
version 1.3c of this license or (at your option) any later
version. This version of this license is in
   https://www.latex-project.org/lppl/lppl-1-3c.txt
and the latest version of this license is in
   https://www.latex-project.org/lppl.txt
and version 1.3 or later is part of all distributions of
LaTeX version 2005/12/01 or later.

This work has the LPPL maintenance status "maintained".

The Current Maintainers of this work are
Heiko Oberdiek and the Oberdiek Package Support Group
https://github.com/ho-tex/transparent/issues


This work consists of the main source file transparent.dtx
and the derived files
   transparent.sty, transparent-nometadata.sty, transparent.pdf,
   transparent.ins, transparent.drv, transparent-example.tex.

\endpreamble
\let\MetaPrefix\DoubleperCent

\generate{%
  \file{transparent.ins}{\from{transparent.dtx}{install}}%
  \file{transparent.drv}{\from{transparent.dtx}{driver}}%
  \usedir{tex/latex/transparent}%
  \file{transparent-nometadata.sty}{\from{transparent.dtx}{package}}%
  \file{transparent.sty}{\from{transparent.dtx}{package-new}}%
  \usedir{doc/latex/transparent}%
  \file{transparent-example.tex}{\from{transparent.dtx}{example}}%
}

\catcode32=13\relax% active space
\let =\space%
\Msg{************************************************************************}
\Msg{*}
\Msg{* To finish the installation you have to move the following}
\Msg{* file into a directory searched by TeX:}
\Msg{*}
\Msg{*     transparent.sty}
\Msg{*     transparent-nometadata.sty}
\Msg{*}
\Msg{* To produce the documentation run the file `transparent.drv'}
\Msg{* through LaTeX.}
\Msg{*}
\Msg{* Happy TeXing!}
\Msg{*}
\Msg{************************************************************************}

\endbatchfile
%</install>
%<*ignore>
\fi
%</ignore>
%<*driver>
\NeedsTeXFormat{LaTeX2e}
\ProvidesFile{transparent.drv}%
  [2022-10-27 v1.5 Transparency via pdfTeX's color stack (HO)]%
\documentclass{ltxdoc}
\usepackage{holtxdoc}[2011/11/22]
\begin{document}
  \DocInput{transparent.dtx}%
\end{document}
%</driver>
% \fi
%
%
%
% \GetFileInfo{transparent.drv}
%
% \title{The \xpackage{transparent} package}
% \date{2022-10-27 v1.5}
% \author{Heiko Oberdiek\thanks
% {Please report any issues at \url{https://github.com/ho-tex/transparent/issues}}}
%
% \maketitle
%
% \begin{abstract}
% \pdfTeX\ and lua\TeX{} supports several color stacks. This
% package shows, how a separate color stack can be used for transparency,
% a property besides color, that also works across page break.
% \end{abstract}
%
% \tableofcontents
%
% \section{User interface}
%
% The package \xpackage{transparent} defines \cs{transparent} and
% \cs{texttransparent}. They are used like \cs{color} and \cs{textcolor}.
% The first argument is the transparency value between 0 and 1 where 0
% is fully transparent and 1 is opaque.
% 
% \section{Engine support}
% If the PDF management is loaded by using \verb+\DocumentMetadata{}+ at the beginning
% of the document the package makes use of \texttt{l3opacity} and
% can be used with all engines supported by \texttt{l3backend}.
% But only with \pdfTeX\ and lua\TeX{} the transparency will also work across page break.
% 
% 
% Without the PDF management, the package works only with \pdfTeX\ and lua\TeX{}.
% It then uses the original code by Heiko Oberdiek. It then uses the
% primitive \cs{pdfpageresources} and can clash with other packages that also use
% \cs{pdfpageresources}.
% 
%
% \section{Example for usage}
%    \begin{macrocode}
%<*example>
\documentclass[12pt]{article}

\usepackage{color}
\usepackage{transparent}

\begin{document}
\colorbox{yellow}{%
  \bfseries
  \color{blue}%
  Blue and %
  \transparent{0.6}%
  transparent blue%
}

\bigskip
Hello World
\texttransparent{0.5}{Hello\newpage World}
Hello World
\end{document}
%</example>
%    \end{macrocode}
%
% \StopEventually{
% }
%
% \section{Implementation}
% \subsection{New version using the methods of the PDF management} 
% Identification
%    \begin{macrocode}
%<*package-new>
\NeedsTeXFormat{LaTeX2e}[2020/10/01]
\ProvidesExplPackage{transparent}{2022-10-27}{1.5}
  {Transparency with color stacks}%
%    \end{macrocode}
% Testing if the pdfmanagement is used
%    \begin{macrocode}
\@ifl@t@r\fmtversion{2022-06-01}
  {
    \IfDocumentMetadataTF {}{\RequirePackage{transparent-nometadata}}
    \IfDocumentMetadataTF {}{\endinput}
  }
  {\RequirePackage{transparent-nometadata}}  
%    \end{macrocode}
%    \begin{macrocode}
\RequirePackage{l3opacity}

\NewDocumentCommand{\transparent} { m }
      {
       \opacity_select:n{\fp_eval:n{ min(max(0,#1),1) } }
      }

\NewDocumentCommand{\texttransparent}{m m}
  {
   \mode_leave_vertical:
   \group_begin:
     \transparent{#1}
     #2
   \group_end:
  }
%</package-new>
%    \end{macrocode}
% \subsection{Old version without the PDF management}
%    \begin{macrocode}
%<*package>
\NeedsTeXFormat{LaTeX2e}
\ProvidesPackage{transparent-nometadata}%
  [2022-10-27 v1.5 Transparency via pdfTeX's color stack (HO)]%
%    \end{macrocode}
%
% \subsection{Initial checks}
%
% \subsubsection{Check for \pdfTeX\ in PDF mode}
%    \begin{macrocode}
\RequirePackage{iftex}
\ifpdf
\else
  \PackageWarningNoLine{transparent}{%
    Loading aborted, because pdfTeX is not running in PDF mode%
  }%
  \expandafter\endinput
\fi
%    \end{macrocode}
%
% \subsubsection{Check \pdfTeX\ or Lua\TeX\ version}
%    \begin{macrocode}
\ifx\pdfextension\@undefined
  \let\TRP@pdfcolorstackinit\pdfcolorstackinit
  \let\TRP@pdfpageresources\pdfpageresources
  \let\TRP@pdfcolorstack\pdfcolorstack
\else
  \def\TRP@pdfcolorstackinit           {\pdffeedback colorstackinit}
  \protected\edef\TRP@pdfpageresources {\pdfvariable pageresources}
  \protected\def\TRP@pdfcolorstack     {\pdfextension colorstack}
\fi
%    \end{macrocode}
%
%    \begin{macrocode}
\ifcsname TRP@pdfcolorstackinit\endcsname\else
  \PackageWarningNoLine{transparent}{%
    Your pdfTeX version does not support color stacks%
  }%
  \expandafter\endinput
\fi
%    \end{macrocode}
%
% \subsection{Compatibility with pgf}
%  https://github.com/ho-tex/transparent/issues/1
%    \begin{macrocode}
\AtBeginDocument
 {%
   \ifcsname pgfutil@addpdfresource@extgs\endcsname
    \let\TRP@addresource\relax
    \pgfutil@addpdfresource@extgs{\TRP@list}%
   \fi
 }
%    \end{macrocode}
% \subsection{Transparency}
%
%    The setting for the different transparency values must
%    be added to the page resources. In the first run the values
%    are recorded in the \xfile{.aux} file. In the second run
%    the values are set and transparency is available.
%    \begin{macrocode}
\RequirePackage{auxhook}
\AddLineBeginAux{%
  \string\providecommand{\string\transparent@use}[1]{}%
}
\gdef\TRP@list{/TRP1<</ca 1/CA 1>>}
\def\transparent@use#1{%
  \@ifundefined{TRP#1}{%
    \g@addto@macro\TRP@list{%
      /TRP#1<</ca #1/CA #1>>%
    }%
    \expandafter\gdef\csname TRP#1\endcsname{/TRP#1 gs}%
  }{%
    % #1 is already known, nothing to do
  }%
}
\AtBeginDocument{%
  \TRP@addresource
  \let\transparent@use\@gobble
}
%    \end{macrocode}
%    Unhappily the interface setting page resources is very
%    poor, only a token register \cs{pdfpageresources}.
%    Thus this package tries to be cooperative in the way that
%    it embeds the previous contents of \cs{pdfpageresources}.
%    However it does not solve the problem, if several packages
%    want to set |/ExtGState|.
%    \begin{macrocode}
\def\TRP@addresource{%
  \begingroup
    \edef\x{\endgroup
      \TRP@pdfpageresources{%
        \the\TRP@pdfpageresources
        /ExtGState<<\TRP@list>>%
      }%
    }%
  \x
}
\newif\ifTRP@rerun
\xdef\TRP@colorstack{%
  \TRP@pdfcolorstackinit page direct{/TRP1 gs}%
}
%    \end{macrocode}
%    \begin{macro}{\transparent}
%    \begin{macrocode}
\newcommand*{\transparent}[1]{%
  \begingroup
    \dimen@=#1\p@\relax
    \ifdim\dimen@>\p@
      \dimen@=\p@
    \fi
    \ifdim\dimen@<\z@
      \dimen@=\z@
    \fi
    \ifdim\dimen@=\p@
      \def\x{1}%
    \else
      \ifdim\dimen@=\z@
        \def\x{0}%
      \else
        \edef\x{\strip@pt\dimen@}%
        \edef\x{\expandafter\@gobble\x}%
      \fi
    \fi
    \if@filesw
      \immediate\write\@auxout{%
        \string\transparent@use{\x}%
      }%
    \fi
    \edef\x{\endgroup
      \def\noexpand\transparent@current{\x}%
    }%
  \x
  \transparent@set
}
%    \end{macrocode}
%    \end{macro}
%    \begin{macrocode}
\AtEndDocument{%
  \ifTRP@rerun
    \PackageWarningNoLine{transparent}{%
      Rerun to get transparencies right%
    }%
  \fi
}
\def\transparent@current{/TRP1 gs}
\def\transparent@set{%
  \@ifundefined{TRP\transparent@current}{%
    \global\TRP@reruntrue
  }{%
    \TRP@pdfcolorstack\TRP@colorstack push{%
      \csname TRP\transparent@current\endcsname
    }%
    \aftergroup\transparent@reset
  }%
}
\def\transparent@reset{%
  \TRP@pdfcolorstack\TRP@colorstack pop\relax
}
%    \end{macrocode}
%    \begin{macro}{\texttransparent}
%    \begin{macrocode}
\newcommand*{\texttransparent}[2]{%
  \protect\leavevmode
  \begingroup
    \transparent{#1}%
    #2%
  \endgroup
}
%    \end{macrocode}
%    \end{macro}
%
%    \begin{macrocode}
%</package>
%    \end{macrocode}
%
% \section{Installation}
%
% \subsection{Download}
%
% \paragraph{Package.} This package is available on
% CTAN\footnote{\CTANpkg{transparent}}:
% \begin{description}
% \item[\CTAN{macros/latex/contrib/transparent/transparent.dtx}] The source file.
% \item[\CTAN{macros/latex/contrib/transparent/transparent.pdf}] Documentation.
% \end{description}
%
%
%
% \subsection{Package installation}
%
% \paragraph{Unpacking.} The \xfile{.dtx} file is a self-extracting
% \docstrip\ archive. The files are extracted by running the
% \xfile{.dtx} through \plainTeX:
% \begin{quote}
%   \verb|tex transparent.dtx|
% \end{quote}
%
% \paragraph{TDS.} Now the different files must be moved into
% the different directories in your installation TDS tree
% (also known as \xfile{texmf} tree):
% \begin{quote}
% \def\t{^^A
% \begin{tabular}{@{}>{\ttfamily}l@{ $\rightarrow$ }>{\ttfamily}l@{}}
%   transparent.sty & tex/latex/transparent/transparent.sty\\
%   transparent-nometadata.sty & tex/latex/transparent/transparent-nometadata.sty\\   
%   transparent.pdf & doc/latex/transparent/transparent.pdf\\
%   transparent-example.tex & doc/latex/transparent/transparent-example.tex\\
%   transparent.dtx & source/latex/transparent/transparent.dtx\\
% \end{tabular}^^A
% }^^A
% \sbox0{\t}^^A
% \ifdim\wd0>\linewidth
%   \begingroup
%     \advance\linewidth by\leftmargin
%     \advance\linewidth by\rightmargin
%   \edef\x{\endgroup
%     \def\noexpand\lw{\the\linewidth}^^A
%   }\x
%   \def\lwbox{^^A
%     \leavevmode
%     \hbox to \linewidth{^^A
%       \kern-\leftmargin\relax
%       \hss
%       \usebox0
%       \hss
%       \kern-\rightmargin\relax
%     }^^A
%   }^^A
%   \ifdim\wd0>\lw
%     \sbox0{\small\t}^^A
%     \ifdim\wd0>\linewidth
%       \ifdim\wd0>\lw
%         \sbox0{\footnotesize\t}^^A
%         \ifdim\wd0>\linewidth
%           \ifdim\wd0>\lw
%             \sbox0{\scriptsize\t}^^A
%             \ifdim\wd0>\linewidth
%               \ifdim\wd0>\lw
%                 \sbox0{\tiny\t}^^A
%                 \ifdim\wd0>\linewidth
%                   \lwbox
%                 \else
%                   \usebox0
%                 \fi
%               \else
%                 \lwbox
%               \fi
%             \else
%               \usebox0
%             \fi
%           \else
%             \lwbox
%           \fi
%         \else
%           \usebox0
%         \fi
%       \else
%         \lwbox
%       \fi
%     \else
%       \usebox0
%     \fi
%   \else
%     \lwbox
%   \fi
% \else
%   \usebox0
% \fi
% \end{quote}
% If you have a \xfile{docstrip.cfg} that configures and enables \docstrip's
% TDS installing feature, then some files can already be in the right
% place, see the documentation of \docstrip.
%
% \subsection{Refresh file name databases}
%
% If your \TeX~distribution
% (\TeX\,Live, \mikTeX, \dots) relies on file name databases, you must refresh
% these. For example, \TeX\,Live\ users run \verb|texhash| or
% \verb|mktexlsr|.
%
% \subsection{Some details for the interested}
%
% \paragraph{Unpacking with \LaTeX.}
% The \xfile{.dtx} chooses its action depending on the format:
% \begin{description}
% \item[\plainTeX:] Run \docstrip\ and extract the files.
% \item[\LaTeX:] Generate the documentation.
% \end{description}
% If you insist on using \LaTeX\ for \docstrip\ (really,
% \docstrip\ does not need \LaTeX), then inform the autodetect routine
% about your intention:
% \begin{quote}
%   \verb|latex \let\install=y% \iffalse meta-comment
%
% File: transparent.dtx
% Version: 2022-10-27 v1.5
% Info: Transparency via pdfTeX's color stack
%
% Copyright (C)
%    2007 Heiko Oberdiek
%    2016-2022 Oberdiek Package Support Group 2018
%    https://github.com/ho-tex/transparent/issues
%
% This work may be distributed and/or modified under the
% conditions of the LaTeX Project Public License, either
% version 1.3c of this license or (at your option) any later
% version. This version of this license is in
%    https://www.latex-project.org/lppl/lppl-1-3c.txt
% and the latest version of this license is in
%    https://www.latex-project.org/lppl.txt
% and version 1.3 or later is part of all distributions of
% LaTeX version 2005/12/01 or later.
%
% This work has the LPPL maintenance status "maintained".
%
% The Current Maintainers of this work are
% Heiko Oberdiek and the Oberdiek Package Support Group
% https://github.com/ho-tex/transparent/issues
%
% This work consists of the main source file transparent.dtx
% and the derived files
%    transparent.sty, transparent-nometadata.sty, transparent.pdf, 
%    transparent.ins, transparent.drv, transparent-example.tex.
%
% Distribution:
%    CTAN:macros/latex/contrib/transparent/transparent.dtx
%    CTAN:macros/latex/contrib/transparent/transparent.pdf
%
% Unpacking:
%    (a) If transparent.ins is present:
%           tex transparent.ins
%    (b) Without transparent.ins:
%           tex transparent.dtx
%    (c) If you insist on using LaTeX
%           latex \let\install=y\input{transparent.dtx}
%        (quote the arguments according to the demands of your shell)
%
% Documentation:
%    (a) If transparent.drv is present:
%           latex transparent.drv
%    (b) Without transparent.drv:
%           latex transparent.dtx; ...
%    The class ltxdoc loads the configuration file ltxdoc.cfg
%    if available. Here you can specify further options, e.g.
%    use A4 as paper format:
%       \PassOptionsToClass{a4paper}{article}
%
%    Programm calls to get the documentation (example):
%       pdflatex transparent.dtx
%       makeindex -s gind.ist transparent.idx
%       pdflatex transparent.dtx
%       makeindex -s gind.ist transparent.idx
%       pdflatex transparent.dtx
%
% Installation:
%    TDS:tex/latex/transparent/transparent.sty
%    TDS:tex/latex/transparent/transparent-nometadata.sty
%    TDS:doc/latex/transparent/transparent.pdf
%    TDS:source/latex/transparent/transparent.dtx
%
%<*ignore>
\begingroup
  \catcode123=1 %
  \catcode125=2 %
  \def\x{LaTeX2e}%
\expandafter\endgroup
\ifcase 0\ifx\install y1\fi\expandafter
         \ifx\csname processbatchFile\endcsname\relax\else1\fi
         \ifx\fmtname\x\else 1\fi\relax
\else\csname fi\endcsname
%</ignore>
%<*install>
\input docstrip.tex
\Msg{************************************************************************}
\Msg{* Installation}
\Msg{* Package: transparent 2022-10-27 v1.5 Transparency via pdfTeX's color stack (HO)}
\Msg{************************************************************************}

\keepsilent
\askforoverwritefalse

\let\MetaPrefix\relax
\preamble

This is a generated file.

Project: transparent
Version: 2022-10-27 v1.5

Copyright (C)
   2007 Heiko Oberdiek
   2016-2022 Oberdiek Package Support Group

This work may be distributed and/or modified under the
conditions of the LaTeX Project Public License, either
version 1.3c of this license or (at your option) any later
version. This version of this license is in
   https://www.latex-project.org/lppl/lppl-1-3c.txt
and the latest version of this license is in
   https://www.latex-project.org/lppl.txt
and version 1.3 or later is part of all distributions of
LaTeX version 2005/12/01 or later.

This work has the LPPL maintenance status "maintained".

The Current Maintainers of this work are
Heiko Oberdiek and the Oberdiek Package Support Group
https://github.com/ho-tex/transparent/issues


This work consists of the main source file transparent.dtx
and the derived files
   transparent.sty, transparent-nometadata.sty, transparent.pdf,
   transparent.ins, transparent.drv, transparent-example.tex.

\endpreamble
\let\MetaPrefix\DoubleperCent

\generate{%
  \file{transparent.ins}{\from{transparent.dtx}{install}}%
  \file{transparent.drv}{\from{transparent.dtx}{driver}}%
  \usedir{tex/latex/transparent}%
  \file{transparent-nometadata.sty}{\from{transparent.dtx}{package}}%
  \file{transparent.sty}{\from{transparent.dtx}{package-new}}%
  \usedir{doc/latex/transparent}%
  \file{transparent-example.tex}{\from{transparent.dtx}{example}}%
}

\catcode32=13\relax% active space
\let =\space%
\Msg{************************************************************************}
\Msg{*}
\Msg{* To finish the installation you have to move the following}
\Msg{* file into a directory searched by TeX:}
\Msg{*}
\Msg{*     transparent.sty}
\Msg{*     transparent-nometadata.sty}
\Msg{*}
\Msg{* To produce the documentation run the file `transparent.drv'}
\Msg{* through LaTeX.}
\Msg{*}
\Msg{* Happy TeXing!}
\Msg{*}
\Msg{************************************************************************}

\endbatchfile
%</install>
%<*ignore>
\fi
%</ignore>
%<*driver>
\NeedsTeXFormat{LaTeX2e}
\ProvidesFile{transparent.drv}%
  [2022-10-27 v1.5 Transparency via pdfTeX's color stack (HO)]%
\documentclass{ltxdoc}
\usepackage{holtxdoc}[2011/11/22]
\begin{document}
  \DocInput{transparent.dtx}%
\end{document}
%</driver>
% \fi
%
%
%
% \GetFileInfo{transparent.drv}
%
% \title{The \xpackage{transparent} package}
% \date{2022-10-27 v1.5}
% \author{Heiko Oberdiek\thanks
% {Please report any issues at \url{https://github.com/ho-tex/transparent/issues}}}
%
% \maketitle
%
% \begin{abstract}
% \pdfTeX\ and lua\TeX{} supports several color stacks. This
% package shows, how a separate color stack can be used for transparency,
% a property besides color, that also works across page break.
% \end{abstract}
%
% \tableofcontents
%
% \section{User interface}
%
% The package \xpackage{transparent} defines \cs{transparent} and
% \cs{texttransparent}. They are used like \cs{color} and \cs{textcolor}.
% The first argument is the transparency value between 0 and 1 where 0
% is fully transparent and 1 is opaque.
% 
% \section{Engine support}
% If the PDF management is loaded by using \verb+\DocumentMetadata{}+ at the beginning
% of the document the package makes use of \texttt{l3opacity} and
% can be used with all engines supported by \texttt{l3backend}.
% But only with \pdfTeX\ and lua\TeX{} the transparency will also work across page break.
% 
% 
% Without the PDF management, the package works only with \pdfTeX\ and lua\TeX{}.
% It then uses the original code by Heiko Oberdiek. It then uses the
% primitive \cs{pdfpageresources} and can clash with other packages that also use
% \cs{pdfpageresources}.
% 
%
% \section{Example for usage}
%    \begin{macrocode}
%<*example>
\documentclass[12pt]{article}

\usepackage{color}
\usepackage{transparent}

\begin{document}
\colorbox{yellow}{%
  \bfseries
  \color{blue}%
  Blue and %
  \transparent{0.6}%
  transparent blue%
}

\bigskip
Hello World
\texttransparent{0.5}{Hello\newpage World}
Hello World
\end{document}
%</example>
%    \end{macrocode}
%
% \StopEventually{
% }
%
% \section{Implementation}
% \subsection{New version using the methods of the PDF management} 
% Identification
%    \begin{macrocode}
%<*package-new>
\NeedsTeXFormat{LaTeX2e}[2020/10/01]
\ProvidesExplPackage{transparent}{2022-10-27}{1.5}
  {Transparency with color stacks}%
%    \end{macrocode}
% Testing if the pdfmanagement is used
%    \begin{macrocode}
\@ifl@t@r\fmtversion{2022-06-01}
  {
    \IfDocumentMetadataTF {}{\RequirePackage{transparent-nometadata}}
    \IfDocumentMetadataTF {}{\endinput}
  }
  {\RequirePackage{transparent-nometadata}}  
%    \end{macrocode}
%    \begin{macrocode}
\RequirePackage{l3opacity}

\NewDocumentCommand{\transparent} { m }
      {
       \opacity_select:n{\fp_eval:n{ min(max(0,#1),1) } }
      }

\NewDocumentCommand{\texttransparent}{m m}
  {
   \mode_leave_vertical:
   \group_begin:
     \transparent{#1}
     #2
   \group_end:
  }
%</package-new>
%    \end{macrocode}
% \subsection{Old version without the PDF management}
%    \begin{macrocode}
%<*package>
\NeedsTeXFormat{LaTeX2e}
\ProvidesPackage{transparent-nometadata}%
  [2022-10-27 v1.5 Transparency via pdfTeX's color stack (HO)]%
%    \end{macrocode}
%
% \subsection{Initial checks}
%
% \subsubsection{Check for \pdfTeX\ in PDF mode}
%    \begin{macrocode}
\RequirePackage{iftex}
\ifpdf
\else
  \PackageWarningNoLine{transparent}{%
    Loading aborted, because pdfTeX is not running in PDF mode%
  }%
  \expandafter\endinput
\fi
%    \end{macrocode}
%
% \subsubsection{Check \pdfTeX\ or Lua\TeX\ version}
%    \begin{macrocode}
\ifx\pdfextension\@undefined
  \let\TRP@pdfcolorstackinit\pdfcolorstackinit
  \let\TRP@pdfpageresources\pdfpageresources
  \let\TRP@pdfcolorstack\pdfcolorstack
\else
  \def\TRP@pdfcolorstackinit           {\pdffeedback colorstackinit}
  \protected\edef\TRP@pdfpageresources {\pdfvariable pageresources}
  \protected\def\TRP@pdfcolorstack     {\pdfextension colorstack}
\fi
%    \end{macrocode}
%
%    \begin{macrocode}
\ifcsname TRP@pdfcolorstackinit\endcsname\else
  \PackageWarningNoLine{transparent}{%
    Your pdfTeX version does not support color stacks%
  }%
  \expandafter\endinput
\fi
%    \end{macrocode}
%
% \subsection{Compatibility with pgf}
%  https://github.com/ho-tex/transparent/issues/1
%    \begin{macrocode}
\AtBeginDocument
 {%
   \ifcsname pgfutil@addpdfresource@extgs\endcsname
    \let\TRP@addresource\relax
    \pgfutil@addpdfresource@extgs{\TRP@list}%
   \fi
 }
%    \end{macrocode}
% \subsection{Transparency}
%
%    The setting for the different transparency values must
%    be added to the page resources. In the first run the values
%    are recorded in the \xfile{.aux} file. In the second run
%    the values are set and transparency is available.
%    \begin{macrocode}
\RequirePackage{auxhook}
\AddLineBeginAux{%
  \string\providecommand{\string\transparent@use}[1]{}%
}
\gdef\TRP@list{/TRP1<</ca 1/CA 1>>}
\def\transparent@use#1{%
  \@ifundefined{TRP#1}{%
    \g@addto@macro\TRP@list{%
      /TRP#1<</ca #1/CA #1>>%
    }%
    \expandafter\gdef\csname TRP#1\endcsname{/TRP#1 gs}%
  }{%
    % #1 is already known, nothing to do
  }%
}
\AtBeginDocument{%
  \TRP@addresource
  \let\transparent@use\@gobble
}
%    \end{macrocode}
%    Unhappily the interface setting page resources is very
%    poor, only a token register \cs{pdfpageresources}.
%    Thus this package tries to be cooperative in the way that
%    it embeds the previous contents of \cs{pdfpageresources}.
%    However it does not solve the problem, if several packages
%    want to set |/ExtGState|.
%    \begin{macrocode}
\def\TRP@addresource{%
  \begingroup
    \edef\x{\endgroup
      \TRP@pdfpageresources{%
        \the\TRP@pdfpageresources
        /ExtGState<<\TRP@list>>%
      }%
    }%
  \x
}
\newif\ifTRP@rerun
\xdef\TRP@colorstack{%
  \TRP@pdfcolorstackinit page direct{/TRP1 gs}%
}
%    \end{macrocode}
%    \begin{macro}{\transparent}
%    \begin{macrocode}
\newcommand*{\transparent}[1]{%
  \begingroup
    \dimen@=#1\p@\relax
    \ifdim\dimen@>\p@
      \dimen@=\p@
    \fi
    \ifdim\dimen@<\z@
      \dimen@=\z@
    \fi
    \ifdim\dimen@=\p@
      \def\x{1}%
    \else
      \ifdim\dimen@=\z@
        \def\x{0}%
      \else
        \edef\x{\strip@pt\dimen@}%
        \edef\x{\expandafter\@gobble\x}%
      \fi
    \fi
    \if@filesw
      \immediate\write\@auxout{%
        \string\transparent@use{\x}%
      }%
    \fi
    \edef\x{\endgroup
      \def\noexpand\transparent@current{\x}%
    }%
  \x
  \transparent@set
}
%    \end{macrocode}
%    \end{macro}
%    \begin{macrocode}
\AtEndDocument{%
  \ifTRP@rerun
    \PackageWarningNoLine{transparent}{%
      Rerun to get transparencies right%
    }%
  \fi
}
\def\transparent@current{/TRP1 gs}
\def\transparent@set{%
  \@ifundefined{TRP\transparent@current}{%
    \global\TRP@reruntrue
  }{%
    \TRP@pdfcolorstack\TRP@colorstack push{%
      \csname TRP\transparent@current\endcsname
    }%
    \aftergroup\transparent@reset
  }%
}
\def\transparent@reset{%
  \TRP@pdfcolorstack\TRP@colorstack pop\relax
}
%    \end{macrocode}
%    \begin{macro}{\texttransparent}
%    \begin{macrocode}
\newcommand*{\texttransparent}[2]{%
  \protect\leavevmode
  \begingroup
    \transparent{#1}%
    #2%
  \endgroup
}
%    \end{macrocode}
%    \end{macro}
%
%    \begin{macrocode}
%</package>
%    \end{macrocode}
%
% \section{Installation}
%
% \subsection{Download}
%
% \paragraph{Package.} This package is available on
% CTAN\footnote{\CTANpkg{transparent}}:
% \begin{description}
% \item[\CTAN{macros/latex/contrib/transparent/transparent.dtx}] The source file.
% \item[\CTAN{macros/latex/contrib/transparent/transparent.pdf}] Documentation.
% \end{description}
%
%
%
% \subsection{Package installation}
%
% \paragraph{Unpacking.} The \xfile{.dtx} file is a self-extracting
% \docstrip\ archive. The files are extracted by running the
% \xfile{.dtx} through \plainTeX:
% \begin{quote}
%   \verb|tex transparent.dtx|
% \end{quote}
%
% \paragraph{TDS.} Now the different files must be moved into
% the different directories in your installation TDS tree
% (also known as \xfile{texmf} tree):
% \begin{quote}
% \def\t{^^A
% \begin{tabular}{@{}>{\ttfamily}l@{ $\rightarrow$ }>{\ttfamily}l@{}}
%   transparent.sty & tex/latex/transparent/transparent.sty\\
%   transparent-nometadata.sty & tex/latex/transparent/transparent-nometadata.sty\\   
%   transparent.pdf & doc/latex/transparent/transparent.pdf\\
%   transparent-example.tex & doc/latex/transparent/transparent-example.tex\\
%   transparent.dtx & source/latex/transparent/transparent.dtx\\
% \end{tabular}^^A
% }^^A
% \sbox0{\t}^^A
% \ifdim\wd0>\linewidth
%   \begingroup
%     \advance\linewidth by\leftmargin
%     \advance\linewidth by\rightmargin
%   \edef\x{\endgroup
%     \def\noexpand\lw{\the\linewidth}^^A
%   }\x
%   \def\lwbox{^^A
%     \leavevmode
%     \hbox to \linewidth{^^A
%       \kern-\leftmargin\relax
%       \hss
%       \usebox0
%       \hss
%       \kern-\rightmargin\relax
%     }^^A
%   }^^A
%   \ifdim\wd0>\lw
%     \sbox0{\small\t}^^A
%     \ifdim\wd0>\linewidth
%       \ifdim\wd0>\lw
%         \sbox0{\footnotesize\t}^^A
%         \ifdim\wd0>\linewidth
%           \ifdim\wd0>\lw
%             \sbox0{\scriptsize\t}^^A
%             \ifdim\wd0>\linewidth
%               \ifdim\wd0>\lw
%                 \sbox0{\tiny\t}^^A
%                 \ifdim\wd0>\linewidth
%                   \lwbox
%                 \else
%                   \usebox0
%                 \fi
%               \else
%                 \lwbox
%               \fi
%             \else
%               \usebox0
%             \fi
%           \else
%             \lwbox
%           \fi
%         \else
%           \usebox0
%         \fi
%       \else
%         \lwbox
%       \fi
%     \else
%       \usebox0
%     \fi
%   \else
%     \lwbox
%   \fi
% \else
%   \usebox0
% \fi
% \end{quote}
% If you have a \xfile{docstrip.cfg} that configures and enables \docstrip's
% TDS installing feature, then some files can already be in the right
% place, see the documentation of \docstrip.
%
% \subsection{Refresh file name databases}
%
% If your \TeX~distribution
% (\TeX\,Live, \mikTeX, \dots) relies on file name databases, you must refresh
% these. For example, \TeX\,Live\ users run \verb|texhash| or
% \verb|mktexlsr|.
%
% \subsection{Some details for the interested}
%
% \paragraph{Unpacking with \LaTeX.}
% The \xfile{.dtx} chooses its action depending on the format:
% \begin{description}
% \item[\plainTeX:] Run \docstrip\ and extract the files.
% \item[\LaTeX:] Generate the documentation.
% \end{description}
% If you insist on using \LaTeX\ for \docstrip\ (really,
% \docstrip\ does not need \LaTeX), then inform the autodetect routine
% about your intention:
% \begin{quote}
%   \verb|latex \let\install=y\input{transparent.dtx}|
% \end{quote}
% Do not forget to quote the argument according to the demands
% of your shell.
%
% \paragraph{Generating the documentation.}
% You can use both the \xfile{.dtx} or the \xfile{.drv} to generate
% the documentation. The process can be configured by the
% configuration file \xfile{ltxdoc.cfg}. For instance, put this
% line into this file, if you want to have A4 as paper format:
% \begin{quote}
%   \verb|\PassOptionsToClass{a4paper}{article}|
% \end{quote}
% An example follows how to generate the
% documentation with pdf\LaTeX:
% \begin{quote}
%\begin{verbatim}
%pdflatex transparent.dtx
%makeindex -s gind.ist transparent.idx
%pdflatex transparent.dtx
%makeindex -s gind.ist transparent.idx
%pdflatex transparent.dtx
%\end{verbatim}
% \end{quote}
%
% \begin{History}
%   \begin{Version}{2007/01/08 v1.0}
%   \item
%     First version.
%   \end{Version}
%   \begin{Version}{2016/05/16 v1.1}
%   \item
%     Documentation updates.
%   \end{Version}
%   \begin{Version}{2018/09/10 v1.2}
%   \item
%     Update for LuaTeX, remove dependency on Pdf\TeX\ command names.
%   \end{Version}
%   \begin{Version}{2018/11/18 v1.3}
%   \item
%     Added code for pgf compatibility, see https://github.com/ho-tex/transparent/issues/1
%   \end{Version}
%   \begin{Version}{2019/11/29 v1.4}
%   \item
%     Documentation updates.
% \item
% Use \xpackage{iftex} package.
%   \end{Version}
%   \begin{Version}{2022-10-27 v1.5}
%   \item Made the package compatible with the PDF management.
%   \end{Version}
% \end{History}
%
% \PrintIndex
%
% \Finale
\endinput
|
% \end{quote}
% Do not forget to quote the argument according to the demands
% of your shell.
%
% \paragraph{Generating the documentation.}
% You can use both the \xfile{.dtx} or the \xfile{.drv} to generate
% the documentation. The process can be configured by the
% configuration file \xfile{ltxdoc.cfg}. For instance, put this
% line into this file, if you want to have A4 as paper format:
% \begin{quote}
%   \verb|\PassOptionsToClass{a4paper}{article}|
% \end{quote}
% An example follows how to generate the
% documentation with pdf\LaTeX:
% \begin{quote}
%\begin{verbatim}
%pdflatex transparent.dtx
%makeindex -s gind.ist transparent.idx
%pdflatex transparent.dtx
%makeindex -s gind.ist transparent.idx
%pdflatex transparent.dtx
%\end{verbatim}
% \end{quote}
%
% \begin{History}
%   \begin{Version}{2007/01/08 v1.0}
%   \item
%     First version.
%   \end{Version}
%   \begin{Version}{2016/05/16 v1.1}
%   \item
%     Documentation updates.
%   \end{Version}
%   \begin{Version}{2018/09/10 v1.2}
%   \item
%     Update for LuaTeX, remove dependency on Pdf\TeX\ command names.
%   \end{Version}
%   \begin{Version}{2018/11/18 v1.3}
%   \item
%     Added code for pgf compatibility, see https://github.com/ho-tex/transparent/issues/1
%   \end{Version}
%   \begin{Version}{2019/11/29 v1.4}
%   \item
%     Documentation updates.
% \item
% Use \xpackage{iftex} package.
%   \end{Version}
%   \begin{Version}{2022-10-27 v1.5}
%   \item Made the package compatible with the PDF management.
%   \end{Version}
% \end{History}
%
% \PrintIndex
%
% \Finale
\endinput

%        (quote the arguments according to the demands of your shell)
%
% Documentation:
%    (a) If transparent.drv is present:
%           latex transparent.drv
%    (b) Without transparent.drv:
%           latex transparent.dtx; ...
%    The class ltxdoc loads the configuration file ltxdoc.cfg
%    if available. Here you can specify further options, e.g.
%    use A4 as paper format:
%       \PassOptionsToClass{a4paper}{article}
%
%    Programm calls to get the documentation (example):
%       pdflatex transparent.dtx
%       makeindex -s gind.ist transparent.idx
%       pdflatex transparent.dtx
%       makeindex -s gind.ist transparent.idx
%       pdflatex transparent.dtx
%
% Installation:
%    TDS:tex/latex/transparent/transparent.sty
%    TDS:tex/latex/transparent/transparent-nometadata.sty
%    TDS:doc/latex/transparent/transparent.pdf
%    TDS:source/latex/transparent/transparent.dtx
%
%<*ignore>
\begingroup
  \catcode123=1 %
  \catcode125=2 %
  \def\x{LaTeX2e}%
\expandafter\endgroup
\ifcase 0\ifx\install y1\fi\expandafter
         \ifx\csname processbatchFile\endcsname\relax\else1\fi
         \ifx\fmtname\x\else 1\fi\relax
\else\csname fi\endcsname
%</ignore>
%<*install>
\input docstrip.tex
\Msg{************************************************************************}
\Msg{* Installation}
\Msg{* Package: transparent 2022-10-27 v1.5 Transparency via pdfTeX's color stack (HO)}
\Msg{************************************************************************}

\keepsilent
\askforoverwritefalse

\let\MetaPrefix\relax
\preamble

This is a generated file.

Project: transparent
Version: 2022-10-27 v1.5

Copyright (C)
   2007 Heiko Oberdiek
   2016-2022 Oberdiek Package Support Group

This work may be distributed and/or modified under the
conditions of the LaTeX Project Public License, either
version 1.3c of this license or (at your option) any later
version. This version of this license is in
   https://www.latex-project.org/lppl/lppl-1-3c.txt
and the latest version of this license is in
   https://www.latex-project.org/lppl.txt
and version 1.3 or later is part of all distributions of
LaTeX version 2005/12/01 or later.

This work has the LPPL maintenance status "maintained".

The Current Maintainers of this work are
Heiko Oberdiek and the Oberdiek Package Support Group
https://github.com/ho-tex/transparent/issues


This work consists of the main source file transparent.dtx
and the derived files
   transparent.sty, transparent-nometadata.sty, transparent.pdf,
   transparent.ins, transparent.drv, transparent-example.tex.

\endpreamble
\let\MetaPrefix\DoubleperCent

\generate{%
  \file{transparent.ins}{\from{transparent.dtx}{install}}%
  \file{transparent.drv}{\from{transparent.dtx}{driver}}%
  \usedir{tex/latex/transparent}%
  \file{transparent-nometadata.sty}{\from{transparent.dtx}{package}}%
  \file{transparent.sty}{\from{transparent.dtx}{package-new}}%
  \usedir{doc/latex/transparent}%
  \file{transparent-example.tex}{\from{transparent.dtx}{example}}%
}

\catcode32=13\relax% active space
\let =\space%
\Msg{************************************************************************}
\Msg{*}
\Msg{* To finish the installation you have to move the following}
\Msg{* file into a directory searched by TeX:}
\Msg{*}
\Msg{*     transparent.sty}
\Msg{*     transparent-nometadata.sty}
\Msg{*}
\Msg{* To produce the documentation run the file `transparent.drv'}
\Msg{* through LaTeX.}
\Msg{*}
\Msg{* Happy TeXing!}
\Msg{*}
\Msg{************************************************************************}

\endbatchfile
%</install>
%<*ignore>
\fi
%</ignore>
%<*driver>
\NeedsTeXFormat{LaTeX2e}
\ProvidesFile{transparent.drv}%
  [2022-10-27 v1.5 Transparency via pdfTeX's color stack (HO)]%
\documentclass{ltxdoc}
\usepackage{holtxdoc}[2011/11/22]
\begin{document}
  \DocInput{transparent.dtx}%
\end{document}
%</driver>
% \fi
%
%
%
% \GetFileInfo{transparent.drv}
%
% \title{The \xpackage{transparent} package}
% \date{2022-10-27 v1.5}
% \author{Heiko Oberdiek\thanks
% {Please report any issues at \url{https://github.com/ho-tex/transparent/issues}}}
%
% \maketitle
%
% \begin{abstract}
% \pdfTeX\ and lua\TeX{} supports several color stacks. This
% package shows, how a separate color stack can be used for transparency,
% a property besides color, that also works across page break.
% \end{abstract}
%
% \tableofcontents
%
% \section{User interface}
%
% The package \xpackage{transparent} defines \cs{transparent} and
% \cs{texttransparent}. They are used like \cs{color} and \cs{textcolor}.
% The first argument is the transparency value between 0 and 1 where 0
% is fully transparent and 1 is opaque.
% 
% \section{Engine support}
% If the PDF management is loaded by using \verb+\DocumentMetadata{}+ at the beginning
% of the document the package makes use of \texttt{l3opacity} and
% can be used with all engines supported by \texttt{l3backend}.
% But only with \pdfTeX\ and lua\TeX{} the transparency will also work across page break.
% 
% 
% Without the PDF management, the package works only with \pdfTeX\ and lua\TeX{}.
% It then uses the original code by Heiko Oberdiek. It then uses the
% primitive \cs{pdfpageresources} and can clash with other packages that also use
% \cs{pdfpageresources}.
% 
%
% \section{Example for usage}
%    \begin{macrocode}
%<*example>
\documentclass[12pt]{article}

\usepackage{color}
\usepackage{transparent}

\begin{document}
\colorbox{yellow}{%
  \bfseries
  \color{blue}%
  Blue and %
  \transparent{0.6}%
  transparent blue%
}

\bigskip
Hello World
\texttransparent{0.5}{Hello\newpage World}
Hello World
\end{document}
%</example>
%    \end{macrocode}
%
% \StopEventually{
% }
%
% \section{Implementation}
% \subsection{New version using the methods of the PDF management} 
% Identification
%    \begin{macrocode}
%<*package-new>
\NeedsTeXFormat{LaTeX2e}[2020/10/01]
\ProvidesExplPackage{transparent}{2022-10-27}{1.5}
  {Transparency with color stacks}%
%    \end{macrocode}
% Testing if the pdfmanagement is used
%    \begin{macrocode}
\@ifl@t@r\fmtversion{2022-06-01}
  {
    \IfDocumentMetadataTF {}{\RequirePackage{transparent-nometadata}}
    \IfDocumentMetadataTF {}{\endinput}
  }
  {\RequirePackage{transparent-nometadata}}  
%    \end{macrocode}
%    \begin{macrocode}
\RequirePackage{l3opacity}

\NewDocumentCommand{\transparent} { m }
      {
       \opacity_select:n{\fp_eval:n{ min(max(0,#1),1) } }
      }

\NewDocumentCommand{\texttransparent}{m m}
  {
   \mode_leave_vertical:
   \group_begin:
     \transparent{#1}
     #2
   \group_end:
  }
%</package-new>
%    \end{macrocode}
% \subsection{Old version without the PDF management}
%    \begin{macrocode}
%<*package>
\NeedsTeXFormat{LaTeX2e}
\ProvidesPackage{transparent-nometadata}%
  [2022-10-27 v1.5 Transparency via pdfTeX's color stack (HO)]%
%    \end{macrocode}
%
% \subsection{Initial checks}
%
% \subsubsection{Check for \pdfTeX\ in PDF mode}
%    \begin{macrocode}
\RequirePackage{iftex}
\ifpdf
\else
  \PackageWarningNoLine{transparent}{%
    Loading aborted, because pdfTeX is not running in PDF mode%
  }%
  \expandafter\endinput
\fi
%    \end{macrocode}
%
% \subsubsection{Check \pdfTeX\ or Lua\TeX\ version}
%    \begin{macrocode}
\ifx\pdfextension\@undefined
  \let\TRP@pdfcolorstackinit\pdfcolorstackinit
  \let\TRP@pdfpageresources\pdfpageresources
  \let\TRP@pdfcolorstack\pdfcolorstack
\else
  \def\TRP@pdfcolorstackinit           {\pdffeedback colorstackinit}
  \protected\edef\TRP@pdfpageresources {\pdfvariable pageresources}
  \protected\def\TRP@pdfcolorstack     {\pdfextension colorstack}
\fi
%    \end{macrocode}
%
%    \begin{macrocode}
\ifcsname TRP@pdfcolorstackinit\endcsname\else
  \PackageWarningNoLine{transparent}{%
    Your pdfTeX version does not support color stacks%
  }%
  \expandafter\endinput
\fi
%    \end{macrocode}
%
% \subsection{Compatibility with pgf}
%  https://github.com/ho-tex/transparent/issues/1
%    \begin{macrocode}
\AtBeginDocument
 {%
   \ifcsname pgfutil@addpdfresource@extgs\endcsname
    \let\TRP@addresource\relax
    \pgfutil@addpdfresource@extgs{\TRP@list}%
   \fi
 }
%    \end{macrocode}
% \subsection{Transparency}
%
%    The setting for the different transparency values must
%    be added to the page resources. In the first run the values
%    are recorded in the \xfile{.aux} file. In the second run
%    the values are set and transparency is available.
%    \begin{macrocode}
\RequirePackage{auxhook}
\AddLineBeginAux{%
  \string\providecommand{\string\transparent@use}[1]{}%
}
\gdef\TRP@list{/TRP1<</ca 1/CA 1>>}
\def\transparent@use#1{%
  \@ifundefined{TRP#1}{%
    \g@addto@macro\TRP@list{%
      /TRP#1<</ca #1/CA #1>>%
    }%
    \expandafter\gdef\csname TRP#1\endcsname{/TRP#1 gs}%
  }{%
    % #1 is already known, nothing to do
  }%
}
\AtBeginDocument{%
  \TRP@addresource
  \let\transparent@use\@gobble
}
%    \end{macrocode}
%    Unhappily the interface setting page resources is very
%    poor, only a token register \cs{pdfpageresources}.
%    Thus this package tries to be cooperative in the way that
%    it embeds the previous contents of \cs{pdfpageresources}.
%    However it does not solve the problem, if several packages
%    want to set |/ExtGState|.
%    \begin{macrocode}
\def\TRP@addresource{%
  \begingroup
    \edef\x{\endgroup
      \TRP@pdfpageresources{%
        \the\TRP@pdfpageresources
        /ExtGState<<\TRP@list>>%
      }%
    }%
  \x
}
\newif\ifTRP@rerun
\xdef\TRP@colorstack{%
  \TRP@pdfcolorstackinit page direct{/TRP1 gs}%
}
%    \end{macrocode}
%    \begin{macro}{\transparent}
%    \begin{macrocode}
\newcommand*{\transparent}[1]{%
  \begingroup
    \dimen@=#1\p@\relax
    \ifdim\dimen@>\p@
      \dimen@=\p@
    \fi
    \ifdim\dimen@<\z@
      \dimen@=\z@
    \fi
    \ifdim\dimen@=\p@
      \def\x{1}%
    \else
      \ifdim\dimen@=\z@
        \def\x{0}%
      \else
        \edef\x{\strip@pt\dimen@}%
        \edef\x{\expandafter\@gobble\x}%
      \fi
    \fi
    \if@filesw
      \immediate\write\@auxout{%
        \string\transparent@use{\x}%
      }%
    \fi
    \edef\x{\endgroup
      \def\noexpand\transparent@current{\x}%
    }%
  \x
  \transparent@set
}
%    \end{macrocode}
%    \end{macro}
%    \begin{macrocode}
\AtEndDocument{%
  \ifTRP@rerun
    \PackageWarningNoLine{transparent}{%
      Rerun to get transparencies right%
    }%
  \fi
}
\def\transparent@current{/TRP1 gs}
\def\transparent@set{%
  \@ifundefined{TRP\transparent@current}{%
    \global\TRP@reruntrue
  }{%
    \TRP@pdfcolorstack\TRP@colorstack push{%
      \csname TRP\transparent@current\endcsname
    }%
    \aftergroup\transparent@reset
  }%
}
\def\transparent@reset{%
  \TRP@pdfcolorstack\TRP@colorstack pop\relax
}
%    \end{macrocode}
%    \begin{macro}{\texttransparent}
%    \begin{macrocode}
\newcommand*{\texttransparent}[2]{%
  \protect\leavevmode
  \begingroup
    \transparent{#1}%
    #2%
  \endgroup
}
%    \end{macrocode}
%    \end{macro}
%
%    \begin{macrocode}
%</package>
%    \end{macrocode}
%
% \section{Installation}
%
% \subsection{Download}
%
% \paragraph{Package.} This package is available on
% CTAN\footnote{\CTANpkg{transparent}}:
% \begin{description}
% \item[\CTAN{macros/latex/contrib/transparent/transparent.dtx}] The source file.
% \item[\CTAN{macros/latex/contrib/transparent/transparent.pdf}] Documentation.
% \end{description}
%
%
%
% \subsection{Package installation}
%
% \paragraph{Unpacking.} The \xfile{.dtx} file is a self-extracting
% \docstrip\ archive. The files are extracted by running the
% \xfile{.dtx} through \plainTeX:
% \begin{quote}
%   \verb|tex transparent.dtx|
% \end{quote}
%
% \paragraph{TDS.} Now the different files must be moved into
% the different directories in your installation TDS tree
% (also known as \xfile{texmf} tree):
% \begin{quote}
% \def\t{^^A
% \begin{tabular}{@{}>{\ttfamily}l@{ $\rightarrow$ }>{\ttfamily}l@{}}
%   transparent.sty & tex/latex/transparent/transparent.sty\\
%   transparent-nometadata.sty & tex/latex/transparent/transparent-nometadata.sty\\   
%   transparent.pdf & doc/latex/transparent/transparent.pdf\\
%   transparent-example.tex & doc/latex/transparent/transparent-example.tex\\
%   transparent.dtx & source/latex/transparent/transparent.dtx\\
% \end{tabular}^^A
% }^^A
% \sbox0{\t}^^A
% \ifdim\wd0>\linewidth
%   \begingroup
%     \advance\linewidth by\leftmargin
%     \advance\linewidth by\rightmargin
%   \edef\x{\endgroup
%     \def\noexpand\lw{\the\linewidth}^^A
%   }\x
%   \def\lwbox{^^A
%     \leavevmode
%     \hbox to \linewidth{^^A
%       \kern-\leftmargin\relax
%       \hss
%       \usebox0
%       \hss
%       \kern-\rightmargin\relax
%     }^^A
%   }^^A
%   \ifdim\wd0>\lw
%     \sbox0{\small\t}^^A
%     \ifdim\wd0>\linewidth
%       \ifdim\wd0>\lw
%         \sbox0{\footnotesize\t}^^A
%         \ifdim\wd0>\linewidth
%           \ifdim\wd0>\lw
%             \sbox0{\scriptsize\t}^^A
%             \ifdim\wd0>\linewidth
%               \ifdim\wd0>\lw
%                 \sbox0{\tiny\t}^^A
%                 \ifdim\wd0>\linewidth
%                   \lwbox
%                 \else
%                   \usebox0
%                 \fi
%               \else
%                 \lwbox
%               \fi
%             \else
%               \usebox0
%             \fi
%           \else
%             \lwbox
%           \fi
%         \else
%           \usebox0
%         \fi
%       \else
%         \lwbox
%       \fi
%     \else
%       \usebox0
%     \fi
%   \else
%     \lwbox
%   \fi
% \else
%   \usebox0
% \fi
% \end{quote}
% If you have a \xfile{docstrip.cfg} that configures and enables \docstrip's
% TDS installing feature, then some files can already be in the right
% place, see the documentation of \docstrip.
%
% \subsection{Refresh file name databases}
%
% If your \TeX~distribution
% (\TeX\,Live, \mikTeX, \dots) relies on file name databases, you must refresh
% these. For example, \TeX\,Live\ users run \verb|texhash| or
% \verb|mktexlsr|.
%
% \subsection{Some details for the interested}
%
% \paragraph{Unpacking with \LaTeX.}
% The \xfile{.dtx} chooses its action depending on the format:
% \begin{description}
% \item[\plainTeX:] Run \docstrip\ and extract the files.
% \item[\LaTeX:] Generate the documentation.
% \end{description}
% If you insist on using \LaTeX\ for \docstrip\ (really,
% \docstrip\ does not need \LaTeX), then inform the autodetect routine
% about your intention:
% \begin{quote}
%   \verb|latex \let\install=y% \iffalse meta-comment
%
% File: transparent.dtx
% Version: 2022-10-27 v1.5
% Info: Transparency via pdfTeX's color stack
%
% Copyright (C)
%    2007 Heiko Oberdiek
%    2016-2022 Oberdiek Package Support Group 2018
%    https://github.com/ho-tex/transparent/issues
%
% This work may be distributed and/or modified under the
% conditions of the LaTeX Project Public License, either
% version 1.3c of this license or (at your option) any later
% version. This version of this license is in
%    https://www.latex-project.org/lppl/lppl-1-3c.txt
% and the latest version of this license is in
%    https://www.latex-project.org/lppl.txt
% and version 1.3 or later is part of all distributions of
% LaTeX version 2005/12/01 or later.
%
% This work has the LPPL maintenance status "maintained".
%
% The Current Maintainers of this work are
% Heiko Oberdiek and the Oberdiek Package Support Group
% https://github.com/ho-tex/transparent/issues
%
% This work consists of the main source file transparent.dtx
% and the derived files
%    transparent.sty, transparent-nometadata.sty, transparent.pdf, 
%    transparent.ins, transparent.drv, transparent-example.tex.
%
% Distribution:
%    CTAN:macros/latex/contrib/transparent/transparent.dtx
%    CTAN:macros/latex/contrib/transparent/transparent.pdf
%
% Unpacking:
%    (a) If transparent.ins is present:
%           tex transparent.ins
%    (b) Without transparent.ins:
%           tex transparent.dtx
%    (c) If you insist on using LaTeX
%           latex \let\install=y% \iffalse meta-comment
%
% File: transparent.dtx
% Version: 2022-10-27 v1.5
% Info: Transparency via pdfTeX's color stack
%
% Copyright (C)
%    2007 Heiko Oberdiek
%    2016-2022 Oberdiek Package Support Group 2018
%    https://github.com/ho-tex/transparent/issues
%
% This work may be distributed and/or modified under the
% conditions of the LaTeX Project Public License, either
% version 1.3c of this license or (at your option) any later
% version. This version of this license is in
%    https://www.latex-project.org/lppl/lppl-1-3c.txt
% and the latest version of this license is in
%    https://www.latex-project.org/lppl.txt
% and version 1.3 or later is part of all distributions of
% LaTeX version 2005/12/01 or later.
%
% This work has the LPPL maintenance status "maintained".
%
% The Current Maintainers of this work are
% Heiko Oberdiek and the Oberdiek Package Support Group
% https://github.com/ho-tex/transparent/issues
%
% This work consists of the main source file transparent.dtx
% and the derived files
%    transparent.sty, transparent-nometadata.sty, transparent.pdf, 
%    transparent.ins, transparent.drv, transparent-example.tex.
%
% Distribution:
%    CTAN:macros/latex/contrib/transparent/transparent.dtx
%    CTAN:macros/latex/contrib/transparent/transparent.pdf
%
% Unpacking:
%    (a) If transparent.ins is present:
%           tex transparent.ins
%    (b) Without transparent.ins:
%           tex transparent.dtx
%    (c) If you insist on using LaTeX
%           latex \let\install=y\input{transparent.dtx}
%        (quote the arguments according to the demands of your shell)
%
% Documentation:
%    (a) If transparent.drv is present:
%           latex transparent.drv
%    (b) Without transparent.drv:
%           latex transparent.dtx; ...
%    The class ltxdoc loads the configuration file ltxdoc.cfg
%    if available. Here you can specify further options, e.g.
%    use A4 as paper format:
%       \PassOptionsToClass{a4paper}{article}
%
%    Programm calls to get the documentation (example):
%       pdflatex transparent.dtx
%       makeindex -s gind.ist transparent.idx
%       pdflatex transparent.dtx
%       makeindex -s gind.ist transparent.idx
%       pdflatex transparent.dtx
%
% Installation:
%    TDS:tex/latex/transparent/transparent.sty
%    TDS:tex/latex/transparent/transparent-nometadata.sty
%    TDS:doc/latex/transparent/transparent.pdf
%    TDS:source/latex/transparent/transparent.dtx
%
%<*ignore>
\begingroup
  \catcode123=1 %
  \catcode125=2 %
  \def\x{LaTeX2e}%
\expandafter\endgroup
\ifcase 0\ifx\install y1\fi\expandafter
         \ifx\csname processbatchFile\endcsname\relax\else1\fi
         \ifx\fmtname\x\else 1\fi\relax
\else\csname fi\endcsname
%</ignore>
%<*install>
\input docstrip.tex
\Msg{************************************************************************}
\Msg{* Installation}
\Msg{* Package: transparent 2022-10-27 v1.5 Transparency via pdfTeX's color stack (HO)}
\Msg{************************************************************************}

\keepsilent
\askforoverwritefalse

\let\MetaPrefix\relax
\preamble

This is a generated file.

Project: transparent
Version: 2022-10-27 v1.5

Copyright (C)
   2007 Heiko Oberdiek
   2016-2022 Oberdiek Package Support Group

This work may be distributed and/or modified under the
conditions of the LaTeX Project Public License, either
version 1.3c of this license or (at your option) any later
version. This version of this license is in
   https://www.latex-project.org/lppl/lppl-1-3c.txt
and the latest version of this license is in
   https://www.latex-project.org/lppl.txt
and version 1.3 or later is part of all distributions of
LaTeX version 2005/12/01 or later.

This work has the LPPL maintenance status "maintained".

The Current Maintainers of this work are
Heiko Oberdiek and the Oberdiek Package Support Group
https://github.com/ho-tex/transparent/issues


This work consists of the main source file transparent.dtx
and the derived files
   transparent.sty, transparent-nometadata.sty, transparent.pdf,
   transparent.ins, transparent.drv, transparent-example.tex.

\endpreamble
\let\MetaPrefix\DoubleperCent

\generate{%
  \file{transparent.ins}{\from{transparent.dtx}{install}}%
  \file{transparent.drv}{\from{transparent.dtx}{driver}}%
  \usedir{tex/latex/transparent}%
  \file{transparent-nometadata.sty}{\from{transparent.dtx}{package}}%
  \file{transparent.sty}{\from{transparent.dtx}{package-new}}%
  \usedir{doc/latex/transparent}%
  \file{transparent-example.tex}{\from{transparent.dtx}{example}}%
}

\catcode32=13\relax% active space
\let =\space%
\Msg{************************************************************************}
\Msg{*}
\Msg{* To finish the installation you have to move the following}
\Msg{* file into a directory searched by TeX:}
\Msg{*}
\Msg{*     transparent.sty}
\Msg{*     transparent-nometadata.sty}
\Msg{*}
\Msg{* To produce the documentation run the file `transparent.drv'}
\Msg{* through LaTeX.}
\Msg{*}
\Msg{* Happy TeXing!}
\Msg{*}
\Msg{************************************************************************}

\endbatchfile
%</install>
%<*ignore>
\fi
%</ignore>
%<*driver>
\NeedsTeXFormat{LaTeX2e}
\ProvidesFile{transparent.drv}%
  [2022-10-27 v1.5 Transparency via pdfTeX's color stack (HO)]%
\documentclass{ltxdoc}
\usepackage{holtxdoc}[2011/11/22]
\begin{document}
  \DocInput{transparent.dtx}%
\end{document}
%</driver>
% \fi
%
%
%
% \GetFileInfo{transparent.drv}
%
% \title{The \xpackage{transparent} package}
% \date{2022-10-27 v1.5}
% \author{Heiko Oberdiek\thanks
% {Please report any issues at \url{https://github.com/ho-tex/transparent/issues}}}
%
% \maketitle
%
% \begin{abstract}
% \pdfTeX\ and lua\TeX{} supports several color stacks. This
% package shows, how a separate color stack can be used for transparency,
% a property besides color, that also works across page break.
% \end{abstract}
%
% \tableofcontents
%
% \section{User interface}
%
% The package \xpackage{transparent} defines \cs{transparent} and
% \cs{texttransparent}. They are used like \cs{color} and \cs{textcolor}.
% The first argument is the transparency value between 0 and 1 where 0
% is fully transparent and 1 is opaque.
% 
% \section{Engine support}
% If the PDF management is loaded by using \verb+\DocumentMetadata{}+ at the beginning
% of the document the package makes use of \texttt{l3opacity} and
% can be used with all engines supported by \texttt{l3backend}.
% But only with \pdfTeX\ and lua\TeX{} the transparency will also work across page break.
% 
% 
% Without the PDF management, the package works only with \pdfTeX\ and lua\TeX{}.
% It then uses the original code by Heiko Oberdiek. It then uses the
% primitive \cs{pdfpageresources} and can clash with other packages that also use
% \cs{pdfpageresources}.
% 
%
% \section{Example for usage}
%    \begin{macrocode}
%<*example>
\documentclass[12pt]{article}

\usepackage{color}
\usepackage{transparent}

\begin{document}
\colorbox{yellow}{%
  \bfseries
  \color{blue}%
  Blue and %
  \transparent{0.6}%
  transparent blue%
}

\bigskip
Hello World
\texttransparent{0.5}{Hello\newpage World}
Hello World
\end{document}
%</example>
%    \end{macrocode}
%
% \StopEventually{
% }
%
% \section{Implementation}
% \subsection{New version using the methods of the PDF management} 
% Identification
%    \begin{macrocode}
%<*package-new>
\NeedsTeXFormat{LaTeX2e}[2020/10/01]
\ProvidesExplPackage{transparent}{2022-10-27}{1.5}
  {Transparency with color stacks}%
%    \end{macrocode}
% Testing if the pdfmanagement is used
%    \begin{macrocode}
\@ifl@t@r\fmtversion{2022-06-01}
  {
    \IfDocumentMetadataTF {}{\RequirePackage{transparent-nometadata}}
    \IfDocumentMetadataTF {}{\endinput}
  }
  {\RequirePackage{transparent-nometadata}}  
%    \end{macrocode}
%    \begin{macrocode}
\RequirePackage{l3opacity}

\NewDocumentCommand{\transparent} { m }
      {
       \opacity_select:n{\fp_eval:n{ min(max(0,#1),1) } }
      }

\NewDocumentCommand{\texttransparent}{m m}
  {
   \mode_leave_vertical:
   \group_begin:
     \transparent{#1}
     #2
   \group_end:
  }
%</package-new>
%    \end{macrocode}
% \subsection{Old version without the PDF management}
%    \begin{macrocode}
%<*package>
\NeedsTeXFormat{LaTeX2e}
\ProvidesPackage{transparent-nometadata}%
  [2022-10-27 v1.5 Transparency via pdfTeX's color stack (HO)]%
%    \end{macrocode}
%
% \subsection{Initial checks}
%
% \subsubsection{Check for \pdfTeX\ in PDF mode}
%    \begin{macrocode}
\RequirePackage{iftex}
\ifpdf
\else
  \PackageWarningNoLine{transparent}{%
    Loading aborted, because pdfTeX is not running in PDF mode%
  }%
  \expandafter\endinput
\fi
%    \end{macrocode}
%
% \subsubsection{Check \pdfTeX\ or Lua\TeX\ version}
%    \begin{macrocode}
\ifx\pdfextension\@undefined
  \let\TRP@pdfcolorstackinit\pdfcolorstackinit
  \let\TRP@pdfpageresources\pdfpageresources
  \let\TRP@pdfcolorstack\pdfcolorstack
\else
  \def\TRP@pdfcolorstackinit           {\pdffeedback colorstackinit}
  \protected\edef\TRP@pdfpageresources {\pdfvariable pageresources}
  \protected\def\TRP@pdfcolorstack     {\pdfextension colorstack}
\fi
%    \end{macrocode}
%
%    \begin{macrocode}
\ifcsname TRP@pdfcolorstackinit\endcsname\else
  \PackageWarningNoLine{transparent}{%
    Your pdfTeX version does not support color stacks%
  }%
  \expandafter\endinput
\fi
%    \end{macrocode}
%
% \subsection{Compatibility with pgf}
%  https://github.com/ho-tex/transparent/issues/1
%    \begin{macrocode}
\AtBeginDocument
 {%
   \ifcsname pgfutil@addpdfresource@extgs\endcsname
    \let\TRP@addresource\relax
    \pgfutil@addpdfresource@extgs{\TRP@list}%
   \fi
 }
%    \end{macrocode}
% \subsection{Transparency}
%
%    The setting for the different transparency values must
%    be added to the page resources. In the first run the values
%    are recorded in the \xfile{.aux} file. In the second run
%    the values are set and transparency is available.
%    \begin{macrocode}
\RequirePackage{auxhook}
\AddLineBeginAux{%
  \string\providecommand{\string\transparent@use}[1]{}%
}
\gdef\TRP@list{/TRP1<</ca 1/CA 1>>}
\def\transparent@use#1{%
  \@ifundefined{TRP#1}{%
    \g@addto@macro\TRP@list{%
      /TRP#1<</ca #1/CA #1>>%
    }%
    \expandafter\gdef\csname TRP#1\endcsname{/TRP#1 gs}%
  }{%
    % #1 is already known, nothing to do
  }%
}
\AtBeginDocument{%
  \TRP@addresource
  \let\transparent@use\@gobble
}
%    \end{macrocode}
%    Unhappily the interface setting page resources is very
%    poor, only a token register \cs{pdfpageresources}.
%    Thus this package tries to be cooperative in the way that
%    it embeds the previous contents of \cs{pdfpageresources}.
%    However it does not solve the problem, if several packages
%    want to set |/ExtGState|.
%    \begin{macrocode}
\def\TRP@addresource{%
  \begingroup
    \edef\x{\endgroup
      \TRP@pdfpageresources{%
        \the\TRP@pdfpageresources
        /ExtGState<<\TRP@list>>%
      }%
    }%
  \x
}
\newif\ifTRP@rerun
\xdef\TRP@colorstack{%
  \TRP@pdfcolorstackinit page direct{/TRP1 gs}%
}
%    \end{macrocode}
%    \begin{macro}{\transparent}
%    \begin{macrocode}
\newcommand*{\transparent}[1]{%
  \begingroup
    \dimen@=#1\p@\relax
    \ifdim\dimen@>\p@
      \dimen@=\p@
    \fi
    \ifdim\dimen@<\z@
      \dimen@=\z@
    \fi
    \ifdim\dimen@=\p@
      \def\x{1}%
    \else
      \ifdim\dimen@=\z@
        \def\x{0}%
      \else
        \edef\x{\strip@pt\dimen@}%
        \edef\x{\expandafter\@gobble\x}%
      \fi
    \fi
    \if@filesw
      \immediate\write\@auxout{%
        \string\transparent@use{\x}%
      }%
    \fi
    \edef\x{\endgroup
      \def\noexpand\transparent@current{\x}%
    }%
  \x
  \transparent@set
}
%    \end{macrocode}
%    \end{macro}
%    \begin{macrocode}
\AtEndDocument{%
  \ifTRP@rerun
    \PackageWarningNoLine{transparent}{%
      Rerun to get transparencies right%
    }%
  \fi
}
\def\transparent@current{/TRP1 gs}
\def\transparent@set{%
  \@ifundefined{TRP\transparent@current}{%
    \global\TRP@reruntrue
  }{%
    \TRP@pdfcolorstack\TRP@colorstack push{%
      \csname TRP\transparent@current\endcsname
    }%
    \aftergroup\transparent@reset
  }%
}
\def\transparent@reset{%
  \TRP@pdfcolorstack\TRP@colorstack pop\relax
}
%    \end{macrocode}
%    \begin{macro}{\texttransparent}
%    \begin{macrocode}
\newcommand*{\texttransparent}[2]{%
  \protect\leavevmode
  \begingroup
    \transparent{#1}%
    #2%
  \endgroup
}
%    \end{macrocode}
%    \end{macro}
%
%    \begin{macrocode}
%</package>
%    \end{macrocode}
%
% \section{Installation}
%
% \subsection{Download}
%
% \paragraph{Package.} This package is available on
% CTAN\footnote{\CTANpkg{transparent}}:
% \begin{description}
% \item[\CTAN{macros/latex/contrib/transparent/transparent.dtx}] The source file.
% \item[\CTAN{macros/latex/contrib/transparent/transparent.pdf}] Documentation.
% \end{description}
%
%
%
% \subsection{Package installation}
%
% \paragraph{Unpacking.} The \xfile{.dtx} file is a self-extracting
% \docstrip\ archive. The files are extracted by running the
% \xfile{.dtx} through \plainTeX:
% \begin{quote}
%   \verb|tex transparent.dtx|
% \end{quote}
%
% \paragraph{TDS.} Now the different files must be moved into
% the different directories in your installation TDS tree
% (also known as \xfile{texmf} tree):
% \begin{quote}
% \def\t{^^A
% \begin{tabular}{@{}>{\ttfamily}l@{ $\rightarrow$ }>{\ttfamily}l@{}}
%   transparent.sty & tex/latex/transparent/transparent.sty\\
%   transparent-nometadata.sty & tex/latex/transparent/transparent-nometadata.sty\\   
%   transparent.pdf & doc/latex/transparent/transparent.pdf\\
%   transparent-example.tex & doc/latex/transparent/transparent-example.tex\\
%   transparent.dtx & source/latex/transparent/transparent.dtx\\
% \end{tabular}^^A
% }^^A
% \sbox0{\t}^^A
% \ifdim\wd0>\linewidth
%   \begingroup
%     \advance\linewidth by\leftmargin
%     \advance\linewidth by\rightmargin
%   \edef\x{\endgroup
%     \def\noexpand\lw{\the\linewidth}^^A
%   }\x
%   \def\lwbox{^^A
%     \leavevmode
%     \hbox to \linewidth{^^A
%       \kern-\leftmargin\relax
%       \hss
%       \usebox0
%       \hss
%       \kern-\rightmargin\relax
%     }^^A
%   }^^A
%   \ifdim\wd0>\lw
%     \sbox0{\small\t}^^A
%     \ifdim\wd0>\linewidth
%       \ifdim\wd0>\lw
%         \sbox0{\footnotesize\t}^^A
%         \ifdim\wd0>\linewidth
%           \ifdim\wd0>\lw
%             \sbox0{\scriptsize\t}^^A
%             \ifdim\wd0>\linewidth
%               \ifdim\wd0>\lw
%                 \sbox0{\tiny\t}^^A
%                 \ifdim\wd0>\linewidth
%                   \lwbox
%                 \else
%                   \usebox0
%                 \fi
%               \else
%                 \lwbox
%               \fi
%             \else
%               \usebox0
%             \fi
%           \else
%             \lwbox
%           \fi
%         \else
%           \usebox0
%         \fi
%       \else
%         \lwbox
%       \fi
%     \else
%       \usebox0
%     \fi
%   \else
%     \lwbox
%   \fi
% \else
%   \usebox0
% \fi
% \end{quote}
% If you have a \xfile{docstrip.cfg} that configures and enables \docstrip's
% TDS installing feature, then some files can already be in the right
% place, see the documentation of \docstrip.
%
% \subsection{Refresh file name databases}
%
% If your \TeX~distribution
% (\TeX\,Live, \mikTeX, \dots) relies on file name databases, you must refresh
% these. For example, \TeX\,Live\ users run \verb|texhash| or
% \verb|mktexlsr|.
%
% \subsection{Some details for the interested}
%
% \paragraph{Unpacking with \LaTeX.}
% The \xfile{.dtx} chooses its action depending on the format:
% \begin{description}
% \item[\plainTeX:] Run \docstrip\ and extract the files.
% \item[\LaTeX:] Generate the documentation.
% \end{description}
% If you insist on using \LaTeX\ for \docstrip\ (really,
% \docstrip\ does not need \LaTeX), then inform the autodetect routine
% about your intention:
% \begin{quote}
%   \verb|latex \let\install=y\input{transparent.dtx}|
% \end{quote}
% Do not forget to quote the argument according to the demands
% of your shell.
%
% \paragraph{Generating the documentation.}
% You can use both the \xfile{.dtx} or the \xfile{.drv} to generate
% the documentation. The process can be configured by the
% configuration file \xfile{ltxdoc.cfg}. For instance, put this
% line into this file, if you want to have A4 as paper format:
% \begin{quote}
%   \verb|\PassOptionsToClass{a4paper}{article}|
% \end{quote}
% An example follows how to generate the
% documentation with pdf\LaTeX:
% \begin{quote}
%\begin{verbatim}
%pdflatex transparent.dtx
%makeindex -s gind.ist transparent.idx
%pdflatex transparent.dtx
%makeindex -s gind.ist transparent.idx
%pdflatex transparent.dtx
%\end{verbatim}
% \end{quote}
%
% \begin{History}
%   \begin{Version}{2007/01/08 v1.0}
%   \item
%     First version.
%   \end{Version}
%   \begin{Version}{2016/05/16 v1.1}
%   \item
%     Documentation updates.
%   \end{Version}
%   \begin{Version}{2018/09/10 v1.2}
%   \item
%     Update for LuaTeX, remove dependency on Pdf\TeX\ command names.
%   \end{Version}
%   \begin{Version}{2018/11/18 v1.3}
%   \item
%     Added code for pgf compatibility, see https://github.com/ho-tex/transparent/issues/1
%   \end{Version}
%   \begin{Version}{2019/11/29 v1.4}
%   \item
%     Documentation updates.
% \item
% Use \xpackage{iftex} package.
%   \end{Version}
%   \begin{Version}{2022-10-27 v1.5}
%   \item Made the package compatible with the PDF management.
%   \end{Version}
% \end{History}
%
% \PrintIndex
%
% \Finale
\endinput

%        (quote the arguments according to the demands of your shell)
%
% Documentation:
%    (a) If transparent.drv is present:
%           latex transparent.drv
%    (b) Without transparent.drv:
%           latex transparent.dtx; ...
%    The class ltxdoc loads the configuration file ltxdoc.cfg
%    if available. Here you can specify further options, e.g.
%    use A4 as paper format:
%       \PassOptionsToClass{a4paper}{article}
%
%    Programm calls to get the documentation (example):
%       pdflatex transparent.dtx
%       makeindex -s gind.ist transparent.idx
%       pdflatex transparent.dtx
%       makeindex -s gind.ist transparent.idx
%       pdflatex transparent.dtx
%
% Installation:
%    TDS:tex/latex/transparent/transparent.sty
%    TDS:tex/latex/transparent/transparent-nometadata.sty
%    TDS:doc/latex/transparent/transparent.pdf
%    TDS:source/latex/transparent/transparent.dtx
%
%<*ignore>
\begingroup
  \catcode123=1 %
  \catcode125=2 %
  \def\x{LaTeX2e}%
\expandafter\endgroup
\ifcase 0\ifx\install y1\fi\expandafter
         \ifx\csname processbatchFile\endcsname\relax\else1\fi
         \ifx\fmtname\x\else 1\fi\relax
\else\csname fi\endcsname
%</ignore>
%<*install>
\input docstrip.tex
\Msg{************************************************************************}
\Msg{* Installation}
\Msg{* Package: transparent 2022-10-27 v1.5 Transparency via pdfTeX's color stack (HO)}
\Msg{************************************************************************}

\keepsilent
\askforoverwritefalse

\let\MetaPrefix\relax
\preamble

This is a generated file.

Project: transparent
Version: 2022-10-27 v1.5

Copyright (C)
   2007 Heiko Oberdiek
   2016-2022 Oberdiek Package Support Group

This work may be distributed and/or modified under the
conditions of the LaTeX Project Public License, either
version 1.3c of this license or (at your option) any later
version. This version of this license is in
   https://www.latex-project.org/lppl/lppl-1-3c.txt
and the latest version of this license is in
   https://www.latex-project.org/lppl.txt
and version 1.3 or later is part of all distributions of
LaTeX version 2005/12/01 or later.

This work has the LPPL maintenance status "maintained".

The Current Maintainers of this work are
Heiko Oberdiek and the Oberdiek Package Support Group
https://github.com/ho-tex/transparent/issues


This work consists of the main source file transparent.dtx
and the derived files
   transparent.sty, transparent-nometadata.sty, transparent.pdf,
   transparent.ins, transparent.drv, transparent-example.tex.

\endpreamble
\let\MetaPrefix\DoubleperCent

\generate{%
  \file{transparent.ins}{\from{transparent.dtx}{install}}%
  \file{transparent.drv}{\from{transparent.dtx}{driver}}%
  \usedir{tex/latex/transparent}%
  \file{transparent-nometadata.sty}{\from{transparent.dtx}{package}}%
  \file{transparent.sty}{\from{transparent.dtx}{package-new}}%
  \usedir{doc/latex/transparent}%
  \file{transparent-example.tex}{\from{transparent.dtx}{example}}%
}

\catcode32=13\relax% active space
\let =\space%
\Msg{************************************************************************}
\Msg{*}
\Msg{* To finish the installation you have to move the following}
\Msg{* file into a directory searched by TeX:}
\Msg{*}
\Msg{*     transparent.sty}
\Msg{*     transparent-nometadata.sty}
\Msg{*}
\Msg{* To produce the documentation run the file `transparent.drv'}
\Msg{* through LaTeX.}
\Msg{*}
\Msg{* Happy TeXing!}
\Msg{*}
\Msg{************************************************************************}

\endbatchfile
%</install>
%<*ignore>
\fi
%</ignore>
%<*driver>
\NeedsTeXFormat{LaTeX2e}
\ProvidesFile{transparent.drv}%
  [2022-10-27 v1.5 Transparency via pdfTeX's color stack (HO)]%
\documentclass{ltxdoc}
\usepackage{holtxdoc}[2011/11/22]
\begin{document}
  \DocInput{transparent.dtx}%
\end{document}
%</driver>
% \fi
%
%
%
% \GetFileInfo{transparent.drv}
%
% \title{The \xpackage{transparent} package}
% \date{2022-10-27 v1.5}
% \author{Heiko Oberdiek\thanks
% {Please report any issues at \url{https://github.com/ho-tex/transparent/issues}}}
%
% \maketitle
%
% \begin{abstract}
% \pdfTeX\ and lua\TeX{} supports several color stacks. This
% package shows, how a separate color stack can be used for transparency,
% a property besides color, that also works across page break.
% \end{abstract}
%
% \tableofcontents
%
% \section{User interface}
%
% The package \xpackage{transparent} defines \cs{transparent} and
% \cs{texttransparent}. They are used like \cs{color} and \cs{textcolor}.
% The first argument is the transparency value between 0 and 1 where 0
% is fully transparent and 1 is opaque.
% 
% \section{Engine support}
% If the PDF management is loaded by using \verb+\DocumentMetadata{}+ at the beginning
% of the document the package makes use of \texttt{l3opacity} and
% can be used with all engines supported by \texttt{l3backend}.
% But only with \pdfTeX\ and lua\TeX{} the transparency will also work across page break.
% 
% 
% Without the PDF management, the package works only with \pdfTeX\ and lua\TeX{}.
% It then uses the original code by Heiko Oberdiek. It then uses the
% primitive \cs{pdfpageresources} and can clash with other packages that also use
% \cs{pdfpageresources}.
% 
%
% \section{Example for usage}
%    \begin{macrocode}
%<*example>
\documentclass[12pt]{article}

\usepackage{color}
\usepackage{transparent}

\begin{document}
\colorbox{yellow}{%
  \bfseries
  \color{blue}%
  Blue and %
  \transparent{0.6}%
  transparent blue%
}

\bigskip
Hello World
\texttransparent{0.5}{Hello\newpage World}
Hello World
\end{document}
%</example>
%    \end{macrocode}
%
% \StopEventually{
% }
%
% \section{Implementation}
% \subsection{New version using the methods of the PDF management} 
% Identification
%    \begin{macrocode}
%<*package-new>
\NeedsTeXFormat{LaTeX2e}[2020/10/01]
\ProvidesExplPackage{transparent}{2022-10-27}{1.5}
  {Transparency with color stacks}%
%    \end{macrocode}
% Testing if the pdfmanagement is used
%    \begin{macrocode}
\@ifl@t@r\fmtversion{2022-06-01}
  {
    \IfDocumentMetadataTF {}{\RequirePackage{transparent-nometadata}}
    \IfDocumentMetadataTF {}{\endinput}
  }
  {\RequirePackage{transparent-nometadata}}  
%    \end{macrocode}
%    \begin{macrocode}
\RequirePackage{l3opacity}

\NewDocumentCommand{\transparent} { m }
      {
       \opacity_select:n{\fp_eval:n{ min(max(0,#1),1) } }
      }

\NewDocumentCommand{\texttransparent}{m m}
  {
   \mode_leave_vertical:
   \group_begin:
     \transparent{#1}
     #2
   \group_end:
  }
%</package-new>
%    \end{macrocode}
% \subsection{Old version without the PDF management}
%    \begin{macrocode}
%<*package>
\NeedsTeXFormat{LaTeX2e}
\ProvidesPackage{transparent-nometadata}%
  [2022-10-27 v1.5 Transparency via pdfTeX's color stack (HO)]%
%    \end{macrocode}
%
% \subsection{Initial checks}
%
% \subsubsection{Check for \pdfTeX\ in PDF mode}
%    \begin{macrocode}
\RequirePackage{iftex}
\ifpdf
\else
  \PackageWarningNoLine{transparent}{%
    Loading aborted, because pdfTeX is not running in PDF mode%
  }%
  \expandafter\endinput
\fi
%    \end{macrocode}
%
% \subsubsection{Check \pdfTeX\ or Lua\TeX\ version}
%    \begin{macrocode}
\ifx\pdfextension\@undefined
  \let\TRP@pdfcolorstackinit\pdfcolorstackinit
  \let\TRP@pdfpageresources\pdfpageresources
  \let\TRP@pdfcolorstack\pdfcolorstack
\else
  \def\TRP@pdfcolorstackinit           {\pdffeedback colorstackinit}
  \protected\edef\TRP@pdfpageresources {\pdfvariable pageresources}
  \protected\def\TRP@pdfcolorstack     {\pdfextension colorstack}
\fi
%    \end{macrocode}
%
%    \begin{macrocode}
\ifcsname TRP@pdfcolorstackinit\endcsname\else
  \PackageWarningNoLine{transparent}{%
    Your pdfTeX version does not support color stacks%
  }%
  \expandafter\endinput
\fi
%    \end{macrocode}
%
% \subsection{Compatibility with pgf}
%  https://github.com/ho-tex/transparent/issues/1
%    \begin{macrocode}
\AtBeginDocument
 {%
   \ifcsname pgfutil@addpdfresource@extgs\endcsname
    \let\TRP@addresource\relax
    \pgfutil@addpdfresource@extgs{\TRP@list}%
   \fi
 }
%    \end{macrocode}
% \subsection{Transparency}
%
%    The setting for the different transparency values must
%    be added to the page resources. In the first run the values
%    are recorded in the \xfile{.aux} file. In the second run
%    the values are set and transparency is available.
%    \begin{macrocode}
\RequirePackage{auxhook}
\AddLineBeginAux{%
  \string\providecommand{\string\transparent@use}[1]{}%
}
\gdef\TRP@list{/TRP1<</ca 1/CA 1>>}
\def\transparent@use#1{%
  \@ifundefined{TRP#1}{%
    \g@addto@macro\TRP@list{%
      /TRP#1<</ca #1/CA #1>>%
    }%
    \expandafter\gdef\csname TRP#1\endcsname{/TRP#1 gs}%
  }{%
    % #1 is already known, nothing to do
  }%
}
\AtBeginDocument{%
  \TRP@addresource
  \let\transparent@use\@gobble
}
%    \end{macrocode}
%    Unhappily the interface setting page resources is very
%    poor, only a token register \cs{pdfpageresources}.
%    Thus this package tries to be cooperative in the way that
%    it embeds the previous contents of \cs{pdfpageresources}.
%    However it does not solve the problem, if several packages
%    want to set |/ExtGState|.
%    \begin{macrocode}
\def\TRP@addresource{%
  \begingroup
    \edef\x{\endgroup
      \TRP@pdfpageresources{%
        \the\TRP@pdfpageresources
        /ExtGState<<\TRP@list>>%
      }%
    }%
  \x
}
\newif\ifTRP@rerun
\xdef\TRP@colorstack{%
  \TRP@pdfcolorstackinit page direct{/TRP1 gs}%
}
%    \end{macrocode}
%    \begin{macro}{\transparent}
%    \begin{macrocode}
\newcommand*{\transparent}[1]{%
  \begingroup
    \dimen@=#1\p@\relax
    \ifdim\dimen@>\p@
      \dimen@=\p@
    \fi
    \ifdim\dimen@<\z@
      \dimen@=\z@
    \fi
    \ifdim\dimen@=\p@
      \def\x{1}%
    \else
      \ifdim\dimen@=\z@
        \def\x{0}%
      \else
        \edef\x{\strip@pt\dimen@}%
        \edef\x{\expandafter\@gobble\x}%
      \fi
    \fi
    \if@filesw
      \immediate\write\@auxout{%
        \string\transparent@use{\x}%
      }%
    \fi
    \edef\x{\endgroup
      \def\noexpand\transparent@current{\x}%
    }%
  \x
  \transparent@set
}
%    \end{macrocode}
%    \end{macro}
%    \begin{macrocode}
\AtEndDocument{%
  \ifTRP@rerun
    \PackageWarningNoLine{transparent}{%
      Rerun to get transparencies right%
    }%
  \fi
}
\def\transparent@current{/TRP1 gs}
\def\transparent@set{%
  \@ifundefined{TRP\transparent@current}{%
    \global\TRP@reruntrue
  }{%
    \TRP@pdfcolorstack\TRP@colorstack push{%
      \csname TRP\transparent@current\endcsname
    }%
    \aftergroup\transparent@reset
  }%
}
\def\transparent@reset{%
  \TRP@pdfcolorstack\TRP@colorstack pop\relax
}
%    \end{macrocode}
%    \begin{macro}{\texttransparent}
%    \begin{macrocode}
\newcommand*{\texttransparent}[2]{%
  \protect\leavevmode
  \begingroup
    \transparent{#1}%
    #2%
  \endgroup
}
%    \end{macrocode}
%    \end{macro}
%
%    \begin{macrocode}
%</package>
%    \end{macrocode}
%
% \section{Installation}
%
% \subsection{Download}
%
% \paragraph{Package.} This package is available on
% CTAN\footnote{\CTANpkg{transparent}}:
% \begin{description}
% \item[\CTAN{macros/latex/contrib/transparent/transparent.dtx}] The source file.
% \item[\CTAN{macros/latex/contrib/transparent/transparent.pdf}] Documentation.
% \end{description}
%
%
%
% \subsection{Package installation}
%
% \paragraph{Unpacking.} The \xfile{.dtx} file is a self-extracting
% \docstrip\ archive. The files are extracted by running the
% \xfile{.dtx} through \plainTeX:
% \begin{quote}
%   \verb|tex transparent.dtx|
% \end{quote}
%
% \paragraph{TDS.} Now the different files must be moved into
% the different directories in your installation TDS tree
% (also known as \xfile{texmf} tree):
% \begin{quote}
% \def\t{^^A
% \begin{tabular}{@{}>{\ttfamily}l@{ $\rightarrow$ }>{\ttfamily}l@{}}
%   transparent.sty & tex/latex/transparent/transparent.sty\\
%   transparent-nometadata.sty & tex/latex/transparent/transparent-nometadata.sty\\   
%   transparent.pdf & doc/latex/transparent/transparent.pdf\\
%   transparent-example.tex & doc/latex/transparent/transparent-example.tex\\
%   transparent.dtx & source/latex/transparent/transparent.dtx\\
% \end{tabular}^^A
% }^^A
% \sbox0{\t}^^A
% \ifdim\wd0>\linewidth
%   \begingroup
%     \advance\linewidth by\leftmargin
%     \advance\linewidth by\rightmargin
%   \edef\x{\endgroup
%     \def\noexpand\lw{\the\linewidth}^^A
%   }\x
%   \def\lwbox{^^A
%     \leavevmode
%     \hbox to \linewidth{^^A
%       \kern-\leftmargin\relax
%       \hss
%       \usebox0
%       \hss
%       \kern-\rightmargin\relax
%     }^^A
%   }^^A
%   \ifdim\wd0>\lw
%     \sbox0{\small\t}^^A
%     \ifdim\wd0>\linewidth
%       \ifdim\wd0>\lw
%         \sbox0{\footnotesize\t}^^A
%         \ifdim\wd0>\linewidth
%           \ifdim\wd0>\lw
%             \sbox0{\scriptsize\t}^^A
%             \ifdim\wd0>\linewidth
%               \ifdim\wd0>\lw
%                 \sbox0{\tiny\t}^^A
%                 \ifdim\wd0>\linewidth
%                   \lwbox
%                 \else
%                   \usebox0
%                 \fi
%               \else
%                 \lwbox
%               \fi
%             \else
%               \usebox0
%             \fi
%           \else
%             \lwbox
%           \fi
%         \else
%           \usebox0
%         \fi
%       \else
%         \lwbox
%       \fi
%     \else
%       \usebox0
%     \fi
%   \else
%     \lwbox
%   \fi
% \else
%   \usebox0
% \fi
% \end{quote}
% If you have a \xfile{docstrip.cfg} that configures and enables \docstrip's
% TDS installing feature, then some files can already be in the right
% place, see the documentation of \docstrip.
%
% \subsection{Refresh file name databases}
%
% If your \TeX~distribution
% (\TeX\,Live, \mikTeX, \dots) relies on file name databases, you must refresh
% these. For example, \TeX\,Live\ users run \verb|texhash| or
% \verb|mktexlsr|.
%
% \subsection{Some details for the interested}
%
% \paragraph{Unpacking with \LaTeX.}
% The \xfile{.dtx} chooses its action depending on the format:
% \begin{description}
% \item[\plainTeX:] Run \docstrip\ and extract the files.
% \item[\LaTeX:] Generate the documentation.
% \end{description}
% If you insist on using \LaTeX\ for \docstrip\ (really,
% \docstrip\ does not need \LaTeX), then inform the autodetect routine
% about your intention:
% \begin{quote}
%   \verb|latex \let\install=y% \iffalse meta-comment
%
% File: transparent.dtx
% Version: 2022-10-27 v1.5
% Info: Transparency via pdfTeX's color stack
%
% Copyright (C)
%    2007 Heiko Oberdiek
%    2016-2022 Oberdiek Package Support Group 2018
%    https://github.com/ho-tex/transparent/issues
%
% This work may be distributed and/or modified under the
% conditions of the LaTeX Project Public License, either
% version 1.3c of this license or (at your option) any later
% version. This version of this license is in
%    https://www.latex-project.org/lppl/lppl-1-3c.txt
% and the latest version of this license is in
%    https://www.latex-project.org/lppl.txt
% and version 1.3 or later is part of all distributions of
% LaTeX version 2005/12/01 or later.
%
% This work has the LPPL maintenance status "maintained".
%
% The Current Maintainers of this work are
% Heiko Oberdiek and the Oberdiek Package Support Group
% https://github.com/ho-tex/transparent/issues
%
% This work consists of the main source file transparent.dtx
% and the derived files
%    transparent.sty, transparent-nometadata.sty, transparent.pdf, 
%    transparent.ins, transparent.drv, transparent-example.tex.
%
% Distribution:
%    CTAN:macros/latex/contrib/transparent/transparent.dtx
%    CTAN:macros/latex/contrib/transparent/transparent.pdf
%
% Unpacking:
%    (a) If transparent.ins is present:
%           tex transparent.ins
%    (b) Without transparent.ins:
%           tex transparent.dtx
%    (c) If you insist on using LaTeX
%           latex \let\install=y\input{transparent.dtx}
%        (quote the arguments according to the demands of your shell)
%
% Documentation:
%    (a) If transparent.drv is present:
%           latex transparent.drv
%    (b) Without transparent.drv:
%           latex transparent.dtx; ...
%    The class ltxdoc loads the configuration file ltxdoc.cfg
%    if available. Here you can specify further options, e.g.
%    use A4 as paper format:
%       \PassOptionsToClass{a4paper}{article}
%
%    Programm calls to get the documentation (example):
%       pdflatex transparent.dtx
%       makeindex -s gind.ist transparent.idx
%       pdflatex transparent.dtx
%       makeindex -s gind.ist transparent.idx
%       pdflatex transparent.dtx
%
% Installation:
%    TDS:tex/latex/transparent/transparent.sty
%    TDS:tex/latex/transparent/transparent-nometadata.sty
%    TDS:doc/latex/transparent/transparent.pdf
%    TDS:source/latex/transparent/transparent.dtx
%
%<*ignore>
\begingroup
  \catcode123=1 %
  \catcode125=2 %
  \def\x{LaTeX2e}%
\expandafter\endgroup
\ifcase 0\ifx\install y1\fi\expandafter
         \ifx\csname processbatchFile\endcsname\relax\else1\fi
         \ifx\fmtname\x\else 1\fi\relax
\else\csname fi\endcsname
%</ignore>
%<*install>
\input docstrip.tex
\Msg{************************************************************************}
\Msg{* Installation}
\Msg{* Package: transparent 2022-10-27 v1.5 Transparency via pdfTeX's color stack (HO)}
\Msg{************************************************************************}

\keepsilent
\askforoverwritefalse

\let\MetaPrefix\relax
\preamble

This is a generated file.

Project: transparent
Version: 2022-10-27 v1.5

Copyright (C)
   2007 Heiko Oberdiek
   2016-2022 Oberdiek Package Support Group

This work may be distributed and/or modified under the
conditions of the LaTeX Project Public License, either
version 1.3c of this license or (at your option) any later
version. This version of this license is in
   https://www.latex-project.org/lppl/lppl-1-3c.txt
and the latest version of this license is in
   https://www.latex-project.org/lppl.txt
and version 1.3 or later is part of all distributions of
LaTeX version 2005/12/01 or later.

This work has the LPPL maintenance status "maintained".

The Current Maintainers of this work are
Heiko Oberdiek and the Oberdiek Package Support Group
https://github.com/ho-tex/transparent/issues


This work consists of the main source file transparent.dtx
and the derived files
   transparent.sty, transparent-nometadata.sty, transparent.pdf,
   transparent.ins, transparent.drv, transparent-example.tex.

\endpreamble
\let\MetaPrefix\DoubleperCent

\generate{%
  \file{transparent.ins}{\from{transparent.dtx}{install}}%
  \file{transparent.drv}{\from{transparent.dtx}{driver}}%
  \usedir{tex/latex/transparent}%
  \file{transparent-nometadata.sty}{\from{transparent.dtx}{package}}%
  \file{transparent.sty}{\from{transparent.dtx}{package-new}}%
  \usedir{doc/latex/transparent}%
  \file{transparent-example.tex}{\from{transparent.dtx}{example}}%
}

\catcode32=13\relax% active space
\let =\space%
\Msg{************************************************************************}
\Msg{*}
\Msg{* To finish the installation you have to move the following}
\Msg{* file into a directory searched by TeX:}
\Msg{*}
\Msg{*     transparent.sty}
\Msg{*     transparent-nometadata.sty}
\Msg{*}
\Msg{* To produce the documentation run the file `transparent.drv'}
\Msg{* through LaTeX.}
\Msg{*}
\Msg{* Happy TeXing!}
\Msg{*}
\Msg{************************************************************************}

\endbatchfile
%</install>
%<*ignore>
\fi
%</ignore>
%<*driver>
\NeedsTeXFormat{LaTeX2e}
\ProvidesFile{transparent.drv}%
  [2022-10-27 v1.5 Transparency via pdfTeX's color stack (HO)]%
\documentclass{ltxdoc}
\usepackage{holtxdoc}[2011/11/22]
\begin{document}
  \DocInput{transparent.dtx}%
\end{document}
%</driver>
% \fi
%
%
%
% \GetFileInfo{transparent.drv}
%
% \title{The \xpackage{transparent} package}
% \date{2022-10-27 v1.5}
% \author{Heiko Oberdiek\thanks
% {Please report any issues at \url{https://github.com/ho-tex/transparent/issues}}}
%
% \maketitle
%
% \begin{abstract}
% \pdfTeX\ and lua\TeX{} supports several color stacks. This
% package shows, how a separate color stack can be used for transparency,
% a property besides color, that also works across page break.
% \end{abstract}
%
% \tableofcontents
%
% \section{User interface}
%
% The package \xpackage{transparent} defines \cs{transparent} and
% \cs{texttransparent}. They are used like \cs{color} and \cs{textcolor}.
% The first argument is the transparency value between 0 and 1 where 0
% is fully transparent and 1 is opaque.
% 
% \section{Engine support}
% If the PDF management is loaded by using \verb+\DocumentMetadata{}+ at the beginning
% of the document the package makes use of \texttt{l3opacity} and
% can be used with all engines supported by \texttt{l3backend}.
% But only with \pdfTeX\ and lua\TeX{} the transparency will also work across page break.
% 
% 
% Without the PDF management, the package works only with \pdfTeX\ and lua\TeX{}.
% It then uses the original code by Heiko Oberdiek. It then uses the
% primitive \cs{pdfpageresources} and can clash with other packages that also use
% \cs{pdfpageresources}.
% 
%
% \section{Example for usage}
%    \begin{macrocode}
%<*example>
\documentclass[12pt]{article}

\usepackage{color}
\usepackage{transparent}

\begin{document}
\colorbox{yellow}{%
  \bfseries
  \color{blue}%
  Blue and %
  \transparent{0.6}%
  transparent blue%
}

\bigskip
Hello World
\texttransparent{0.5}{Hello\newpage World}
Hello World
\end{document}
%</example>
%    \end{macrocode}
%
% \StopEventually{
% }
%
% \section{Implementation}
% \subsection{New version using the methods of the PDF management} 
% Identification
%    \begin{macrocode}
%<*package-new>
\NeedsTeXFormat{LaTeX2e}[2020/10/01]
\ProvidesExplPackage{transparent}{2022-10-27}{1.5}
  {Transparency with color stacks}%
%    \end{macrocode}
% Testing if the pdfmanagement is used
%    \begin{macrocode}
\@ifl@t@r\fmtversion{2022-06-01}
  {
    \IfDocumentMetadataTF {}{\RequirePackage{transparent-nometadata}}
    \IfDocumentMetadataTF {}{\endinput}
  }
  {\RequirePackage{transparent-nometadata}}  
%    \end{macrocode}
%    \begin{macrocode}
\RequirePackage{l3opacity}

\NewDocumentCommand{\transparent} { m }
      {
       \opacity_select:n{\fp_eval:n{ min(max(0,#1),1) } }
      }

\NewDocumentCommand{\texttransparent}{m m}
  {
   \mode_leave_vertical:
   \group_begin:
     \transparent{#1}
     #2
   \group_end:
  }
%</package-new>
%    \end{macrocode}
% \subsection{Old version without the PDF management}
%    \begin{macrocode}
%<*package>
\NeedsTeXFormat{LaTeX2e}
\ProvidesPackage{transparent-nometadata}%
  [2022-10-27 v1.5 Transparency via pdfTeX's color stack (HO)]%
%    \end{macrocode}
%
% \subsection{Initial checks}
%
% \subsubsection{Check for \pdfTeX\ in PDF mode}
%    \begin{macrocode}
\RequirePackage{iftex}
\ifpdf
\else
  \PackageWarningNoLine{transparent}{%
    Loading aborted, because pdfTeX is not running in PDF mode%
  }%
  \expandafter\endinput
\fi
%    \end{macrocode}
%
% \subsubsection{Check \pdfTeX\ or Lua\TeX\ version}
%    \begin{macrocode}
\ifx\pdfextension\@undefined
  \let\TRP@pdfcolorstackinit\pdfcolorstackinit
  \let\TRP@pdfpageresources\pdfpageresources
  \let\TRP@pdfcolorstack\pdfcolorstack
\else
  \def\TRP@pdfcolorstackinit           {\pdffeedback colorstackinit}
  \protected\edef\TRP@pdfpageresources {\pdfvariable pageresources}
  \protected\def\TRP@pdfcolorstack     {\pdfextension colorstack}
\fi
%    \end{macrocode}
%
%    \begin{macrocode}
\ifcsname TRP@pdfcolorstackinit\endcsname\else
  \PackageWarningNoLine{transparent}{%
    Your pdfTeX version does not support color stacks%
  }%
  \expandafter\endinput
\fi
%    \end{macrocode}
%
% \subsection{Compatibility with pgf}
%  https://github.com/ho-tex/transparent/issues/1
%    \begin{macrocode}
\AtBeginDocument
 {%
   \ifcsname pgfutil@addpdfresource@extgs\endcsname
    \let\TRP@addresource\relax
    \pgfutil@addpdfresource@extgs{\TRP@list}%
   \fi
 }
%    \end{macrocode}
% \subsection{Transparency}
%
%    The setting for the different transparency values must
%    be added to the page resources. In the first run the values
%    are recorded in the \xfile{.aux} file. In the second run
%    the values are set and transparency is available.
%    \begin{macrocode}
\RequirePackage{auxhook}
\AddLineBeginAux{%
  \string\providecommand{\string\transparent@use}[1]{}%
}
\gdef\TRP@list{/TRP1<</ca 1/CA 1>>}
\def\transparent@use#1{%
  \@ifundefined{TRP#1}{%
    \g@addto@macro\TRP@list{%
      /TRP#1<</ca #1/CA #1>>%
    }%
    \expandafter\gdef\csname TRP#1\endcsname{/TRP#1 gs}%
  }{%
    % #1 is already known, nothing to do
  }%
}
\AtBeginDocument{%
  \TRP@addresource
  \let\transparent@use\@gobble
}
%    \end{macrocode}
%    Unhappily the interface setting page resources is very
%    poor, only a token register \cs{pdfpageresources}.
%    Thus this package tries to be cooperative in the way that
%    it embeds the previous contents of \cs{pdfpageresources}.
%    However it does not solve the problem, if several packages
%    want to set |/ExtGState|.
%    \begin{macrocode}
\def\TRP@addresource{%
  \begingroup
    \edef\x{\endgroup
      \TRP@pdfpageresources{%
        \the\TRP@pdfpageresources
        /ExtGState<<\TRP@list>>%
      }%
    }%
  \x
}
\newif\ifTRP@rerun
\xdef\TRP@colorstack{%
  \TRP@pdfcolorstackinit page direct{/TRP1 gs}%
}
%    \end{macrocode}
%    \begin{macro}{\transparent}
%    \begin{macrocode}
\newcommand*{\transparent}[1]{%
  \begingroup
    \dimen@=#1\p@\relax
    \ifdim\dimen@>\p@
      \dimen@=\p@
    \fi
    \ifdim\dimen@<\z@
      \dimen@=\z@
    \fi
    \ifdim\dimen@=\p@
      \def\x{1}%
    \else
      \ifdim\dimen@=\z@
        \def\x{0}%
      \else
        \edef\x{\strip@pt\dimen@}%
        \edef\x{\expandafter\@gobble\x}%
      \fi
    \fi
    \if@filesw
      \immediate\write\@auxout{%
        \string\transparent@use{\x}%
      }%
    \fi
    \edef\x{\endgroup
      \def\noexpand\transparent@current{\x}%
    }%
  \x
  \transparent@set
}
%    \end{macrocode}
%    \end{macro}
%    \begin{macrocode}
\AtEndDocument{%
  \ifTRP@rerun
    \PackageWarningNoLine{transparent}{%
      Rerun to get transparencies right%
    }%
  \fi
}
\def\transparent@current{/TRP1 gs}
\def\transparent@set{%
  \@ifundefined{TRP\transparent@current}{%
    \global\TRP@reruntrue
  }{%
    \TRP@pdfcolorstack\TRP@colorstack push{%
      \csname TRP\transparent@current\endcsname
    }%
    \aftergroup\transparent@reset
  }%
}
\def\transparent@reset{%
  \TRP@pdfcolorstack\TRP@colorstack pop\relax
}
%    \end{macrocode}
%    \begin{macro}{\texttransparent}
%    \begin{macrocode}
\newcommand*{\texttransparent}[2]{%
  \protect\leavevmode
  \begingroup
    \transparent{#1}%
    #2%
  \endgroup
}
%    \end{macrocode}
%    \end{macro}
%
%    \begin{macrocode}
%</package>
%    \end{macrocode}
%
% \section{Installation}
%
% \subsection{Download}
%
% \paragraph{Package.} This package is available on
% CTAN\footnote{\CTANpkg{transparent}}:
% \begin{description}
% \item[\CTAN{macros/latex/contrib/transparent/transparent.dtx}] The source file.
% \item[\CTAN{macros/latex/contrib/transparent/transparent.pdf}] Documentation.
% \end{description}
%
%
%
% \subsection{Package installation}
%
% \paragraph{Unpacking.} The \xfile{.dtx} file is a self-extracting
% \docstrip\ archive. The files are extracted by running the
% \xfile{.dtx} through \plainTeX:
% \begin{quote}
%   \verb|tex transparent.dtx|
% \end{quote}
%
% \paragraph{TDS.} Now the different files must be moved into
% the different directories in your installation TDS tree
% (also known as \xfile{texmf} tree):
% \begin{quote}
% \def\t{^^A
% \begin{tabular}{@{}>{\ttfamily}l@{ $\rightarrow$ }>{\ttfamily}l@{}}
%   transparent.sty & tex/latex/transparent/transparent.sty\\
%   transparent-nometadata.sty & tex/latex/transparent/transparent-nometadata.sty\\   
%   transparent.pdf & doc/latex/transparent/transparent.pdf\\
%   transparent-example.tex & doc/latex/transparent/transparent-example.tex\\
%   transparent.dtx & source/latex/transparent/transparent.dtx\\
% \end{tabular}^^A
% }^^A
% \sbox0{\t}^^A
% \ifdim\wd0>\linewidth
%   \begingroup
%     \advance\linewidth by\leftmargin
%     \advance\linewidth by\rightmargin
%   \edef\x{\endgroup
%     \def\noexpand\lw{\the\linewidth}^^A
%   }\x
%   \def\lwbox{^^A
%     \leavevmode
%     \hbox to \linewidth{^^A
%       \kern-\leftmargin\relax
%       \hss
%       \usebox0
%       \hss
%       \kern-\rightmargin\relax
%     }^^A
%   }^^A
%   \ifdim\wd0>\lw
%     \sbox0{\small\t}^^A
%     \ifdim\wd0>\linewidth
%       \ifdim\wd0>\lw
%         \sbox0{\footnotesize\t}^^A
%         \ifdim\wd0>\linewidth
%           \ifdim\wd0>\lw
%             \sbox0{\scriptsize\t}^^A
%             \ifdim\wd0>\linewidth
%               \ifdim\wd0>\lw
%                 \sbox0{\tiny\t}^^A
%                 \ifdim\wd0>\linewidth
%                   \lwbox
%                 \else
%                   \usebox0
%                 \fi
%               \else
%                 \lwbox
%               \fi
%             \else
%               \usebox0
%             \fi
%           \else
%             \lwbox
%           \fi
%         \else
%           \usebox0
%         \fi
%       \else
%         \lwbox
%       \fi
%     \else
%       \usebox0
%     \fi
%   \else
%     \lwbox
%   \fi
% \else
%   \usebox0
% \fi
% \end{quote}
% If you have a \xfile{docstrip.cfg} that configures and enables \docstrip's
% TDS installing feature, then some files can already be in the right
% place, see the documentation of \docstrip.
%
% \subsection{Refresh file name databases}
%
% If your \TeX~distribution
% (\TeX\,Live, \mikTeX, \dots) relies on file name databases, you must refresh
% these. For example, \TeX\,Live\ users run \verb|texhash| or
% \verb|mktexlsr|.
%
% \subsection{Some details for the interested}
%
% \paragraph{Unpacking with \LaTeX.}
% The \xfile{.dtx} chooses its action depending on the format:
% \begin{description}
% \item[\plainTeX:] Run \docstrip\ and extract the files.
% \item[\LaTeX:] Generate the documentation.
% \end{description}
% If you insist on using \LaTeX\ for \docstrip\ (really,
% \docstrip\ does not need \LaTeX), then inform the autodetect routine
% about your intention:
% \begin{quote}
%   \verb|latex \let\install=y\input{transparent.dtx}|
% \end{quote}
% Do not forget to quote the argument according to the demands
% of your shell.
%
% \paragraph{Generating the documentation.}
% You can use both the \xfile{.dtx} or the \xfile{.drv} to generate
% the documentation. The process can be configured by the
% configuration file \xfile{ltxdoc.cfg}. For instance, put this
% line into this file, if you want to have A4 as paper format:
% \begin{quote}
%   \verb|\PassOptionsToClass{a4paper}{article}|
% \end{quote}
% An example follows how to generate the
% documentation with pdf\LaTeX:
% \begin{quote}
%\begin{verbatim}
%pdflatex transparent.dtx
%makeindex -s gind.ist transparent.idx
%pdflatex transparent.dtx
%makeindex -s gind.ist transparent.idx
%pdflatex transparent.dtx
%\end{verbatim}
% \end{quote}
%
% \begin{History}
%   \begin{Version}{2007/01/08 v1.0}
%   \item
%     First version.
%   \end{Version}
%   \begin{Version}{2016/05/16 v1.1}
%   \item
%     Documentation updates.
%   \end{Version}
%   \begin{Version}{2018/09/10 v1.2}
%   \item
%     Update for LuaTeX, remove dependency on Pdf\TeX\ command names.
%   \end{Version}
%   \begin{Version}{2018/11/18 v1.3}
%   \item
%     Added code for pgf compatibility, see https://github.com/ho-tex/transparent/issues/1
%   \end{Version}
%   \begin{Version}{2019/11/29 v1.4}
%   \item
%     Documentation updates.
% \item
% Use \xpackage{iftex} package.
%   \end{Version}
%   \begin{Version}{2022-10-27 v1.5}
%   \item Made the package compatible with the PDF management.
%   \end{Version}
% \end{History}
%
% \PrintIndex
%
% \Finale
\endinput
|
% \end{quote}
% Do not forget to quote the argument according to the demands
% of your shell.
%
% \paragraph{Generating the documentation.}
% You can use both the \xfile{.dtx} or the \xfile{.drv} to generate
% the documentation. The process can be configured by the
% configuration file \xfile{ltxdoc.cfg}. For instance, put this
% line into this file, if you want to have A4 as paper format:
% \begin{quote}
%   \verb|\PassOptionsToClass{a4paper}{article}|
% \end{quote}
% An example follows how to generate the
% documentation with pdf\LaTeX:
% \begin{quote}
%\begin{verbatim}
%pdflatex transparent.dtx
%makeindex -s gind.ist transparent.idx
%pdflatex transparent.dtx
%makeindex -s gind.ist transparent.idx
%pdflatex transparent.dtx
%\end{verbatim}
% \end{quote}
%
% \begin{History}
%   \begin{Version}{2007/01/08 v1.0}
%   \item
%     First version.
%   \end{Version}
%   \begin{Version}{2016/05/16 v1.1}
%   \item
%     Documentation updates.
%   \end{Version}
%   \begin{Version}{2018/09/10 v1.2}
%   \item
%     Update for LuaTeX, remove dependency on Pdf\TeX\ command names.
%   \end{Version}
%   \begin{Version}{2018/11/18 v1.3}
%   \item
%     Added code for pgf compatibility, see https://github.com/ho-tex/transparent/issues/1
%   \end{Version}
%   \begin{Version}{2019/11/29 v1.4}
%   \item
%     Documentation updates.
% \item
% Use \xpackage{iftex} package.
%   \end{Version}
%   \begin{Version}{2022-10-27 v1.5}
%   \item Made the package compatible with the PDF management.
%   \end{Version}
% \end{History}
%
% \PrintIndex
%
% \Finale
\endinput
|
% \end{quote}
% Do not forget to quote the argument according to the demands
% of your shell.
%
% \paragraph{Generating the documentation.}
% You can use both the \xfile{.dtx} or the \xfile{.drv} to generate
% the documentation. The process can be configured by the
% configuration file \xfile{ltxdoc.cfg}. For instance, put this
% line into this file, if you want to have A4 as paper format:
% \begin{quote}
%   \verb|\PassOptionsToClass{a4paper}{article}|
% \end{quote}
% An example follows how to generate the
% documentation with pdf\LaTeX:
% \begin{quote}
%\begin{verbatim}
%pdflatex transparent.dtx
%makeindex -s gind.ist transparent.idx
%pdflatex transparent.dtx
%makeindex -s gind.ist transparent.idx
%pdflatex transparent.dtx
%\end{verbatim}
% \end{quote}
%
% \begin{History}
%   \begin{Version}{2007/01/08 v1.0}
%   \item
%     First version.
%   \end{Version}
%   \begin{Version}{2016/05/16 v1.1}
%   \item
%     Documentation updates.
%   \end{Version}
%   \begin{Version}{2018/09/10 v1.2}
%   \item
%     Update for LuaTeX, remove dependency on Pdf\TeX\ command names.
%   \end{Version}
%   \begin{Version}{2018/11/18 v1.3}
%   \item
%     Added code for pgf compatibility, see https://github.com/ho-tex/transparent/issues/1
%   \end{Version}
%   \begin{Version}{2019/11/29 v1.4}
%   \item
%     Documentation updates.
% \item
% Use \xpackage{iftex} package.
%   \end{Version}
%   \begin{Version}{2022-10-27 v1.5}
%   \item Made the package compatible with the PDF management.
%   \end{Version}
% \end{History}
%
% \PrintIndex
%
% \Finale
\endinput

%        (quote the arguments according to the demands of your shell)
%
% Documentation:
%    (a) If transparent.drv is present:
%           latex transparent.drv
%    (b) Without transparent.drv:
%           latex transparent.dtx; ...
%    The class ltxdoc loads the configuration file ltxdoc.cfg
%    if available. Here you can specify further options, e.g.
%    use A4 as paper format:
%       \PassOptionsToClass{a4paper}{article}
%
%    Programm calls to get the documentation (example):
%       pdflatex transparent.dtx
%       makeindex -s gind.ist transparent.idx
%       pdflatex transparent.dtx
%       makeindex -s gind.ist transparent.idx
%       pdflatex transparent.dtx
%
% Installation:
%    TDS:tex/latex/transparent/transparent.sty
%    TDS:tex/latex/transparent/transparent-nometadata.sty
%    TDS:doc/latex/transparent/transparent.pdf
%    TDS:source/latex/transparent/transparent.dtx
%
%<*ignore>
\begingroup
  \catcode123=1 %
  \catcode125=2 %
  \def\x{LaTeX2e}%
\expandafter\endgroup
\ifcase 0\ifx\install y1\fi\expandafter
         \ifx\csname processbatchFile\endcsname\relax\else1\fi
         \ifx\fmtname\x\else 1\fi\relax
\else\csname fi\endcsname
%</ignore>
%<*install>
\input docstrip.tex
\Msg{************************************************************************}
\Msg{* Installation}
\Msg{* Package: transparent 2022-10-27 v1.5 Transparency via pdfTeX's color stack (HO)}
\Msg{************************************************************************}

\keepsilent
\askforoverwritefalse

\let\MetaPrefix\relax
\preamble

This is a generated file.

Project: transparent
Version: 2022-10-27 v1.5

Copyright (C)
   2007 Heiko Oberdiek
   2016-2022 Oberdiek Package Support Group

This work may be distributed and/or modified under the
conditions of the LaTeX Project Public License, either
version 1.3c of this license or (at your option) any later
version. This version of this license is in
   https://www.latex-project.org/lppl/lppl-1-3c.txt
and the latest version of this license is in
   https://www.latex-project.org/lppl.txt
and version 1.3 or later is part of all distributions of
LaTeX version 2005/12/01 or later.

This work has the LPPL maintenance status "maintained".

The Current Maintainers of this work are
Heiko Oberdiek and the Oberdiek Package Support Group
https://github.com/ho-tex/transparent/issues


This work consists of the main source file transparent.dtx
and the derived files
   transparent.sty, transparent-nometadata.sty, transparent.pdf,
   transparent.ins, transparent.drv, transparent-example.tex.

\endpreamble
\let\MetaPrefix\DoubleperCent

\generate{%
  \file{transparent.ins}{\from{transparent.dtx}{install}}%
  \file{transparent.drv}{\from{transparent.dtx}{driver}}%
  \usedir{tex/latex/transparent}%
  \file{transparent-nometadata.sty}{\from{transparent.dtx}{package}}%
  \file{transparent.sty}{\from{transparent.dtx}{package-new}}%
  \usedir{doc/latex/transparent}%
  \file{transparent-example.tex}{\from{transparent.dtx}{example}}%
}

\catcode32=13\relax% active space
\let =\space%
\Msg{************************************************************************}
\Msg{*}
\Msg{* To finish the installation you have to move the following}
\Msg{* file into a directory searched by TeX:}
\Msg{*}
\Msg{*     transparent.sty}
\Msg{*     transparent-nometadata.sty}
\Msg{*}
\Msg{* To produce the documentation run the file `transparent.drv'}
\Msg{* through LaTeX.}
\Msg{*}
\Msg{* Happy TeXing!}
\Msg{*}
\Msg{************************************************************************}

\endbatchfile
%</install>
%<*ignore>
\fi
%</ignore>
%<*driver>
\NeedsTeXFormat{LaTeX2e}
\ProvidesFile{transparent.drv}%
  [2022-10-27 v1.5 Transparency via pdfTeX's color stack (HO)]%
\documentclass{ltxdoc}
\usepackage{holtxdoc}[2011/11/22]
\begin{document}
  \DocInput{transparent.dtx}%
\end{document}
%</driver>
% \fi
%
%
%
% \GetFileInfo{transparent.drv}
%
% \title{The \xpackage{transparent} package}
% \date{2022-10-27 v1.5}
% \author{Heiko Oberdiek\thanks
% {Please report any issues at \url{https://github.com/ho-tex/transparent/issues}}}
%
% \maketitle
%
% \begin{abstract}
% \pdfTeX\ and lua\TeX{} supports several color stacks. This
% package shows, how a separate color stack can be used for transparency,
% a property besides color, that also works across page break.
% \end{abstract}
%
% \tableofcontents
%
% \section{User interface}
%
% The package \xpackage{transparent} defines \cs{transparent} and
% \cs{texttransparent}. They are used like \cs{color} and \cs{textcolor}.
% The first argument is the transparency value between 0 and 1 where 0
% is fully transparent and 1 is opaque.
% 
% \section{Engine support}
% If the PDF management is loaded by using \verb+\DocumentMetadata{}+ at the beginning
% of the document the package makes use of \texttt{l3opacity} and
% can be used with all engines supported by \texttt{l3backend}.
% But only with \pdfTeX\ and lua\TeX{} the transparency will also work across page break.
% 
% 
% Without the PDF management, the package works only with \pdfTeX\ and lua\TeX{}.
% It then uses the original code by Heiko Oberdiek. It then uses the
% primitive \cs{pdfpageresources} and can clash with other packages that also use
% \cs{pdfpageresources}.
% 
%
% \section{Example for usage}
%    \begin{macrocode}
%<*example>
\documentclass[12pt]{article}

\usepackage{color}
\usepackage{transparent}

\begin{document}
\colorbox{yellow}{%
  \bfseries
  \color{blue}%
  Blue and %
  \transparent{0.6}%
  transparent blue%
}

\bigskip
Hello World
\texttransparent{0.5}{Hello\newpage World}
Hello World
\end{document}
%</example>
%    \end{macrocode}
%
% \StopEventually{
% }
%
% \section{Implementation}
% \subsection{New version using the methods of the PDF management} 
% Identification
%    \begin{macrocode}
%<*package-new>
\NeedsTeXFormat{LaTeX2e}[2020/10/01]
\ProvidesExplPackage{transparent}{2022-10-27}{1.5}
  {Transparency with color stacks}%
%    \end{macrocode}
% Testing if the pdfmanagement is used
%    \begin{macrocode}
\@ifl@t@r\fmtversion{2022-06-01}
  {
    \IfDocumentMetadataTF {}{\RequirePackage{transparent-nometadata}}
    \IfDocumentMetadataTF {}{\endinput}
  }
  {\RequirePackage{transparent-nometadata}}  
%    \end{macrocode}
%    \begin{macrocode}
\RequirePackage{l3opacity}

\NewDocumentCommand{\transparent} { m }
      {
       \opacity_select:n{\fp_eval:n{ min(max(0,#1),1) } }
      }

\NewDocumentCommand{\texttransparent}{m m}
  {
   \mode_leave_vertical:
   \group_begin:
     \transparent{#1}
     #2
   \group_end:
  }
%</package-new>
%    \end{macrocode}
% \subsection{Old version without the PDF management}
%    \begin{macrocode}
%<*package>
\NeedsTeXFormat{LaTeX2e}
\ProvidesPackage{transparent-nometadata}%
  [2022-10-27 v1.5 Transparency via pdfTeX's color stack (HO)]%
%    \end{macrocode}
%
% \subsection{Initial checks}
%
% \subsubsection{Check for \pdfTeX\ in PDF mode}
%    \begin{macrocode}
\RequirePackage{iftex}
\ifpdf
\else
  \PackageWarningNoLine{transparent}{%
    Loading aborted, because pdfTeX is not running in PDF mode%
  }%
  \expandafter\endinput
\fi
%    \end{macrocode}
%
% \subsubsection{Check \pdfTeX\ or Lua\TeX\ version}
%    \begin{macrocode}
\ifx\pdfextension\@undefined
  \let\TRP@pdfcolorstackinit\pdfcolorstackinit
  \let\TRP@pdfpageresources\pdfpageresources
  \let\TRP@pdfcolorstack\pdfcolorstack
\else
  \def\TRP@pdfcolorstackinit           {\pdffeedback colorstackinit}
  \protected\edef\TRP@pdfpageresources {\pdfvariable pageresources}
  \protected\def\TRP@pdfcolorstack     {\pdfextension colorstack}
\fi
%    \end{macrocode}
%
%    \begin{macrocode}
\ifcsname TRP@pdfcolorstackinit\endcsname\else
  \PackageWarningNoLine{transparent}{%
    Your pdfTeX version does not support color stacks%
  }%
  \expandafter\endinput
\fi
%    \end{macrocode}
%
% \subsection{Compatibility with pgf}
%  https://github.com/ho-tex/transparent/issues/1
%    \begin{macrocode}
\AtBeginDocument
 {%
   \ifcsname pgfutil@addpdfresource@extgs\endcsname
    \let\TRP@addresource\relax
    \pgfutil@addpdfresource@extgs{\TRP@list}%
   \fi
 }
%    \end{macrocode}
% \subsection{Transparency}
%
%    The setting for the different transparency values must
%    be added to the page resources. In the first run the values
%    are recorded in the \xfile{.aux} file. In the second run
%    the values are set and transparency is available.
%    \begin{macrocode}
\RequirePackage{auxhook}
\AddLineBeginAux{%
  \string\providecommand{\string\transparent@use}[1]{}%
}
\gdef\TRP@list{/TRP1<</ca 1/CA 1>>}
\def\transparent@use#1{%
  \@ifundefined{TRP#1}{%
    \g@addto@macro\TRP@list{%
      /TRP#1<</ca #1/CA #1>>%
    }%
    \expandafter\gdef\csname TRP#1\endcsname{/TRP#1 gs}%
  }{%
    % #1 is already known, nothing to do
  }%
}
\AtBeginDocument{%
  \TRP@addresource
  \let\transparent@use\@gobble
}
%    \end{macrocode}
%    Unhappily the interface setting page resources is very
%    poor, only a token register \cs{pdfpageresources}.
%    Thus this package tries to be cooperative in the way that
%    it embeds the previous contents of \cs{pdfpageresources}.
%    However it does not solve the problem, if several packages
%    want to set |/ExtGState|.
%    \begin{macrocode}
\def\TRP@addresource{%
  \begingroup
    \edef\x{\endgroup
      \TRP@pdfpageresources{%
        \the\TRP@pdfpageresources
        /ExtGState<<\TRP@list>>%
      }%
    }%
  \x
}
\newif\ifTRP@rerun
\xdef\TRP@colorstack{%
  \TRP@pdfcolorstackinit page direct{/TRP1 gs}%
}
%    \end{macrocode}
%    \begin{macro}{\transparent}
%    \begin{macrocode}
\newcommand*{\transparent}[1]{%
  \begingroup
    \dimen@=#1\p@\relax
    \ifdim\dimen@>\p@
      \dimen@=\p@
    \fi
    \ifdim\dimen@<\z@
      \dimen@=\z@
    \fi
    \ifdim\dimen@=\p@
      \def\x{1}%
    \else
      \ifdim\dimen@=\z@
        \def\x{0}%
      \else
        \edef\x{\strip@pt\dimen@}%
        \edef\x{\expandafter\@gobble\x}%
      \fi
    \fi
    \if@filesw
      \immediate\write\@auxout{%
        \string\transparent@use{\x}%
      }%
    \fi
    \edef\x{\endgroup
      \def\noexpand\transparent@current{\x}%
    }%
  \x
  \transparent@set
}
%    \end{macrocode}
%    \end{macro}
%    \begin{macrocode}
\AtEndDocument{%
  \ifTRP@rerun
    \PackageWarningNoLine{transparent}{%
      Rerun to get transparencies right%
    }%
  \fi
}
\def\transparent@current{/TRP1 gs}
\def\transparent@set{%
  \@ifundefined{TRP\transparent@current}{%
    \global\TRP@reruntrue
  }{%
    \TRP@pdfcolorstack\TRP@colorstack push{%
      \csname TRP\transparent@current\endcsname
    }%
    \aftergroup\transparent@reset
  }%
}
\def\transparent@reset{%
  \TRP@pdfcolorstack\TRP@colorstack pop\relax
}
%    \end{macrocode}
%    \begin{macro}{\texttransparent}
%    \begin{macrocode}
\newcommand*{\texttransparent}[2]{%
  \protect\leavevmode
  \begingroup
    \transparent{#1}%
    #2%
  \endgroup
}
%    \end{macrocode}
%    \end{macro}
%
%    \begin{macrocode}
%</package>
%    \end{macrocode}
%
% \section{Installation}
%
% \subsection{Download}
%
% \paragraph{Package.} This package is available on
% CTAN\footnote{\CTANpkg{transparent}}:
% \begin{description}
% \item[\CTAN{macros/latex/contrib/transparent/transparent.dtx}] The source file.
% \item[\CTAN{macros/latex/contrib/transparent/transparent.pdf}] Documentation.
% \end{description}
%
%
%
% \subsection{Package installation}
%
% \paragraph{Unpacking.} The \xfile{.dtx} file is a self-extracting
% \docstrip\ archive. The files are extracted by running the
% \xfile{.dtx} through \plainTeX:
% \begin{quote}
%   \verb|tex transparent.dtx|
% \end{quote}
%
% \paragraph{TDS.} Now the different files must be moved into
% the different directories in your installation TDS tree
% (also known as \xfile{texmf} tree):
% \begin{quote}
% \def\t{^^A
% \begin{tabular}{@{}>{\ttfamily}l@{ $\rightarrow$ }>{\ttfamily}l@{}}
%   transparent.sty & tex/latex/transparent/transparent.sty\\
%   transparent-nometadata.sty & tex/latex/transparent/transparent-nometadata.sty\\   
%   transparent.pdf & doc/latex/transparent/transparent.pdf\\
%   transparent-example.tex & doc/latex/transparent/transparent-example.tex\\
%   transparent.dtx & source/latex/transparent/transparent.dtx\\
% \end{tabular}^^A
% }^^A
% \sbox0{\t}^^A
% \ifdim\wd0>\linewidth
%   \begingroup
%     \advance\linewidth by\leftmargin
%     \advance\linewidth by\rightmargin
%   \edef\x{\endgroup
%     \def\noexpand\lw{\the\linewidth}^^A
%   }\x
%   \def\lwbox{^^A
%     \leavevmode
%     \hbox to \linewidth{^^A
%       \kern-\leftmargin\relax
%       \hss
%       \usebox0
%       \hss
%       \kern-\rightmargin\relax
%     }^^A
%   }^^A
%   \ifdim\wd0>\lw
%     \sbox0{\small\t}^^A
%     \ifdim\wd0>\linewidth
%       \ifdim\wd0>\lw
%         \sbox0{\footnotesize\t}^^A
%         \ifdim\wd0>\linewidth
%           \ifdim\wd0>\lw
%             \sbox0{\scriptsize\t}^^A
%             \ifdim\wd0>\linewidth
%               \ifdim\wd0>\lw
%                 \sbox0{\tiny\t}^^A
%                 \ifdim\wd0>\linewidth
%                   \lwbox
%                 \else
%                   \usebox0
%                 \fi
%               \else
%                 \lwbox
%               \fi
%             \else
%               \usebox0
%             \fi
%           \else
%             \lwbox
%           \fi
%         \else
%           \usebox0
%         \fi
%       \else
%         \lwbox
%       \fi
%     \else
%       \usebox0
%     \fi
%   \else
%     \lwbox
%   \fi
% \else
%   \usebox0
% \fi
% \end{quote}
% If you have a \xfile{docstrip.cfg} that configures and enables \docstrip's
% TDS installing feature, then some files can already be in the right
% place, see the documentation of \docstrip.
%
% \subsection{Refresh file name databases}
%
% If your \TeX~distribution
% (\TeX\,Live, \mikTeX, \dots) relies on file name databases, you must refresh
% these. For example, \TeX\,Live\ users run \verb|texhash| or
% \verb|mktexlsr|.
%
% \subsection{Some details for the interested}
%
% \paragraph{Unpacking with \LaTeX.}
% The \xfile{.dtx} chooses its action depending on the format:
% \begin{description}
% \item[\plainTeX:] Run \docstrip\ and extract the files.
% \item[\LaTeX:] Generate the documentation.
% \end{description}
% If you insist on using \LaTeX\ for \docstrip\ (really,
% \docstrip\ does not need \LaTeX), then inform the autodetect routine
% about your intention:
% \begin{quote}
%   \verb|latex \let\install=y% \iffalse meta-comment
%
% File: transparent.dtx
% Version: 2022-10-27 v1.5
% Info: Transparency via pdfTeX's color stack
%
% Copyright (C)
%    2007 Heiko Oberdiek
%    2016-2022 Oberdiek Package Support Group 2018
%    https://github.com/ho-tex/transparent/issues
%
% This work may be distributed and/or modified under the
% conditions of the LaTeX Project Public License, either
% version 1.3c of this license or (at your option) any later
% version. This version of this license is in
%    https://www.latex-project.org/lppl/lppl-1-3c.txt
% and the latest version of this license is in
%    https://www.latex-project.org/lppl.txt
% and version 1.3 or later is part of all distributions of
% LaTeX version 2005/12/01 or later.
%
% This work has the LPPL maintenance status "maintained".
%
% The Current Maintainers of this work are
% Heiko Oberdiek and the Oberdiek Package Support Group
% https://github.com/ho-tex/transparent/issues
%
% This work consists of the main source file transparent.dtx
% and the derived files
%    transparent.sty, transparent-nometadata.sty, transparent.pdf, 
%    transparent.ins, transparent.drv, transparent-example.tex.
%
% Distribution:
%    CTAN:macros/latex/contrib/transparent/transparent.dtx
%    CTAN:macros/latex/contrib/transparent/transparent.pdf
%
% Unpacking:
%    (a) If transparent.ins is present:
%           tex transparent.ins
%    (b) Without transparent.ins:
%           tex transparent.dtx
%    (c) If you insist on using LaTeX
%           latex \let\install=y% \iffalse meta-comment
%
% File: transparent.dtx
% Version: 2022-10-27 v1.5
% Info: Transparency via pdfTeX's color stack
%
% Copyright (C)
%    2007 Heiko Oberdiek
%    2016-2022 Oberdiek Package Support Group 2018
%    https://github.com/ho-tex/transparent/issues
%
% This work may be distributed and/or modified under the
% conditions of the LaTeX Project Public License, either
% version 1.3c of this license or (at your option) any later
% version. This version of this license is in
%    https://www.latex-project.org/lppl/lppl-1-3c.txt
% and the latest version of this license is in
%    https://www.latex-project.org/lppl.txt
% and version 1.3 or later is part of all distributions of
% LaTeX version 2005/12/01 or later.
%
% This work has the LPPL maintenance status "maintained".
%
% The Current Maintainers of this work are
% Heiko Oberdiek and the Oberdiek Package Support Group
% https://github.com/ho-tex/transparent/issues
%
% This work consists of the main source file transparent.dtx
% and the derived files
%    transparent.sty, transparent-nometadata.sty, transparent.pdf, 
%    transparent.ins, transparent.drv, transparent-example.tex.
%
% Distribution:
%    CTAN:macros/latex/contrib/transparent/transparent.dtx
%    CTAN:macros/latex/contrib/transparent/transparent.pdf
%
% Unpacking:
%    (a) If transparent.ins is present:
%           tex transparent.ins
%    (b) Without transparent.ins:
%           tex transparent.dtx
%    (c) If you insist on using LaTeX
%           latex \let\install=y% \iffalse meta-comment
%
% File: transparent.dtx
% Version: 2022-10-27 v1.5
% Info: Transparency via pdfTeX's color stack
%
% Copyright (C)
%    2007 Heiko Oberdiek
%    2016-2022 Oberdiek Package Support Group 2018
%    https://github.com/ho-tex/transparent/issues
%
% This work may be distributed and/or modified under the
% conditions of the LaTeX Project Public License, either
% version 1.3c of this license or (at your option) any later
% version. This version of this license is in
%    https://www.latex-project.org/lppl/lppl-1-3c.txt
% and the latest version of this license is in
%    https://www.latex-project.org/lppl.txt
% and version 1.3 or later is part of all distributions of
% LaTeX version 2005/12/01 or later.
%
% This work has the LPPL maintenance status "maintained".
%
% The Current Maintainers of this work are
% Heiko Oberdiek and the Oberdiek Package Support Group
% https://github.com/ho-tex/transparent/issues
%
% This work consists of the main source file transparent.dtx
% and the derived files
%    transparent.sty, transparent-nometadata.sty, transparent.pdf, 
%    transparent.ins, transparent.drv, transparent-example.tex.
%
% Distribution:
%    CTAN:macros/latex/contrib/transparent/transparent.dtx
%    CTAN:macros/latex/contrib/transparent/transparent.pdf
%
% Unpacking:
%    (a) If transparent.ins is present:
%           tex transparent.ins
%    (b) Without transparent.ins:
%           tex transparent.dtx
%    (c) If you insist on using LaTeX
%           latex \let\install=y\input{transparent.dtx}
%        (quote the arguments according to the demands of your shell)
%
% Documentation:
%    (a) If transparent.drv is present:
%           latex transparent.drv
%    (b) Without transparent.drv:
%           latex transparent.dtx; ...
%    The class ltxdoc loads the configuration file ltxdoc.cfg
%    if available. Here you can specify further options, e.g.
%    use A4 as paper format:
%       \PassOptionsToClass{a4paper}{article}
%
%    Programm calls to get the documentation (example):
%       pdflatex transparent.dtx
%       makeindex -s gind.ist transparent.idx
%       pdflatex transparent.dtx
%       makeindex -s gind.ist transparent.idx
%       pdflatex transparent.dtx
%
% Installation:
%    TDS:tex/latex/transparent/transparent.sty
%    TDS:tex/latex/transparent/transparent-nometadata.sty
%    TDS:doc/latex/transparent/transparent.pdf
%    TDS:source/latex/transparent/transparent.dtx
%
%<*ignore>
\begingroup
  \catcode123=1 %
  \catcode125=2 %
  \def\x{LaTeX2e}%
\expandafter\endgroup
\ifcase 0\ifx\install y1\fi\expandafter
         \ifx\csname processbatchFile\endcsname\relax\else1\fi
         \ifx\fmtname\x\else 1\fi\relax
\else\csname fi\endcsname
%</ignore>
%<*install>
\input docstrip.tex
\Msg{************************************************************************}
\Msg{* Installation}
\Msg{* Package: transparent 2022-10-27 v1.5 Transparency via pdfTeX's color stack (HO)}
\Msg{************************************************************************}

\keepsilent
\askforoverwritefalse

\let\MetaPrefix\relax
\preamble

This is a generated file.

Project: transparent
Version: 2022-10-27 v1.5

Copyright (C)
   2007 Heiko Oberdiek
   2016-2022 Oberdiek Package Support Group

This work may be distributed and/or modified under the
conditions of the LaTeX Project Public License, either
version 1.3c of this license or (at your option) any later
version. This version of this license is in
   https://www.latex-project.org/lppl/lppl-1-3c.txt
and the latest version of this license is in
   https://www.latex-project.org/lppl.txt
and version 1.3 or later is part of all distributions of
LaTeX version 2005/12/01 or later.

This work has the LPPL maintenance status "maintained".

The Current Maintainers of this work are
Heiko Oberdiek and the Oberdiek Package Support Group
https://github.com/ho-tex/transparent/issues


This work consists of the main source file transparent.dtx
and the derived files
   transparent.sty, transparent-nometadata.sty, transparent.pdf,
   transparent.ins, transparent.drv, transparent-example.tex.

\endpreamble
\let\MetaPrefix\DoubleperCent

\generate{%
  \file{transparent.ins}{\from{transparent.dtx}{install}}%
  \file{transparent.drv}{\from{transparent.dtx}{driver}}%
  \usedir{tex/latex/transparent}%
  \file{transparent-nometadata.sty}{\from{transparent.dtx}{package}}%
  \file{transparent.sty}{\from{transparent.dtx}{package-new}}%
  \usedir{doc/latex/transparent}%
  \file{transparent-example.tex}{\from{transparent.dtx}{example}}%
}

\catcode32=13\relax% active space
\let =\space%
\Msg{************************************************************************}
\Msg{*}
\Msg{* To finish the installation you have to move the following}
\Msg{* file into a directory searched by TeX:}
\Msg{*}
\Msg{*     transparent.sty}
\Msg{*     transparent-nometadata.sty}
\Msg{*}
\Msg{* To produce the documentation run the file `transparent.drv'}
\Msg{* through LaTeX.}
\Msg{*}
\Msg{* Happy TeXing!}
\Msg{*}
\Msg{************************************************************************}

\endbatchfile
%</install>
%<*ignore>
\fi
%</ignore>
%<*driver>
\NeedsTeXFormat{LaTeX2e}
\ProvidesFile{transparent.drv}%
  [2022-10-27 v1.5 Transparency via pdfTeX's color stack (HO)]%
\documentclass{ltxdoc}
\usepackage{holtxdoc}[2011/11/22]
\begin{document}
  \DocInput{transparent.dtx}%
\end{document}
%</driver>
% \fi
%
%
%
% \GetFileInfo{transparent.drv}
%
% \title{The \xpackage{transparent} package}
% \date{2022-10-27 v1.5}
% \author{Heiko Oberdiek\thanks
% {Please report any issues at \url{https://github.com/ho-tex/transparent/issues}}}
%
% \maketitle
%
% \begin{abstract}
% \pdfTeX\ and lua\TeX{} supports several color stacks. This
% package shows, how a separate color stack can be used for transparency,
% a property besides color, that also works across page break.
% \end{abstract}
%
% \tableofcontents
%
% \section{User interface}
%
% The package \xpackage{transparent} defines \cs{transparent} and
% \cs{texttransparent}. They are used like \cs{color} and \cs{textcolor}.
% The first argument is the transparency value between 0 and 1 where 0
% is fully transparent and 1 is opaque.
% 
% \section{Engine support}
% If the PDF management is loaded by using \verb+\DocumentMetadata{}+ at the beginning
% of the document the package makes use of \texttt{l3opacity} and
% can be used with all engines supported by \texttt{l3backend}.
% But only with \pdfTeX\ and lua\TeX{} the transparency will also work across page break.
% 
% 
% Without the PDF management, the package works only with \pdfTeX\ and lua\TeX{}.
% It then uses the original code by Heiko Oberdiek. It then uses the
% primitive \cs{pdfpageresources} and can clash with other packages that also use
% \cs{pdfpageresources}.
% 
%
% \section{Example for usage}
%    \begin{macrocode}
%<*example>
\documentclass[12pt]{article}

\usepackage{color}
\usepackage{transparent}

\begin{document}
\colorbox{yellow}{%
  \bfseries
  \color{blue}%
  Blue and %
  \transparent{0.6}%
  transparent blue%
}

\bigskip
Hello World
\texttransparent{0.5}{Hello\newpage World}
Hello World
\end{document}
%</example>
%    \end{macrocode}
%
% \StopEventually{
% }
%
% \section{Implementation}
% \subsection{New version using the methods of the PDF management} 
% Identification
%    \begin{macrocode}
%<*package-new>
\NeedsTeXFormat{LaTeX2e}[2020/10/01]
\ProvidesExplPackage{transparent}{2022-10-27}{1.5}
  {Transparency with color stacks}%
%    \end{macrocode}
% Testing if the pdfmanagement is used
%    \begin{macrocode}
\@ifl@t@r\fmtversion{2022-06-01}
  {
    \IfDocumentMetadataTF {}{\RequirePackage{transparent-nometadata}}
    \IfDocumentMetadataTF {}{\endinput}
  }
  {\RequirePackage{transparent-nometadata}}  
%    \end{macrocode}
%    \begin{macrocode}
\RequirePackage{l3opacity}

\NewDocumentCommand{\transparent} { m }
      {
       \opacity_select:n{\fp_eval:n{ min(max(0,#1),1) } }
      }

\NewDocumentCommand{\texttransparent}{m m}
  {
   \mode_leave_vertical:
   \group_begin:
     \transparent{#1}
     #2
   \group_end:
  }
%</package-new>
%    \end{macrocode}
% \subsection{Old version without the PDF management}
%    \begin{macrocode}
%<*package>
\NeedsTeXFormat{LaTeX2e}
\ProvidesPackage{transparent-nometadata}%
  [2022-10-27 v1.5 Transparency via pdfTeX's color stack (HO)]%
%    \end{macrocode}
%
% \subsection{Initial checks}
%
% \subsubsection{Check for \pdfTeX\ in PDF mode}
%    \begin{macrocode}
\RequirePackage{iftex}
\ifpdf
\else
  \PackageWarningNoLine{transparent}{%
    Loading aborted, because pdfTeX is not running in PDF mode%
  }%
  \expandafter\endinput
\fi
%    \end{macrocode}
%
% \subsubsection{Check \pdfTeX\ or Lua\TeX\ version}
%    \begin{macrocode}
\ifx\pdfextension\@undefined
  \let\TRP@pdfcolorstackinit\pdfcolorstackinit
  \let\TRP@pdfpageresources\pdfpageresources
  \let\TRP@pdfcolorstack\pdfcolorstack
\else
  \def\TRP@pdfcolorstackinit           {\pdffeedback colorstackinit}
  \protected\edef\TRP@pdfpageresources {\pdfvariable pageresources}
  \protected\def\TRP@pdfcolorstack     {\pdfextension colorstack}
\fi
%    \end{macrocode}
%
%    \begin{macrocode}
\ifcsname TRP@pdfcolorstackinit\endcsname\else
  \PackageWarningNoLine{transparent}{%
    Your pdfTeX version does not support color stacks%
  }%
  \expandafter\endinput
\fi
%    \end{macrocode}
%
% \subsection{Compatibility with pgf}
%  https://github.com/ho-tex/transparent/issues/1
%    \begin{macrocode}
\AtBeginDocument
 {%
   \ifcsname pgfutil@addpdfresource@extgs\endcsname
    \let\TRP@addresource\relax
    \pgfutil@addpdfresource@extgs{\TRP@list}%
   \fi
 }
%    \end{macrocode}
% \subsection{Transparency}
%
%    The setting for the different transparency values must
%    be added to the page resources. In the first run the values
%    are recorded in the \xfile{.aux} file. In the second run
%    the values are set and transparency is available.
%    \begin{macrocode}
\RequirePackage{auxhook}
\AddLineBeginAux{%
  \string\providecommand{\string\transparent@use}[1]{}%
}
\gdef\TRP@list{/TRP1<</ca 1/CA 1>>}
\def\transparent@use#1{%
  \@ifundefined{TRP#1}{%
    \g@addto@macro\TRP@list{%
      /TRP#1<</ca #1/CA #1>>%
    }%
    \expandafter\gdef\csname TRP#1\endcsname{/TRP#1 gs}%
  }{%
    % #1 is already known, nothing to do
  }%
}
\AtBeginDocument{%
  \TRP@addresource
  \let\transparent@use\@gobble
}
%    \end{macrocode}
%    Unhappily the interface setting page resources is very
%    poor, only a token register \cs{pdfpageresources}.
%    Thus this package tries to be cooperative in the way that
%    it embeds the previous contents of \cs{pdfpageresources}.
%    However it does not solve the problem, if several packages
%    want to set |/ExtGState|.
%    \begin{macrocode}
\def\TRP@addresource{%
  \begingroup
    \edef\x{\endgroup
      \TRP@pdfpageresources{%
        \the\TRP@pdfpageresources
        /ExtGState<<\TRP@list>>%
      }%
    }%
  \x
}
\newif\ifTRP@rerun
\xdef\TRP@colorstack{%
  \TRP@pdfcolorstackinit page direct{/TRP1 gs}%
}
%    \end{macrocode}
%    \begin{macro}{\transparent}
%    \begin{macrocode}
\newcommand*{\transparent}[1]{%
  \begingroup
    \dimen@=#1\p@\relax
    \ifdim\dimen@>\p@
      \dimen@=\p@
    \fi
    \ifdim\dimen@<\z@
      \dimen@=\z@
    \fi
    \ifdim\dimen@=\p@
      \def\x{1}%
    \else
      \ifdim\dimen@=\z@
        \def\x{0}%
      \else
        \edef\x{\strip@pt\dimen@}%
        \edef\x{\expandafter\@gobble\x}%
      \fi
    \fi
    \if@filesw
      \immediate\write\@auxout{%
        \string\transparent@use{\x}%
      }%
    \fi
    \edef\x{\endgroup
      \def\noexpand\transparent@current{\x}%
    }%
  \x
  \transparent@set
}
%    \end{macrocode}
%    \end{macro}
%    \begin{macrocode}
\AtEndDocument{%
  \ifTRP@rerun
    \PackageWarningNoLine{transparent}{%
      Rerun to get transparencies right%
    }%
  \fi
}
\def\transparent@current{/TRP1 gs}
\def\transparent@set{%
  \@ifundefined{TRP\transparent@current}{%
    \global\TRP@reruntrue
  }{%
    \TRP@pdfcolorstack\TRP@colorstack push{%
      \csname TRP\transparent@current\endcsname
    }%
    \aftergroup\transparent@reset
  }%
}
\def\transparent@reset{%
  \TRP@pdfcolorstack\TRP@colorstack pop\relax
}
%    \end{macrocode}
%    \begin{macro}{\texttransparent}
%    \begin{macrocode}
\newcommand*{\texttransparent}[2]{%
  \protect\leavevmode
  \begingroup
    \transparent{#1}%
    #2%
  \endgroup
}
%    \end{macrocode}
%    \end{macro}
%
%    \begin{macrocode}
%</package>
%    \end{macrocode}
%
% \section{Installation}
%
% \subsection{Download}
%
% \paragraph{Package.} This package is available on
% CTAN\footnote{\CTANpkg{transparent}}:
% \begin{description}
% \item[\CTAN{macros/latex/contrib/transparent/transparent.dtx}] The source file.
% \item[\CTAN{macros/latex/contrib/transparent/transparent.pdf}] Documentation.
% \end{description}
%
%
%
% \subsection{Package installation}
%
% \paragraph{Unpacking.} The \xfile{.dtx} file is a self-extracting
% \docstrip\ archive. The files are extracted by running the
% \xfile{.dtx} through \plainTeX:
% \begin{quote}
%   \verb|tex transparent.dtx|
% \end{quote}
%
% \paragraph{TDS.} Now the different files must be moved into
% the different directories in your installation TDS tree
% (also known as \xfile{texmf} tree):
% \begin{quote}
% \def\t{^^A
% \begin{tabular}{@{}>{\ttfamily}l@{ $\rightarrow$ }>{\ttfamily}l@{}}
%   transparent.sty & tex/latex/transparent/transparent.sty\\
%   transparent-nometadata.sty & tex/latex/transparent/transparent-nometadata.sty\\   
%   transparent.pdf & doc/latex/transparent/transparent.pdf\\
%   transparent-example.tex & doc/latex/transparent/transparent-example.tex\\
%   transparent.dtx & source/latex/transparent/transparent.dtx\\
% \end{tabular}^^A
% }^^A
% \sbox0{\t}^^A
% \ifdim\wd0>\linewidth
%   \begingroup
%     \advance\linewidth by\leftmargin
%     \advance\linewidth by\rightmargin
%   \edef\x{\endgroup
%     \def\noexpand\lw{\the\linewidth}^^A
%   }\x
%   \def\lwbox{^^A
%     \leavevmode
%     \hbox to \linewidth{^^A
%       \kern-\leftmargin\relax
%       \hss
%       \usebox0
%       \hss
%       \kern-\rightmargin\relax
%     }^^A
%   }^^A
%   \ifdim\wd0>\lw
%     \sbox0{\small\t}^^A
%     \ifdim\wd0>\linewidth
%       \ifdim\wd0>\lw
%         \sbox0{\footnotesize\t}^^A
%         \ifdim\wd0>\linewidth
%           \ifdim\wd0>\lw
%             \sbox0{\scriptsize\t}^^A
%             \ifdim\wd0>\linewidth
%               \ifdim\wd0>\lw
%                 \sbox0{\tiny\t}^^A
%                 \ifdim\wd0>\linewidth
%                   \lwbox
%                 \else
%                   \usebox0
%                 \fi
%               \else
%                 \lwbox
%               \fi
%             \else
%               \usebox0
%             \fi
%           \else
%             \lwbox
%           \fi
%         \else
%           \usebox0
%         \fi
%       \else
%         \lwbox
%       \fi
%     \else
%       \usebox0
%     \fi
%   \else
%     \lwbox
%   \fi
% \else
%   \usebox0
% \fi
% \end{quote}
% If you have a \xfile{docstrip.cfg} that configures and enables \docstrip's
% TDS installing feature, then some files can already be in the right
% place, see the documentation of \docstrip.
%
% \subsection{Refresh file name databases}
%
% If your \TeX~distribution
% (\TeX\,Live, \mikTeX, \dots) relies on file name databases, you must refresh
% these. For example, \TeX\,Live\ users run \verb|texhash| or
% \verb|mktexlsr|.
%
% \subsection{Some details for the interested}
%
% \paragraph{Unpacking with \LaTeX.}
% The \xfile{.dtx} chooses its action depending on the format:
% \begin{description}
% \item[\plainTeX:] Run \docstrip\ and extract the files.
% \item[\LaTeX:] Generate the documentation.
% \end{description}
% If you insist on using \LaTeX\ for \docstrip\ (really,
% \docstrip\ does not need \LaTeX), then inform the autodetect routine
% about your intention:
% \begin{quote}
%   \verb|latex \let\install=y\input{transparent.dtx}|
% \end{quote}
% Do not forget to quote the argument according to the demands
% of your shell.
%
% \paragraph{Generating the documentation.}
% You can use both the \xfile{.dtx} or the \xfile{.drv} to generate
% the documentation. The process can be configured by the
% configuration file \xfile{ltxdoc.cfg}. For instance, put this
% line into this file, if you want to have A4 as paper format:
% \begin{quote}
%   \verb|\PassOptionsToClass{a4paper}{article}|
% \end{quote}
% An example follows how to generate the
% documentation with pdf\LaTeX:
% \begin{quote}
%\begin{verbatim}
%pdflatex transparent.dtx
%makeindex -s gind.ist transparent.idx
%pdflatex transparent.dtx
%makeindex -s gind.ist transparent.idx
%pdflatex transparent.dtx
%\end{verbatim}
% \end{quote}
%
% \begin{History}
%   \begin{Version}{2007/01/08 v1.0}
%   \item
%     First version.
%   \end{Version}
%   \begin{Version}{2016/05/16 v1.1}
%   \item
%     Documentation updates.
%   \end{Version}
%   \begin{Version}{2018/09/10 v1.2}
%   \item
%     Update for LuaTeX, remove dependency on Pdf\TeX\ command names.
%   \end{Version}
%   \begin{Version}{2018/11/18 v1.3}
%   \item
%     Added code for pgf compatibility, see https://github.com/ho-tex/transparent/issues/1
%   \end{Version}
%   \begin{Version}{2019/11/29 v1.4}
%   \item
%     Documentation updates.
% \item
% Use \xpackage{iftex} package.
%   \end{Version}
%   \begin{Version}{2022-10-27 v1.5}
%   \item Made the package compatible with the PDF management.
%   \end{Version}
% \end{History}
%
% \PrintIndex
%
% \Finale
\endinput

%        (quote the arguments according to the demands of your shell)
%
% Documentation:
%    (a) If transparent.drv is present:
%           latex transparent.drv
%    (b) Without transparent.drv:
%           latex transparent.dtx; ...
%    The class ltxdoc loads the configuration file ltxdoc.cfg
%    if available. Here you can specify further options, e.g.
%    use A4 as paper format:
%       \PassOptionsToClass{a4paper}{article}
%
%    Programm calls to get the documentation (example):
%       pdflatex transparent.dtx
%       makeindex -s gind.ist transparent.idx
%       pdflatex transparent.dtx
%       makeindex -s gind.ist transparent.idx
%       pdflatex transparent.dtx
%
% Installation:
%    TDS:tex/latex/transparent/transparent.sty
%    TDS:tex/latex/transparent/transparent-nometadata.sty
%    TDS:doc/latex/transparent/transparent.pdf
%    TDS:source/latex/transparent/transparent.dtx
%
%<*ignore>
\begingroup
  \catcode123=1 %
  \catcode125=2 %
  \def\x{LaTeX2e}%
\expandafter\endgroup
\ifcase 0\ifx\install y1\fi\expandafter
         \ifx\csname processbatchFile\endcsname\relax\else1\fi
         \ifx\fmtname\x\else 1\fi\relax
\else\csname fi\endcsname
%</ignore>
%<*install>
\input docstrip.tex
\Msg{************************************************************************}
\Msg{* Installation}
\Msg{* Package: transparent 2022-10-27 v1.5 Transparency via pdfTeX's color stack (HO)}
\Msg{************************************************************************}

\keepsilent
\askforoverwritefalse

\let\MetaPrefix\relax
\preamble

This is a generated file.

Project: transparent
Version: 2022-10-27 v1.5

Copyright (C)
   2007 Heiko Oberdiek
   2016-2022 Oberdiek Package Support Group

This work may be distributed and/or modified under the
conditions of the LaTeX Project Public License, either
version 1.3c of this license or (at your option) any later
version. This version of this license is in
   https://www.latex-project.org/lppl/lppl-1-3c.txt
and the latest version of this license is in
   https://www.latex-project.org/lppl.txt
and version 1.3 or later is part of all distributions of
LaTeX version 2005/12/01 or later.

This work has the LPPL maintenance status "maintained".

The Current Maintainers of this work are
Heiko Oberdiek and the Oberdiek Package Support Group
https://github.com/ho-tex/transparent/issues


This work consists of the main source file transparent.dtx
and the derived files
   transparent.sty, transparent-nometadata.sty, transparent.pdf,
   transparent.ins, transparent.drv, transparent-example.tex.

\endpreamble
\let\MetaPrefix\DoubleperCent

\generate{%
  \file{transparent.ins}{\from{transparent.dtx}{install}}%
  \file{transparent.drv}{\from{transparent.dtx}{driver}}%
  \usedir{tex/latex/transparent}%
  \file{transparent-nometadata.sty}{\from{transparent.dtx}{package}}%
  \file{transparent.sty}{\from{transparent.dtx}{package-new}}%
  \usedir{doc/latex/transparent}%
  \file{transparent-example.tex}{\from{transparent.dtx}{example}}%
}

\catcode32=13\relax% active space
\let =\space%
\Msg{************************************************************************}
\Msg{*}
\Msg{* To finish the installation you have to move the following}
\Msg{* file into a directory searched by TeX:}
\Msg{*}
\Msg{*     transparent.sty}
\Msg{*     transparent-nometadata.sty}
\Msg{*}
\Msg{* To produce the documentation run the file `transparent.drv'}
\Msg{* through LaTeX.}
\Msg{*}
\Msg{* Happy TeXing!}
\Msg{*}
\Msg{************************************************************************}

\endbatchfile
%</install>
%<*ignore>
\fi
%</ignore>
%<*driver>
\NeedsTeXFormat{LaTeX2e}
\ProvidesFile{transparent.drv}%
  [2022-10-27 v1.5 Transparency via pdfTeX's color stack (HO)]%
\documentclass{ltxdoc}
\usepackage{holtxdoc}[2011/11/22]
\begin{document}
  \DocInput{transparent.dtx}%
\end{document}
%</driver>
% \fi
%
%
%
% \GetFileInfo{transparent.drv}
%
% \title{The \xpackage{transparent} package}
% \date{2022-10-27 v1.5}
% \author{Heiko Oberdiek\thanks
% {Please report any issues at \url{https://github.com/ho-tex/transparent/issues}}}
%
% \maketitle
%
% \begin{abstract}
% \pdfTeX\ and lua\TeX{} supports several color stacks. This
% package shows, how a separate color stack can be used for transparency,
% a property besides color, that also works across page break.
% \end{abstract}
%
% \tableofcontents
%
% \section{User interface}
%
% The package \xpackage{transparent} defines \cs{transparent} and
% \cs{texttransparent}. They are used like \cs{color} and \cs{textcolor}.
% The first argument is the transparency value between 0 and 1 where 0
% is fully transparent and 1 is opaque.
% 
% \section{Engine support}
% If the PDF management is loaded by using \verb+\DocumentMetadata{}+ at the beginning
% of the document the package makes use of \texttt{l3opacity} and
% can be used with all engines supported by \texttt{l3backend}.
% But only with \pdfTeX\ and lua\TeX{} the transparency will also work across page break.
% 
% 
% Without the PDF management, the package works only with \pdfTeX\ and lua\TeX{}.
% It then uses the original code by Heiko Oberdiek. It then uses the
% primitive \cs{pdfpageresources} and can clash with other packages that also use
% \cs{pdfpageresources}.
% 
%
% \section{Example for usage}
%    \begin{macrocode}
%<*example>
\documentclass[12pt]{article}

\usepackage{color}
\usepackage{transparent}

\begin{document}
\colorbox{yellow}{%
  \bfseries
  \color{blue}%
  Blue and %
  \transparent{0.6}%
  transparent blue%
}

\bigskip
Hello World
\texttransparent{0.5}{Hello\newpage World}
Hello World
\end{document}
%</example>
%    \end{macrocode}
%
% \StopEventually{
% }
%
% \section{Implementation}
% \subsection{New version using the methods of the PDF management} 
% Identification
%    \begin{macrocode}
%<*package-new>
\NeedsTeXFormat{LaTeX2e}[2020/10/01]
\ProvidesExplPackage{transparent}{2022-10-27}{1.5}
  {Transparency with color stacks}%
%    \end{macrocode}
% Testing if the pdfmanagement is used
%    \begin{macrocode}
\@ifl@t@r\fmtversion{2022-06-01}
  {
    \IfDocumentMetadataTF {}{\RequirePackage{transparent-nometadata}}
    \IfDocumentMetadataTF {}{\endinput}
  }
  {\RequirePackage{transparent-nometadata}}  
%    \end{macrocode}
%    \begin{macrocode}
\RequirePackage{l3opacity}

\NewDocumentCommand{\transparent} { m }
      {
       \opacity_select:n{\fp_eval:n{ min(max(0,#1),1) } }
      }

\NewDocumentCommand{\texttransparent}{m m}
  {
   \mode_leave_vertical:
   \group_begin:
     \transparent{#1}
     #2
   \group_end:
  }
%</package-new>
%    \end{macrocode}
% \subsection{Old version without the PDF management}
%    \begin{macrocode}
%<*package>
\NeedsTeXFormat{LaTeX2e}
\ProvidesPackage{transparent-nometadata}%
  [2022-10-27 v1.5 Transparency via pdfTeX's color stack (HO)]%
%    \end{macrocode}
%
% \subsection{Initial checks}
%
% \subsubsection{Check for \pdfTeX\ in PDF mode}
%    \begin{macrocode}
\RequirePackage{iftex}
\ifpdf
\else
  \PackageWarningNoLine{transparent}{%
    Loading aborted, because pdfTeX is not running in PDF mode%
  }%
  \expandafter\endinput
\fi
%    \end{macrocode}
%
% \subsubsection{Check \pdfTeX\ or Lua\TeX\ version}
%    \begin{macrocode}
\ifx\pdfextension\@undefined
  \let\TRP@pdfcolorstackinit\pdfcolorstackinit
  \let\TRP@pdfpageresources\pdfpageresources
  \let\TRP@pdfcolorstack\pdfcolorstack
\else
  \def\TRP@pdfcolorstackinit           {\pdffeedback colorstackinit}
  \protected\edef\TRP@pdfpageresources {\pdfvariable pageresources}
  \protected\def\TRP@pdfcolorstack     {\pdfextension colorstack}
\fi
%    \end{macrocode}
%
%    \begin{macrocode}
\ifcsname TRP@pdfcolorstackinit\endcsname\else
  \PackageWarningNoLine{transparent}{%
    Your pdfTeX version does not support color stacks%
  }%
  \expandafter\endinput
\fi
%    \end{macrocode}
%
% \subsection{Compatibility with pgf}
%  https://github.com/ho-tex/transparent/issues/1
%    \begin{macrocode}
\AtBeginDocument
 {%
   \ifcsname pgfutil@addpdfresource@extgs\endcsname
    \let\TRP@addresource\relax
    \pgfutil@addpdfresource@extgs{\TRP@list}%
   \fi
 }
%    \end{macrocode}
% \subsection{Transparency}
%
%    The setting for the different transparency values must
%    be added to the page resources. In the first run the values
%    are recorded in the \xfile{.aux} file. In the second run
%    the values are set and transparency is available.
%    \begin{macrocode}
\RequirePackage{auxhook}
\AddLineBeginAux{%
  \string\providecommand{\string\transparent@use}[1]{}%
}
\gdef\TRP@list{/TRP1<</ca 1/CA 1>>}
\def\transparent@use#1{%
  \@ifundefined{TRP#1}{%
    \g@addto@macro\TRP@list{%
      /TRP#1<</ca #1/CA #1>>%
    }%
    \expandafter\gdef\csname TRP#1\endcsname{/TRP#1 gs}%
  }{%
    % #1 is already known, nothing to do
  }%
}
\AtBeginDocument{%
  \TRP@addresource
  \let\transparent@use\@gobble
}
%    \end{macrocode}
%    Unhappily the interface setting page resources is very
%    poor, only a token register \cs{pdfpageresources}.
%    Thus this package tries to be cooperative in the way that
%    it embeds the previous contents of \cs{pdfpageresources}.
%    However it does not solve the problem, if several packages
%    want to set |/ExtGState|.
%    \begin{macrocode}
\def\TRP@addresource{%
  \begingroup
    \edef\x{\endgroup
      \TRP@pdfpageresources{%
        \the\TRP@pdfpageresources
        /ExtGState<<\TRP@list>>%
      }%
    }%
  \x
}
\newif\ifTRP@rerun
\xdef\TRP@colorstack{%
  \TRP@pdfcolorstackinit page direct{/TRP1 gs}%
}
%    \end{macrocode}
%    \begin{macro}{\transparent}
%    \begin{macrocode}
\newcommand*{\transparent}[1]{%
  \begingroup
    \dimen@=#1\p@\relax
    \ifdim\dimen@>\p@
      \dimen@=\p@
    \fi
    \ifdim\dimen@<\z@
      \dimen@=\z@
    \fi
    \ifdim\dimen@=\p@
      \def\x{1}%
    \else
      \ifdim\dimen@=\z@
        \def\x{0}%
      \else
        \edef\x{\strip@pt\dimen@}%
        \edef\x{\expandafter\@gobble\x}%
      \fi
    \fi
    \if@filesw
      \immediate\write\@auxout{%
        \string\transparent@use{\x}%
      }%
    \fi
    \edef\x{\endgroup
      \def\noexpand\transparent@current{\x}%
    }%
  \x
  \transparent@set
}
%    \end{macrocode}
%    \end{macro}
%    \begin{macrocode}
\AtEndDocument{%
  \ifTRP@rerun
    \PackageWarningNoLine{transparent}{%
      Rerun to get transparencies right%
    }%
  \fi
}
\def\transparent@current{/TRP1 gs}
\def\transparent@set{%
  \@ifundefined{TRP\transparent@current}{%
    \global\TRP@reruntrue
  }{%
    \TRP@pdfcolorstack\TRP@colorstack push{%
      \csname TRP\transparent@current\endcsname
    }%
    \aftergroup\transparent@reset
  }%
}
\def\transparent@reset{%
  \TRP@pdfcolorstack\TRP@colorstack pop\relax
}
%    \end{macrocode}
%    \begin{macro}{\texttransparent}
%    \begin{macrocode}
\newcommand*{\texttransparent}[2]{%
  \protect\leavevmode
  \begingroup
    \transparent{#1}%
    #2%
  \endgroup
}
%    \end{macrocode}
%    \end{macro}
%
%    \begin{macrocode}
%</package>
%    \end{macrocode}
%
% \section{Installation}
%
% \subsection{Download}
%
% \paragraph{Package.} This package is available on
% CTAN\footnote{\CTANpkg{transparent}}:
% \begin{description}
% \item[\CTAN{macros/latex/contrib/transparent/transparent.dtx}] The source file.
% \item[\CTAN{macros/latex/contrib/transparent/transparent.pdf}] Documentation.
% \end{description}
%
%
%
% \subsection{Package installation}
%
% \paragraph{Unpacking.} The \xfile{.dtx} file is a self-extracting
% \docstrip\ archive. The files are extracted by running the
% \xfile{.dtx} through \plainTeX:
% \begin{quote}
%   \verb|tex transparent.dtx|
% \end{quote}
%
% \paragraph{TDS.} Now the different files must be moved into
% the different directories in your installation TDS tree
% (also known as \xfile{texmf} tree):
% \begin{quote}
% \def\t{^^A
% \begin{tabular}{@{}>{\ttfamily}l@{ $\rightarrow$ }>{\ttfamily}l@{}}
%   transparent.sty & tex/latex/transparent/transparent.sty\\
%   transparent-nometadata.sty & tex/latex/transparent/transparent-nometadata.sty\\   
%   transparent.pdf & doc/latex/transparent/transparent.pdf\\
%   transparent-example.tex & doc/latex/transparent/transparent-example.tex\\
%   transparent.dtx & source/latex/transparent/transparent.dtx\\
% \end{tabular}^^A
% }^^A
% \sbox0{\t}^^A
% \ifdim\wd0>\linewidth
%   \begingroup
%     \advance\linewidth by\leftmargin
%     \advance\linewidth by\rightmargin
%   \edef\x{\endgroup
%     \def\noexpand\lw{\the\linewidth}^^A
%   }\x
%   \def\lwbox{^^A
%     \leavevmode
%     \hbox to \linewidth{^^A
%       \kern-\leftmargin\relax
%       \hss
%       \usebox0
%       \hss
%       \kern-\rightmargin\relax
%     }^^A
%   }^^A
%   \ifdim\wd0>\lw
%     \sbox0{\small\t}^^A
%     \ifdim\wd0>\linewidth
%       \ifdim\wd0>\lw
%         \sbox0{\footnotesize\t}^^A
%         \ifdim\wd0>\linewidth
%           \ifdim\wd0>\lw
%             \sbox0{\scriptsize\t}^^A
%             \ifdim\wd0>\linewidth
%               \ifdim\wd0>\lw
%                 \sbox0{\tiny\t}^^A
%                 \ifdim\wd0>\linewidth
%                   \lwbox
%                 \else
%                   \usebox0
%                 \fi
%               \else
%                 \lwbox
%               \fi
%             \else
%               \usebox0
%             \fi
%           \else
%             \lwbox
%           \fi
%         \else
%           \usebox0
%         \fi
%       \else
%         \lwbox
%       \fi
%     \else
%       \usebox0
%     \fi
%   \else
%     \lwbox
%   \fi
% \else
%   \usebox0
% \fi
% \end{quote}
% If you have a \xfile{docstrip.cfg} that configures and enables \docstrip's
% TDS installing feature, then some files can already be in the right
% place, see the documentation of \docstrip.
%
% \subsection{Refresh file name databases}
%
% If your \TeX~distribution
% (\TeX\,Live, \mikTeX, \dots) relies on file name databases, you must refresh
% these. For example, \TeX\,Live\ users run \verb|texhash| or
% \verb|mktexlsr|.
%
% \subsection{Some details for the interested}
%
% \paragraph{Unpacking with \LaTeX.}
% The \xfile{.dtx} chooses its action depending on the format:
% \begin{description}
% \item[\plainTeX:] Run \docstrip\ and extract the files.
% \item[\LaTeX:] Generate the documentation.
% \end{description}
% If you insist on using \LaTeX\ for \docstrip\ (really,
% \docstrip\ does not need \LaTeX), then inform the autodetect routine
% about your intention:
% \begin{quote}
%   \verb|latex \let\install=y% \iffalse meta-comment
%
% File: transparent.dtx
% Version: 2022-10-27 v1.5
% Info: Transparency via pdfTeX's color stack
%
% Copyright (C)
%    2007 Heiko Oberdiek
%    2016-2022 Oberdiek Package Support Group 2018
%    https://github.com/ho-tex/transparent/issues
%
% This work may be distributed and/or modified under the
% conditions of the LaTeX Project Public License, either
% version 1.3c of this license or (at your option) any later
% version. This version of this license is in
%    https://www.latex-project.org/lppl/lppl-1-3c.txt
% and the latest version of this license is in
%    https://www.latex-project.org/lppl.txt
% and version 1.3 or later is part of all distributions of
% LaTeX version 2005/12/01 or later.
%
% This work has the LPPL maintenance status "maintained".
%
% The Current Maintainers of this work are
% Heiko Oberdiek and the Oberdiek Package Support Group
% https://github.com/ho-tex/transparent/issues
%
% This work consists of the main source file transparent.dtx
% and the derived files
%    transparent.sty, transparent-nometadata.sty, transparent.pdf, 
%    transparent.ins, transparent.drv, transparent-example.tex.
%
% Distribution:
%    CTAN:macros/latex/contrib/transparent/transparent.dtx
%    CTAN:macros/latex/contrib/transparent/transparent.pdf
%
% Unpacking:
%    (a) If transparent.ins is present:
%           tex transparent.ins
%    (b) Without transparent.ins:
%           tex transparent.dtx
%    (c) If you insist on using LaTeX
%           latex \let\install=y\input{transparent.dtx}
%        (quote the arguments according to the demands of your shell)
%
% Documentation:
%    (a) If transparent.drv is present:
%           latex transparent.drv
%    (b) Without transparent.drv:
%           latex transparent.dtx; ...
%    The class ltxdoc loads the configuration file ltxdoc.cfg
%    if available. Here you can specify further options, e.g.
%    use A4 as paper format:
%       \PassOptionsToClass{a4paper}{article}
%
%    Programm calls to get the documentation (example):
%       pdflatex transparent.dtx
%       makeindex -s gind.ist transparent.idx
%       pdflatex transparent.dtx
%       makeindex -s gind.ist transparent.idx
%       pdflatex transparent.dtx
%
% Installation:
%    TDS:tex/latex/transparent/transparent.sty
%    TDS:tex/latex/transparent/transparent-nometadata.sty
%    TDS:doc/latex/transparent/transparent.pdf
%    TDS:source/latex/transparent/transparent.dtx
%
%<*ignore>
\begingroup
  \catcode123=1 %
  \catcode125=2 %
  \def\x{LaTeX2e}%
\expandafter\endgroup
\ifcase 0\ifx\install y1\fi\expandafter
         \ifx\csname processbatchFile\endcsname\relax\else1\fi
         \ifx\fmtname\x\else 1\fi\relax
\else\csname fi\endcsname
%</ignore>
%<*install>
\input docstrip.tex
\Msg{************************************************************************}
\Msg{* Installation}
\Msg{* Package: transparent 2022-10-27 v1.5 Transparency via pdfTeX's color stack (HO)}
\Msg{************************************************************************}

\keepsilent
\askforoverwritefalse

\let\MetaPrefix\relax
\preamble

This is a generated file.

Project: transparent
Version: 2022-10-27 v1.5

Copyright (C)
   2007 Heiko Oberdiek
   2016-2022 Oberdiek Package Support Group

This work may be distributed and/or modified under the
conditions of the LaTeX Project Public License, either
version 1.3c of this license or (at your option) any later
version. This version of this license is in
   https://www.latex-project.org/lppl/lppl-1-3c.txt
and the latest version of this license is in
   https://www.latex-project.org/lppl.txt
and version 1.3 or later is part of all distributions of
LaTeX version 2005/12/01 or later.

This work has the LPPL maintenance status "maintained".

The Current Maintainers of this work are
Heiko Oberdiek and the Oberdiek Package Support Group
https://github.com/ho-tex/transparent/issues


This work consists of the main source file transparent.dtx
and the derived files
   transparent.sty, transparent-nometadata.sty, transparent.pdf,
   transparent.ins, transparent.drv, transparent-example.tex.

\endpreamble
\let\MetaPrefix\DoubleperCent

\generate{%
  \file{transparent.ins}{\from{transparent.dtx}{install}}%
  \file{transparent.drv}{\from{transparent.dtx}{driver}}%
  \usedir{tex/latex/transparent}%
  \file{transparent-nometadata.sty}{\from{transparent.dtx}{package}}%
  \file{transparent.sty}{\from{transparent.dtx}{package-new}}%
  \usedir{doc/latex/transparent}%
  \file{transparent-example.tex}{\from{transparent.dtx}{example}}%
}

\catcode32=13\relax% active space
\let =\space%
\Msg{************************************************************************}
\Msg{*}
\Msg{* To finish the installation you have to move the following}
\Msg{* file into a directory searched by TeX:}
\Msg{*}
\Msg{*     transparent.sty}
\Msg{*     transparent-nometadata.sty}
\Msg{*}
\Msg{* To produce the documentation run the file `transparent.drv'}
\Msg{* through LaTeX.}
\Msg{*}
\Msg{* Happy TeXing!}
\Msg{*}
\Msg{************************************************************************}

\endbatchfile
%</install>
%<*ignore>
\fi
%</ignore>
%<*driver>
\NeedsTeXFormat{LaTeX2e}
\ProvidesFile{transparent.drv}%
  [2022-10-27 v1.5 Transparency via pdfTeX's color stack (HO)]%
\documentclass{ltxdoc}
\usepackage{holtxdoc}[2011/11/22]
\begin{document}
  \DocInput{transparent.dtx}%
\end{document}
%</driver>
% \fi
%
%
%
% \GetFileInfo{transparent.drv}
%
% \title{The \xpackage{transparent} package}
% \date{2022-10-27 v1.5}
% \author{Heiko Oberdiek\thanks
% {Please report any issues at \url{https://github.com/ho-tex/transparent/issues}}}
%
% \maketitle
%
% \begin{abstract}
% \pdfTeX\ and lua\TeX{} supports several color stacks. This
% package shows, how a separate color stack can be used for transparency,
% a property besides color, that also works across page break.
% \end{abstract}
%
% \tableofcontents
%
% \section{User interface}
%
% The package \xpackage{transparent} defines \cs{transparent} and
% \cs{texttransparent}. They are used like \cs{color} and \cs{textcolor}.
% The first argument is the transparency value between 0 and 1 where 0
% is fully transparent and 1 is opaque.
% 
% \section{Engine support}
% If the PDF management is loaded by using \verb+\DocumentMetadata{}+ at the beginning
% of the document the package makes use of \texttt{l3opacity} and
% can be used with all engines supported by \texttt{l3backend}.
% But only with \pdfTeX\ and lua\TeX{} the transparency will also work across page break.
% 
% 
% Without the PDF management, the package works only with \pdfTeX\ and lua\TeX{}.
% It then uses the original code by Heiko Oberdiek. It then uses the
% primitive \cs{pdfpageresources} and can clash with other packages that also use
% \cs{pdfpageresources}.
% 
%
% \section{Example for usage}
%    \begin{macrocode}
%<*example>
\documentclass[12pt]{article}

\usepackage{color}
\usepackage{transparent}

\begin{document}
\colorbox{yellow}{%
  \bfseries
  \color{blue}%
  Blue and %
  \transparent{0.6}%
  transparent blue%
}

\bigskip
Hello World
\texttransparent{0.5}{Hello\newpage World}
Hello World
\end{document}
%</example>
%    \end{macrocode}
%
% \StopEventually{
% }
%
% \section{Implementation}
% \subsection{New version using the methods of the PDF management} 
% Identification
%    \begin{macrocode}
%<*package-new>
\NeedsTeXFormat{LaTeX2e}[2020/10/01]
\ProvidesExplPackage{transparent}{2022-10-27}{1.5}
  {Transparency with color stacks}%
%    \end{macrocode}
% Testing if the pdfmanagement is used
%    \begin{macrocode}
\@ifl@t@r\fmtversion{2022-06-01}
  {
    \IfDocumentMetadataTF {}{\RequirePackage{transparent-nometadata}}
    \IfDocumentMetadataTF {}{\endinput}
  }
  {\RequirePackage{transparent-nometadata}}  
%    \end{macrocode}
%    \begin{macrocode}
\RequirePackage{l3opacity}

\NewDocumentCommand{\transparent} { m }
      {
       \opacity_select:n{\fp_eval:n{ min(max(0,#1),1) } }
      }

\NewDocumentCommand{\texttransparent}{m m}
  {
   \mode_leave_vertical:
   \group_begin:
     \transparent{#1}
     #2
   \group_end:
  }
%</package-new>
%    \end{macrocode}
% \subsection{Old version without the PDF management}
%    \begin{macrocode}
%<*package>
\NeedsTeXFormat{LaTeX2e}
\ProvidesPackage{transparent-nometadata}%
  [2022-10-27 v1.5 Transparency via pdfTeX's color stack (HO)]%
%    \end{macrocode}
%
% \subsection{Initial checks}
%
% \subsubsection{Check for \pdfTeX\ in PDF mode}
%    \begin{macrocode}
\RequirePackage{iftex}
\ifpdf
\else
  \PackageWarningNoLine{transparent}{%
    Loading aborted, because pdfTeX is not running in PDF mode%
  }%
  \expandafter\endinput
\fi
%    \end{macrocode}
%
% \subsubsection{Check \pdfTeX\ or Lua\TeX\ version}
%    \begin{macrocode}
\ifx\pdfextension\@undefined
  \let\TRP@pdfcolorstackinit\pdfcolorstackinit
  \let\TRP@pdfpageresources\pdfpageresources
  \let\TRP@pdfcolorstack\pdfcolorstack
\else
  \def\TRP@pdfcolorstackinit           {\pdffeedback colorstackinit}
  \protected\edef\TRP@pdfpageresources {\pdfvariable pageresources}
  \protected\def\TRP@pdfcolorstack     {\pdfextension colorstack}
\fi
%    \end{macrocode}
%
%    \begin{macrocode}
\ifcsname TRP@pdfcolorstackinit\endcsname\else
  \PackageWarningNoLine{transparent}{%
    Your pdfTeX version does not support color stacks%
  }%
  \expandafter\endinput
\fi
%    \end{macrocode}
%
% \subsection{Compatibility with pgf}
%  https://github.com/ho-tex/transparent/issues/1
%    \begin{macrocode}
\AtBeginDocument
 {%
   \ifcsname pgfutil@addpdfresource@extgs\endcsname
    \let\TRP@addresource\relax
    \pgfutil@addpdfresource@extgs{\TRP@list}%
   \fi
 }
%    \end{macrocode}
% \subsection{Transparency}
%
%    The setting for the different transparency values must
%    be added to the page resources. In the first run the values
%    are recorded in the \xfile{.aux} file. In the second run
%    the values are set and transparency is available.
%    \begin{macrocode}
\RequirePackage{auxhook}
\AddLineBeginAux{%
  \string\providecommand{\string\transparent@use}[1]{}%
}
\gdef\TRP@list{/TRP1<</ca 1/CA 1>>}
\def\transparent@use#1{%
  \@ifundefined{TRP#1}{%
    \g@addto@macro\TRP@list{%
      /TRP#1<</ca #1/CA #1>>%
    }%
    \expandafter\gdef\csname TRP#1\endcsname{/TRP#1 gs}%
  }{%
    % #1 is already known, nothing to do
  }%
}
\AtBeginDocument{%
  \TRP@addresource
  \let\transparent@use\@gobble
}
%    \end{macrocode}
%    Unhappily the interface setting page resources is very
%    poor, only a token register \cs{pdfpageresources}.
%    Thus this package tries to be cooperative in the way that
%    it embeds the previous contents of \cs{pdfpageresources}.
%    However it does not solve the problem, if several packages
%    want to set |/ExtGState|.
%    \begin{macrocode}
\def\TRP@addresource{%
  \begingroup
    \edef\x{\endgroup
      \TRP@pdfpageresources{%
        \the\TRP@pdfpageresources
        /ExtGState<<\TRP@list>>%
      }%
    }%
  \x
}
\newif\ifTRP@rerun
\xdef\TRP@colorstack{%
  \TRP@pdfcolorstackinit page direct{/TRP1 gs}%
}
%    \end{macrocode}
%    \begin{macro}{\transparent}
%    \begin{macrocode}
\newcommand*{\transparent}[1]{%
  \begingroup
    \dimen@=#1\p@\relax
    \ifdim\dimen@>\p@
      \dimen@=\p@
    \fi
    \ifdim\dimen@<\z@
      \dimen@=\z@
    \fi
    \ifdim\dimen@=\p@
      \def\x{1}%
    \else
      \ifdim\dimen@=\z@
        \def\x{0}%
      \else
        \edef\x{\strip@pt\dimen@}%
        \edef\x{\expandafter\@gobble\x}%
      \fi
    \fi
    \if@filesw
      \immediate\write\@auxout{%
        \string\transparent@use{\x}%
      }%
    \fi
    \edef\x{\endgroup
      \def\noexpand\transparent@current{\x}%
    }%
  \x
  \transparent@set
}
%    \end{macrocode}
%    \end{macro}
%    \begin{macrocode}
\AtEndDocument{%
  \ifTRP@rerun
    \PackageWarningNoLine{transparent}{%
      Rerun to get transparencies right%
    }%
  \fi
}
\def\transparent@current{/TRP1 gs}
\def\transparent@set{%
  \@ifundefined{TRP\transparent@current}{%
    \global\TRP@reruntrue
  }{%
    \TRP@pdfcolorstack\TRP@colorstack push{%
      \csname TRP\transparent@current\endcsname
    }%
    \aftergroup\transparent@reset
  }%
}
\def\transparent@reset{%
  \TRP@pdfcolorstack\TRP@colorstack pop\relax
}
%    \end{macrocode}
%    \begin{macro}{\texttransparent}
%    \begin{macrocode}
\newcommand*{\texttransparent}[2]{%
  \protect\leavevmode
  \begingroup
    \transparent{#1}%
    #2%
  \endgroup
}
%    \end{macrocode}
%    \end{macro}
%
%    \begin{macrocode}
%</package>
%    \end{macrocode}
%
% \section{Installation}
%
% \subsection{Download}
%
% \paragraph{Package.} This package is available on
% CTAN\footnote{\CTANpkg{transparent}}:
% \begin{description}
% \item[\CTAN{macros/latex/contrib/transparent/transparent.dtx}] The source file.
% \item[\CTAN{macros/latex/contrib/transparent/transparent.pdf}] Documentation.
% \end{description}
%
%
%
% \subsection{Package installation}
%
% \paragraph{Unpacking.} The \xfile{.dtx} file is a self-extracting
% \docstrip\ archive. The files are extracted by running the
% \xfile{.dtx} through \plainTeX:
% \begin{quote}
%   \verb|tex transparent.dtx|
% \end{quote}
%
% \paragraph{TDS.} Now the different files must be moved into
% the different directories in your installation TDS tree
% (also known as \xfile{texmf} tree):
% \begin{quote}
% \def\t{^^A
% \begin{tabular}{@{}>{\ttfamily}l@{ $\rightarrow$ }>{\ttfamily}l@{}}
%   transparent.sty & tex/latex/transparent/transparent.sty\\
%   transparent-nometadata.sty & tex/latex/transparent/transparent-nometadata.sty\\   
%   transparent.pdf & doc/latex/transparent/transparent.pdf\\
%   transparent-example.tex & doc/latex/transparent/transparent-example.tex\\
%   transparent.dtx & source/latex/transparent/transparent.dtx\\
% \end{tabular}^^A
% }^^A
% \sbox0{\t}^^A
% \ifdim\wd0>\linewidth
%   \begingroup
%     \advance\linewidth by\leftmargin
%     \advance\linewidth by\rightmargin
%   \edef\x{\endgroup
%     \def\noexpand\lw{\the\linewidth}^^A
%   }\x
%   \def\lwbox{^^A
%     \leavevmode
%     \hbox to \linewidth{^^A
%       \kern-\leftmargin\relax
%       \hss
%       \usebox0
%       \hss
%       \kern-\rightmargin\relax
%     }^^A
%   }^^A
%   \ifdim\wd0>\lw
%     \sbox0{\small\t}^^A
%     \ifdim\wd0>\linewidth
%       \ifdim\wd0>\lw
%         \sbox0{\footnotesize\t}^^A
%         \ifdim\wd0>\linewidth
%           \ifdim\wd0>\lw
%             \sbox0{\scriptsize\t}^^A
%             \ifdim\wd0>\linewidth
%               \ifdim\wd0>\lw
%                 \sbox0{\tiny\t}^^A
%                 \ifdim\wd0>\linewidth
%                   \lwbox
%                 \else
%                   \usebox0
%                 \fi
%               \else
%                 \lwbox
%               \fi
%             \else
%               \usebox0
%             \fi
%           \else
%             \lwbox
%           \fi
%         \else
%           \usebox0
%         \fi
%       \else
%         \lwbox
%       \fi
%     \else
%       \usebox0
%     \fi
%   \else
%     \lwbox
%   \fi
% \else
%   \usebox0
% \fi
% \end{quote}
% If you have a \xfile{docstrip.cfg} that configures and enables \docstrip's
% TDS installing feature, then some files can already be in the right
% place, see the documentation of \docstrip.
%
% \subsection{Refresh file name databases}
%
% If your \TeX~distribution
% (\TeX\,Live, \mikTeX, \dots) relies on file name databases, you must refresh
% these. For example, \TeX\,Live\ users run \verb|texhash| or
% \verb|mktexlsr|.
%
% \subsection{Some details for the interested}
%
% \paragraph{Unpacking with \LaTeX.}
% The \xfile{.dtx} chooses its action depending on the format:
% \begin{description}
% \item[\plainTeX:] Run \docstrip\ and extract the files.
% \item[\LaTeX:] Generate the documentation.
% \end{description}
% If you insist on using \LaTeX\ for \docstrip\ (really,
% \docstrip\ does not need \LaTeX), then inform the autodetect routine
% about your intention:
% \begin{quote}
%   \verb|latex \let\install=y\input{transparent.dtx}|
% \end{quote}
% Do not forget to quote the argument according to the demands
% of your shell.
%
% \paragraph{Generating the documentation.}
% You can use both the \xfile{.dtx} or the \xfile{.drv} to generate
% the documentation. The process can be configured by the
% configuration file \xfile{ltxdoc.cfg}. For instance, put this
% line into this file, if you want to have A4 as paper format:
% \begin{quote}
%   \verb|\PassOptionsToClass{a4paper}{article}|
% \end{quote}
% An example follows how to generate the
% documentation with pdf\LaTeX:
% \begin{quote}
%\begin{verbatim}
%pdflatex transparent.dtx
%makeindex -s gind.ist transparent.idx
%pdflatex transparent.dtx
%makeindex -s gind.ist transparent.idx
%pdflatex transparent.dtx
%\end{verbatim}
% \end{quote}
%
% \begin{History}
%   \begin{Version}{2007/01/08 v1.0}
%   \item
%     First version.
%   \end{Version}
%   \begin{Version}{2016/05/16 v1.1}
%   \item
%     Documentation updates.
%   \end{Version}
%   \begin{Version}{2018/09/10 v1.2}
%   \item
%     Update for LuaTeX, remove dependency on Pdf\TeX\ command names.
%   \end{Version}
%   \begin{Version}{2018/11/18 v1.3}
%   \item
%     Added code for pgf compatibility, see https://github.com/ho-tex/transparent/issues/1
%   \end{Version}
%   \begin{Version}{2019/11/29 v1.4}
%   \item
%     Documentation updates.
% \item
% Use \xpackage{iftex} package.
%   \end{Version}
%   \begin{Version}{2022-10-27 v1.5}
%   \item Made the package compatible with the PDF management.
%   \end{Version}
% \end{History}
%
% \PrintIndex
%
% \Finale
\endinput
|
% \end{quote}
% Do not forget to quote the argument according to the demands
% of your shell.
%
% \paragraph{Generating the documentation.}
% You can use both the \xfile{.dtx} or the \xfile{.drv} to generate
% the documentation. The process can be configured by the
% configuration file \xfile{ltxdoc.cfg}. For instance, put this
% line into this file, if you want to have A4 as paper format:
% \begin{quote}
%   \verb|\PassOptionsToClass{a4paper}{article}|
% \end{quote}
% An example follows how to generate the
% documentation with pdf\LaTeX:
% \begin{quote}
%\begin{verbatim}
%pdflatex transparent.dtx
%makeindex -s gind.ist transparent.idx
%pdflatex transparent.dtx
%makeindex -s gind.ist transparent.idx
%pdflatex transparent.dtx
%\end{verbatim}
% \end{quote}
%
% \begin{History}
%   \begin{Version}{2007/01/08 v1.0}
%   \item
%     First version.
%   \end{Version}
%   \begin{Version}{2016/05/16 v1.1}
%   \item
%     Documentation updates.
%   \end{Version}
%   \begin{Version}{2018/09/10 v1.2}
%   \item
%     Update for LuaTeX, remove dependency on Pdf\TeX\ command names.
%   \end{Version}
%   \begin{Version}{2018/11/18 v1.3}
%   \item
%     Added code for pgf compatibility, see https://github.com/ho-tex/transparent/issues/1
%   \end{Version}
%   \begin{Version}{2019/11/29 v1.4}
%   \item
%     Documentation updates.
% \item
% Use \xpackage{iftex} package.
%   \end{Version}
%   \begin{Version}{2022-10-27 v1.5}
%   \item Made the package compatible with the PDF management.
%   \end{Version}
% \end{History}
%
% \PrintIndex
%
% \Finale
\endinput

%        (quote the arguments according to the demands of your shell)
%
% Documentation:
%    (a) If transparent.drv is present:
%           latex transparent.drv
%    (b) Without transparent.drv:
%           latex transparent.dtx; ...
%    The class ltxdoc loads the configuration file ltxdoc.cfg
%    if available. Here you can specify further options, e.g.
%    use A4 as paper format:
%       \PassOptionsToClass{a4paper}{article}
%
%    Programm calls to get the documentation (example):
%       pdflatex transparent.dtx
%       makeindex -s gind.ist transparent.idx
%       pdflatex transparent.dtx
%       makeindex -s gind.ist transparent.idx
%       pdflatex transparent.dtx
%
% Installation:
%    TDS:tex/latex/transparent/transparent.sty
%    TDS:tex/latex/transparent/transparent-nometadata.sty
%    TDS:doc/latex/transparent/transparent.pdf
%    TDS:source/latex/transparent/transparent.dtx
%
%<*ignore>
\begingroup
  \catcode123=1 %
  \catcode125=2 %
  \def\x{LaTeX2e}%
\expandafter\endgroup
\ifcase 0\ifx\install y1\fi\expandafter
         \ifx\csname processbatchFile\endcsname\relax\else1\fi
         \ifx\fmtname\x\else 1\fi\relax
\else\csname fi\endcsname
%</ignore>
%<*install>
\input docstrip.tex
\Msg{************************************************************************}
\Msg{* Installation}
\Msg{* Package: transparent 2022-10-27 v1.5 Transparency via pdfTeX's color stack (HO)}
\Msg{************************************************************************}

\keepsilent
\askforoverwritefalse

\let\MetaPrefix\relax
\preamble

This is a generated file.

Project: transparent
Version: 2022-10-27 v1.5

Copyright (C)
   2007 Heiko Oberdiek
   2016-2022 Oberdiek Package Support Group

This work may be distributed and/or modified under the
conditions of the LaTeX Project Public License, either
version 1.3c of this license or (at your option) any later
version. This version of this license is in
   https://www.latex-project.org/lppl/lppl-1-3c.txt
and the latest version of this license is in
   https://www.latex-project.org/lppl.txt
and version 1.3 or later is part of all distributions of
LaTeX version 2005/12/01 or later.

This work has the LPPL maintenance status "maintained".

The Current Maintainers of this work are
Heiko Oberdiek and the Oberdiek Package Support Group
https://github.com/ho-tex/transparent/issues


This work consists of the main source file transparent.dtx
and the derived files
   transparent.sty, transparent-nometadata.sty, transparent.pdf,
   transparent.ins, transparent.drv, transparent-example.tex.

\endpreamble
\let\MetaPrefix\DoubleperCent

\generate{%
  \file{transparent.ins}{\from{transparent.dtx}{install}}%
  \file{transparent.drv}{\from{transparent.dtx}{driver}}%
  \usedir{tex/latex/transparent}%
  \file{transparent-nometadata.sty}{\from{transparent.dtx}{package}}%
  \file{transparent.sty}{\from{transparent.dtx}{package-new}}%
  \usedir{doc/latex/transparent}%
  \file{transparent-example.tex}{\from{transparent.dtx}{example}}%
}

\catcode32=13\relax% active space
\let =\space%
\Msg{************************************************************************}
\Msg{*}
\Msg{* To finish the installation you have to move the following}
\Msg{* file into a directory searched by TeX:}
\Msg{*}
\Msg{*     transparent.sty}
\Msg{*     transparent-nometadata.sty}
\Msg{*}
\Msg{* To produce the documentation run the file `transparent.drv'}
\Msg{* through LaTeX.}
\Msg{*}
\Msg{* Happy TeXing!}
\Msg{*}
\Msg{************************************************************************}

\endbatchfile
%</install>
%<*ignore>
\fi
%</ignore>
%<*driver>
\NeedsTeXFormat{LaTeX2e}
\ProvidesFile{transparent.drv}%
  [2022-10-27 v1.5 Transparency via pdfTeX's color stack (HO)]%
\documentclass{ltxdoc}
\usepackage{holtxdoc}[2011/11/22]
\begin{document}
  \DocInput{transparent.dtx}%
\end{document}
%</driver>
% \fi
%
%
%
% \GetFileInfo{transparent.drv}
%
% \title{The \xpackage{transparent} package}
% \date{2022-10-27 v1.5}
% \author{Heiko Oberdiek\thanks
% {Please report any issues at \url{https://github.com/ho-tex/transparent/issues}}}
%
% \maketitle
%
% \begin{abstract}
% \pdfTeX\ and lua\TeX{} supports several color stacks. This
% package shows, how a separate color stack can be used for transparency,
% a property besides color, that also works across page break.
% \end{abstract}
%
% \tableofcontents
%
% \section{User interface}
%
% The package \xpackage{transparent} defines \cs{transparent} and
% \cs{texttransparent}. They are used like \cs{color} and \cs{textcolor}.
% The first argument is the transparency value between 0 and 1 where 0
% is fully transparent and 1 is opaque.
% 
% \section{Engine support}
% If the PDF management is loaded by using \verb+\DocumentMetadata{}+ at the beginning
% of the document the package makes use of \texttt{l3opacity} and
% can be used with all engines supported by \texttt{l3backend}.
% But only with \pdfTeX\ and lua\TeX{} the transparency will also work across page break.
% 
% 
% Without the PDF management, the package works only with \pdfTeX\ and lua\TeX{}.
% It then uses the original code by Heiko Oberdiek. It then uses the
% primitive \cs{pdfpageresources} and can clash with other packages that also use
% \cs{pdfpageresources}.
% 
%
% \section{Example for usage}
%    \begin{macrocode}
%<*example>
\documentclass[12pt]{article}

\usepackage{color}
\usepackage{transparent}

\begin{document}
\colorbox{yellow}{%
  \bfseries
  \color{blue}%
  Blue and %
  \transparent{0.6}%
  transparent blue%
}

\bigskip
Hello World
\texttransparent{0.5}{Hello\newpage World}
Hello World
\end{document}
%</example>
%    \end{macrocode}
%
% \StopEventually{
% }
%
% \section{Implementation}
% \subsection{New version using the methods of the PDF management} 
% Identification
%    \begin{macrocode}
%<*package-new>
\NeedsTeXFormat{LaTeX2e}[2020/10/01]
\ProvidesExplPackage{transparent}{2022-10-27}{1.5}
  {Transparency with color stacks}%
%    \end{macrocode}
% Testing if the pdfmanagement is used
%    \begin{macrocode}
\@ifl@t@r\fmtversion{2022-06-01}
  {
    \IfDocumentMetadataTF {}{\RequirePackage{transparent-nometadata}}
    \IfDocumentMetadataTF {}{\endinput}
  }
  {\RequirePackage{transparent-nometadata}}  
%    \end{macrocode}
%    \begin{macrocode}
\RequirePackage{l3opacity}

\NewDocumentCommand{\transparent} { m }
      {
       \opacity_select:n{\fp_eval:n{ min(max(0,#1),1) } }
      }

\NewDocumentCommand{\texttransparent}{m m}
  {
   \mode_leave_vertical:
   \group_begin:
     \transparent{#1}
     #2
   \group_end:
  }
%</package-new>
%    \end{macrocode}
% \subsection{Old version without the PDF management}
%    \begin{macrocode}
%<*package>
\NeedsTeXFormat{LaTeX2e}
\ProvidesPackage{transparent-nometadata}%
  [2022-10-27 v1.5 Transparency via pdfTeX's color stack (HO)]%
%    \end{macrocode}
%
% \subsection{Initial checks}
%
% \subsubsection{Check for \pdfTeX\ in PDF mode}
%    \begin{macrocode}
\RequirePackage{iftex}
\ifpdf
\else
  \PackageWarningNoLine{transparent}{%
    Loading aborted, because pdfTeX is not running in PDF mode%
  }%
  \expandafter\endinput
\fi
%    \end{macrocode}
%
% \subsubsection{Check \pdfTeX\ or Lua\TeX\ version}
%    \begin{macrocode}
\ifx\pdfextension\@undefined
  \let\TRP@pdfcolorstackinit\pdfcolorstackinit
  \let\TRP@pdfpageresources\pdfpageresources
  \let\TRP@pdfcolorstack\pdfcolorstack
\else
  \def\TRP@pdfcolorstackinit           {\pdffeedback colorstackinit}
  \protected\edef\TRP@pdfpageresources {\pdfvariable pageresources}
  \protected\def\TRP@pdfcolorstack     {\pdfextension colorstack}
\fi
%    \end{macrocode}
%
%    \begin{macrocode}
\ifcsname TRP@pdfcolorstackinit\endcsname\else
  \PackageWarningNoLine{transparent}{%
    Your pdfTeX version does not support color stacks%
  }%
  \expandafter\endinput
\fi
%    \end{macrocode}
%
% \subsection{Compatibility with pgf}
%  https://github.com/ho-tex/transparent/issues/1
%    \begin{macrocode}
\AtBeginDocument
 {%
   \ifcsname pgfutil@addpdfresource@extgs\endcsname
    \let\TRP@addresource\relax
    \pgfutil@addpdfresource@extgs{\TRP@list}%
   \fi
 }
%    \end{macrocode}
% \subsection{Transparency}
%
%    The setting for the different transparency values must
%    be added to the page resources. In the first run the values
%    are recorded in the \xfile{.aux} file. In the second run
%    the values are set and transparency is available.
%    \begin{macrocode}
\RequirePackage{auxhook}
\AddLineBeginAux{%
  \string\providecommand{\string\transparent@use}[1]{}%
}
\gdef\TRP@list{/TRP1<</ca 1/CA 1>>}
\def\transparent@use#1{%
  \@ifundefined{TRP#1}{%
    \g@addto@macro\TRP@list{%
      /TRP#1<</ca #1/CA #1>>%
    }%
    \expandafter\gdef\csname TRP#1\endcsname{/TRP#1 gs}%
  }{%
    % #1 is already known, nothing to do
  }%
}
\AtBeginDocument{%
  \TRP@addresource
  \let\transparent@use\@gobble
}
%    \end{macrocode}
%    Unhappily the interface setting page resources is very
%    poor, only a token register \cs{pdfpageresources}.
%    Thus this package tries to be cooperative in the way that
%    it embeds the previous contents of \cs{pdfpageresources}.
%    However it does not solve the problem, if several packages
%    want to set |/ExtGState|.
%    \begin{macrocode}
\def\TRP@addresource{%
  \begingroup
    \edef\x{\endgroup
      \TRP@pdfpageresources{%
        \the\TRP@pdfpageresources
        /ExtGState<<\TRP@list>>%
      }%
    }%
  \x
}
\newif\ifTRP@rerun
\xdef\TRP@colorstack{%
  \TRP@pdfcolorstackinit page direct{/TRP1 gs}%
}
%    \end{macrocode}
%    \begin{macro}{\transparent}
%    \begin{macrocode}
\newcommand*{\transparent}[1]{%
  \begingroup
    \dimen@=#1\p@\relax
    \ifdim\dimen@>\p@
      \dimen@=\p@
    \fi
    \ifdim\dimen@<\z@
      \dimen@=\z@
    \fi
    \ifdim\dimen@=\p@
      \def\x{1}%
    \else
      \ifdim\dimen@=\z@
        \def\x{0}%
      \else
        \edef\x{\strip@pt\dimen@}%
        \edef\x{\expandafter\@gobble\x}%
      \fi
    \fi
    \if@filesw
      \immediate\write\@auxout{%
        \string\transparent@use{\x}%
      }%
    \fi
    \edef\x{\endgroup
      \def\noexpand\transparent@current{\x}%
    }%
  \x
  \transparent@set
}
%    \end{macrocode}
%    \end{macro}
%    \begin{macrocode}
\AtEndDocument{%
  \ifTRP@rerun
    \PackageWarningNoLine{transparent}{%
      Rerun to get transparencies right%
    }%
  \fi
}
\def\transparent@current{/TRP1 gs}
\def\transparent@set{%
  \@ifundefined{TRP\transparent@current}{%
    \global\TRP@reruntrue
  }{%
    \TRP@pdfcolorstack\TRP@colorstack push{%
      \csname TRP\transparent@current\endcsname
    }%
    \aftergroup\transparent@reset
  }%
}
\def\transparent@reset{%
  \TRP@pdfcolorstack\TRP@colorstack pop\relax
}
%    \end{macrocode}
%    \begin{macro}{\texttransparent}
%    \begin{macrocode}
\newcommand*{\texttransparent}[2]{%
  \protect\leavevmode
  \begingroup
    \transparent{#1}%
    #2%
  \endgroup
}
%    \end{macrocode}
%    \end{macro}
%
%    \begin{macrocode}
%</package>
%    \end{macrocode}
%
% \section{Installation}
%
% \subsection{Download}
%
% \paragraph{Package.} This package is available on
% CTAN\footnote{\CTANpkg{transparent}}:
% \begin{description}
% \item[\CTAN{macros/latex/contrib/transparent/transparent.dtx}] The source file.
% \item[\CTAN{macros/latex/contrib/transparent/transparent.pdf}] Documentation.
% \end{description}
%
%
%
% \subsection{Package installation}
%
% \paragraph{Unpacking.} The \xfile{.dtx} file is a self-extracting
% \docstrip\ archive. The files are extracted by running the
% \xfile{.dtx} through \plainTeX:
% \begin{quote}
%   \verb|tex transparent.dtx|
% \end{quote}
%
% \paragraph{TDS.} Now the different files must be moved into
% the different directories in your installation TDS tree
% (also known as \xfile{texmf} tree):
% \begin{quote}
% \def\t{^^A
% \begin{tabular}{@{}>{\ttfamily}l@{ $\rightarrow$ }>{\ttfamily}l@{}}
%   transparent.sty & tex/latex/transparent/transparent.sty\\
%   transparent-nometadata.sty & tex/latex/transparent/transparent-nometadata.sty\\   
%   transparent.pdf & doc/latex/transparent/transparent.pdf\\
%   transparent-example.tex & doc/latex/transparent/transparent-example.tex\\
%   transparent.dtx & source/latex/transparent/transparent.dtx\\
% \end{tabular}^^A
% }^^A
% \sbox0{\t}^^A
% \ifdim\wd0>\linewidth
%   \begingroup
%     \advance\linewidth by\leftmargin
%     \advance\linewidth by\rightmargin
%   \edef\x{\endgroup
%     \def\noexpand\lw{\the\linewidth}^^A
%   }\x
%   \def\lwbox{^^A
%     \leavevmode
%     \hbox to \linewidth{^^A
%       \kern-\leftmargin\relax
%       \hss
%       \usebox0
%       \hss
%       \kern-\rightmargin\relax
%     }^^A
%   }^^A
%   \ifdim\wd0>\lw
%     \sbox0{\small\t}^^A
%     \ifdim\wd0>\linewidth
%       \ifdim\wd0>\lw
%         \sbox0{\footnotesize\t}^^A
%         \ifdim\wd0>\linewidth
%           \ifdim\wd0>\lw
%             \sbox0{\scriptsize\t}^^A
%             \ifdim\wd0>\linewidth
%               \ifdim\wd0>\lw
%                 \sbox0{\tiny\t}^^A
%                 \ifdim\wd0>\linewidth
%                   \lwbox
%                 \else
%                   \usebox0
%                 \fi
%               \else
%                 \lwbox
%               \fi
%             \else
%               \usebox0
%             \fi
%           \else
%             \lwbox
%           \fi
%         \else
%           \usebox0
%         \fi
%       \else
%         \lwbox
%       \fi
%     \else
%       \usebox0
%     \fi
%   \else
%     \lwbox
%   \fi
% \else
%   \usebox0
% \fi
% \end{quote}
% If you have a \xfile{docstrip.cfg} that configures and enables \docstrip's
% TDS installing feature, then some files can already be in the right
% place, see the documentation of \docstrip.
%
% \subsection{Refresh file name databases}
%
% If your \TeX~distribution
% (\TeX\,Live, \mikTeX, \dots) relies on file name databases, you must refresh
% these. For example, \TeX\,Live\ users run \verb|texhash| or
% \verb|mktexlsr|.
%
% \subsection{Some details for the interested}
%
% \paragraph{Unpacking with \LaTeX.}
% The \xfile{.dtx} chooses its action depending on the format:
% \begin{description}
% \item[\plainTeX:] Run \docstrip\ and extract the files.
% \item[\LaTeX:] Generate the documentation.
% \end{description}
% If you insist on using \LaTeX\ for \docstrip\ (really,
% \docstrip\ does not need \LaTeX), then inform the autodetect routine
% about your intention:
% \begin{quote}
%   \verb|latex \let\install=y% \iffalse meta-comment
%
% File: transparent.dtx
% Version: 2022-10-27 v1.5
% Info: Transparency via pdfTeX's color stack
%
% Copyright (C)
%    2007 Heiko Oberdiek
%    2016-2022 Oberdiek Package Support Group 2018
%    https://github.com/ho-tex/transparent/issues
%
% This work may be distributed and/or modified under the
% conditions of the LaTeX Project Public License, either
% version 1.3c of this license or (at your option) any later
% version. This version of this license is in
%    https://www.latex-project.org/lppl/lppl-1-3c.txt
% and the latest version of this license is in
%    https://www.latex-project.org/lppl.txt
% and version 1.3 or later is part of all distributions of
% LaTeX version 2005/12/01 or later.
%
% This work has the LPPL maintenance status "maintained".
%
% The Current Maintainers of this work are
% Heiko Oberdiek and the Oberdiek Package Support Group
% https://github.com/ho-tex/transparent/issues
%
% This work consists of the main source file transparent.dtx
% and the derived files
%    transparent.sty, transparent-nometadata.sty, transparent.pdf, 
%    transparent.ins, transparent.drv, transparent-example.tex.
%
% Distribution:
%    CTAN:macros/latex/contrib/transparent/transparent.dtx
%    CTAN:macros/latex/contrib/transparent/transparent.pdf
%
% Unpacking:
%    (a) If transparent.ins is present:
%           tex transparent.ins
%    (b) Without transparent.ins:
%           tex transparent.dtx
%    (c) If you insist on using LaTeX
%           latex \let\install=y% \iffalse meta-comment
%
% File: transparent.dtx
% Version: 2022-10-27 v1.5
% Info: Transparency via pdfTeX's color stack
%
% Copyright (C)
%    2007 Heiko Oberdiek
%    2016-2022 Oberdiek Package Support Group 2018
%    https://github.com/ho-tex/transparent/issues
%
% This work may be distributed and/or modified under the
% conditions of the LaTeX Project Public License, either
% version 1.3c of this license or (at your option) any later
% version. This version of this license is in
%    https://www.latex-project.org/lppl/lppl-1-3c.txt
% and the latest version of this license is in
%    https://www.latex-project.org/lppl.txt
% and version 1.3 or later is part of all distributions of
% LaTeX version 2005/12/01 or later.
%
% This work has the LPPL maintenance status "maintained".
%
% The Current Maintainers of this work are
% Heiko Oberdiek and the Oberdiek Package Support Group
% https://github.com/ho-tex/transparent/issues
%
% This work consists of the main source file transparent.dtx
% and the derived files
%    transparent.sty, transparent-nometadata.sty, transparent.pdf, 
%    transparent.ins, transparent.drv, transparent-example.tex.
%
% Distribution:
%    CTAN:macros/latex/contrib/transparent/transparent.dtx
%    CTAN:macros/latex/contrib/transparent/transparent.pdf
%
% Unpacking:
%    (a) If transparent.ins is present:
%           tex transparent.ins
%    (b) Without transparent.ins:
%           tex transparent.dtx
%    (c) If you insist on using LaTeX
%           latex \let\install=y\input{transparent.dtx}
%        (quote the arguments according to the demands of your shell)
%
% Documentation:
%    (a) If transparent.drv is present:
%           latex transparent.drv
%    (b) Without transparent.drv:
%           latex transparent.dtx; ...
%    The class ltxdoc loads the configuration file ltxdoc.cfg
%    if available. Here you can specify further options, e.g.
%    use A4 as paper format:
%       \PassOptionsToClass{a4paper}{article}
%
%    Programm calls to get the documentation (example):
%       pdflatex transparent.dtx
%       makeindex -s gind.ist transparent.idx
%       pdflatex transparent.dtx
%       makeindex -s gind.ist transparent.idx
%       pdflatex transparent.dtx
%
% Installation:
%    TDS:tex/latex/transparent/transparent.sty
%    TDS:tex/latex/transparent/transparent-nometadata.sty
%    TDS:doc/latex/transparent/transparent.pdf
%    TDS:source/latex/transparent/transparent.dtx
%
%<*ignore>
\begingroup
  \catcode123=1 %
  \catcode125=2 %
  \def\x{LaTeX2e}%
\expandafter\endgroup
\ifcase 0\ifx\install y1\fi\expandafter
         \ifx\csname processbatchFile\endcsname\relax\else1\fi
         \ifx\fmtname\x\else 1\fi\relax
\else\csname fi\endcsname
%</ignore>
%<*install>
\input docstrip.tex
\Msg{************************************************************************}
\Msg{* Installation}
\Msg{* Package: transparent 2022-10-27 v1.5 Transparency via pdfTeX's color stack (HO)}
\Msg{************************************************************************}

\keepsilent
\askforoverwritefalse

\let\MetaPrefix\relax
\preamble

This is a generated file.

Project: transparent
Version: 2022-10-27 v1.5

Copyright (C)
   2007 Heiko Oberdiek
   2016-2022 Oberdiek Package Support Group

This work may be distributed and/or modified under the
conditions of the LaTeX Project Public License, either
version 1.3c of this license or (at your option) any later
version. This version of this license is in
   https://www.latex-project.org/lppl/lppl-1-3c.txt
and the latest version of this license is in
   https://www.latex-project.org/lppl.txt
and version 1.3 or later is part of all distributions of
LaTeX version 2005/12/01 or later.

This work has the LPPL maintenance status "maintained".

The Current Maintainers of this work are
Heiko Oberdiek and the Oberdiek Package Support Group
https://github.com/ho-tex/transparent/issues


This work consists of the main source file transparent.dtx
and the derived files
   transparent.sty, transparent-nometadata.sty, transparent.pdf,
   transparent.ins, transparent.drv, transparent-example.tex.

\endpreamble
\let\MetaPrefix\DoubleperCent

\generate{%
  \file{transparent.ins}{\from{transparent.dtx}{install}}%
  \file{transparent.drv}{\from{transparent.dtx}{driver}}%
  \usedir{tex/latex/transparent}%
  \file{transparent-nometadata.sty}{\from{transparent.dtx}{package}}%
  \file{transparent.sty}{\from{transparent.dtx}{package-new}}%
  \usedir{doc/latex/transparent}%
  \file{transparent-example.tex}{\from{transparent.dtx}{example}}%
}

\catcode32=13\relax% active space
\let =\space%
\Msg{************************************************************************}
\Msg{*}
\Msg{* To finish the installation you have to move the following}
\Msg{* file into a directory searched by TeX:}
\Msg{*}
\Msg{*     transparent.sty}
\Msg{*     transparent-nometadata.sty}
\Msg{*}
\Msg{* To produce the documentation run the file `transparent.drv'}
\Msg{* through LaTeX.}
\Msg{*}
\Msg{* Happy TeXing!}
\Msg{*}
\Msg{************************************************************************}

\endbatchfile
%</install>
%<*ignore>
\fi
%</ignore>
%<*driver>
\NeedsTeXFormat{LaTeX2e}
\ProvidesFile{transparent.drv}%
  [2022-10-27 v1.5 Transparency via pdfTeX's color stack (HO)]%
\documentclass{ltxdoc}
\usepackage{holtxdoc}[2011/11/22]
\begin{document}
  \DocInput{transparent.dtx}%
\end{document}
%</driver>
% \fi
%
%
%
% \GetFileInfo{transparent.drv}
%
% \title{The \xpackage{transparent} package}
% \date{2022-10-27 v1.5}
% \author{Heiko Oberdiek\thanks
% {Please report any issues at \url{https://github.com/ho-tex/transparent/issues}}}
%
% \maketitle
%
% \begin{abstract}
% \pdfTeX\ and lua\TeX{} supports several color stacks. This
% package shows, how a separate color stack can be used for transparency,
% a property besides color, that also works across page break.
% \end{abstract}
%
% \tableofcontents
%
% \section{User interface}
%
% The package \xpackage{transparent} defines \cs{transparent} and
% \cs{texttransparent}. They are used like \cs{color} and \cs{textcolor}.
% The first argument is the transparency value between 0 and 1 where 0
% is fully transparent and 1 is opaque.
% 
% \section{Engine support}
% If the PDF management is loaded by using \verb+\DocumentMetadata{}+ at the beginning
% of the document the package makes use of \texttt{l3opacity} and
% can be used with all engines supported by \texttt{l3backend}.
% But only with \pdfTeX\ and lua\TeX{} the transparency will also work across page break.
% 
% 
% Without the PDF management, the package works only with \pdfTeX\ and lua\TeX{}.
% It then uses the original code by Heiko Oberdiek. It then uses the
% primitive \cs{pdfpageresources} and can clash with other packages that also use
% \cs{pdfpageresources}.
% 
%
% \section{Example for usage}
%    \begin{macrocode}
%<*example>
\documentclass[12pt]{article}

\usepackage{color}
\usepackage{transparent}

\begin{document}
\colorbox{yellow}{%
  \bfseries
  \color{blue}%
  Blue and %
  \transparent{0.6}%
  transparent blue%
}

\bigskip
Hello World
\texttransparent{0.5}{Hello\newpage World}
Hello World
\end{document}
%</example>
%    \end{macrocode}
%
% \StopEventually{
% }
%
% \section{Implementation}
% \subsection{New version using the methods of the PDF management} 
% Identification
%    \begin{macrocode}
%<*package-new>
\NeedsTeXFormat{LaTeX2e}[2020/10/01]
\ProvidesExplPackage{transparent}{2022-10-27}{1.5}
  {Transparency with color stacks}%
%    \end{macrocode}
% Testing if the pdfmanagement is used
%    \begin{macrocode}
\@ifl@t@r\fmtversion{2022-06-01}
  {
    \IfDocumentMetadataTF {}{\RequirePackage{transparent-nometadata}}
    \IfDocumentMetadataTF {}{\endinput}
  }
  {\RequirePackage{transparent-nometadata}}  
%    \end{macrocode}
%    \begin{macrocode}
\RequirePackage{l3opacity}

\NewDocumentCommand{\transparent} { m }
      {
       \opacity_select:n{\fp_eval:n{ min(max(0,#1),1) } }
      }

\NewDocumentCommand{\texttransparent}{m m}
  {
   \mode_leave_vertical:
   \group_begin:
     \transparent{#1}
     #2
   \group_end:
  }
%</package-new>
%    \end{macrocode}
% \subsection{Old version without the PDF management}
%    \begin{macrocode}
%<*package>
\NeedsTeXFormat{LaTeX2e}
\ProvidesPackage{transparent-nometadata}%
  [2022-10-27 v1.5 Transparency via pdfTeX's color stack (HO)]%
%    \end{macrocode}
%
% \subsection{Initial checks}
%
% \subsubsection{Check for \pdfTeX\ in PDF mode}
%    \begin{macrocode}
\RequirePackage{iftex}
\ifpdf
\else
  \PackageWarningNoLine{transparent}{%
    Loading aborted, because pdfTeX is not running in PDF mode%
  }%
  \expandafter\endinput
\fi
%    \end{macrocode}
%
% \subsubsection{Check \pdfTeX\ or Lua\TeX\ version}
%    \begin{macrocode}
\ifx\pdfextension\@undefined
  \let\TRP@pdfcolorstackinit\pdfcolorstackinit
  \let\TRP@pdfpageresources\pdfpageresources
  \let\TRP@pdfcolorstack\pdfcolorstack
\else
  \def\TRP@pdfcolorstackinit           {\pdffeedback colorstackinit}
  \protected\edef\TRP@pdfpageresources {\pdfvariable pageresources}
  \protected\def\TRP@pdfcolorstack     {\pdfextension colorstack}
\fi
%    \end{macrocode}
%
%    \begin{macrocode}
\ifcsname TRP@pdfcolorstackinit\endcsname\else
  \PackageWarningNoLine{transparent}{%
    Your pdfTeX version does not support color stacks%
  }%
  \expandafter\endinput
\fi
%    \end{macrocode}
%
% \subsection{Compatibility with pgf}
%  https://github.com/ho-tex/transparent/issues/1
%    \begin{macrocode}
\AtBeginDocument
 {%
   \ifcsname pgfutil@addpdfresource@extgs\endcsname
    \let\TRP@addresource\relax
    \pgfutil@addpdfresource@extgs{\TRP@list}%
   \fi
 }
%    \end{macrocode}
% \subsection{Transparency}
%
%    The setting for the different transparency values must
%    be added to the page resources. In the first run the values
%    are recorded in the \xfile{.aux} file. In the second run
%    the values are set and transparency is available.
%    \begin{macrocode}
\RequirePackage{auxhook}
\AddLineBeginAux{%
  \string\providecommand{\string\transparent@use}[1]{}%
}
\gdef\TRP@list{/TRP1<</ca 1/CA 1>>}
\def\transparent@use#1{%
  \@ifundefined{TRP#1}{%
    \g@addto@macro\TRP@list{%
      /TRP#1<</ca #1/CA #1>>%
    }%
    \expandafter\gdef\csname TRP#1\endcsname{/TRP#1 gs}%
  }{%
    % #1 is already known, nothing to do
  }%
}
\AtBeginDocument{%
  \TRP@addresource
  \let\transparent@use\@gobble
}
%    \end{macrocode}
%    Unhappily the interface setting page resources is very
%    poor, only a token register \cs{pdfpageresources}.
%    Thus this package tries to be cooperative in the way that
%    it embeds the previous contents of \cs{pdfpageresources}.
%    However it does not solve the problem, if several packages
%    want to set |/ExtGState|.
%    \begin{macrocode}
\def\TRP@addresource{%
  \begingroup
    \edef\x{\endgroup
      \TRP@pdfpageresources{%
        \the\TRP@pdfpageresources
        /ExtGState<<\TRP@list>>%
      }%
    }%
  \x
}
\newif\ifTRP@rerun
\xdef\TRP@colorstack{%
  \TRP@pdfcolorstackinit page direct{/TRP1 gs}%
}
%    \end{macrocode}
%    \begin{macro}{\transparent}
%    \begin{macrocode}
\newcommand*{\transparent}[1]{%
  \begingroup
    \dimen@=#1\p@\relax
    \ifdim\dimen@>\p@
      \dimen@=\p@
    \fi
    \ifdim\dimen@<\z@
      \dimen@=\z@
    \fi
    \ifdim\dimen@=\p@
      \def\x{1}%
    \else
      \ifdim\dimen@=\z@
        \def\x{0}%
      \else
        \edef\x{\strip@pt\dimen@}%
        \edef\x{\expandafter\@gobble\x}%
      \fi
    \fi
    \if@filesw
      \immediate\write\@auxout{%
        \string\transparent@use{\x}%
      }%
    \fi
    \edef\x{\endgroup
      \def\noexpand\transparent@current{\x}%
    }%
  \x
  \transparent@set
}
%    \end{macrocode}
%    \end{macro}
%    \begin{macrocode}
\AtEndDocument{%
  \ifTRP@rerun
    \PackageWarningNoLine{transparent}{%
      Rerun to get transparencies right%
    }%
  \fi
}
\def\transparent@current{/TRP1 gs}
\def\transparent@set{%
  \@ifundefined{TRP\transparent@current}{%
    \global\TRP@reruntrue
  }{%
    \TRP@pdfcolorstack\TRP@colorstack push{%
      \csname TRP\transparent@current\endcsname
    }%
    \aftergroup\transparent@reset
  }%
}
\def\transparent@reset{%
  \TRP@pdfcolorstack\TRP@colorstack pop\relax
}
%    \end{macrocode}
%    \begin{macro}{\texttransparent}
%    \begin{macrocode}
\newcommand*{\texttransparent}[2]{%
  \protect\leavevmode
  \begingroup
    \transparent{#1}%
    #2%
  \endgroup
}
%    \end{macrocode}
%    \end{macro}
%
%    \begin{macrocode}
%</package>
%    \end{macrocode}
%
% \section{Installation}
%
% \subsection{Download}
%
% \paragraph{Package.} This package is available on
% CTAN\footnote{\CTANpkg{transparent}}:
% \begin{description}
% \item[\CTAN{macros/latex/contrib/transparent/transparent.dtx}] The source file.
% \item[\CTAN{macros/latex/contrib/transparent/transparent.pdf}] Documentation.
% \end{description}
%
%
%
% \subsection{Package installation}
%
% \paragraph{Unpacking.} The \xfile{.dtx} file is a self-extracting
% \docstrip\ archive. The files are extracted by running the
% \xfile{.dtx} through \plainTeX:
% \begin{quote}
%   \verb|tex transparent.dtx|
% \end{quote}
%
% \paragraph{TDS.} Now the different files must be moved into
% the different directories in your installation TDS tree
% (also known as \xfile{texmf} tree):
% \begin{quote}
% \def\t{^^A
% \begin{tabular}{@{}>{\ttfamily}l@{ $\rightarrow$ }>{\ttfamily}l@{}}
%   transparent.sty & tex/latex/transparent/transparent.sty\\
%   transparent-nometadata.sty & tex/latex/transparent/transparent-nometadata.sty\\   
%   transparent.pdf & doc/latex/transparent/transparent.pdf\\
%   transparent-example.tex & doc/latex/transparent/transparent-example.tex\\
%   transparent.dtx & source/latex/transparent/transparent.dtx\\
% \end{tabular}^^A
% }^^A
% \sbox0{\t}^^A
% \ifdim\wd0>\linewidth
%   \begingroup
%     \advance\linewidth by\leftmargin
%     \advance\linewidth by\rightmargin
%   \edef\x{\endgroup
%     \def\noexpand\lw{\the\linewidth}^^A
%   }\x
%   \def\lwbox{^^A
%     \leavevmode
%     \hbox to \linewidth{^^A
%       \kern-\leftmargin\relax
%       \hss
%       \usebox0
%       \hss
%       \kern-\rightmargin\relax
%     }^^A
%   }^^A
%   \ifdim\wd0>\lw
%     \sbox0{\small\t}^^A
%     \ifdim\wd0>\linewidth
%       \ifdim\wd0>\lw
%         \sbox0{\footnotesize\t}^^A
%         \ifdim\wd0>\linewidth
%           \ifdim\wd0>\lw
%             \sbox0{\scriptsize\t}^^A
%             \ifdim\wd0>\linewidth
%               \ifdim\wd0>\lw
%                 \sbox0{\tiny\t}^^A
%                 \ifdim\wd0>\linewidth
%                   \lwbox
%                 \else
%                   \usebox0
%                 \fi
%               \else
%                 \lwbox
%               \fi
%             \else
%               \usebox0
%             \fi
%           \else
%             \lwbox
%           \fi
%         \else
%           \usebox0
%         \fi
%       \else
%         \lwbox
%       \fi
%     \else
%       \usebox0
%     \fi
%   \else
%     \lwbox
%   \fi
% \else
%   \usebox0
% \fi
% \end{quote}
% If you have a \xfile{docstrip.cfg} that configures and enables \docstrip's
% TDS installing feature, then some files can already be in the right
% place, see the documentation of \docstrip.
%
% \subsection{Refresh file name databases}
%
% If your \TeX~distribution
% (\TeX\,Live, \mikTeX, \dots) relies on file name databases, you must refresh
% these. For example, \TeX\,Live\ users run \verb|texhash| or
% \verb|mktexlsr|.
%
% \subsection{Some details for the interested}
%
% \paragraph{Unpacking with \LaTeX.}
% The \xfile{.dtx} chooses its action depending on the format:
% \begin{description}
% \item[\plainTeX:] Run \docstrip\ and extract the files.
% \item[\LaTeX:] Generate the documentation.
% \end{description}
% If you insist on using \LaTeX\ for \docstrip\ (really,
% \docstrip\ does not need \LaTeX), then inform the autodetect routine
% about your intention:
% \begin{quote}
%   \verb|latex \let\install=y\input{transparent.dtx}|
% \end{quote}
% Do not forget to quote the argument according to the demands
% of your shell.
%
% \paragraph{Generating the documentation.}
% You can use both the \xfile{.dtx} or the \xfile{.drv} to generate
% the documentation. The process can be configured by the
% configuration file \xfile{ltxdoc.cfg}. For instance, put this
% line into this file, if you want to have A4 as paper format:
% \begin{quote}
%   \verb|\PassOptionsToClass{a4paper}{article}|
% \end{quote}
% An example follows how to generate the
% documentation with pdf\LaTeX:
% \begin{quote}
%\begin{verbatim}
%pdflatex transparent.dtx
%makeindex -s gind.ist transparent.idx
%pdflatex transparent.dtx
%makeindex -s gind.ist transparent.idx
%pdflatex transparent.dtx
%\end{verbatim}
% \end{quote}
%
% \begin{History}
%   \begin{Version}{2007/01/08 v1.0}
%   \item
%     First version.
%   \end{Version}
%   \begin{Version}{2016/05/16 v1.1}
%   \item
%     Documentation updates.
%   \end{Version}
%   \begin{Version}{2018/09/10 v1.2}
%   \item
%     Update for LuaTeX, remove dependency on Pdf\TeX\ command names.
%   \end{Version}
%   \begin{Version}{2018/11/18 v1.3}
%   \item
%     Added code for pgf compatibility, see https://github.com/ho-tex/transparent/issues/1
%   \end{Version}
%   \begin{Version}{2019/11/29 v1.4}
%   \item
%     Documentation updates.
% \item
% Use \xpackage{iftex} package.
%   \end{Version}
%   \begin{Version}{2022-10-27 v1.5}
%   \item Made the package compatible with the PDF management.
%   \end{Version}
% \end{History}
%
% \PrintIndex
%
% \Finale
\endinput

%        (quote the arguments according to the demands of your shell)
%
% Documentation:
%    (a) If transparent.drv is present:
%           latex transparent.drv
%    (b) Without transparent.drv:
%           latex transparent.dtx; ...
%    The class ltxdoc loads the configuration file ltxdoc.cfg
%    if available. Here you can specify further options, e.g.
%    use A4 as paper format:
%       \PassOptionsToClass{a4paper}{article}
%
%    Programm calls to get the documentation (example):
%       pdflatex transparent.dtx
%       makeindex -s gind.ist transparent.idx
%       pdflatex transparent.dtx
%       makeindex -s gind.ist transparent.idx
%       pdflatex transparent.dtx
%
% Installation:
%    TDS:tex/latex/transparent/transparent.sty
%    TDS:tex/latex/transparent/transparent-nometadata.sty
%    TDS:doc/latex/transparent/transparent.pdf
%    TDS:source/latex/transparent/transparent.dtx
%
%<*ignore>
\begingroup
  \catcode123=1 %
  \catcode125=2 %
  \def\x{LaTeX2e}%
\expandafter\endgroup
\ifcase 0\ifx\install y1\fi\expandafter
         \ifx\csname processbatchFile\endcsname\relax\else1\fi
         \ifx\fmtname\x\else 1\fi\relax
\else\csname fi\endcsname
%</ignore>
%<*install>
\input docstrip.tex
\Msg{************************************************************************}
\Msg{* Installation}
\Msg{* Package: transparent 2022-10-27 v1.5 Transparency via pdfTeX's color stack (HO)}
\Msg{************************************************************************}

\keepsilent
\askforoverwritefalse

\let\MetaPrefix\relax
\preamble

This is a generated file.

Project: transparent
Version: 2022-10-27 v1.5

Copyright (C)
   2007 Heiko Oberdiek
   2016-2022 Oberdiek Package Support Group

This work may be distributed and/or modified under the
conditions of the LaTeX Project Public License, either
version 1.3c of this license or (at your option) any later
version. This version of this license is in
   https://www.latex-project.org/lppl/lppl-1-3c.txt
and the latest version of this license is in
   https://www.latex-project.org/lppl.txt
and version 1.3 or later is part of all distributions of
LaTeX version 2005/12/01 or later.

This work has the LPPL maintenance status "maintained".

The Current Maintainers of this work are
Heiko Oberdiek and the Oberdiek Package Support Group
https://github.com/ho-tex/transparent/issues


This work consists of the main source file transparent.dtx
and the derived files
   transparent.sty, transparent-nometadata.sty, transparent.pdf,
   transparent.ins, transparent.drv, transparent-example.tex.

\endpreamble
\let\MetaPrefix\DoubleperCent

\generate{%
  \file{transparent.ins}{\from{transparent.dtx}{install}}%
  \file{transparent.drv}{\from{transparent.dtx}{driver}}%
  \usedir{tex/latex/transparent}%
  \file{transparent-nometadata.sty}{\from{transparent.dtx}{package}}%
  \file{transparent.sty}{\from{transparent.dtx}{package-new}}%
  \usedir{doc/latex/transparent}%
  \file{transparent-example.tex}{\from{transparent.dtx}{example}}%
}

\catcode32=13\relax% active space
\let =\space%
\Msg{************************************************************************}
\Msg{*}
\Msg{* To finish the installation you have to move the following}
\Msg{* file into a directory searched by TeX:}
\Msg{*}
\Msg{*     transparent.sty}
\Msg{*     transparent-nometadata.sty}
\Msg{*}
\Msg{* To produce the documentation run the file `transparent.drv'}
\Msg{* through LaTeX.}
\Msg{*}
\Msg{* Happy TeXing!}
\Msg{*}
\Msg{************************************************************************}

\endbatchfile
%</install>
%<*ignore>
\fi
%</ignore>
%<*driver>
\NeedsTeXFormat{LaTeX2e}
\ProvidesFile{transparent.drv}%
  [2022-10-27 v1.5 Transparency via pdfTeX's color stack (HO)]%
\documentclass{ltxdoc}
\usepackage{holtxdoc}[2011/11/22]
\begin{document}
  \DocInput{transparent.dtx}%
\end{document}
%</driver>
% \fi
%
%
%
% \GetFileInfo{transparent.drv}
%
% \title{The \xpackage{transparent} package}
% \date{2022-10-27 v1.5}
% \author{Heiko Oberdiek\thanks
% {Please report any issues at \url{https://github.com/ho-tex/transparent/issues}}}
%
% \maketitle
%
% \begin{abstract}
% \pdfTeX\ and lua\TeX{} supports several color stacks. This
% package shows, how a separate color stack can be used for transparency,
% a property besides color, that also works across page break.
% \end{abstract}
%
% \tableofcontents
%
% \section{User interface}
%
% The package \xpackage{transparent} defines \cs{transparent} and
% \cs{texttransparent}. They are used like \cs{color} and \cs{textcolor}.
% The first argument is the transparency value between 0 and 1 where 0
% is fully transparent and 1 is opaque.
% 
% \section{Engine support}
% If the PDF management is loaded by using \verb+\DocumentMetadata{}+ at the beginning
% of the document the package makes use of \texttt{l3opacity} and
% can be used with all engines supported by \texttt{l3backend}.
% But only with \pdfTeX\ and lua\TeX{} the transparency will also work across page break.
% 
% 
% Without the PDF management, the package works only with \pdfTeX\ and lua\TeX{}.
% It then uses the original code by Heiko Oberdiek. It then uses the
% primitive \cs{pdfpageresources} and can clash with other packages that also use
% \cs{pdfpageresources}.
% 
%
% \section{Example for usage}
%    \begin{macrocode}
%<*example>
\documentclass[12pt]{article}

\usepackage{color}
\usepackage{transparent}

\begin{document}
\colorbox{yellow}{%
  \bfseries
  \color{blue}%
  Blue and %
  \transparent{0.6}%
  transparent blue%
}

\bigskip
Hello World
\texttransparent{0.5}{Hello\newpage World}
Hello World
\end{document}
%</example>
%    \end{macrocode}
%
% \StopEventually{
% }
%
% \section{Implementation}
% \subsection{New version using the methods of the PDF management} 
% Identification
%    \begin{macrocode}
%<*package-new>
\NeedsTeXFormat{LaTeX2e}[2020/10/01]
\ProvidesExplPackage{transparent}{2022-10-27}{1.5}
  {Transparency with color stacks}%
%    \end{macrocode}
% Testing if the pdfmanagement is used
%    \begin{macrocode}
\@ifl@t@r\fmtversion{2022-06-01}
  {
    \IfDocumentMetadataTF {}{\RequirePackage{transparent-nometadata}}
    \IfDocumentMetadataTF {}{\endinput}
  }
  {\RequirePackage{transparent-nometadata}}  
%    \end{macrocode}
%    \begin{macrocode}
\RequirePackage{l3opacity}

\NewDocumentCommand{\transparent} { m }
      {
       \opacity_select:n{\fp_eval:n{ min(max(0,#1),1) } }
      }

\NewDocumentCommand{\texttransparent}{m m}
  {
   \mode_leave_vertical:
   \group_begin:
     \transparent{#1}
     #2
   \group_end:
  }
%</package-new>
%    \end{macrocode}
% \subsection{Old version without the PDF management}
%    \begin{macrocode}
%<*package>
\NeedsTeXFormat{LaTeX2e}
\ProvidesPackage{transparent-nometadata}%
  [2022-10-27 v1.5 Transparency via pdfTeX's color stack (HO)]%
%    \end{macrocode}
%
% \subsection{Initial checks}
%
% \subsubsection{Check for \pdfTeX\ in PDF mode}
%    \begin{macrocode}
\RequirePackage{iftex}
\ifpdf
\else
  \PackageWarningNoLine{transparent}{%
    Loading aborted, because pdfTeX is not running in PDF mode%
  }%
  \expandafter\endinput
\fi
%    \end{macrocode}
%
% \subsubsection{Check \pdfTeX\ or Lua\TeX\ version}
%    \begin{macrocode}
\ifx\pdfextension\@undefined
  \let\TRP@pdfcolorstackinit\pdfcolorstackinit
  \let\TRP@pdfpageresources\pdfpageresources
  \let\TRP@pdfcolorstack\pdfcolorstack
\else
  \def\TRP@pdfcolorstackinit           {\pdffeedback colorstackinit}
  \protected\edef\TRP@pdfpageresources {\pdfvariable pageresources}
  \protected\def\TRP@pdfcolorstack     {\pdfextension colorstack}
\fi
%    \end{macrocode}
%
%    \begin{macrocode}
\ifcsname TRP@pdfcolorstackinit\endcsname\else
  \PackageWarningNoLine{transparent}{%
    Your pdfTeX version does not support color stacks%
  }%
  \expandafter\endinput
\fi
%    \end{macrocode}
%
% \subsection{Compatibility with pgf}
%  https://github.com/ho-tex/transparent/issues/1
%    \begin{macrocode}
\AtBeginDocument
 {%
   \ifcsname pgfutil@addpdfresource@extgs\endcsname
    \let\TRP@addresource\relax
    \pgfutil@addpdfresource@extgs{\TRP@list}%
   \fi
 }
%    \end{macrocode}
% \subsection{Transparency}
%
%    The setting for the different transparency values must
%    be added to the page resources. In the first run the values
%    are recorded in the \xfile{.aux} file. In the second run
%    the values are set and transparency is available.
%    \begin{macrocode}
\RequirePackage{auxhook}
\AddLineBeginAux{%
  \string\providecommand{\string\transparent@use}[1]{}%
}
\gdef\TRP@list{/TRP1<</ca 1/CA 1>>}
\def\transparent@use#1{%
  \@ifundefined{TRP#1}{%
    \g@addto@macro\TRP@list{%
      /TRP#1<</ca #1/CA #1>>%
    }%
    \expandafter\gdef\csname TRP#1\endcsname{/TRP#1 gs}%
  }{%
    % #1 is already known, nothing to do
  }%
}
\AtBeginDocument{%
  \TRP@addresource
  \let\transparent@use\@gobble
}
%    \end{macrocode}
%    Unhappily the interface setting page resources is very
%    poor, only a token register \cs{pdfpageresources}.
%    Thus this package tries to be cooperative in the way that
%    it embeds the previous contents of \cs{pdfpageresources}.
%    However it does not solve the problem, if several packages
%    want to set |/ExtGState|.
%    \begin{macrocode}
\def\TRP@addresource{%
  \begingroup
    \edef\x{\endgroup
      \TRP@pdfpageresources{%
        \the\TRP@pdfpageresources
        /ExtGState<<\TRP@list>>%
      }%
    }%
  \x
}
\newif\ifTRP@rerun
\xdef\TRP@colorstack{%
  \TRP@pdfcolorstackinit page direct{/TRP1 gs}%
}
%    \end{macrocode}
%    \begin{macro}{\transparent}
%    \begin{macrocode}
\newcommand*{\transparent}[1]{%
  \begingroup
    \dimen@=#1\p@\relax
    \ifdim\dimen@>\p@
      \dimen@=\p@
    \fi
    \ifdim\dimen@<\z@
      \dimen@=\z@
    \fi
    \ifdim\dimen@=\p@
      \def\x{1}%
    \else
      \ifdim\dimen@=\z@
        \def\x{0}%
      \else
        \edef\x{\strip@pt\dimen@}%
        \edef\x{\expandafter\@gobble\x}%
      \fi
    \fi
    \if@filesw
      \immediate\write\@auxout{%
        \string\transparent@use{\x}%
      }%
    \fi
    \edef\x{\endgroup
      \def\noexpand\transparent@current{\x}%
    }%
  \x
  \transparent@set
}
%    \end{macrocode}
%    \end{macro}
%    \begin{macrocode}
\AtEndDocument{%
  \ifTRP@rerun
    \PackageWarningNoLine{transparent}{%
      Rerun to get transparencies right%
    }%
  \fi
}
\def\transparent@current{/TRP1 gs}
\def\transparent@set{%
  \@ifundefined{TRP\transparent@current}{%
    \global\TRP@reruntrue
  }{%
    \TRP@pdfcolorstack\TRP@colorstack push{%
      \csname TRP\transparent@current\endcsname
    }%
    \aftergroup\transparent@reset
  }%
}
\def\transparent@reset{%
  \TRP@pdfcolorstack\TRP@colorstack pop\relax
}
%    \end{macrocode}
%    \begin{macro}{\texttransparent}
%    \begin{macrocode}
\newcommand*{\texttransparent}[2]{%
  \protect\leavevmode
  \begingroup
    \transparent{#1}%
    #2%
  \endgroup
}
%    \end{macrocode}
%    \end{macro}
%
%    \begin{macrocode}
%</package>
%    \end{macrocode}
%
% \section{Installation}
%
% \subsection{Download}
%
% \paragraph{Package.} This package is available on
% CTAN\footnote{\CTANpkg{transparent}}:
% \begin{description}
% \item[\CTAN{macros/latex/contrib/transparent/transparent.dtx}] The source file.
% \item[\CTAN{macros/latex/contrib/transparent/transparent.pdf}] Documentation.
% \end{description}
%
%
%
% \subsection{Package installation}
%
% \paragraph{Unpacking.} The \xfile{.dtx} file is a self-extracting
% \docstrip\ archive. The files are extracted by running the
% \xfile{.dtx} through \plainTeX:
% \begin{quote}
%   \verb|tex transparent.dtx|
% \end{quote}
%
% \paragraph{TDS.} Now the different files must be moved into
% the different directories in your installation TDS tree
% (also known as \xfile{texmf} tree):
% \begin{quote}
% \def\t{^^A
% \begin{tabular}{@{}>{\ttfamily}l@{ $\rightarrow$ }>{\ttfamily}l@{}}
%   transparent.sty & tex/latex/transparent/transparent.sty\\
%   transparent-nometadata.sty & tex/latex/transparent/transparent-nometadata.sty\\   
%   transparent.pdf & doc/latex/transparent/transparent.pdf\\
%   transparent-example.tex & doc/latex/transparent/transparent-example.tex\\
%   transparent.dtx & source/latex/transparent/transparent.dtx\\
% \end{tabular}^^A
% }^^A
% \sbox0{\t}^^A
% \ifdim\wd0>\linewidth
%   \begingroup
%     \advance\linewidth by\leftmargin
%     \advance\linewidth by\rightmargin
%   \edef\x{\endgroup
%     \def\noexpand\lw{\the\linewidth}^^A
%   }\x
%   \def\lwbox{^^A
%     \leavevmode
%     \hbox to \linewidth{^^A
%       \kern-\leftmargin\relax
%       \hss
%       \usebox0
%       \hss
%       \kern-\rightmargin\relax
%     }^^A
%   }^^A
%   \ifdim\wd0>\lw
%     \sbox0{\small\t}^^A
%     \ifdim\wd0>\linewidth
%       \ifdim\wd0>\lw
%         \sbox0{\footnotesize\t}^^A
%         \ifdim\wd0>\linewidth
%           \ifdim\wd0>\lw
%             \sbox0{\scriptsize\t}^^A
%             \ifdim\wd0>\linewidth
%               \ifdim\wd0>\lw
%                 \sbox0{\tiny\t}^^A
%                 \ifdim\wd0>\linewidth
%                   \lwbox
%                 \else
%                   \usebox0
%                 \fi
%               \else
%                 \lwbox
%               \fi
%             \else
%               \usebox0
%             \fi
%           \else
%             \lwbox
%           \fi
%         \else
%           \usebox0
%         \fi
%       \else
%         \lwbox
%       \fi
%     \else
%       \usebox0
%     \fi
%   \else
%     \lwbox
%   \fi
% \else
%   \usebox0
% \fi
% \end{quote}
% If you have a \xfile{docstrip.cfg} that configures and enables \docstrip's
% TDS installing feature, then some files can already be in the right
% place, see the documentation of \docstrip.
%
% \subsection{Refresh file name databases}
%
% If your \TeX~distribution
% (\TeX\,Live, \mikTeX, \dots) relies on file name databases, you must refresh
% these. For example, \TeX\,Live\ users run \verb|texhash| or
% \verb|mktexlsr|.
%
% \subsection{Some details for the interested}
%
% \paragraph{Unpacking with \LaTeX.}
% The \xfile{.dtx} chooses its action depending on the format:
% \begin{description}
% \item[\plainTeX:] Run \docstrip\ and extract the files.
% \item[\LaTeX:] Generate the documentation.
% \end{description}
% If you insist on using \LaTeX\ for \docstrip\ (really,
% \docstrip\ does not need \LaTeX), then inform the autodetect routine
% about your intention:
% \begin{quote}
%   \verb|latex \let\install=y% \iffalse meta-comment
%
% File: transparent.dtx
% Version: 2022-10-27 v1.5
% Info: Transparency via pdfTeX's color stack
%
% Copyright (C)
%    2007 Heiko Oberdiek
%    2016-2022 Oberdiek Package Support Group 2018
%    https://github.com/ho-tex/transparent/issues
%
% This work may be distributed and/or modified under the
% conditions of the LaTeX Project Public License, either
% version 1.3c of this license or (at your option) any later
% version. This version of this license is in
%    https://www.latex-project.org/lppl/lppl-1-3c.txt
% and the latest version of this license is in
%    https://www.latex-project.org/lppl.txt
% and version 1.3 or later is part of all distributions of
% LaTeX version 2005/12/01 or later.
%
% This work has the LPPL maintenance status "maintained".
%
% The Current Maintainers of this work are
% Heiko Oberdiek and the Oberdiek Package Support Group
% https://github.com/ho-tex/transparent/issues
%
% This work consists of the main source file transparent.dtx
% and the derived files
%    transparent.sty, transparent-nometadata.sty, transparent.pdf, 
%    transparent.ins, transparent.drv, transparent-example.tex.
%
% Distribution:
%    CTAN:macros/latex/contrib/transparent/transparent.dtx
%    CTAN:macros/latex/contrib/transparent/transparent.pdf
%
% Unpacking:
%    (a) If transparent.ins is present:
%           tex transparent.ins
%    (b) Without transparent.ins:
%           tex transparent.dtx
%    (c) If you insist on using LaTeX
%           latex \let\install=y\input{transparent.dtx}
%        (quote the arguments according to the demands of your shell)
%
% Documentation:
%    (a) If transparent.drv is present:
%           latex transparent.drv
%    (b) Without transparent.drv:
%           latex transparent.dtx; ...
%    The class ltxdoc loads the configuration file ltxdoc.cfg
%    if available. Here you can specify further options, e.g.
%    use A4 as paper format:
%       \PassOptionsToClass{a4paper}{article}
%
%    Programm calls to get the documentation (example):
%       pdflatex transparent.dtx
%       makeindex -s gind.ist transparent.idx
%       pdflatex transparent.dtx
%       makeindex -s gind.ist transparent.idx
%       pdflatex transparent.dtx
%
% Installation:
%    TDS:tex/latex/transparent/transparent.sty
%    TDS:tex/latex/transparent/transparent-nometadata.sty
%    TDS:doc/latex/transparent/transparent.pdf
%    TDS:source/latex/transparent/transparent.dtx
%
%<*ignore>
\begingroup
  \catcode123=1 %
  \catcode125=2 %
  \def\x{LaTeX2e}%
\expandafter\endgroup
\ifcase 0\ifx\install y1\fi\expandafter
         \ifx\csname processbatchFile\endcsname\relax\else1\fi
         \ifx\fmtname\x\else 1\fi\relax
\else\csname fi\endcsname
%</ignore>
%<*install>
\input docstrip.tex
\Msg{************************************************************************}
\Msg{* Installation}
\Msg{* Package: transparent 2022-10-27 v1.5 Transparency via pdfTeX's color stack (HO)}
\Msg{************************************************************************}

\keepsilent
\askforoverwritefalse

\let\MetaPrefix\relax
\preamble

This is a generated file.

Project: transparent
Version: 2022-10-27 v1.5

Copyright (C)
   2007 Heiko Oberdiek
   2016-2022 Oberdiek Package Support Group

This work may be distributed and/or modified under the
conditions of the LaTeX Project Public License, either
version 1.3c of this license or (at your option) any later
version. This version of this license is in
   https://www.latex-project.org/lppl/lppl-1-3c.txt
and the latest version of this license is in
   https://www.latex-project.org/lppl.txt
and version 1.3 or later is part of all distributions of
LaTeX version 2005/12/01 or later.

This work has the LPPL maintenance status "maintained".

The Current Maintainers of this work are
Heiko Oberdiek and the Oberdiek Package Support Group
https://github.com/ho-tex/transparent/issues


This work consists of the main source file transparent.dtx
and the derived files
   transparent.sty, transparent-nometadata.sty, transparent.pdf,
   transparent.ins, transparent.drv, transparent-example.tex.

\endpreamble
\let\MetaPrefix\DoubleperCent

\generate{%
  \file{transparent.ins}{\from{transparent.dtx}{install}}%
  \file{transparent.drv}{\from{transparent.dtx}{driver}}%
  \usedir{tex/latex/transparent}%
  \file{transparent-nometadata.sty}{\from{transparent.dtx}{package}}%
  \file{transparent.sty}{\from{transparent.dtx}{package-new}}%
  \usedir{doc/latex/transparent}%
  \file{transparent-example.tex}{\from{transparent.dtx}{example}}%
}

\catcode32=13\relax% active space
\let =\space%
\Msg{************************************************************************}
\Msg{*}
\Msg{* To finish the installation you have to move the following}
\Msg{* file into a directory searched by TeX:}
\Msg{*}
\Msg{*     transparent.sty}
\Msg{*     transparent-nometadata.sty}
\Msg{*}
\Msg{* To produce the documentation run the file `transparent.drv'}
\Msg{* through LaTeX.}
\Msg{*}
\Msg{* Happy TeXing!}
\Msg{*}
\Msg{************************************************************************}

\endbatchfile
%</install>
%<*ignore>
\fi
%</ignore>
%<*driver>
\NeedsTeXFormat{LaTeX2e}
\ProvidesFile{transparent.drv}%
  [2022-10-27 v1.5 Transparency via pdfTeX's color stack (HO)]%
\documentclass{ltxdoc}
\usepackage{holtxdoc}[2011/11/22]
\begin{document}
  \DocInput{transparent.dtx}%
\end{document}
%</driver>
% \fi
%
%
%
% \GetFileInfo{transparent.drv}
%
% \title{The \xpackage{transparent} package}
% \date{2022-10-27 v1.5}
% \author{Heiko Oberdiek\thanks
% {Please report any issues at \url{https://github.com/ho-tex/transparent/issues}}}
%
% \maketitle
%
% \begin{abstract}
% \pdfTeX\ and lua\TeX{} supports several color stacks. This
% package shows, how a separate color stack can be used for transparency,
% a property besides color, that also works across page break.
% \end{abstract}
%
% \tableofcontents
%
% \section{User interface}
%
% The package \xpackage{transparent} defines \cs{transparent} and
% \cs{texttransparent}. They are used like \cs{color} and \cs{textcolor}.
% The first argument is the transparency value between 0 and 1 where 0
% is fully transparent and 1 is opaque.
% 
% \section{Engine support}
% If the PDF management is loaded by using \verb+\DocumentMetadata{}+ at the beginning
% of the document the package makes use of \texttt{l3opacity} and
% can be used with all engines supported by \texttt{l3backend}.
% But only with \pdfTeX\ and lua\TeX{} the transparency will also work across page break.
% 
% 
% Without the PDF management, the package works only with \pdfTeX\ and lua\TeX{}.
% It then uses the original code by Heiko Oberdiek. It then uses the
% primitive \cs{pdfpageresources} and can clash with other packages that also use
% \cs{pdfpageresources}.
% 
%
% \section{Example for usage}
%    \begin{macrocode}
%<*example>
\documentclass[12pt]{article}

\usepackage{color}
\usepackage{transparent}

\begin{document}
\colorbox{yellow}{%
  \bfseries
  \color{blue}%
  Blue and %
  \transparent{0.6}%
  transparent blue%
}

\bigskip
Hello World
\texttransparent{0.5}{Hello\newpage World}
Hello World
\end{document}
%</example>
%    \end{macrocode}
%
% \StopEventually{
% }
%
% \section{Implementation}
% \subsection{New version using the methods of the PDF management} 
% Identification
%    \begin{macrocode}
%<*package-new>
\NeedsTeXFormat{LaTeX2e}[2020/10/01]
\ProvidesExplPackage{transparent}{2022-10-27}{1.5}
  {Transparency with color stacks}%
%    \end{macrocode}
% Testing if the pdfmanagement is used
%    \begin{macrocode}
\@ifl@t@r\fmtversion{2022-06-01}
  {
    \IfDocumentMetadataTF {}{\RequirePackage{transparent-nometadata}}
    \IfDocumentMetadataTF {}{\endinput}
  }
  {\RequirePackage{transparent-nometadata}}  
%    \end{macrocode}
%    \begin{macrocode}
\RequirePackage{l3opacity}

\NewDocumentCommand{\transparent} { m }
      {
       \opacity_select:n{\fp_eval:n{ min(max(0,#1),1) } }
      }

\NewDocumentCommand{\texttransparent}{m m}
  {
   \mode_leave_vertical:
   \group_begin:
     \transparent{#1}
     #2
   \group_end:
  }
%</package-new>
%    \end{macrocode}
% \subsection{Old version without the PDF management}
%    \begin{macrocode}
%<*package>
\NeedsTeXFormat{LaTeX2e}
\ProvidesPackage{transparent-nometadata}%
  [2022-10-27 v1.5 Transparency via pdfTeX's color stack (HO)]%
%    \end{macrocode}
%
% \subsection{Initial checks}
%
% \subsubsection{Check for \pdfTeX\ in PDF mode}
%    \begin{macrocode}
\RequirePackage{iftex}
\ifpdf
\else
  \PackageWarningNoLine{transparent}{%
    Loading aborted, because pdfTeX is not running in PDF mode%
  }%
  \expandafter\endinput
\fi
%    \end{macrocode}
%
% \subsubsection{Check \pdfTeX\ or Lua\TeX\ version}
%    \begin{macrocode}
\ifx\pdfextension\@undefined
  \let\TRP@pdfcolorstackinit\pdfcolorstackinit
  \let\TRP@pdfpageresources\pdfpageresources
  \let\TRP@pdfcolorstack\pdfcolorstack
\else
  \def\TRP@pdfcolorstackinit           {\pdffeedback colorstackinit}
  \protected\edef\TRP@pdfpageresources {\pdfvariable pageresources}
  \protected\def\TRP@pdfcolorstack     {\pdfextension colorstack}
\fi
%    \end{macrocode}
%
%    \begin{macrocode}
\ifcsname TRP@pdfcolorstackinit\endcsname\else
  \PackageWarningNoLine{transparent}{%
    Your pdfTeX version does not support color stacks%
  }%
  \expandafter\endinput
\fi
%    \end{macrocode}
%
% \subsection{Compatibility with pgf}
%  https://github.com/ho-tex/transparent/issues/1
%    \begin{macrocode}
\AtBeginDocument
 {%
   \ifcsname pgfutil@addpdfresource@extgs\endcsname
    \let\TRP@addresource\relax
    \pgfutil@addpdfresource@extgs{\TRP@list}%
   \fi
 }
%    \end{macrocode}
% \subsection{Transparency}
%
%    The setting for the different transparency values must
%    be added to the page resources. In the first run the values
%    are recorded in the \xfile{.aux} file. In the second run
%    the values are set and transparency is available.
%    \begin{macrocode}
\RequirePackage{auxhook}
\AddLineBeginAux{%
  \string\providecommand{\string\transparent@use}[1]{}%
}
\gdef\TRP@list{/TRP1<</ca 1/CA 1>>}
\def\transparent@use#1{%
  \@ifundefined{TRP#1}{%
    \g@addto@macro\TRP@list{%
      /TRP#1<</ca #1/CA #1>>%
    }%
    \expandafter\gdef\csname TRP#1\endcsname{/TRP#1 gs}%
  }{%
    % #1 is already known, nothing to do
  }%
}
\AtBeginDocument{%
  \TRP@addresource
  \let\transparent@use\@gobble
}
%    \end{macrocode}
%    Unhappily the interface setting page resources is very
%    poor, only a token register \cs{pdfpageresources}.
%    Thus this package tries to be cooperative in the way that
%    it embeds the previous contents of \cs{pdfpageresources}.
%    However it does not solve the problem, if several packages
%    want to set |/ExtGState|.
%    \begin{macrocode}
\def\TRP@addresource{%
  \begingroup
    \edef\x{\endgroup
      \TRP@pdfpageresources{%
        \the\TRP@pdfpageresources
        /ExtGState<<\TRP@list>>%
      }%
    }%
  \x
}
\newif\ifTRP@rerun
\xdef\TRP@colorstack{%
  \TRP@pdfcolorstackinit page direct{/TRP1 gs}%
}
%    \end{macrocode}
%    \begin{macro}{\transparent}
%    \begin{macrocode}
\newcommand*{\transparent}[1]{%
  \begingroup
    \dimen@=#1\p@\relax
    \ifdim\dimen@>\p@
      \dimen@=\p@
    \fi
    \ifdim\dimen@<\z@
      \dimen@=\z@
    \fi
    \ifdim\dimen@=\p@
      \def\x{1}%
    \else
      \ifdim\dimen@=\z@
        \def\x{0}%
      \else
        \edef\x{\strip@pt\dimen@}%
        \edef\x{\expandafter\@gobble\x}%
      \fi
    \fi
    \if@filesw
      \immediate\write\@auxout{%
        \string\transparent@use{\x}%
      }%
    \fi
    \edef\x{\endgroup
      \def\noexpand\transparent@current{\x}%
    }%
  \x
  \transparent@set
}
%    \end{macrocode}
%    \end{macro}
%    \begin{macrocode}
\AtEndDocument{%
  \ifTRP@rerun
    \PackageWarningNoLine{transparent}{%
      Rerun to get transparencies right%
    }%
  \fi
}
\def\transparent@current{/TRP1 gs}
\def\transparent@set{%
  \@ifundefined{TRP\transparent@current}{%
    \global\TRP@reruntrue
  }{%
    \TRP@pdfcolorstack\TRP@colorstack push{%
      \csname TRP\transparent@current\endcsname
    }%
    \aftergroup\transparent@reset
  }%
}
\def\transparent@reset{%
  \TRP@pdfcolorstack\TRP@colorstack pop\relax
}
%    \end{macrocode}
%    \begin{macro}{\texttransparent}
%    \begin{macrocode}
\newcommand*{\texttransparent}[2]{%
  \protect\leavevmode
  \begingroup
    \transparent{#1}%
    #2%
  \endgroup
}
%    \end{macrocode}
%    \end{macro}
%
%    \begin{macrocode}
%</package>
%    \end{macrocode}
%
% \section{Installation}
%
% \subsection{Download}
%
% \paragraph{Package.} This package is available on
% CTAN\footnote{\CTANpkg{transparent}}:
% \begin{description}
% \item[\CTAN{macros/latex/contrib/transparent/transparent.dtx}] The source file.
% \item[\CTAN{macros/latex/contrib/transparent/transparent.pdf}] Documentation.
% \end{description}
%
%
%
% \subsection{Package installation}
%
% \paragraph{Unpacking.} The \xfile{.dtx} file is a self-extracting
% \docstrip\ archive. The files are extracted by running the
% \xfile{.dtx} through \plainTeX:
% \begin{quote}
%   \verb|tex transparent.dtx|
% \end{quote}
%
% \paragraph{TDS.} Now the different files must be moved into
% the different directories in your installation TDS tree
% (also known as \xfile{texmf} tree):
% \begin{quote}
% \def\t{^^A
% \begin{tabular}{@{}>{\ttfamily}l@{ $\rightarrow$ }>{\ttfamily}l@{}}
%   transparent.sty & tex/latex/transparent/transparent.sty\\
%   transparent-nometadata.sty & tex/latex/transparent/transparent-nometadata.sty\\   
%   transparent.pdf & doc/latex/transparent/transparent.pdf\\
%   transparent-example.tex & doc/latex/transparent/transparent-example.tex\\
%   transparent.dtx & source/latex/transparent/transparent.dtx\\
% \end{tabular}^^A
% }^^A
% \sbox0{\t}^^A
% \ifdim\wd0>\linewidth
%   \begingroup
%     \advance\linewidth by\leftmargin
%     \advance\linewidth by\rightmargin
%   \edef\x{\endgroup
%     \def\noexpand\lw{\the\linewidth}^^A
%   }\x
%   \def\lwbox{^^A
%     \leavevmode
%     \hbox to \linewidth{^^A
%       \kern-\leftmargin\relax
%       \hss
%       \usebox0
%       \hss
%       \kern-\rightmargin\relax
%     }^^A
%   }^^A
%   \ifdim\wd0>\lw
%     \sbox0{\small\t}^^A
%     \ifdim\wd0>\linewidth
%       \ifdim\wd0>\lw
%         \sbox0{\footnotesize\t}^^A
%         \ifdim\wd0>\linewidth
%           \ifdim\wd0>\lw
%             \sbox0{\scriptsize\t}^^A
%             \ifdim\wd0>\linewidth
%               \ifdim\wd0>\lw
%                 \sbox0{\tiny\t}^^A
%                 \ifdim\wd0>\linewidth
%                   \lwbox
%                 \else
%                   \usebox0
%                 \fi
%               \else
%                 \lwbox
%               \fi
%             \else
%               \usebox0
%             \fi
%           \else
%             \lwbox
%           \fi
%         \else
%           \usebox0
%         \fi
%       \else
%         \lwbox
%       \fi
%     \else
%       \usebox0
%     \fi
%   \else
%     \lwbox
%   \fi
% \else
%   \usebox0
% \fi
% \end{quote}
% If you have a \xfile{docstrip.cfg} that configures and enables \docstrip's
% TDS installing feature, then some files can already be in the right
% place, see the documentation of \docstrip.
%
% \subsection{Refresh file name databases}
%
% If your \TeX~distribution
% (\TeX\,Live, \mikTeX, \dots) relies on file name databases, you must refresh
% these. For example, \TeX\,Live\ users run \verb|texhash| or
% \verb|mktexlsr|.
%
% \subsection{Some details for the interested}
%
% \paragraph{Unpacking with \LaTeX.}
% The \xfile{.dtx} chooses its action depending on the format:
% \begin{description}
% \item[\plainTeX:] Run \docstrip\ and extract the files.
% \item[\LaTeX:] Generate the documentation.
% \end{description}
% If you insist on using \LaTeX\ for \docstrip\ (really,
% \docstrip\ does not need \LaTeX), then inform the autodetect routine
% about your intention:
% \begin{quote}
%   \verb|latex \let\install=y\input{transparent.dtx}|
% \end{quote}
% Do not forget to quote the argument according to the demands
% of your shell.
%
% \paragraph{Generating the documentation.}
% You can use both the \xfile{.dtx} or the \xfile{.drv} to generate
% the documentation. The process can be configured by the
% configuration file \xfile{ltxdoc.cfg}. For instance, put this
% line into this file, if you want to have A4 as paper format:
% \begin{quote}
%   \verb|\PassOptionsToClass{a4paper}{article}|
% \end{quote}
% An example follows how to generate the
% documentation with pdf\LaTeX:
% \begin{quote}
%\begin{verbatim}
%pdflatex transparent.dtx
%makeindex -s gind.ist transparent.idx
%pdflatex transparent.dtx
%makeindex -s gind.ist transparent.idx
%pdflatex transparent.dtx
%\end{verbatim}
% \end{quote}
%
% \begin{History}
%   \begin{Version}{2007/01/08 v1.0}
%   \item
%     First version.
%   \end{Version}
%   \begin{Version}{2016/05/16 v1.1}
%   \item
%     Documentation updates.
%   \end{Version}
%   \begin{Version}{2018/09/10 v1.2}
%   \item
%     Update for LuaTeX, remove dependency on Pdf\TeX\ command names.
%   \end{Version}
%   \begin{Version}{2018/11/18 v1.3}
%   \item
%     Added code for pgf compatibility, see https://github.com/ho-tex/transparent/issues/1
%   \end{Version}
%   \begin{Version}{2019/11/29 v1.4}
%   \item
%     Documentation updates.
% \item
% Use \xpackage{iftex} package.
%   \end{Version}
%   \begin{Version}{2022-10-27 v1.5}
%   \item Made the package compatible with the PDF management.
%   \end{Version}
% \end{History}
%
% \PrintIndex
%
% \Finale
\endinput
|
% \end{quote}
% Do not forget to quote the argument according to the demands
% of your shell.
%
% \paragraph{Generating the documentation.}
% You can use both the \xfile{.dtx} or the \xfile{.drv} to generate
% the documentation. The process can be configured by the
% configuration file \xfile{ltxdoc.cfg}. For instance, put this
% line into this file, if you want to have A4 as paper format:
% \begin{quote}
%   \verb|\PassOptionsToClass{a4paper}{article}|
% \end{quote}
% An example follows how to generate the
% documentation with pdf\LaTeX:
% \begin{quote}
%\begin{verbatim}
%pdflatex transparent.dtx
%makeindex -s gind.ist transparent.idx
%pdflatex transparent.dtx
%makeindex -s gind.ist transparent.idx
%pdflatex transparent.dtx
%\end{verbatim}
% \end{quote}
%
% \begin{History}
%   \begin{Version}{2007/01/08 v1.0}
%   \item
%     First version.
%   \end{Version}
%   \begin{Version}{2016/05/16 v1.1}
%   \item
%     Documentation updates.
%   \end{Version}
%   \begin{Version}{2018/09/10 v1.2}
%   \item
%     Update for LuaTeX, remove dependency on Pdf\TeX\ command names.
%   \end{Version}
%   \begin{Version}{2018/11/18 v1.3}
%   \item
%     Added code for pgf compatibility, see https://github.com/ho-tex/transparent/issues/1
%   \end{Version}
%   \begin{Version}{2019/11/29 v1.4}
%   \item
%     Documentation updates.
% \item
% Use \xpackage{iftex} package.
%   \end{Version}
%   \begin{Version}{2022-10-27 v1.5}
%   \item Made the package compatible with the PDF management.
%   \end{Version}
% \end{History}
%
% \PrintIndex
%
% \Finale
\endinput
|
% \end{quote}
% Do not forget to quote the argument according to the demands
% of your shell.
%
% \paragraph{Generating the documentation.}
% You can use both the \xfile{.dtx} or the \xfile{.drv} to generate
% the documentation. The process can be configured by the
% configuration file \xfile{ltxdoc.cfg}. For instance, put this
% line into this file, if you want to have A4 as paper format:
% \begin{quote}
%   \verb|\PassOptionsToClass{a4paper}{article}|
% \end{quote}
% An example follows how to generate the
% documentation with pdf\LaTeX:
% \begin{quote}
%\begin{verbatim}
%pdflatex transparent.dtx
%makeindex -s gind.ist transparent.idx
%pdflatex transparent.dtx
%makeindex -s gind.ist transparent.idx
%pdflatex transparent.dtx
%\end{verbatim}
% \end{quote}
%
% \begin{History}
%   \begin{Version}{2007/01/08 v1.0}
%   \item
%     First version.
%   \end{Version}
%   \begin{Version}{2016/05/16 v1.1}
%   \item
%     Documentation updates.
%   \end{Version}
%   \begin{Version}{2018/09/10 v1.2}
%   \item
%     Update for LuaTeX, remove dependency on Pdf\TeX\ command names.
%   \end{Version}
%   \begin{Version}{2018/11/18 v1.3}
%   \item
%     Added code for pgf compatibility, see https://github.com/ho-tex/transparent/issues/1
%   \end{Version}
%   \begin{Version}{2019/11/29 v1.4}
%   \item
%     Documentation updates.
% \item
% Use \xpackage{iftex} package.
%   \end{Version}
%   \begin{Version}{2022-10-27 v1.5}
%   \item Made the package compatible with the PDF management.
%   \end{Version}
% \end{History}
%
% \PrintIndex
%
% \Finale
\endinput
|
% \end{quote}
% Do not forget to quote the argument according to the demands
% of your shell.
%
% \paragraph{Generating the documentation.}
% You can use both the \xfile{.dtx} or the \xfile{.drv} to generate
% the documentation. The process can be configured by the
% configuration file \xfile{ltxdoc.cfg}. For instance, put this
% line into this file, if you want to have A4 as paper format:
% \begin{quote}
%   \verb|\PassOptionsToClass{a4paper}{article}|
% \end{quote}
% An example follows how to generate the
% documentation with pdf\LaTeX:
% \begin{quote}
%\begin{verbatim}
%pdflatex transparent.dtx
%makeindex -s gind.ist transparent.idx
%pdflatex transparent.dtx
%makeindex -s gind.ist transparent.idx
%pdflatex transparent.dtx
%\end{verbatim}
% \end{quote}
%
% \begin{History}
%   \begin{Version}{2007/01/08 v1.0}
%   \item
%     First version.
%   \end{Version}
%   \begin{Version}{2016/05/16 v1.1}
%   \item
%     Documentation updates.
%   \end{Version}
%   \begin{Version}{2018/09/10 v1.2}
%   \item
%     Update for LuaTeX, remove dependency on Pdf\TeX\ command names.
%   \end{Version}
%   \begin{Version}{2018/11/18 v1.3}
%   \item
%     Added code for pgf compatibility, see https://github.com/ho-tex/transparent/issues/1
%   \end{Version}
%   \begin{Version}{2019/11/29 v1.4}
%   \item
%     Documentation updates.
% \item
% Use \xpackage{iftex} package.
%   \end{Version}
%   \begin{Version}{2022-10-27 v1.5}
%   \item Made the package compatible with the PDF management.
%   \end{Version}
% \end{History}
%
% \PrintIndex
%
% \Finale
\endinput
