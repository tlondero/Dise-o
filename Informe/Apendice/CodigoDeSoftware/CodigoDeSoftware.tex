\Subsubsection{BH1750}
\begin{lstlisting}[language=Python]
import board
import adafruit_bh1750
import Clases.Database_Lum

from RTC import getTime

import time

i2c = board.I2C()
sensor = adafruit_bh1750.BH1750(i2c)

db_Lum = Clases.Database_Lum.Database_Lum(using_file_db=True,
	foldername='db')
db_Lum.insert_lum_meas(sensor.lux,getTime())
\end{lstlisting}
\Subsubsection{DHT22}

\vspace{1mm}
\noindent{\begin{minipage}{\linewidth}
\begin{lstlisting}[language=Python]
import Adafruit_DHT
import Clases.Database_TyH
from RTC import getTime

DHT_SENSOR = Adafruit_DHT.DHT22
DHT_PIN = 4
TEMPERATURE_OUTLIER_RANGE = 50
HUMIDITY_OUTLIER = 101

waiting_measurement = True
while waiting_measurement:
    humidity, temperature = Adafruit_DHT.read_retry(DHT_SENSOR, DHT_PIN)

    if humidity is not None and temperature is not None:
        waiting_measurement = False
    else:
        print("Failed to retrieve data from humidity sensor")

db = os.getenv('SIMFS_DATA_DB_PATH', 'db')

print("Temp={0:0.1f}*C  Humidity={1:0.1f}%".format(temperature, humidity))

tempOutlier = temperature < TEMPERATURE_OUTLIER_RANGE
humOutlier = humidity < HUMIDITY_OUTLIER

if( tempOutlier and humOutlier ):
    db_TyH = Clases.Database_TyH.Database_TyH(using_file_db=True,
    	foldername='db')
    db_TyH.insert_tyh_meas(temperature, humidity,getTime())
\end{lstlisting}
\end{minipage}}

\Subsubsection{getRTMeasurement}

\begin{lstlisting}[language=Python]
import Adafruit_DHT
import board
import adafruit_bh1750
import Clases.Database_Lum
import Clases.Database_TyH
import json
import os

DHT_SENSOR = Adafruit_DHT.DHT22(4)
DHT_PIN = 4
TEMPERATURE_OUTLIER_RANGE = 50
HUMIDITY_OUTLIER = 101

i2c = board.I2C()
sensor = adafruit_bh1750.BH1750(i2c)

waiting_measurement=True

while waiting_measurement:
    humidity, temperature = Adafruit_DHT.read_retry(DHT_SENSOR, DHT_PIN)

    if humidity is not None and temperature is not None:
        waiting_measurement = False
    else:
        print("Failed to retrieve data from humidity sensor")
        waiting_measurement = False
        humidity = 999
        temperature = 999

lum = round(sensor.lux,1)
temp = round(temperature,1)
hum = round(humidity,1)
print("Luminosity = {0:0.1f} lux".format(lum))

tempOutlier = temperature > TEMPERATURE_OUTLIER_RANGE
humOutlier = humidity > HUMIDITY_OUTLIER

if( tempOutlier and humOutlier ):
    temp = 999
    hum = 999
    
print("Temp={0:0.1f}*C  Humidity={1:0.1f}%".format(temperature, humidity))
print("lum={0:0.1f} lux".format(lum))

data = {
    "lum": lum,
    "temp": temp,
    "hum": hum
}

json_data = json.dumps(data)
with open("../../LastData/last_data.json", "w") as outfile:
  outfile.write(json_data)
\end{lstlisting}
\Subsubsection{getTime}

\begin{lstlisting}[language=Python]
from RTC import getTime

print(getTime().strip())
\end{lstlisting}

\Subsubsection{RTC}

\vspace{1mm}
\noindent{\begin{minipage}{\linewidth}
\begin{lstlisting}[language=Python]
import board
import adafruit_ds1307
import time 

def getTime():
   i2c = board.I2C()
   rtc = adafruit_ds1307.DS1307(i2c)
   t = rtc.datetime

   day = t.tm_mday
   if day < 10:
      day = '0' + str(day)

   mon = t.tm_mon
   if mon < 10:
      mon = '0' + str(mon)

   year = t.tm_year

   hour = t.tm_hour
   if hour < 10:
      hour = '0' + str(hour)

   min = t.tm_min
   if min < 10:
      min = '0' + str(min)

   sec = t.tm_sec
   if sec < 10:
      sec = '0' + str(sec)

   return f"{year}-{mon}-{day}_{hour}:{min}:{sec}"
   
def setTime(year,month,day,hour,minute):
   i2c = board.I2C()
   rtc = adafruit_ds1307.DS1307(i2c)
   rtc.datetime = time.struct_time((year,month,day,hour,minute,0,0,9,-1))
\end{lstlisting}
\end{minipage}}

\Subsubsection{DatabaseLum}

\begin{lstlisting}[language=Python]
import sqlite3
import os

class Database_Lum:
    """Luminosity database class for sqlite3"""

    def __init__(self, db_dir: str, using_file_db=True) -> None:
        if using_file_db is True:
            self.foldername = db_dir
            self.usingfolder = True
            if(os.path.isdir(db_dir) == 0):
                os.makedirs(db_dir)
            self.conn = sqlite3.connect(db_dir + '\Lum.db')
        else:
            self.usingfolder = False
            self.conn = sqlite3.connect(':memory:')
            
        self.c = self.conn.cursor()
        self.c.execute(
            ''' SELECT count(name) FROM sqlite_master WHERE type='table'
            	AND name='measurements' ''')
        if self.c.fetchone()[0] == 0:
            {                
                self.c.execute("""CREATE TABLE measurements ( 
                        date_meas TEXT, lum_meas REAL)""")
            }

    def insert_lum_meas(self, lum_,date) -> None:
        if(self.usingfolder):
            with self.conn:
                self.c.execute("INSERT INTO measurements VALUES
                	(:date_, :luminosity_meas)", 
                {'luminosity_meas': lum_, 'date_': date})

    def get_meas_in_range(self, start_time: str, end_time: str):
        if (self.usingfolder):
            self.c.execute("SELECT * FROM measurements WHERE date_meas
            	BETWEEN (:start_) AND (:end_)",
                           {'start_': start_time, 'end_': end_time})
            return self.c.fetchall()

    def remove_meas_in_range(self, start_time: str, end_time: str):
        with self.conn:
            self.c.execute("DELETE FROM measurements WHERE date_meas
            	BETWEEN (:start_) AND (:end_)", {
                           'start_': start_time, 'end_': end_time})
    def get_all_meas(self):
        if (self.usingfolder):
            self.c.execute("SELECT * FROM measurements "
                          )
            return self.c.fetchall()
            
    def __del__(self):
        self.conn.close()
\end{lstlisting}

\Subsubsection{DatabaseTyH}

\begin{lstlisting}[language=Python]
import sqlite3
import os

class Database_TyH:
    """Temperature and Humidity database class for sqlite3"""

    def __init__(self, db_dir: str, using_file_db=True) -> None:
        if using_file_db is True:
            self.foldername = db_dir
            self.usingfolder = True
            if(os.path.isdir(db_dir) == 0):
                os.makedirs(db_dir)
            self.conn_temp_n_hum = sqlite3.connect(
                db_dir + 'Temp_and_hum.db')
        else:
            self.usingfolder = False
            self.conn_temp_n_hum = sqlite3.connect(':memory:')

        self.ctyh = self.conn_temp_n_hum.cursor()
        self.ctyh.execute(
            ''' SELECT count(name) FROM sqlite_master WHERE type='table'
            	AND name='measurements' ''')
        if self.ctyh.fetchone()[0] == 0:
            {
                self.ctyh.execute("""CREATE TABLE measurements (
                        date_meas TEXT, temperature_meas REAL,
                        humidity_meas REAL )""")
            }

    def insert_tyh_meas(self, temp_, hum_,date) -> None:
        if(self.usingfolder):

            with self.conn_temp_n_hum:
                self.ctyh.execute("INSERT INTO measurements VALUES
                	(:date_, :temperature_meas, :humidity_meas)", {
                	'temperature_meas': temp_, 'humidity_meas': hum_,
                	'date_': date})

    def get_meas_in_range(self, start_time, end_time):
        if (self.usingfolder):
            self.ctyh.execute("SELECT * FROM measurements WHERE
            	date_meas BETWEEN (:start_) AND (:end_)", {
            		'start_': start_time, 'end_': end_time})
            return self.ctyh.fetchall()

    def get_all_meas(self):
        if (self.usingfolder):
            self.ctyh.execute("SELECT * FROM measurements")
            return self.ctyh.fetchall()

    def remove_meas_in_range(self, start_time: str, end_time: str):
        with self.conn_temp_n_hum:
            self.ctyh.execute("DELETE FROM measurements WHERE
            	date_meas BETWEEN (:start_) AND (:end_)", {
                              'start_': start_time, 'end_': end_time})

    def __del__(self):
        self.conn_temp_n_hum.close()
\end{lstlisting}

\Subsubsection{FileMakerDB}

\vspace{1mm}
\noindent{\begin{minipage}{\linewidth}
\begin{lstlisting}[language=Python]
import Database_Lum
import Database_TyH
import sys
import csv

def write_TyH(start_time: str, end_time: str):
    db_TyH = Database_TyH.Database_TyH(using_file_db=True, db_dir='../db')

    data = db_TyH.get_meas_in_range(start_time, end_time)
    csv_name = start_time[0:10] + '_' + end_time[0:10]
    db_export_path = "/home/pi/simfs_export_data/CSV/TemperaturaYHumedad_"
    filename = db_export_path + csv_name + '.csv'
    with open(filename, 'w') as f:
        writer = csv.writer(f, lineterminator='\n')
        writer.writerow(['Time',
                         'Temperature', 'Humidity'])
        for tup in data:
            writer.writerow(tup)

def write_Lum(start_time: str, end_time: str):
    db_Lum = Database_Lum.Database_Lum(using_file_db=True, db_dir='../db')

    data = db_Lum.get_meas_in_range(start_time, end_time)
    csv_name = '../CSV/Luminosidad_' + start_time[0:10]
    	+ '_' + end_time[0:10]
    db_export_path = "/home/pi/simfs_export_data/CSV/TemperaturaYHumedad_"
    filename = db_export_path + csv_name + '.csv'
    with open(filename, 'w') as f:
        writer = csv.writer(f, lineterminator='\n')
        writer.writerow(['Time',
                         'Lux'])
        for tup in data:
            writer.writerow(tup)

if __name__ == "__main__":
    argv = sys.argv[1:]
    if(len(argv) != 3):
        print('Wrong amount of parameters!')
    else:
        if(argv[0].lower() == '-l'):
            if((len(argv[1]) != 16) or (len(argv[2]) != 16)):
                print('Bad date formating!')
            else:
                write_Lum(argv[1], argv[2])
                print('Writing CSV')
        elif(argv[0].lower() == '-tyh'):
            if((len(argv[1]) != 16) or (len(argv[2]) != 16)):
                print('Bad date formating!')
            else:
                print('Writing CSV')
                write_TyH(argv[1], argv[2])
        else:
            print('Wrong key word!')
\end{lstlisting}
\end{minipage}}

\Subsubsection{makeCsvCli}

\vspace{1mm}
\noindent{\begin{minipage}{\linewidth}
\begin{lstlisting}[language=Python]
import argparse
from enum import Enum
from File_maker_DB import write_Lum, write_TyH

class MeasDevices(Enum):
    TEMP_AND_HUM_DEV = 1
    LUX_DEV = 2

if __name__ == '__main__':
    parser = argparse.ArgumentParser(description='Generate csv data')
    parser.add_argument('--start_time', dest='start_time',
                        help='csv start_time', type=str)
    parser.add_argument('--end_time', dest='end_time',
                        help='csv end_time', type=str)

    parser.add_argument('--meas_type', type=int,
                        help="""Select information source
                                1: Temperature and Humidity
                                2: Luminosity""")

    args = parser.parse_args()
    start_time = args.start_time
    end_time = args.end_time
    print("Generating CSV ")
    print(f"from: {start_time}")
    print(f"to: {end_time}")
    print(f"for device {MeasDevices.TEMP_AND_HUM_DEV}")
    try:
        if args.meas_type == MeasDevices.TEMP_AND_HUM_DEV.value:
            print("Doing temp")
            write_TyH(start_time, end_time)
        elif args.meas_type == MeasDevices.LUX_DEV.value:
            print("Doing light")
            print("start_time", start_time)
            print("end_time", end_time)
            write_Lum(start_time, end_time)
    except Exception as e:
        print("Generating CSV failed")
        print(e)
\end{lstlisting}
\end{minipage}}

\Subsubsection{presenceDetector}

\vspace{1mm}
\noindent{\begin{minipage}{\linewidth}
\begin{lstlisting}[language=Python]
from bluepy.btle import Scanner
import numpy as np
import os
RETRIES = 3

def resetBL():
    os.system("sudo hciconfig hci0 down && sudo hciconfig hci0 up")

def IsInNest(mac, timeout, minDistance = -50):
    avg = 0
    scan_time = 0.1
    iterations = int(timeout/0.1)
    for i in range(iterations):
        try:
            ble_list = Scanner().scan(0.1)
            for dev in ble_list:
                if mac.lower() == dev.addr:
                    avg = avg + 0.2*(dev.rssi - avg) 
                    	.format(dev.rssi,dev.addr, dev['scanData'][8]))
        except:
            continue
    if 0.0 > avg >= minDistance:
        return True, avg
    else:
        return False, avg

if __name__ == "__main__":
    devices = []
    mac = "8C:AA:B5:87:6F:12"
    isBirdInNest = False
    secondsToScan = 5
    for attempt in range(RETRIES):
        try:
            isBirdInNest, value = IsInNest(mac, secondsToScan)
            if value == 0:
                resetBL()
                continue
            else:
                break
        except BTLEManagementError:
           resetBL()
        except KeyboardInterrupt:
            Scanner().stop()
            raise SystemExit
        except:
            raise Exception("Error occured")
    
    print(isBirdInNest)
\end{lstlisting}
\end{minipage}}

\Subsubsection{printMeasures}

\vspace{1mm}
\noindent{\begin{minipage}{\linewidth}
\begin{lstlisting}[language=Python]
import Database_TyH
import Database_Lum

db_TyH = Database_TyH.Database_TyH(using_file = True,foldername = '../db')
data = db_TyH.get_all_meas();
print("========Temperature and Humidity measures========")
print(data)
db_Lum = Database_Lum.Database_Lum(using_file = True,foldername = '../db')
data = db_Lum.get_all_meas();
print("========Luminosity measures========")
print(data)
db_Pres = Database_Pres.Database_Pres(using_file = True,
	foldername = '../db')
data = db_Pres.get_all_meas();
print("========Presence measures========")
print(data)
\end{lstlisting}
\end{minipage}}