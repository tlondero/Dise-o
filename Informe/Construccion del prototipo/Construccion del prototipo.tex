\documentclass[a4paper]{article}
\usepackage[utf8]{inputenc}
\usepackage[spanish, es-tabla, es-noshorthands]{babel}
\usepackage[table,xcdraw]{xcolor}
\usepackage[a4paper, footnotesep=1.25cm, headheight=1.25cm, top=2.54cm, left=2.54cm, bottom=2.54cm, right=2.54cm]{geometry}
%\geometry{showframe}
%VERIFICAR EL HEAD Y EL FOOT EN
%https://ctan.dcc.uchile.cl/macros/latex/contrib/geometry/geometry.pdf

%\usepackage{lipsum}			%LOREM IPSUM

%\usepackage{wrapfig}		%Wrap figure in text
\usepackage[export]{adjustbox}	%Move images
\usepackage{changepage}			%Move tables

%Font
\usepackage{anyfontsize}	%Font size
% #1 = size, #2 = text
\newcommand{\setparagraphsize}[2]{{\fontsize{#1}{6}\selectfont#2 \par}}		%Cambia el size de todo el parrafo
\newcommand{\setlinesize}[2]{{\fontsize{#1}{6}\selectfont#2}}				%Cambia el font de una oración


%FONTS (IMPORTANTE): Compilar en XeLaTex o LuaLaTeX
\usepackage{fontspec}
%Si sigue sin andar comentar \usepackage[utf8]{inputenc}
%https://ctan.dcc.uchile.cl/macros/unicodetex/latex/fontspec/fontspec.pdf
%https://www.overleaf.com/learn/latex/XeLaTeX

\usepackage{tikz}
\usepackage{amsmath}
\usepackage{amsfonts}
\usepackage{amssymb}
\usepackage{float}
\usepackage{graphicx}
\usepackage{caption}
\usepackage{subcaption}
\usepackage{multicol}
\usepackage{multirow}
\setlength{\doublerulesep}{\arrayrulewidth}
\usepackage{booktabs}

\usepackage{hyperref}
\hypersetup{
    colorlinks=true,
    linkcolor=black,
    filecolor=magenta,      
    urlcolor=blue,
    citecolor=blue,    
}

\newcommand{\captionsection}{\setcounter{figure}{0} \renewcommand{\thetable}{\arabic{section}.\arabic{table}} \renewcommand{\thefigure}{\arabic{section}.\arabic{figure}}}

\newcommand{\captionsubsection}{\setcounter{figure}{0} \renewcommand{\thetable}{\arabic{section}.\arabic{subsection}.\arabic{table}} \renewcommand{\thefigure}{\arabic{section}.\arabic{subsection}.\arabic{figure}}}

\newcommand{\captionsubsubsection}{\setcounter{figure}{0} \renewcommand{\thetable}{\arabic{section}.\arabic{subsection}.\arabic{subsubsection}.\arabic{table}} \renewcommand{\thefigure}{\arabic{section}.\arabic{subsection}.\arabic{subsubsection}.\arabic{figure}}}

%PICTURES AND TABLE INDEX:
\newcommand{\Section}[1]{ \section{#1}\captionsection }
\newcommand{\Subsection}[1]{ \subsection{#1}\captionsubsection }
\newcommand{\Subsubsection}[1]{ \subsubsection{#1}\captionsubsubsection }

%NOTAS GRANDES
\newcommand{\note}[1]{
	\begin{center}
		\huge{ \textcolor{red}{#1} }
	\end{center}
}
%Notas pequeñas
\newcommand{\lnote}[1]{{\fontsize{20}{6}\selectfont\textcolor{green}{#1}}}

\newcommand{\quotes}[1]{``#1''}
\usepackage{array}
\newcolumntype{C}[1]{>{\centering\let\newline\\\arraybackslash\hspace{0pt}}m{#1}}
\usepackage[american]{circuitikz}
\usetikzlibrary{calc}
\usepackage{fancyhdr}
\usepackage{units} 

\graphicspath{{../Prefacio/}{../Acronimos/}{../Introduccion/}{../Objetivos/}{../Definicion/}{../Plan de validacion/}{../Resumen/}}

%COLORES
\definecolor{AzulFoot}{rgb}{0.682,0.809,0.926}	%RGB	%{174,206,235}
\definecolor{AzulInfo}{rgb}{0.180,0.455,0.710}	%RGB	%{46,116,181}
\definecolor{AzulTable}{rgb}{0.302,0.507,0.871}	%RGB	%{68,114,196}
\definecolor{PName}{rgb}{0.353,0.353,0.353}	%RGB	%{90,90,90}

\usepackage{xcolor}
\usepackage{sectsty}
\chapterfont{\color{AzulInfo}}  % sets colour of chapters
\sectionfont{\color{AzulInfo}}  % sets colour of sections
\subsectionfont{\color{AzulInfo}}
\subsubsectionfont{\color{AzulInfo}}

%Header y footer
\usepackage{etoolbox}

\pagestyle{fancy}
\fancyhf{}
\rfoot{\thepage}
\renewcommand{\footrulewidth}{4pt}
\renewcommand{\headrulewidth}{0pt}
\patchcmd{\footrule}{\hrule}{\color{AzulFoot}\hrule}{}{}

\begin{document}

Un parametró fundamental a definir es el ancho que deben tener las paredes del nido prototipo, si bien el menor ancho que tienen la cavidad real está determinado por (referencia a el paper de la biologa). Si se calcula la profundidad de penetración de la onda electromagnética se obtiene que.
\begin{equation}
\delta \ = \ \frac{1}{\alpha} \ =\ \frac{1}{\sqrt{\pi f \mu \sigma}}  
\end{equation}
donde ademas se sabe que los valores de las constantes son:
\begin{itemize}
\item f = 915 MHz
\item $\mu$ = 1.25663760  $\mu \frac{ Hy}{m}$ (Permeabilidad magnética de la madera)
% Fuente
% https://www.engineeringtoolbox.com/permeability-d_1923.html
\item $\sigma$ $\approx$   1 m$\frac{S}{m}$ (Conductividad de la madera)
% Fuente
% https://www.thoughtco.com/table-of-electrical-resistivity-conductivity-608499

\end{itemize}
Computando se obtiene que $\delta$ = 526 mm.


Ahora se calcula el coeficiente de reflexi\'on de la onda electromagn\'etica.
Sabiendo que 
\begin{itemize}
\item $\varepsilon_{Madera}$ = 17.7 $10^{-12}$ $\frac{F}{m}$
% Fuente
% https://www.engineeringtoolbox.com/relative-permittivity-d_1660.html
\item $\varepsilon_{Aire}$ = 8.8595003 $10^{-12}$ $\frac{F}{m} $
\item $\mu_{Aire}$ = 1.25663753 $\mu \frac{ Hy}{m}$ 
\end{itemize}
y los valores en el vacio. Se obtiene que:
\begin{equation}
\eta_{Madera}=\sqrt{\frac{\mu_{Madera}}{\varepsilon_{Madera}}}
\end{equation}
El coeficiente de reflexi\'on se define como:
\begin{equation}
\Gamma = \frac{\eta_2 - \eta_1  }{\eta_2 + \eta_1} = -0.1714\footnote{El signo negativa implica una inversi\'on de fase en la onda electromagn\'etica.}
\end{equation}

Adem\'as la potencia de la onda reflejada ser\'a $P_{ref}=|\Gamma^2|$ = 0.029. Por lo que menos de un 3\%  de la potencia ser\'a reflejada. Esto es algo bueno debido a que no habr\'a una interferencia significativa con la transmisi\'on original.
\end{document}
