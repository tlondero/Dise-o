\Subsection{Prototipo Nido-Sustrato}
Para la construcción del prototipo del nido se tuvo en cuenta que el sustrato sea madera, y que las medidas coincidan con las medias tomadas del paper \cite{ref:varepsilon_madera}, adicionalmente se contempla un lugar donde apoyar la electrónica en el nido, y el hueco de entrada del carpintero.
El diseño del prototipo de nido, se puede observar en los planos especificados en (\ref{fig:base_nido_plano}), (\ref{fig:tapa_nido_plano}) y (\ref{fig:explotado_nido_plano}).

\Subsection{Prototipo Nido-Electrónica}

\Subsubsection{Raspberry Pi}
\TBC

\Subsubsection{Sensores}
\TBC

\Subsection{Prototipo Nido-Potencia}
Para el prototipo del sistema de potencia del nido, se cuenta con la batería \TBC y la fuente simuladora de paneles solares \TBC.

\Subsection{Prototipo Nido-Comunicaciones }
Existen dos comunicaciones en el prototipo, la encargada de Node-Red para comunicación con usuarios, y la comunicación con la base principal de seguimiento para recolección de datos del ave.
