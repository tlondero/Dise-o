%\documentclass[a4paper]{article}
\usepackage[utf8]{inputenc}
\usepackage[spanish, es-tabla, es-noshorthands]{babel}
\usepackage[table,xcdraw]{xcolor}
\usepackage[a4paper, footnotesep=1.25cm, headheight=1.25cm, top=2.54cm, left=2.54cm, bottom=2.54cm, right=2.54cm]{geometry}
%\geometry{showframe}
%VERIFICAR EL HEAD Y EL FOOT EN
%https://ctan.dcc.uchile.cl/macros/latex/contrib/geometry/geometry.pdf

%\usepackage{lipsum}			%LOREM IPSUM

%\usepackage{wrapfig}		%Wrap figure in text
\usepackage[export]{adjustbox}	%Move images
\usepackage{changepage}			%Move tables

%Font
\usepackage{anyfontsize}	%Font size
% #1 = size, #2 = text
\newcommand{\setparagraphsize}[2]{{\fontsize{#1}{6}\selectfont#2 \par}}		%Cambia el size de todo el parrafo
\newcommand{\setlinesize}[2]{{\fontsize{#1}{6}\selectfont#2}}				%Cambia el font de una oración


%FONTS (IMPORTANTE): Compilar en XeLaTex o LuaLaTeX
\usepackage{fontspec}
%Si sigue sin andar comentar \usepackage[utf8]{inputenc}
%https://ctan.dcc.uchile.cl/macros/unicodetex/latex/fontspec/fontspec.pdf
%https://www.overleaf.com/learn/latex/XeLaTeX

\usepackage{tikz}
\usepackage{amsmath}
\usepackage{amsfonts}
\usepackage{amssymb}
\usepackage{float}
\usepackage{graphicx}
\usepackage{caption}
\usepackage{subcaption}
\usepackage{multicol}
\usepackage{multirow}
\setlength{\doublerulesep}{\arrayrulewidth}
\usepackage{booktabs}

\usepackage{hyperref}
\hypersetup{
    colorlinks=true,
    linkcolor=black,
    filecolor=magenta,      
    urlcolor=blue,
    citecolor=blue,    
}

\newcommand{\captionsection}{\setcounter{figure}{0} \renewcommand{\thetable}{\arabic{section}.\arabic{table}} \renewcommand{\thefigure}{\arabic{section}.\arabic{figure}}}

\newcommand{\captionsubsection}{\setcounter{figure}{0} \renewcommand{\thetable}{\arabic{section}.\arabic{subsection}.\arabic{table}} \renewcommand{\thefigure}{\arabic{section}.\arabic{subsection}.\arabic{figure}}}

\newcommand{\captionsubsubsection}{\setcounter{figure}{0} \renewcommand{\thetable}{\arabic{section}.\arabic{subsection}.\arabic{subsubsection}.\arabic{table}} \renewcommand{\thefigure}{\arabic{section}.\arabic{subsection}.\arabic{subsubsection}.\arabic{figure}}}

%PICTURES AND TABLE INDEX:
\newcommand{\Section}[1]{ \section{#1}\captionsection }
\newcommand{\Subsection}[1]{ \subsection{#1}\captionsubsection }
\newcommand{\Subsubsection}[1]{ \subsubsection{#1}\captionsubsubsection }

%NOTAS GRANDES
\newcommand{\note}[1]{
	\begin{center}
		\huge{ \textcolor{red}{#1} }
	\end{center}
}
%Notas pequeñas
\newcommand{\lnote}[1]{{\fontsize{20}{6}\selectfont\textcolor{green}{#1}}}

\newcommand{\quotes}[1]{``#1''}
\usepackage{array}
\newcolumntype{C}[1]{>{\centering\let\newline\\\arraybackslash\hspace{0pt}}m{#1}}
\usepackage[american]{circuitikz}
\usetikzlibrary{calc}
\usepackage{fancyhdr}
\usepackage{units} 

\graphicspath{{../Prefacio/}{../Acronimos/}{../Introduccion/}{../Objetivos/}{../Definicion/}{../Plan de validacion/}{../Resumen/}}

%COLORES
\definecolor{AzulFoot}{rgb}{0.682,0.809,0.926}	%RGB	%{174,206,235}
\definecolor{AzulInfo}{rgb}{0.180,0.455,0.710}	%RGB	%{46,116,181}
\definecolor{AzulTable}{rgb}{0.302,0.507,0.871}	%RGB	%{68,114,196}
\definecolor{PName}{rgb}{0.353,0.353,0.353}	%RGB	%{90,90,90}

\usepackage{xcolor}
\usepackage{sectsty}
\chapterfont{\color{AzulInfo}}  % sets colour of chapters
\sectionfont{\color{AzulInfo}}  % sets colour of sections
\subsectionfont{\color{AzulInfo}}
\subsubsectionfont{\color{AzulInfo}}

%Header y footer
\usepackage{etoolbox}

\pagestyle{fancy}
\fancyhf{}
\rfoot{\thepage}
\renewcommand{\footrulewidth}{4pt}
\renewcommand{\headrulewidth}{0pt}
\patchcmd{\footrule}{\hrule}{\color{AzulFoot}\hrule}{}{}


Al momento de comenzar la contradicción del prototipo, un parámetro fundamental a definir es el ancho que tienen las paredes del nido prototipo. Se sabe que el menor valor que tiene la cavidad real es de \lnote{XX} \cite{ref:PaperValeriaOjeda}.

Mediante el uso de la Equación (\ref{eq:penetracion}) se calcula la profundidad de penetración de la onda electromagnética.
\begin{equation}
\delta \ = \ \frac{1}{\alpha} \ =\ \frac{1}{\sqrt{\pi f \mu \sigma}}
\label{eq:penetracion}
\end{equation}

Para ello se sabe que los valores de las constantes son los siguientes:
\begin{itemize}
	\item Frecuencia $f = 915 \ MHz$.
	\item Permeabilidad magnética de la madera \cite{ref:permeabilidad_madera}: $\mu_{Madera} = 1.25663760 \ \mu \frac{Hy}{m}$.
	\item Conductividad de la madera \cite{ref:conductividad_madera}:$\sigma \approx 1 \ m\frac{S}{m}$.
\end{itemize}

Con los valores mencionados, reemplazando en la Ecuación (\ref{eq:penetracion}) se obtiene que $\delta = 526 \ mm$.

Se procede a calcular el coeficiente de reflexión de la onda electromagnética. Para ello se sabe también que las permitividades de la madera y aire son:
\begin{itemize}
	\item $\varepsilon_{Madera} = 17.7 10^{-12} \ \frac{F}{m}$ \cite{ref:varepsilon_madera}.
	\item $\varepsilon_{Aire} = 8.8595003 10^{-12} \ \frac{F}{m}$
	\item $\mu_{Aire} = 1.25663753 \ \mu \frac{ Hy}{m}$ 
\end{itemize}

También se calcula la impedancia de la madera como:
\begin{equation}
	\eta_{Madera}=\sqrt{\frac{\mu_{Madera}}{\varepsilon_{Madera}}}
\end{equation}

Es así que el coeficiente de reflexión se calcula de la siguiente forma:
\begin{equation}
	\Gamma = \frac{\eta_2 - \eta_1  }{\eta_2 + \eta_1} = -0.1714
	\label{eq:reflexion_madera}
\end{equation}

Se observa en la Equiación (\ref{eq:reflexion_madera}) que el resultado presenta un valor negativo. Lo que esto implica es una inversión de fase en la onda electromagnética.

Además se puede calcular la potencia de la onda reflejada, siendo esta $P_{ref} = |\Gamma^2| = 0.029$. Es así que se nota que menos de un $3\%$ de la potencia sera reflejada. De esta forma se puede afirmar que no habrá una interferencia significativa con la transmisión original. \lnote{(Creo que hay que ver mejor la forma de decir esto. Nos vana  decir que como podemos afirmar que un 3\% es \quotes{no significativo}).}

