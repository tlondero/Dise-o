Este proyecto involucrará el diseño de un dispositivo capaz de recolectar información para ser utilizado en el ámbito de la investigación. Luego, se deberá efectuar una comunicación tanto con los datos que posea el animal como con una persona en la base del árbol.

Además, el producto deberá ser capaz de lograr la carga de la UBM colocada sobre el pájaro. En una primer instancia, se buscará lograr dicha carga a través unidades de carga de RF, mientras el ave se encuentre en el nido. En caso de que no sea viable cargar inalámbricamente, se dispondrá de sistemas complementarios que permitan el funcionamiento de la mochila de manera autónoma, tales como tecnologías de carga piezoeléctrica o solar. 

En consecuencia, se necesitará alimentar a los sensores instalados en el nido, a la batería que posea el ave y a los elementos relacionados a la comunicación. Dado que el dispositivo estará destinado para utilizarse en zonas remotas, se deberá conseguir energía mediante medios propios del entrono, como lo puede ser el uso de paneles solares para recolectar energía y baterías para almacenarla.

Se realizarán los análisis relevantes para asegurar la viabilidad financiera del proyecto según los requisitos del cliente. La verificación de la calidad del diseño estará basada en un único prototipo no comercial, el cual buscará cumplir los requerimientos definidos y adquirir las validaciones posibles dentro del marco económico actual y las limitaciones del cliente. 

%Por otro lado, en este trabajo no se contemplará la instalación del producto final in situ; la electrónica que irá situada en la mochila, exceptuando el receptor de energía y un prototipo con el cual se comprobarán las funcionalidades del nido; ni el procesamiento de los datos recibidos, solo aquellos relacionados con el almacenamiento y retransmisión.

Por otro lado, para que quede definido el límite del proyecto, se enumeran ciertos aspectos que no se contemplarán en este trabajo:
\begin{itemize}
	\item La instalación del producto final in situ.
	\item El procesamiento de los datos recibidos. Solo se realizaran aquellos procesos relacionados con el almacenamiento y retransmisión.
	\item La electrónica que irá situada en la mochila, tales como sensores, almacenamiento o señales de control. Solo se desarrollará lo pertinente con la carga y, en caso de ser necesario, prototipos que únicamente serán destinados para ser empleados en bancos de pruebas de la unidad del nido. 
\end{itemize}

\observacion{\verObs}{VERIFICAR QUE: \quotes{Capacidad de recargar completamente de manera inalámbrica en 6 horas las baterías de un dispositivo ajeno al proyecto que irá sobre el ave} y \quotes{El sistema de carga del dispositivo del ave debe entregar al menos 7.5 mW y hasta 10 mW} esté explicitado.}

%Por ser un proyecto con bajo volumen de producción no se considera el desarrollo de tecnologías de adquisición de datos, almacenamiento o extracción de energía. Tampoco se busca desarrollar un plan de desarrollo económico.