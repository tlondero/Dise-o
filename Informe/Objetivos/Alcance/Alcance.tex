El alcance de este proyecto es realizar un prototipo funcional no comercial de un dispositivo capaz de recolectar información para ser utilizado en el ámbito de la investigación. Cuenta con la capacidad de comunicarse por medio de WiFi con la finalidad de habilitar una página web local para uso de las biólogas. En dicha red se puede descargar la información recolectada. Además, cuenta con comunicación \textit{Bluetooth} para recibir los datos recolectados por un proyecto ajeno a S.I.M.F.S., aunque complementario.

Se analiza la factibilidad tecnológica de cargar de manera inalámbrica la UBM colocada sobre el proyecto ajeno. En caso de que el estudio de factibilidad dicte que no sea viable la carga, se debe garantizar la obtención de datos de posición determinadas por el dispositivo montado sobre el ave, cumpliendo aún con los requisitos de peso, dimensiones y transmisión de información de esta.

En consecuencia, se necesita alimentar a los sensores instalados en el nido y a los elementos relacionados a la comunicación.

Dado que el dispositivo está destinado para utilizarse en zonas remotas, se debe conseguir energía mediante medios propios del entorno, como utilizando paneles solares para recolectar energía y baterías para almacenarla.

Se realizan los análisis relevantes para asegurar la viabilidad financiera del proyecto según los requisitos del cliente. La verificación de la calidad del diseño está basada en un único prototipo no comercial, el cual busca cumplir los requerimientos definidos y adquirir las validaciones posibles dentro del marco económico actual y las limitaciones del cliente. 

Por ser un proyecto con bajo volumen de producción no se considera el desarrollo de tecnologías de adquisición de datos, almacenamiento o extracción de energía. Tampoco se busca desarrollar un plan de desarrollo económico.