Este proyecto involucra el diseño de un dispositivo capaz de recolectar información para ser utilizado en el ámbito de la investigación. Luego, se efectúa una comunicación tanto con los datos que posea el animal como con una persona en la base del árbol.

Luego, se realiza una comunicación entre el dispositivo del animal y la infraestructura del nido; y entre el dicha electrónica con una persona en la base del árbol. Dicha comunicación debe se caracteriza por ser una transmisión de datos de manera unidireccional e inalámbrica.

Además, se analiza la factibilidad de cargar de manera inalámbrica la UBM colocada sobre el pájaro. En una primer instancia, se busca lograr dicha carga a través de RF, mientras el ave se encuentre en el nido. En caso de que el estudio de factibilidad tecnológica dicte que no sea viable la carga, se debe garantizar el la obtención de datos de posición determinadas por la UBM cumpliendo aún con los requisitos de peso, dimensiones y transmisión de información de esta.

En consecuencia, se necesita alimentar a los sensores instalados en el nido, a la batería que posea el ave (en caso de que sea posible) y a los elementos relacionados a la comunicación. En el caso de que no se pueda realizar la carga inalámbrica, se necesita buscar una forma de garantizar la medición de las variables de interés.

Dado que el dispositivo esta destinado para utilizarse en zonas remotas, se debe conseguir energía mediante medios propios del entorno, como utilizando paneles solares para recolectar energía y baterías para almacenarla.

Se realizan los análisis relevantes para asegurar la viabilidad financiera del proyecto según los requisitos del cliente. La verificación de la calidad del diseño esta basada en un único prototipo no comercial, el cual busca cumplir los requerimientos definidos y adquirir las validaciones posibles dentro del marco económico actual y las limitaciones del cliente. 

Por otro lado, para que quede definido el límite del proyecto, se enumeran ciertos aspectos que no se contemplarán en este trabajo:
\begin{enumerate}
	\item La instalación del producto final in situ.
	\item El procesamiento de los datos recibidos. Solo se realizaran aquellos procesos relacionados con el almacenamiento y retransmisión.
	\item La electrónica situada en la mochila, tales como sensores, almacenamiento o señales de control. Solo se desarrolla lo pertinente con el abastecimiento energético y, en caso de ser necesario, prototipos que únicamente serán destinados para ser empleados en bancos de pruebas de la unidad del nido. 
\end{enumerate}

Por ser un proyecto con bajo volumen de producción no se considera el desarrollo de tecnologías de adquisición de datos, almacenamiento o extracción de energía. Tampoco se busca desarrollar un plan de desarrollo económico.