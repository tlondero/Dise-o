El alcance de este proyecto es realizar un prototipo funcional no comercial de un dispositivo capaz de recolectar información para ser utilizado en el ámbito de la investigación. Con capacidad de  comunicación por medio de WiFi con la finalidad de habilitar una pagina web para uso de las biólogas donde pueden descargar la información recolectada. Así también comunicación Bluetooth para recibir los datos recolectados por un proyecto ajeno al nuestro, aunque complementario.

Además, se analiza la factibilidad tecnológica de cargar de manera inalámbrica la UBM colocada sobre el proyecto ajeno. En caso de que el estudio de factibilidad dicte que no sea viable la carga, se debe garantizar la obtención de datos de posición determinadas por el dispositivo montado sobre el ave, cumpliendo aún con los requisitos de peso, dimensiones y transmisión de información de esta.

En consecuencia, se necesita alimentar a los sensores instalados en el nido y a los elementos relacionados a la comunicación.

Dado que el dispositivo esta destinado para utilizarse en zonas remotas, se debe conseguir energía mediante medios propios del entorno, como utilizando paneles solares para recolectar energía y baterías para almacenarla.

Se realizan los análisis relevantes para asegurar la viabilidad financiera del proyecto según los requisitos del cliente. La verificación de la calidad del diseño esta basada en un único prototipo no comercial, el cual busca cumplir los requerimientos definidos y adquirir las validaciones posibles dentro del marco económico actual y las limitaciones del cliente. 

%Por otro lado, para que quede definido el límite del proyecto, se enumeran ciertos aspectos que no se contemplarán en este trabajo:
%\begin{enumerate}
%	\item La instalación del producto final in situ.
%	\item El procesamiento de los datos recibidos. Solo se realizaran aquellos procesos relacionados con el almacenamiento y retransmisión.
%	\item La electrónica situada en la mochila, exceptuando:
%	\begin{itemize}
%		\item El desarrollo de lo pertinente con el abastecimiento energético, en caso de ser posible.
%		\item Una propuesta viable para lograr la obtención de posición, en caso de que no sea factible la carga.
%		\item Prototipos únicamente destinados para ser empleados en bancos de pruebas de la unidad del nido, en caso de que sean necesarios.	
%	\end{itemize}	
%\end{enumerate}
Por ser un proyecto con bajo volumen de producción no se considera el desarrollo de tecnologías de adquisición de datos, almacenamiento o extracción de energía. Tampoco se busca desarrollar un plan de desarrollo económico.