Este proyecto involucrará el diseño de un dispositivo capaz de recolectar información para ser utilizado en el ámbito de la investigación. Luego, se deberá efectuar una comunicación tanto con los datos que posea el animal como con una persona en la base del árbol. 

En consecuencia, se necesitará alimentar a los sensores instalados en el nido, a la batería que posea el ave y a los elementos relacionados a la comunicación. Dado que el dispositivo estará destinado para utilizarse en zonas remotas, se deberá conseguir energía mediante medios propios del entrono, como lo puede ser el uso de paneles solares y baterías.

Se realizarán los análisis relevantes para asegurar la viabilidad financiera del proyecto según los requisitos del cliente. La verificación de la calidad del diseño estará basada en un único prototipo no comercial, el cual buscará cumplir los requerimientos definidos y adquirir las validaciones posibles dentro del marco económico actual y las limitaciones del cliente. 

Por otro lado, en este trabajo no se contemplará la instalación del producto final in situ; la electrónica que irá situada en la mochila, exceptuando el receptor de energía y un prototipo con el cual se comprobarán las funcionalidades del nido; ni el procesamiento de los datos recibidos, solo aquellos relacionados con el almacenamiento y retransmisión.

%Por ser un proyecto con bajo volumen de producción no se considera el desarrollo de tecnologías de adquisición de datos, almacenamiento o extracción de energía. Tampoco se busca desarrollar un plan de desarrollo económico.