En este proyecto se busca desarrollar un equipo electrónico que permita a un grupo de ornitólogas del CONICET realizar un estudio sobre el ave Campephilus Magellanicus.

La ornitología, el estudio de las aves, es una rama muy importante de la biología, con varios aportes diversos al conocimiento colectivo como conceptos claves sobre la evolución, comportamiento y la conservación de ecosistemas. Esta última es de especial importancia, dado que las aves controlan las poblaciones de roedores e insectos, dispersan semillas que ayudan a la conservación de bosques, son fuente de alimento de otras especies y son indicadores de la calidad de un ecosistema.

El producto propuesto debe involucrar la adquisición y disposición de distintos parámetros de la vida del ave y de su entorno. Dentro de lo que abarca propiamente el nido, se encuentran como datos elementales la temperatura, la luminosidad, la humedad y la detección de presencia del ave en el nido. A estas variables se le suma la necesidad de almacenar y transmitir en tiempo real imágenes del interior.

Se pactó con el grupo de ornitólogas que, dos veces por semana, se acercará una persona a la base del árbol para obtener las mediciones. Este proceso se repite durante todo el proceso de anidamiento del espécimen en cuestión. Se concluyó que la manera menos invasiva para lograr este objetivo es descargando de manera inalámbrica todos los datos almacenados en el equipo del nido. 

Por otro lado, este proyecto se complementa a su vez con otro dispositivo, el cual se encuentra montado sobre el ave. El elemento mencionado se emplea para poder captar la posición del pájaro durante el día. Como también se debe dar acceso a esos datos, no solo se debe poder almacenar las mediciones captados en el nido, sino que también se debe incorporar la posibilidad de recibir de manera inalámbrica aquellos datos que obtenga el equipo del ave y permitirle acceso al usuario. 