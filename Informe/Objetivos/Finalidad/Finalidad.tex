En este proyecto se busca desarrollar un equipo electrónico encargado por el CIDEI que permita a un grupo de ornitólogas del CONICET realizar un estudio sobre el ave Campephilus Magellanicus. Este equipo adquiere distintas variables del entorno del nido y del ave, sin perturbar la vida normal de este e independizándose de la necesidad de conectarse a una red eléctrica. 

El producto brindará al campo de la investigación de la ornitología vital información acerca de uno de los vectores de referencia más importantes del bosque patagónico argentino.

%\red{La ornitología, el estudio de las aves, es una rama muy importante de la biología, con varios aportes diversos al conocimiento colectivo como conceptos claves sobre la evolución, comportamiento y la conservación de ecosistemas. Esta última es de especial importancia, dado que las aves controlan las poblaciones de roedores e insectos, dispersan semillas que ayudan a la conservación de bosques, son fuente de alimento de otras especies y son indicadores de la calidad de un ecosistema.(esto me parece que es contexto del proyecto, no finalidad.)}