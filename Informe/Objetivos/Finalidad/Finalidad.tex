La ornitología, el estudio de las aves, es una rama muy importante de la biología, con varios aportes diversos al conocimiento colectivo como conceptos claves sobre la evolución, comportamiento y conservamiento de ecosistemas. Siendo esta última de especial importancia, dado que las aves controlan las poblaciones de roedores e insectos, dispersan semillas que ayuda a la conservación de bosques, son fuente de alimento de otras especies y son indicadores de la calidad de un ecosistema.

%Este proyecto, el equipo electrónico le permitirá a un grupo de ornitólogas del CONICET realizar un estudio sobre las aves del territorio argentino, especialmente pero no limitado a las de la especie Campephilus Magellanicus.

En este proyecto se busca desarrollar un equipo electrónico que permitirá a un grupo de ornitólogas del CONICET realizar un estudio sobre el ave Campephilus Magellanicus. Esto consiste en un dispositivo capaz de recolectar datos, transmitirlos, funcionar autónomamente y abastecer otros dispositivos.