Nuestro producto deberá involucrar la adquisición de distintos parámetros de la vida del ave. Estos parámetros forman parte tanto del entorno del ave (variables dentro o en las cercanías del nido). 

Si bien existe un dispositivo que se encontrará en el montado sobre el ave, es necesario complementar con otro elemento que permita tener datos del nido. Además, el primer componente mencionado también es capaz de recolectar datos. Es por esto que no solo se debe poder almacenar datos recolectados por el producto a desarrollar, sino que también se debe incorporar al dispositivo la posibilidad de recibir datos de manera inalámbrica desde el equipo del ave. 

Con anterioridad se pactó con el grupo de ornitólogas que, una vez por semana, se acercará una persona a la base del árbol para descargar de manera inalámbrica todos los datos almacenados en el equipo del nido, para no afectar en el comportamiento al ave. Este proceso se repetirá durante todo el proceso de anidamiento del espécimen en cuestión.

Por otra parte, el equipo del ave contará con una fuente de alimentación, la cual consta de baterías que deberán poder ser recargadas mientras el pájaro se encuentre dentro del nido. Es por esto que el sistema debe ser capaz de recargar dichas baterías. Por lo general, el carpintero gigante macho suele dormir entre seis y ocho horas en el nido, para luego tomar turnos de dos a tres horas con la hembra para cuidar a los pichones.

Una gran limitación del proyecto se basa en que las aves suelen hacer mantenimiento del nido, picoteando las paredes y el suelo de este para tapar los restos de comida o las heces de los pichones. Esto imposibilita la colocación de electrónica en el suelo o las paredes del hábitat. Sin embargo, la excepción de esto es la bóveda o techo, la cual es excavada primero para permitir luego la progresión hacia abajo.

El desafío del trabajo se centra en la complejidad de las condiciones de uso del dispositivo dado por el comportamiento destructivo de las aves dentro del nido, la necesidad de transferencia de energía inalámbrica y el requisito de lograr mantener energizado al sistema sin intervención humana durante todo el periodo de anidamiento del ave, sin la posibilidad de una conexión a la red eléctrica.