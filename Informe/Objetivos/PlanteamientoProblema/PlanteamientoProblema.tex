\observacion{\verObs}{Escribieron bastante y no han dejado muy claro qué problema deben resolver ustedes. Hay que intentarlo nuevamente escribiendo más claro, tratando de independizar el trabajo que ustedes deben hacer de la investigación principal.}

\observacion{\verObs}{Ustedes no hacen la investigación, van a diseñar un equipo para soporte de esa investigación, hablen de lo que ustedes van a realizar.}

\observacion{\verObs}{Tema importante para el lector: Hasta ahora no sabemos cuáles son esos distintos parámetros. Sólo sabemos que los productos disponibles en el mercado no los tienen en cuenta.}

Nuestro producto deberá involucrar la adquisición de distintos parámetros de la vida del ave, siendo estos la temperatura, la luminosidad y la humedad del nido. También es pertinente obtener imágenes del interior. 

Si bien existe un dispositivo que se encontrará montado sobre el ave, es necesario complementar con otro elemento que permita tener datos del nido. Además, el primer componente mencionado también es capaz de recolectar información. Es por esto que no solo se debe poder almacenar lo recolectado por el producto a desarrollar, sino que también se debe incorporar al dispositivo la posibilidad de recibir de manera inalámbrica aquellos datos que obtenga el equipo del ave. 

Con anterioridad se pactó con el grupo de ornitólogas que, una vez por semana, se acercará una persona a la base del árbol para descargar de manera inalámbrica todos los datos almacenados en el equipo del nido. Este proceso se repetirá durante todo el proceso de anidamiento del espécimen en cuestión.

Por otra parte, el equipo del ave contará con una fuente de alimentación, la cual consta de baterías que deberán poder ser recargadas mientras el pájaro se encuentre dentro del nido. Es por esto que el sistema debe ser capaz de lograr dicha recarga. Por lo general, el carpintero gigante macho suele dormir entre seis y ocho horas en el nido, para luego tomar turnos de dos a tres horas con la hembra para cuidar a los pichones.

Una gran limitación del proyecto se basa en que las aves suelen hacer mantenimiento del nido, picoteando las paredes y el suelo de este para tapar los restos de comida o las heces de los pichones. Esto imposibilita la colocación de electrónica en el suelo o las paredes del hábitat. Sin embargo, la excepción de esto es la bóveda o techo, la cual es excavada primero para permitir luego la progresión hacia abajo.

Finalmente, se contempla que el pájaro carpintero habita en zonas urbanas, suburbanas, rurales e intangibles (es decir zonas a las que no puede acceder el turista, donde se protege la biodiversidad). Este factor limita las fuentes de alimentación que se emplean para poder mantener funcionando al sistema. Es por ello que el uso de la red eléctrica no es una opción.

El desafío del trabajo se centra en la complejidad de las condiciones de uso del dispositivo dado por el comportamiento destructivo de las aves dentro del nido, la necesidad de transferencia de energía inalámbrica y el requisito de lograr mantener energizado al sistema sin intervención humana durante todo el periodo de anidamiento del ave, sin la posibilidad de una conexión a la red eléctrica.