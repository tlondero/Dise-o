El pájaro carpintero habita en zonas urbanas, suburbanas, rurales e intangibles\footnote{Se definen las zonas intangibles como aquellas zonas a las que no puede acceder el turista, donde se protege la biodiversidad.}. Este factor limita las fuentes de alimentación que se emplean para poder mantener funcionando al sistema. Es por ello que el uso de la red eléctrica no es una opción.

Tanto la electrónica del nido como la del ave deben contar con una fuente de alimentación para poder garantizar las mediciones mencionadas durante el proceso de investigación.

El desafío del trabajo se centra en el requisito de lograr mantener energizado ambos sistemas (dispositivo en el nido y dispositivo sobre el ave) sin intervención humana durante todo el periodo de anidamiento del ave, sin la posibilidad de una conexión a la red eléctrica. A pesar de ello, la mayor dificultad se encuentra en el segundo dispositivo debido a su tamaño reducido, a que no se encuentra fijo en un lugar y que no puede ser cargado utilizando cables.

Es por ello que un planteamiento del proyecto consiste en determinar la factibilidad de la carga inalámbrica de esta segunda batería. En caso de obtener una respuesta favorable, debe ser llevado a la práctica. Caso contrario, se debe garantizar una solución viable que permita obtener la posición del ave durante el día, garantizando un dicha variable en un entorno mínimo.


