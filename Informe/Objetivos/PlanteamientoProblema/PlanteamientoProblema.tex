Este producto debe involucrar la adquisición y disposición de distintos parámetros de la vida del ave y de su entorno. Dentro de lo que abarca propiamente el nido, se encuentran como datos elementales la temperatura, la luminosidad y la humedad de este entorno. A estas variables se le suma la necesidad de obtener imágenes del interior.

Se pactó con el grupo de ornitólogas que, dos veces por semana, se acercará una persona a la base del árbol para obtener las mediciones. Este proceso se repite durante todo el proceso de anidamiento del espécimen en cuestión. Se concluyó que la manera menos invasiva para lograr este objetivo es descargando de manera inalámbrica todos los datos almacenados en el equipo del nido. 

Por otro lado, en este proyecto se complementa a su vez con otro dispositivo, el cual se encuentra montado sobre el ave. El elemento mencionado se emplea para poder captar la posición del pájaro durante el día. Como también se debe dar acceso a esos datos, no solo se debe poder almacenar las mediciones captados en el nido, sino que también se debe incorporar la posibilidad de recibir de manera inalámbrica aquellos datos que obtenga el equipo del ave y permitirle acceso al usuario. 

El pájaro carpintero habita en zonas urbanas, suburbanas, rurales e intangibles\footnote{Se definen las zonas intangibles como aquellas zonas a las que no puede acceder el turista, donde se protege la biodiversidad.}. Este factor limita las fuentes de alimentación que se emplean para poder mantener funcionando al sistema. Es por ello que el uso de la red eléctrica no es una opción.

Tanto la electrónica del nido como la del ave deben contar con una fuente de alimentación para poder garantizar las mediciones mencionadas durante el proceso de investigación. El desafío del trabajo se centra en el requisito de lograr mantener energizado ambos sistemas (dispositivo en el nido y dispositivo sobre el ave) sin intervención humana durante todo el periodo de anidamiento del ave, sin la posibilidad de una conexión a la red eléctrica. A pesar de ello, la mayor dificultad se encuentra en el segundo dispositivo debido a su tamaño reducido, a que no se encuentra fijo en un lugar y que no puede ser cargado utilizando cables.

Es por ello que un planteamiento del proyecto consiste en determinar la factibilidad de la carga inalámbrica de esta segunda batería. En caso de obtener una respuesta favorable, debe ser llevado a la práctica. Caso contrario, se debe garantizar una solución viable que permita obtener la posición del ave durante el día, garantizando un dicha variable en un entorno mínimo.


