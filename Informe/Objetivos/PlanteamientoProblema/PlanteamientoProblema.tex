El desafío del trabajo se centra en la complejidad de las condiciones de uso del dispositivo dado por el comportamiento destructivo de las aves dentro del nido, la ubicación de los nidos dentro de lo alto de los árboles y el grado de autonomía del sistema. 

El proyecto consiste de desarrollar un nido inteligente que sea capaz de adquirir diversos datos y entregarle energía a un dispositivo que estará colocado sobre el ave a ser estudiada. Todas las condiciones planteadas anteriormente generan una gran problemática a la hora de desarrollar la electrónica de carga del dispositivo del ave, debido a que no se pueden colocar bobinas en la base o las paredes del nido ya que esto puede ocasionar disturbios en el comportamiento de los pájaros, como por ejemplo picoteo excesivo al instrumental o \textcolor{red}{(TBC)}. Además, como por estudios anteriores se conoce, las aves duermen en distintas posiciones dentro del nido, lo cual dificulta la recepción del campo electromagnético generado por las bobinas emisoras de la carga inalámbrica, por lo que se debe diseñar un arreglo de bobinas multiaxial.

%El desafío del trabajo se centra en la sinergia e integración entre distintas tecnologías para aplicarlas sobre la investigación. Además, existe una complejidad en las condiciones de uso del dispositivo dado por las temperaturas y condiciones del ambiente. 

Es por esto que se establecen como aspectos fundamentales los siguientes puntos:
\begin{itemize}
	\item Determinación de la viabilidad de una base de carga inalámbrica multiaxial dentro o en los alrededores del nido teniendo en cuenta las condiciones impuestas por el animal.
	\item Eventual desarrollo de tecnologías bajo los requisitos máximos planteados.
	\item Integración de las diversas tecnologías.
	\item Validación del producto.
\end{itemize}

%Las pruebas de validación serán de extrema relevancia para asegurar un producto de la calidad que se requiere.