Nuestro producto deberá involucrar la adquisición de distintos parámetros de la vida del ave, siendo estos la temperatura, la luminosidad y la humedad del nido. También es pertinente obtener imágenes del interior. 

Si bien existe un dispositivo que se encontrará montado sobre el ave, es necesario complementar con otro elemento que permita obtener datos del nido. Además, el primer componente mencionado también es capaz de recolectar información. Es por esto que no solo se debe poder almacenar lo recolectado por el producto a desarrollar, sino que también se debe incorporar al dispositivo la posibilidad de recibir de manera inalámbrica aquellos datos que obtenga el equipo del ave. 

Con anterioridad se pactó con el grupo de ornitólogas que, una vez por semana, se acercará una persona a la base del árbol para descargar de manera inalámbrica todos los datos almacenados en el equipo del nido. Este proceso se repetirá durante todo el proceso de anidamiento del espécimen en cuestión.

Por otra parte, el equipo del ave contará con una fuente de alimentación, la cual consta de baterías que deberán poder ser recargadas mientras el pájaro se encuentre dentro del nido.

Se contempla que el pájaro carpintero habita en zonas urbanas, suburbanas, rurales e intangibles (es decir zonas a las que no puede acceder el turista, donde se protege la biodiversidad). Este factor limita las fuentes de alimentación que se emplean para poder mantener funcionando al sistema. Es por ello que el uso de la red eléctrica no es una opción.

El desafío del trabajo se centra en la  transferencia de energía inalámbrica y el requisito de lograr mantener energizado al sistema sin intervención humana durante todo el periodo de anidamiento del ave, sin la posibilidad de una conexión a la red eléctrica.
