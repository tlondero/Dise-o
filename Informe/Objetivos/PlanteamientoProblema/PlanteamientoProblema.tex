El estudio de investigación involucra la adquisición de distintos parámetros de la vida del ave. Estos parámetros forman parte tanto del entorno del ave (variables dentro o en las cercanías del nido) como del comportamiento mismo de esta (tiempo de vuelo, ubicación a lo largo del tiempo, etc.). El primer grupo de parámetros deberá ser adquirido por nuestro producto, mientras que el segundo grupo será adquirido por un dispositivo ajeno al proyecto que irá sujetado al ave en todo momento.

Para librar al dispositivo del ave de limitaciones dimensionales o de peso, la mayor parte del almacenamiento de datos se hará en el nido, siendo necesario por consiguiente incorporar al producto la posibilidad de recibir datos de manera inalámbrica desde el equipo del ave. Con anterioridad se pactó con el grupo de ornitólogas que, una vez por semana, se acercará una persona a la base del árbol para descargar de manera inalámbrica todos los datos almacenados en el equipo del nido, para no afectar en el comportamiento al ave.

Por otra parte, como el equipo del ave debe ser lo más pequeño posible, su fuente de alimentación constará solamente de baterías que deberán poder ser recargadas mientras el pájaro se encuentre dentro del nido. Por lo general, el carpintero gigante macho suele dormir entre seis y ocho horas en el nido, para luego tomar turnos de dos a tres horas con la hembra para cuidar a los pichones.

Una gran limitación del proyecto se basa en que las aves suelen hacer mantenimiento del nido, picoteando las paredes y el suelo de este para tapar los restos de comida o las heces de los pichones. Esto imposibilita la colocación de electrónica en el suelo o las paredes del hábitat. Sin embargo, la excepción de esto es la bóveda o techo, la cual es excavada primero para permitir luego la progresión hacia abajo.

El desafío del trabajo se centra en la complejidad de las condiciones de uso del dispositivo dado por el comportamiento destructivo de las aves dentro del nido, la necesidad de transferencia de energía inalámbrica y el requisito de lograr mantener energizado al sistema sin intervención humana durante todo el periodo de anidamiento del ave, sin la posibilidad de una conexión a la red eléctrica. 