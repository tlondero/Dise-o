\Subsubsection{Especificaciones Funcionales}
\begin{table}[H]
\centering
\begin{tabular}{|c|c|}
\hline
\multicolumn{2}{|c|}{\textbf{Leyenda para especificaciones}}    \\ \hline
\textbf{Aplicabilidad}             & \textbf{Validación}        \\ \hline
\multirow{2}{*}{P: Prototipo}      & I: Inspección Visual       \\ \cline{2-2} 
                                   & D: Documentación de Diseño \\ \hline
\multirow{2}{*}{F: Producto Final} & S: Simulación              \\ \cline{2-2} 
                                   & T: Test                    \\ \hline
\end{tabular}
\caption{Leyendas para las especificaciones.}
\end{table}

\observacion{\verObs}{INT-FUN-03: Acá no entiendo, si no pierde alimentación cuál es el problema. Qué quisieron decir??? Creo que va a ser importante que expliquen qué entienden como \quotes{FUENTE DE ENERGÍA PRINCIPAL}. Es un panel solar? Es una Batería externa que se carga con un panel solar? Es otra batería interna del \quotes{Nido}?}

\begin{table}[H]
\centering
\begin{tabular}{|l|l|l|c|}
\hline
\multicolumn{1}{|c|}{\textbf{ID}} & \multicolumn{1}{c|}{\textbf{Descripción}}                                                                                                                                                                                                                                                                                                                          & \multicolumn{1}{c|}{\textbf{Origen}} & \textbf{\begin{tabular}[c]{@{}c@{}}Aplicabilidad\\ Validación\end{tabular}} \\ \hline
INT-FUN-01                        & \begin{tabular}[c]{@{}l@{}}El dispositivo deberá tener un espacio de\\ almacenamiento de datos de por lo menos\\ 32 GBy, equivalente a la suma de los\\ datos adquiridos en el nido y por el\\ dispositivo del ave a lo largo de siete días.\end{tabular}                                                                                                        & REQ-06, REQ-04                       & P F - D                                                                     \\ \hline
INT-FUN-02                        & \begin{tabular}[c]{@{}l@{}}El producto deberá poder recuperarse\\ totalmente de una pérdida de alimentación\\ eléctrica sin intervención humana y sin\\ pérdida de datos almacenados. Se entiende\\ por pérdida de alimentación eléctrica como\\ tensión o corriente de entrada\\ menor a la mínima definida.\end{tabular}                               & REQ-02, REQ-03                       & F - I T                                                                     \\ \hline
INT-FUN-03                        & \begin{tabular}[c]{@{}l@{}}El producto deberá poder almacenar\\ suficiente energía como para poder seguir\\ funcionando correctamente sin pérdida de\\ alimentación (según lo definido en\\ INT-FUN-02) por \TBD días, cuando la\\ fuente de energía principal se encuentre\\ en condiciones de hasta un \TBD\%\\ inferiores a las mínimas definidas.\end{tabular} & REQ-02, REQ-03                       & P F - D                                                                     \\ \hline
INT-FUN-04                        & \begin{tabular}[c]{@{}l@{}}El producto debe ser capaz de obtener\\ la temperatura del entorno.\end{tabular}                                                                                                                                                                                                                                         & REQ-04                               & P F - I D T                                                                 \\ \hline
INT-FUN-05                        & \begin{tabular}[c]{@{}l@{}}El producto debe ser capaz de obtener la\\ humedad del entorno.\end{tabular}                                                                                                                                                                                                                                             & REQ-04                               & P F - I D T                                                                 \\ \hline
\end{tabular}
\caption{Especificaciones funcionales (Parte 1).}
\end{table}

\observacion{\verObs}{INT-FUN-08: INT-COM2 01 ó 02. Las leí y no me dicen mucho, salvo por la distancia}
\observacion{\verObs}{INT-FUN-09: Idem anterior.}

\begin{table}[H]
\centering
\begin{tabular}{|l|l|l|c|}
\hline
\multicolumn{1}{|c|}{\textbf{ID}} & \multicolumn{1}{c|}{\textbf{Descripción}}                                                                                                                                                                                                                                                                                                                          & \multicolumn{1}{c|}{\textbf{Origen}} & \textbf{\begin{tabular}[c]{@{}c@{}}Aplicabilidad\\ Validación\end{tabular}} \\ \hline
INT-FUN-06                        & \begin{tabular}[c]{@{}l@{}}El producto debe ser capaz de obtener\\ la intensidad lumínica del interior del nido.\end{tabular}                                                                                                                                                                                                                                         & REQ-04                               & P F - I D T                                                                 \\ \hline
INT-FUN-07                        & \begin{tabular}[c]{@{}l@{}}El producto debe ser capaz de obtener \\ imagenes del interior del nido.\end{tabular}                                                                                                                                                                                                                                             & REQ-04                               & P F - I D T                                                                 \\ \hline
INT-FUN-08                        & \begin{tabular}[c]{@{}l@{}}El producto debe poder transmitir de\\ manera inalámbrica los datos almacenados\\ en el nido a un dispositivo según todas\\ las especificaciones de la tabla INT-COM2.\end{tabular}                                                                                                                                                                       & REQ-07                               & P F - I D T                                                                 \\ \hline
INT-FUN-09                        & \begin{tabular}[c]{@{}l@{}}El producto debe poder recibir de manera\\ inalámbrica datos almacenados en un\\ dispositivo ajeno al proyecto que irá sobre\\ el ave según todas las especificaciones\\ de la tabla INT-COM1.\end{tabular}                                                                                                                                                 & REQ-05                               & P F - I D T                                                                 \\ \hline
INT-FUN-10                        & \begin{tabular}[c]{@{}l@{}}Capacidad de recargar completamente de\\ manera inalámbrica en 6 horas las baterías\\ de un dispositivo ajeno al proyecto que\\ irá sobre el ave.\end{tabular}                                                                                                                                                                          & REQ-14                               & P F - I D T                                                                 \\ \hline
\end{tabular}
\caption{Especificaciones funcionales (Parte 2).}
\end{table}

\begin{table}[H]
\centering
\begin{tabular}{|l|l|l|c|}
\hline
\multicolumn{1}{|c|}{\textbf{ID}} & \multicolumn{1}{c|}{\textbf{Descripción}}                                                                                                                                                                                                                                                                                                                          & \multicolumn{1}{c|}{\textbf{Origen}} & \textbf{\begin{tabular}[c]{@{}c@{}}Aplicabilidad\\ Validación\end{tabular}} \\ \hline
INT-FUN-11                        & \begin{tabular}[c]{@{}l@{}}El sistema de carga del dispositivo del\\ ave debe entregar al menos 7.5 mW y\\ hasta 10 mW.\end{tabular}                                                                                                                                                                                                                               & REQ-14                               & P F - I D                                                                   \\ \hline
INT-FUN-12                        & \begin{tabular}[c]{@{}l@{}}El sistema obtendrá valores del sensor\\ de temperatura cada 20 minutos.\end{tabular}                                                                                                                                                                                                                                                 & REQ-17                               & P F - I D T                                                                 \\ \hline
INT-FUN-13                        & \begin{tabular}[c]{@{}l@{}}El sistema obtendrá valores del sensor\\ de humedad cada 20 minutos.\end{tabular}                                                                                                                                                                                                                                                     & REQ-17                               & P F - I D T                                                                 \\ \hline
INT-FUN-14                        & \begin{tabular}[c]{@{}l@{}}El sistema obtendrá valores del sensor\\ de luminosidad cada 20 minutos.\end{tabular}                                                                                                                                                                                                                                                 & REQ-17                               & P F - I D T                                                                 \\ \hline
INT-FUN-15                        & \begin{tabular}[c]{@{}l@{}}El sistema obtendrá imagenes cada\\ 30 minutos.\end{tabular}                                                                                                                                                                                                                                                     					   & REQ-17                               & P F - I D T                                                                 \\ \hline
INT-FUN-16                        & \begin{tabular}[c]{@{}l@{}}El sistema utilizará \TBD paneles solares\\ para cargar una batería principal de \TBD\\ tecnología\end{tabular}                                                                                                                                                                                                                         & REQ-02                               & P F - I D T                                                                 \\ \hline
\end{tabular}
\caption{Especificaciones funcionales (Parte 3).}
\end{table}

\Subsubsection{Especificaciones de Interfaz}

\begin{table}[H]
\centering
\begin{tabular}{|l|l|l|c|}
\hline
\multicolumn{1}{|c|}{\textbf{ID}} & \multicolumn{1}{c|}{\textbf{Descripción}}                                                                                            & \multicolumn{1}{c|}{\textbf{Origen}} & \textbf{\begin{tabular}[c]{@{}c@{}}Aplicabilidad\\ Validación\end{tabular}} \\ \hline
INT-MEC-01                        & \begin{tabular}[c]{@{}l@{}}El equipo deberá poder sujetarse del árbol\\ con \TBD distanciados entre sí \TBD\end{tabular}   & REQ-01, REQ-10                       & F - I D                                                                     \\ \hline
INT-MEC-02                        & \begin{tabular}[c]{@{}l@{}}El sistema de montaje de la unidad de energía\\ deberá ser capaz de soportar un peso de \TBD\end{tabular} & REQ-01, REQ-10                       & F - D T                                                                     \\ \hline
\end{tabular}
\caption{Especificaciones de interfaz MEC.}
\end{table}

\begin{table}[H]
\centering
\begin{tabular}{|l|l|l|c|}
\hline
\multicolumn{1}{|c|}{\textbf{ID}} & \multicolumn{1}{c|}{\textbf{Descripción}}                                                                                                                                                                                                    & \multicolumn{1}{c|}{\textbf{Origen}} & \textbf{\begin{tabular}[c]{@{}c@{}}Aplicabilidad\\ Validación\end{tabular}} \\ \hline
INT-COM1-01                       & \begin{tabular}[c]{@{}l@{}}La transmisión de datos desde el ave al nido debe\\ poder ser interrumpida en cualquier momento sin\\ pérdidas de información equivalente a los últimos\\ \TBD minutos de recolección.\end{tabular}                                                         & REQ-05, REQ-15                       & P F - I D T                                                                 \\ \hline
INT-COM1-02                       & \begin{tabular}[c]{@{}l@{}}La transmisión de datos deberá tener un alcance\\ mínimo de 50 cm.\end{tabular}                                                                                                                                   & REQ-05                               & P F - I D T                                                                 \\ \hline
INT-COM1-03                       & \begin{tabular}[c]{@{}l@{}}La transmisión de datos deberá comenzar de\\ manera automática en cuanto el dispositivo del ave\\ se encuentre dentro del alcance.\end{tabular} & REQ-05, REQ-15                       & P F - I D T                                                                 \\ \hline
INT-COM1-04                       & \begin{tabular}[c]{@{}l@{}}La transmisión de datos deberá efectuarse por\\ medio del protocolo \TBD.\end{tabular}                                                                                                                            & REQ-05                               & P F - I D T                                                                 \\ \hline
\end{tabular}
\caption{Especificaciones de interfaz COM1.}
\end{table}

\begin{table}[H]
\centering
\begin{tabular}{|l|l|l|c|}
\hline
\multicolumn{1}{|c|}{\textbf{ID}} & \multicolumn{1}{c|}{\textbf{Descripción}}                                                                                                                                                                                                                                                     & \multicolumn{1}{c|}{\textbf{Origen}} & \textbf{\begin{tabular}[c]{@{}c@{}}Aplicabilidad\\ Validación\end{tabular}} \\ \hline
INT-COM2-01                       & \begin{tabular}[c]{@{}l@{}}La transmisión de datos desde el nido hacia una\\ persona deberá ser del tipo flush, descargándose al\\ dispositivo de la persona todos los datos almacenados\\ en el nido, liberando a la vez todo el espacio de\\ almacenamiento de datos del nido.\end{tabular} & REQ-07                               & P F - I D T                                                                 \\ \hline
INT-COM2-02                       & \begin{tabular}[c]{@{}l@{}}La transmisión de datos deberá tener un alcance\\ de 15 a 30 metros.\end{tabular}                                                                                & REQ-15                               & P F - I D T                                                                 \\ \hline
INT-COM2-03                       & \begin{tabular}[c]{@{}l@{}}La transmisión de datos deberá ser inicializada\\ por la persona de modo manual.\end{tabular}                                                                                                                                                                                     & REQ-07                               & P F - I D T                                                                 \\ \hline
INT-COM2-04                       & \begin{tabular}[c]{@{}l@{}}La transmisión de datos deberá efectuarse por\\ medio del protocolo \TBD.\end{tabular}                                                                                                                                                                             & REQ-07                               & P F - I D T                                                                 \\ \hline
INT-COM2-05                       & \begin{tabular}[c]{@{}l@{}}Una vez completa la transmisión de datos sin\\ interrupciones prematuras se indicará que se\\ finalizó dicha transmisión.\end{tabular}                                                                                                                           & REQ-07                               & P F - I D T                                                                 \\ \hline
INT-COM2-06                       & \begin{tabular}[c]{@{}l@{}}El descarte de los datos almacenados en el nido\\ sucederá una vez completa la transmisión sin\\ interrupciones prematuras.\end{tabular}                                                                                                                           & REQ-07                               & P F - I D T                                                                 \\ \hline
INT-COM2-07                       & \begin{tabular}[c]{@{}l@{}}Se le indicará a la persona que los datos fueron\\ transmitidos al finalizar dicho proceso.\end{tabular}                                                                                                                        & REQ-07                               & P F - I D T                                                                 \\ \hline
%INT-COM2-07                       & \begin{tabular}[c]{@{}l@{}}Ante una interrupción prematura de la comunicación,\\ la persona deberá reiniciar la transmisión de datos\\ desde el comienzo.\end{tabular}                                                                                                                        & REQ-07                               & P F - I D T                                                                 \\ \hline
\end{tabular}
\caption{Especificaciones de interfaz COM2.}
\end{table}

\Subsubsection{Especificaciones de Performance}

\begin{table}[H]
\centering
\begin{tabular}{|l|l|l|c|}
\hline
\multicolumn{1}{|c|}{\textbf{ID}} & \multicolumn{1}{c|}{\textbf{Descripción}}                                                                                                                                 & \multicolumn{1}{c|}{\textbf{Origen}} & \textbf{\begin{tabular}[c]{@{}c@{}}Aplicabilidad\\ Validación\end{tabular}} \\ \hline
%PER-01                            & \begin{tabular}[c]{@{}l@{}}El equipo deberá realizar la carga de la batería del\\ ave con una eficiencia no menor al \TBD\%.\end{tabular}                                 & INT-FUN-10                               & F - D T                                                                     \\ \hline
PER-01                            & \begin{tabular}[c]{@{}l@{}}El equipo no deberá consumir más de \TBD Watts\\ mientras no se esté recargando al dispositivo\\ del ave ni transmitiendo datos.\end{tabular}  & REQ-02                               & F - D T                                                                     \\ \hline
PER-02                            & \begin{tabular}[c]{@{}l@{}}El equipo no deberá consumir más de \TBD Watts\\ mientras se está recargando al dispositivo\\ del ave y recibiendo datos de este.\end{tabular} & REQ-02, REQ-14                       & F - D T                                                                     \\ \hline
PER-03                            & \begin{tabular}[c]{@{}l@{}}El equipo no deberá consumir más de \TBD Watts\\ mientras se está transmitiendo datos al\\ dispositivo de la persona.\end{tabular}             & REQ-02, REQ-7                        & F - D T                                                                     \\ \hline
\end{tabular}
\caption{Especificaciones de performance.}
\end{table}

\begin{table}[H]
\centering
\begin{tabular}{|l|l|l|c|}
\hline
\multicolumn{1}{|c|}{\textbf{ID}} & \multicolumn{1}{c|}{\textbf{Descripción}}                                                                                                                         & \multicolumn{1}{c|}{\textbf{Origen}} & \textbf{\begin{tabular}[c]{@{}c@{}}Aplicabilidad\\ Validación\end{tabular}} \\ \hline
IMP-DIM-01                        & \begin{tabular}[c]{@{}l@{}}El dispositivo del nido no deberá exceder las\\ siguientes dimensiones\\ Largo < 26 cm\\ Ancho < 8,79 cm\\ Alto < 4,55 cm\end{tabular} & REQ-01, REQ-09                              & F - I D T                                                                   \\ \hline
IMP-DIM-02                        & \begin{tabular}[c]{@{}l@{}}La unidad de energía no deberá exceder las\\ siguientes dimensiones\\ Largo < \TBD \\ Ancho < \TBD \\ Alto < \TBD\end{tabular}                    & REQ-01, REQ-09                       & F - I D T                                                                   \\ \hline
IMP-DIM-03                        & \begin{tabular}[c]{@{}l@{}}El equipo dentro del nido no deberá exceder\\ los \TBD gramos.\end{tabular}                                                             & REQ-01, REQ-09                       & F - I D T                                                                   \\ \hline
IMP-DIM-04                        & \begin{tabular}[c]{@{}l@{}}La unidad de energía no deberá exceder\\ los \TBD kilos.\end{tabular}                                                                   & REQ-01, REQ-09                       & F - I D T                                                                   \\ \hline
\end{tabular}
\caption{Especificaciones dimensionales y de peso.}
\end{table}

\Subsubsection{Especificaciones de Implementación}

\begin{table}[H]
\centering
\begin{tabular}{|c|l|l|c|}
\hline
\textbf{ID} & \multicolumn{1}{c|}{\textbf{Descripción}}                                                                                                                                                                                                                                                                                                                                                 & \multicolumn{1}{c|}{\textbf{Origen}} & \textbf{\begin{tabular}[c]{@{}c@{}}Aplicabilidad\\ Validación\end{tabular}} \\ \hline
IMP-OPE-01  & \begin{tabular}[c]{@{}l@{}}El sistema deberá poder operar normalmente cuando\\ la temperatura  ambiente sea -20°C < $\text{T}_{\text{AMB}}$ < 30°C.\end{tabular} & REQ-10                               & F - I D                                                                     \\ \hline
IMP-OPE-02  & \begin{tabular}[c]{@{}l@{}}Deberá poder operar normalmente cuando la humedad\\ sea: 0\% < RH < 100\%.\end{tabular}                                                                                                                                                                                                                         & REQ-10                               & F - I D                                                                     \\ \hline
IMP-OPE-03  & \begin{tabular}[c]{@{}l@{}}El dispositivo deberá poder operar normalmente cuando\\ la presión atmosférica sea: 84 kPa < $P_{ATM}$ < 101 kPa.\\ Esto equivale a 1500 m de altura para el mínimo de\\ presión, y un máximo de 0 m .\end{tabular}                                                                                                                                                  & REQ-10                               & F - I D                                                                     \\ \hline
IMP-OPE-04  & \begin{tabular}[c]{@{}l@{}}El dispositivo deberá tener un grado de\\ protección IPXX \TBD \end{tabular}                                                                                                                                                                                                                                                                                                                                                                                                                                                                               & REQ-10                               & F - I D                                                                     \\ \hline
\end{tabular}
\caption{Especificaciones de operación.}
\end{table}


\begin{table}[H]
\centering
\begin{tabular}{|l|l|l|l|}
\hline
\multicolumn{1}{|c|}{\textbf{ID}} & \multicolumn{1}{c|}{\textbf{Descripción}                                                                                                                                                                                } & \multicolumn{1}{c|}{\textbf{Origen}}                       & \textbf{\begin{tabular}[c]{@{}c@{}}Aplicabilidad\\ Validación\end{tabular}} \\ \hline
IMP-AYT-01  & \begin{tabular}[c]{@{}l@{}}No se deberán sufrir daños cuando, estando\\ desenergizado, la temperatura ambiente\\ sea -20°C \textless \ $\text{T}_{\text{AMB}}$ \textless \ 40°C.\end{tabular} 			& REQ-10, REQ-16 & P F - I D                                                                   \\ \hline
IMP-AYT-02  & \begin{tabular}[c]{@{}l@{}}No se deberán sufrir daños cuando, estando\\ desenergizado, la humedad\\ sea 0\% \textless \ RH \textless \ 100\%.\end{tabular}                 				& REQ-10, REQ-16 & P\TBD F - I D                                                              \\ \hline
IMP-AYT-03  & \begin{tabular}[c]{@{}l@{}}No se deberán sufrir daños cuando, estando\\ desenergizado, la presión atmosférica\\ sea 84 kPa \textless \ $\text{P}_{\text{ATM}}$ \textless \ 101 kPa.\end{tabular}        & REQ-10, REQ-16 & P\TBD F - I D                                                              \\ \hline
%IMP-AYT-04  & \begin{tabular}[c]{@{}l@{}}El equipo deberá tolerar vibraciones mecánicas\\ del siguiente modo \TBD \end{tabular}	& REQ-16, REQ-10 & P\TBD F - I D	\\ \hline
\end{tabular}
\caption{Especificaciones de almacenamiento y transporte.}
\end{table}

%\begin{table}[H]
%\centering
%\begin{tabular}{|l|l|l|l|}
%\hline
%\multicolumn{1}{|c|}{\textbf{ID}} & \multicolumn{1}{c|}{\textbf{Descripción}                                                                                                                                                                                } & \multicolumn{1}{c|}{\textbf{Origen}}                       & \textbf{\begin{tabular}[c]{@{}c@{}}Aplicabilidad\\ Validación\end{tabular}} \\ \hline
%IMP-COS-01  & El costo del producto no deberá ser superior a \TBD USD.  & REQ-11 & F - D		\\ \hline
%IMP-COS-02  & El costo del prototipo no deberá ser superior a \TBD USD. & REQ-11 & P - D	\\ \hline
%\end{tabular}
%\caption{Especificaciones de costos.}
%\end{table}

%\begin{table}[H]
%\centering
%\begin{tabular}{|l|l|l|c|}
%\hline
%\multicolumn{1}{|c|}{\textbf{ID}} & \multicolumn{1}{c|}{\textbf{Descripción}}                                                                                                                   & \multicolumn{1}{c|}{\textbf{Origen}} & \textbf{\begin{tabular}[c]{@{}c@{}}Aplicabilidad\\ Validación\end{tabular}} \\ \hline
%IMP-EMC-01                        & \begin{tabular}[c]{@{}l@{}}El dispositivo deberá poder operar normalmente con inmunidad al \\ ruido electromagnético de acuerdo a la norma \TBD\end{tabular} & REQ-13                               & F - D                                                                       \\ \hline
%\end{tabular}
%\caption{Especificaciones de compatibilidad electromagnética.}
%\end{table}

\Subsubsection{Especificaciones de Servicio (RAMS)}
\begin{table}[H]
\centering
\begin{tabular}{|l|l|l|c|}
\hline
\multicolumn{1}{|c|}{\textbf{ID}} 	& \multicolumn{1}{c|}{\textbf{Descripción}}                                                                                                                        	& \multicolumn{1}{c|}{\textbf{Origen}} & \textbf{\begin{tabular}[c]{@{}c@{}}Aplicabilidad\\ Validación\end{tabular}} \\ \hline
RAM-SEG-01                        	& \begin{tabular}[c]{@{}l@{}}La máxima temperatura que podrá tener la carcasa\\ será de \TBD °C.\end{tabular}                                                       & REQ-12, REQ-15                       & P F - I D T \TBD                                                            \\ \hline
%RAM-SEG-02                        	& \begin{tabular}[c]{@{}l@{}}Si en algún lugar (accesible o no) hay tensiones\\ peligrosas, deberá haber un cartel que lo advierta.\end{tabular}                   	& \TBC                                 & P \TBD F - I                                                                \\ \hline
RAM-SEG-02                        	& \begin{tabular}[c]{@{}l@{}}El dispositivo contará con un sistema de autenticación\\ ante el pedido de transmisión de datos definido\\ por INT-COM2.\end{tabular} 	& REQ-07                               & F - I D T \TBD                                                              \\ \hline
%RAM-SEG-04                        	& \TBD según la norma \TBD: seguridad eléctrica.                                                                                                                   	& REQ-12                               & \multicolumn{1}{l|}{\TBD}                                                   \\ \hline
\end{tabular}
\caption{Especificaciones de seguridad.}
\end{table}

\begin{table}[H]
\centering
\begin{tabular}{|l|l|l|l|}
\hline
\multicolumn{1}{|c|}{\textbf{ID}} & \multicolumn{1}{c|}{\textbf{Descripción}                                                                                                                                                                                } & \multicolumn{1}{c|}{\textbf{Origen}}                       & \textbf{\begin{tabular}[c]{@{}c@{}}Aplicabilidad\\ Validación\end{tabular}} \\ \hline
RAM-MAN-01  & \begin{tabular}[c]{@{}l@{}}En caso de utilizar software o firmware, deberá ser posible para\\ técnicos calificados realizar actualizaciones del mismo.\end{tabular}    & \TBC & \TBC
  \\ \hline
RAM-MAN-02  & \begin{tabular}[c]{@{}l@{}}En caso de utilizar software o firmware, no deberá ser posible\\ que sea modificado por el usuario.\end{tabular}                             & \TBC & \TBC \\ \hline
RAM-MAN-03  & \begin{tabular}[c]{@{}l@{}}El equipo deberá contener la siguiente documentación:\\ Manual de Usuario\\ Esquemáticos de circuitos\\ Esquemáticos de placas\end{tabular} & \TBC & \TBC                                                   \\ \hline
\end{tabular}
\caption{Especificaciones de mantenibilidad.}
\end{table}

\begin{table}[H]
\centering
\begin{tabular}{|l|l|l|l|}
\hline
\multicolumn{1}{|c|}{\textbf{ID}} & \multicolumn{1}{c|}{\textbf{Descripción}                                                                                                                                                                                } & \multicolumn{1}{c|}{\textbf{Origen}}                       & \textbf{\begin{tabular}[c]{@{}c@{}}Aplicabilidad\\ Validación\end{tabular}} \\ \hline
RAM-DIS-01  & \TBD & \TBD & \TBD                                                  \\ \hline
\end{tabular}
\caption{Especificaciones de disponibilidad.}
\end{table}

\begin{table}[H]
\centering
\begin{tabular}{|l|l|l|l|}
\hline
\multicolumn{1}{|c|}{\textbf{ID}} & \multicolumn{1}{c|}{\textbf{Descripción}                                                                                                                                                                                } & \multicolumn{1}{c|}{\textbf{Origen}}                       & \textbf{\begin{tabular}[c]{@{}c@{}}Aplicabilidad\\ Validación\end{tabular}} \\ \hline
RAM-CON-01  & El producto deberá tener una vida útil no menor a 2 años. & REQ-17          & P F - D                                                                     \\ \hline
\end{tabular}
\caption{Especificaciones de confiabilidad.}
\end{table}