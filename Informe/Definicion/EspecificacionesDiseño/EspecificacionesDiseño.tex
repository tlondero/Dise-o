\Subsubsection{Especificaciones Funcionales}
\begin{table}[H]
\centering
\begin{tabular}{|c|c|}
\hline
\multicolumn{2}{|c|}{\textbf{Leyenda para especificaciones}}    \\ \hline
\textbf{Aplicabilidad}             & \textbf{Validación}        \\ \hline
\multirow{2}{*}{P: Prototipo}      & I: Inspección Visual       \\ \cline{2-2} 
                                   & D: Documentación de Diseño \\ \hline
\multirow{2}{*}{F: Producto Final} & S: Simulación              \\ \cline{2-2} 
                                   & T: Test                    \\ \hline
\end{tabular}
\caption{Leyendas para las especificaciones.}
\end{table}

\begin{table}[H]
\centering
\begin{tabular}{|c|l|c|}
\hline
\textbf{ID} & \textbf{Descripción}                                                                                                                                                                                                                        & \textbf{Origen} \\ \hline
REQ-01      & \begin{tabular}[c]{@{}l@{}}El producto esta colgado de un árbol a una altura de 4 a 14\\ metros y se instala parcialmente dentro del nido del ave sin\\ comprometer la integridad del árbol.\end{tabular}                                   & Cliente         \\ \hline
REQ-02      & \begin{tabular}[c]{@{}l@{}}El producto debe mantenerse energizado sin\\ intervención humana.\end{tabular}                                                                                                                                   & Cliente         \\ \hline
REQ-03      & \begin{tabular}[c]{@{}l@{}}El producto no debe requerir conexión a la red eléctrica \\ para su funcionamiento.\end{tabular}                                                                                                                 & Tácito          \\ \hline
REQ-04      & \begin{tabular}[c]{@{}l@{}}El producto debe ser capaz de adquirir los siguientes \\ datos dentro del nido: imágenes, temperatura, humedad,\\ nivel de luz y presencia del ave.\end{tabular}                                                 & Cliente         \\ \hline
REQ-05      & El producto debe ser capaz de saber la fecha y hora.                                                                                                                                                                                        & Tácito          \\ \hline
REQ-06      & \begin{tabular}[c]{@{}l@{}}Aquellos datos recolectados por dispositivos ajenos al\\ nido deberán ser transmitidos a este por medio del\\ protocolo BLE.\end{tabular}                                                                        & Cliente         \\ \hline
REQ-07      & El producto debe almacenar los datos adquiridos.                                                                                                                                                                                            & Tácito          \\ \hline
REQ-08      & \begin{tabular}[c]{@{}l@{}}Una persona debe recibir los datos almacenados en el\\ nido a la distancia a través de una computadora.\end{tabular}                                                                                             & Cliente         \\ \hline
REQ-09      & \begin{tabular}[c]{@{}l@{}}El producto no debe llamar la atención de humanos \\ desde el nivel del piso.\end{tabular}                                                                                                                       & Cliente         \\ \hline
REQ-10      & \begin{tabular}[c]{@{}l@{}}El producto debe soportar las condiciones meteorológicas del\\ sur Argentino, específicamente los alrededores de Bariloche,\\ Rio Negro.\end{tabular}                                                            & Tácito          \\ \hline
REQ-11      & El costo del producto debe ser menor o igual a \precio USD.                                                                                                                                                                  & Tácito          \\ \hline
REQ-12      & \begin{tabular}[c]{@{}l@{}}Analizar la factibilidad tecnológica de la recarga inalámbrica\\ de la batería de la mochila del pájaro.\end{tabular}                                                                                            & Cliente         \\ \hline
REQ-13      & \begin{tabular}[c]{@{}l@{}}El producto desarmado debe soportar las condiciones de\\ translado impuestas por los caminos rurales hasta llegar\\ a la zona de instalación.\end{tabular}                                                       & Estado          \\ \hline
REQ-14      & La vida útil del producto debe ser de por lo menos dos años.                                                                                                                                                                                & Estado          \\ \hline
REQ-15      & \begin{tabular}[c]{@{}l@{}}En caso que no sea factible una recarga inalámbrica\\ de la batería de la mochila, proponer una solución alternativa\\ que permita adquirir los datos de posición del ave a lo largo\\ del estudio.\end{tabular} & Cliente         \\ \hline
\end{tabular}
\caption{Especificaciones funcionales.}
\label{tab:intFun}
\end{table}

\Subsubsection{Especificaciones de Interfaz}
\begin{table}[H]
\centering
\begin{tabular}{|c|l|c|c|}
\hline
\textbf{ID} & \textbf{Descripción}                                                                                                                                                                                       & \textbf{Origen} & \textbf{\begin{tabular}[c]{@{}c@{}}Aplicabilidad\\ Validación\end{tabular}} \\ \hline
INT-COM1-01 & \begin{tabular}[c]{@{}l@{}}La transmisión de datos entre el nido y la base principal\\ de seguimiento debe tener un alcance de entre por lo\\ menos uno y cinco metros.\end{tabular}                       & REQ-06          & P F - T                                                                     \\ \hline
INT-COM1-02 & \begin{tabular}[c]{@{}l@{}}La transmisión de datos se efectua una vez al día con\\ la información almacenada en la base principal de\\ seguimiento.\end{tabular}                                           & REQ-06          & F - D                                                                       \\ \hline
INT-COM1-03 & \begin{tabular}[c]{@{}l@{}}La transmisión de datos debe efectuarse por medio del\\ protocolo BLE desde el la base principal de seguimiento\\ hacia la base del nido de manera unidireccional.\end{tabular} & REQ-06          & P F - T                                                                     \\ \hline
\end{tabular}
\caption{Especificaciones de interfaz COM1.}
\end{table}

\begin{table}[H]
\centering
\begin{tabular}{|c|l|c|c|}
\hline
\textbf{ID} & \textbf{Descripción}                                                                                                                                                                  & \textbf{Origen} & \textbf{\begin{tabular}[c]{@{}c@{}}Aplicabilidad\\ Validación\end{tabular}} \\ \hline
INT-COM2-01 & \begin{tabular}[c]{@{}l@{}}Esta comunicación debe descargar al dispositivo de la\\ persona todos los datos seleccionados almacenados\\ en el nido.\end{tabular}                       & REQ-08          & P F - T                                                                     \\ \hline
INT-COM2-02 & \begin{tabular}[c]{@{}l@{}}La transmisión de datos debe tener un alcance que por\\ lo menos abarque entre 3.3 y 17 metros.\end{tabular}                                               & REQ-08          & P F - T                                                                     \\ \hline
INT-COM2-03 & \begin{tabular}[c]{@{}l@{}}La transmisión de datos debe ser inicializada\\ por la persona de modo manual.\end{tabular}                                                                & REQ-08          & P F - D                                                                     \\ \hline
INT-COM2-04 & \begin{tabular}[c]{@{}l@{}}La transmisión de datos debe efectuarse por\\ medio del protocolo WiFi.\end{tabular}                                                                       & REQ-08          & P F - D                                                                     \\ \hline
INT-COM2-05 & \begin{tabular}[c]{@{}l@{}}Los datos descargados por el usuario son descartados\\ de los datos almacenados en el nido una vez el usuario\\ se desconecte de la red WiFi.\end{tabular} & REQ-08          & F - D T                                                                     \\ \hline
\end{tabular}
\caption{Especificaciones de interfaz COM2.}
\label{tab:esp_com_wifi}
\end{table}

\Subsubsection{Especificaciones de Dimensionales}
\begin{table}[H]
\centering
\begin{tabular}{|c|l|c|c|}
\hline
\textbf{ID} & \textbf{Descripción}                                                                                                                                                                       & \textbf{Origen} & \textbf{\begin{tabular}[c]{@{}c@{}}Aplicabilidad\\ Validación\end{tabular}} \\ \hline
IMP-DIM-01  & \begin{tabular}[c]{@{}l@{}}El dispositivo del nido no deberá exceder las\\ siguientes dimensiones:\\ Largo \textless 26 cm\\ Ancho \textless 8,79 cm\\ Alto \textless 4,55 cm\end{tabular} & REQ-01          & F - T                                                                       \\ \hline
IMP-DIM-02  & \begin{tabular}[c]{@{}l@{}}La unidad de energía no deberá exceder las\\ siguientes dimensiones:\\ Largo \textless 1 m\\ Ancho \textless 50 cm\\ Alto \textless  1 m\end{tabular}           & REQ-01          & F - D T                                                                     \\ \hline
IMP-DIM-03  & \begin{tabular}[c]{@{}l@{}}El equipo dentro del nido no deberá exceder\\ los 500 gramos.\end{tabular}                                                                                      & REQ-01          & F - D T                                                                     \\ \hline
IMP-DIM-04  & \begin{tabular}[c]{@{}l@{}}La unidad de energía no deberá exceder\\ los 20 kilos.\end{tabular}                                                                                             & REQ-01          & F - D T                                                                     \\ \hline
\end{tabular}
\caption{Especificaciones dimensionales y de peso.}
\end{table}

\Subsubsection{Especificaciones de Implementación}
\begin{table}[H]
\centering
\begin{tabular}{|c|l|c|c|}
\hline
\textbf{ID} & \multicolumn{1}{c|}{\textbf{Descripción}}                                                                                                                                                                                                                             & \textbf{Origen} & \textbf{\begin{tabular}[c]{@{}c@{}}Aplicabilidad\\ Validación\end{tabular}} \\ \hline
IMP-OPE-01  & \begin{tabular}[c]{@{}l@{}}El sistema debe operar normalmente cuando la \\ temperatura  ambiente sea -20°C \textless $\text{T}_{\text{AMB}}$ \textless 30°C.\end{tabular}                                                                                           & REQ-10          & F - D                                                                       \\ \hline
IMP-OPE-02  & \begin{tabular}[c]{@{}l@{}}Debe operar normalmente cuando la humedad  sea: \\ 0\% \textless RH \textless 100\%\end{tabular}                                                                                                                                           & REQ-10          & F - D                                                                       \\ \hline
IMP-OPE-03  & \begin{tabular}[c]{@{}l@{}}El dispositivo debe operar normalmente cuando la presión \\ atmosférica sea: 84 kPa \textless $\text{P}_{\text{ATM}}$ \textless 90 kPa.\\ Esto equivale a 1500 m de altura para el mínimo de\\ presión, y un máximo de 0 m.\end{tabular} & REQ-10          & F - D                                                                       \\ \hline
IMP-DIM-04  & \begin{tabular}[c]{@{}l@{}}La unidad de energía no deberá exceder\\ los 20 kilos.\end{tabular}                                                                                                                                                                        & REQ-01          & F - D T                                                                     \\ \hline
\end{tabular}
\caption{Especificaciones de operación.}
\end{table}

\begin{table}[H]
\centering
\begin{tabular}{|c|l|c|c|}
\hline
\textbf{ID} & \multicolumn{1}{c|}{\textbf{Descripción}}                                                                                                                                                    & \textbf{Origen}                                         & \textbf{\begin{tabular}[c]{@{}c@{}}Aplicabilidad\\ Validación\end{tabular}} \\ \hline
IMP-AYT-01  & \begin{tabular}[c]{@{}l@{}}No se deben sufrir daños cuando, estando\\ desenergizado, la temperatura ambiente\\ sea -20°C \textless $\text{T}_{\text{AMB}}$ \textless 40°C.\end{tabular}    & \begin{tabular}[c]{@{}c@{}}REQ-10\\ REQ-13\end{tabular} & F - D                                                                       \\ \hline
IMP-AYT-02  & \begin{tabular}[c]{@{}l@{}}No se deben sufrir daños cuando, estando\\ desenergizado, la humedad\\ sea 0\% \textless RH \textless 100\%.\end{tabular}                                         & \begin{tabular}[c]{@{}c@{}}REQ-10\\ REQ-13\end{tabular} & F - D                                                                       \\ \hline
IMP-AYT-03  & \begin{tabular}[c]{@{}l@{}}No se deben sufrir daños cuando, estando\\ desenergizado, la presión atmosférica\\ sea 84 kPa \textless $\text{P}_{\text{ATM}}$ \textless 101 kPa.\end{tabular} & \begin{tabular}[c]{@{}c@{}}REQ-10\\ REQ-13\end{tabular} & F - D                                                                       \\ \hline
\end{tabular}
\caption{Especificaciones de almacenamiento y transporte.}
\end{table}

\begin{table}[H]
\centering
\begin{tabular}{|c|l|c|c|}
\hline
\textbf{ID} & \textbf{Descripción}                                                                                 & \textbf{Origen} & \textbf{\begin{tabular}[c]{@{}c@{}}Aplicabilidad\\ Validación\end{tabular}} \\ \hline
RAM-CON-01  & \begin{tabular}[c]{@{}l@{}}El producto debe tener una vida útil no\\ menor a tres años.\end{tabular} & REQ-14          & F - D                                                                       \\ \hline
\end{tabular}
\caption{Especificaciones de confiabilidad.}
\end{table}

\begin{table}[H]
\centering
\begin{tabular}{|c|l|c|c|}
\hline
\textbf{ID} & \textbf{Descripción}                                                                                                                                            & \textbf{Origen} & \textbf{\begin{tabular}[c]{@{}c@{}}Aplicabilidad\\ Validación\end{tabular}} \\ \hline
RAM-SEG-01  & \begin{tabular}[c]{@{}l@{}}El dispositivo cuenta con un sistema de autenticación\\ ante el pedido de transmisión de datos definido\\ por INT-COM2.\end{tabular} & REQ-08          & P F - I D                                                                   \\ \hline
RAM-SEG-02  & \begin{tabular}[c]{@{}l@{}}El dispositivo no pose luces ni elementos reflectantes\\ para no ser percibido.\end{tabular}                                         & REQ-09          & F - I                                                                       \\ \hline
\end{tabular}
\caption{Especificaciones de seguridad.}
\end{table}

\begin{table}[H]
\centering
\begin{tabular}{|c|l|c|c|}
\hline
\textbf{ID} & \textbf{Descripción}                                                                                                 & \textbf{Origen} & \textbf{\begin{tabular}[c]{@{}c@{}}Aplicabilidad\\ Validación\end{tabular}} \\ \hline
IMP-COS-01  & \begin{tabular}[c]{@{}l@{}}El costo del producto debe ser menor o igual\\ a \precio USD.\end{tabular} & REQ-11          & F - D                                                                       \\ \hline
\end{tabular}
\caption{Especificaciones de costo.}
\end{table}
