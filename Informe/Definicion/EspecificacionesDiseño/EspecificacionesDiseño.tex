\Subsubsection{Especificaciones Funcionales}
\begin{table}[H]
\centering
\begin{tabular}{|c|c|}
\hline
\multicolumn{2}{|c|}{\textbf{Leyenda para especificaciones}}    \\ \hline
\textbf{Aplicabilidad}             & \textbf{Validación}        \\ \hline
\multirow{2}{*}{P: Prototipo}      & I: Inspección Visual       \\ \cline{2-2} 
                                   & D: Documentación de Diseño \\ \hline
\multirow{2}{*}{F: Producto Final} & S: Simulación              \\ \cline{2-2} 
                                   & T: Test                    \\ \hline
\end{tabular}
\caption{Leyendas para las especificaciones.}
\end{table}

\begin{table}[H]
\centering
\begin{tabular}{|l|l|l|c|}
\hline
\textbf{ID} & \textbf{Descripción}                                                                                                                                                                                                                                                                                                                                             & \textbf{Origen} & \multicolumn{1}{l|}{\textbf{\begin{tabular}[c]{@{}l@{}}Aplicabilidad\\ Validación\end{tabular}}} \\ \hline
INT-FUN-01  & \begin{tabular}[c]{@{}l@{}}El dispositivo deberá tener un espacio de almacenamiento de\\ datos de por lo menos \TBD GBy, equivalente a la suma de los\\ datos adquiridos en el nido y por el dispositivo del ave a lo largo\\ de siete días.\end{tabular}                                                                                                       & REQ-06, REQ-04  & P F - D                                                                                          \\ \hline
INT-FUN-02  & \begin{tabular}[c]{@{}l@{}}El producto deberá funcionar correctamente con alimentación\\ eléctrica de como mínimo \TBD Watts y \TBD Volts y como \\ máximo \TBD Watts y \TBD Volts.\end{tabular}                                                                                                                                                             & REQ-02, REQ-03  & P F - I D T                                                                                      \\ \hline
INT-FUN-03  & \begin{tabular}[c]{@{}l@{}}El producto deberá poder recuperarse totalmente de una \\ pérdida de alimentación eléctrica sin intervención humana y sin \\ pérdida de datos almacenados.\\ Se entiende por pérdida de alimentación eléctrica como tensión \\ de entrada o potencia de entrada menor a la mínima definida.\end{tabular}                              & REQ-02, REQ-03  & P F - I T                                                                                        \\ \hline
INT-FUN-04  & \begin{tabular}[c]{@{}l@{}}El producto deberá poder almacenar suficiente energía como\\ para poder seguir funcionando correctamente sin pérdida de \\ alimentación (según lo definido en INT-FUN-03) por \TBD días, \\ cuando la fuente de energía principal se encuentre en \\ condiciones de hasta un \TBD\% inferiores a las mínimas definidas.\end{tabular} & REQ-02, REQ-03  & P F - D                                                                                          \\ \hline
INT-FUN-05  & \begin{tabular}[c]{@{}l@{}}El producto debe ser capaz de obtener los siguientes datos\\ del entorno: Imágenes, temperatura, humedad, \TBD \end{tabular}                                                                                                                                                                                                          & REQ-04          & P F - I D T                                                                                      \\ \hline
INT-FUN-06  & \begin{tabular}[c]{@{}l@{}}El producto debe poder transmitir de manera inalámbrica \\ los datos almacenados en el nido a un dispositivo según las \\ especificaciones INT-COM2.\end{tabular}                                                                                                                                                                     & REQ-07          & P F - I D T                                                                                      \\ \hline
INT-FUN-07  & \begin{tabular}[c]{@{}l@{}}El producto debe poder recibir de manera inalámbrica\\ datos almacenados en un dispositivo ajeno al proyecto \\ que irá sobre el ave según las especificaciones INT-COM1.\end{tabular}                                                                                                                                                & REQ-05          & P F - I D T                                                                                      \\ \hline
INT-FUN-08  & \begin{tabular}[c]{@{}l@{}}Capacidad de recargar completamente de manera \\ inalámbrica en 6 horas las baterías de un dispositivo \\ ajeno al proyecto que irá sobre el ave.\end{tabular}                                                                                                                                                                        & REQ-15          & P F - I D T                                                                                      \\ \hline
INT-FUN-09  & \begin{tabular}[c]{@{}l@{}}El sistema de carga del dispositivo del ave debe entregar \\ al menos 7.5 mW y hasta 10 mW.\end{tabular}                                                                                                                                                                                                                              & REQ-15          & P F - I D T                                                                                      \\ \hline
\end{tabular}
\caption{Especificaciones funcionales.}
\end{table}

\Subsubsection{Especificaciones de Interfaz}

\begin{table}[H]
\centering
\begin{tabular}{|l|l|l|l|}
\hline
\textbf{ID} & \textbf{Descripción}                                                                                                                  & \textbf{Origen}                       & \textbf{\begin{tabular}[c]{@{}l@{}}Aplicabilidad\\ Validación\end{tabular}} \\ \hline
INT-MEC-01  & \begin{tabular}[c]{@{}l@{}}El equipo deberá poder sujetarse con \TBD \\ tornillos tipo \TBD distanciados entre sí \TBD \end{tabular} & REQ-01, REQ-10 & F - I D                                                                     \\ \hline
INT-MEC-02  & \begin{tabular}[c]{@{}l@{}}El sistema de montaje de la unidad de energía\\ deberá ser capaz de soportar un peso de \TBD \end{tabular} & REQ-01, REQ-10 & F - D T                                                                     \\ \hline
\end{tabular}
\caption{Especificaciones de interfaz MEC.}
\end{table}

\begin{table}[H]
\centering
\begin{tabular}{|l|l|l|l|}
\hline
\textbf{ID} & \textbf{Descripción}                                                                                                                                                                                                                         & \textbf{Origen} & \textbf{\begin{tabular}[c]{@{}l@{}}Aplicabilidad\\ Validación\end{tabular}} \\ \hline
INT-COM1-01 & \begin{tabular}[c]{@{}l@{}}La transmisión de datos desde el ave al nido debe\\ poder ser interrumpida en cualquier momento sin\\ pérdidas de información considerables.\end{tabular}                                                         & REQ-05, REQ-16  & P F - I D T                                                                 \\ \hline
INT-COM1-02 & \begin{tabular}[c]{@{}l@{}}La transmisión de datos deberá tener un alcance\\ mínimo de 50cm.\end{tabular}                                                                                                                                    & REQ-05          & P F - I D T                                                                 \\ \hline
INT-COM1-03 & \begin{tabular}[c]{@{}l@{}}La transmisión de datos deberá comenzar de\\ manera automática en cuanto el dispositivo del ave\\ se encuentre dentro del alcance y con el nivel de\\ carga suficiente para sostener la transmisión.\end{tabular} & REQ-05, REQ-16  & P F - I D T                                                                 \\ \hline
INT-COM1-04 & \begin{tabular}[c]{@{}l@{}}La transmisión de datos deberá efectuarse por\\ medio del protocolo TBD.\end{tabular}                                                                                                                             & REQ-05          & P F - I D T                                                                 \\ \hline
\end{tabular}
\caption{Especificaciones de interfaz COM1.}
\end{table}

\begin{table}[H]
\centering
\begin{tabular}{|l|l|l|l|}
\hline
\textbf{ID} & \textbf{Descripción}                                                                                                                                                                                                                                                                          & \textbf{Origen} & \textbf{\begin{tabular}[c]{@{}l@{}}Aplicabilidad\\ Validación\end{tabular}} \\ \hline
INT-COM2-01 & \begin{tabular}[c]{@{}l@{}}La transmisión de datos desde el nido hacia una\\ persona deberá ser del tipo flush, descargándose al\\ dispositivo de la persona todos los datos almacenados\\ en el nido, liberando a la vez todo el espacio de\\ almacenamiento de datos del nido.\end{tabular} & REQ-07          & P F - I D T                                                                 \\ \hline
INT-COM2-02 & \begin{tabular}[c]{@{}l@{}}La transmisión de datos deberá tener un alcance\\ mínimo de 15 metros, la cual concuerda con la altura\\ máxima observada de los nidos del\\ Campephilus Magellanicus.\end{tabular}                                                                                & REQ-16          & P F - I D T                                                                 \\ \hline
INT-COM2-03 & \begin{tabular}[c]{@{}l@{}}La transmisión de datos deberá ser inicializada\\ por la persona.\end{tabular}                                                                                                                                                                                     & REQ-07          & P F - I D T                                                                 \\ \hline
INT-COM2-04 & \begin{tabular}[c]{@{}l@{}}La transmisión de datos deberá efectuarse por\\ medio del protocolo \TBD .\end{tabular}                                                                                                                                                                              & REQ-07          & P F - I D T                                                                 \\ \hline
INT-COM2-05 & \begin{tabular}[c]{@{}l@{}}El descarte de los datos almacenados en el nido\\ sucederá una vez completa la transmisión sin\\ interrupciones prematuras.\end{tabular}                                                                                                                           & REQ-07          & P F - I D T                                                                 \\ \hline
INT-COM2-06 & \begin{tabular}[c]{@{}l@{}}Ante una interrupción prematura de la comunicación,\\ la persona deberá reiniciar la transmisión de datos\\ desde el comienzo.\end{tabular}                                                                                                                        & REQ-07          & P F - I D T                                                                 \\ \hline
\end{tabular}
\caption{Especificaciones de interfaz COM2.}
\end{table}

\Subsubsection{Especificaciones de Performance}

\begin{table}[H]
\centering
\begin{tabular}{|l|l|l|l|}
\hline
\textbf{ID} & \textbf{Descripción}                                                                                                                                                    & \textbf{Origen}                       & \textbf{\begin{tabular}[c]{@{}l@{}}Aplicabilidad\\ Validación\end{tabular}} \\ \hline
PER-01      & \begin{tabular}[c]{@{}l@{}}El equipo deberá realizar la carga de la batería del ave con una \\ eficiencia no menor al \TBD \%.\end{tabular}                               & REQ-15         & P F - D T                                                                   \\ \hline
PER-02      & \begin{tabular}[c]{@{}l@{}}El equipo no deberá consumir más de \TBD Watts mientras no se \\ esté recargando al dispositivo del ave ni transmitiendo datos.\end{tabular}  & REQ-02         & P F - D T                                                                   \\ \hline
PER-03      & \begin{tabular}[c]{@{}l@{}}El equipo no deberá consumir más de \TBD Watts mientras se está \\ recargando al dispositivo del ave y recibiendo datos de este.\end{tabular} & REQ-02, REQ-15 & P F - D T                                                                   \\ \hline
PER-04      & \begin{tabular}[c]{@{}l@{}}El equipo no deberá consumir más de \TBD Watts mientras se está \\ transmitiendo datos al dispositivo de la persona.\end{tabular}             & REQ-02, REQ-7  & P F - D T                                                                   \\ \hline
\end{tabular}
\caption{Especificaciones de performance.}
\end{table}

\begin{table}[H]
\centering
\begin{tabular}{|l|l|l|l|}
\hline
\textbf{ID} & \textbf{Descripción}                                                                                                                                                                      & \textbf{Origen}                       & \textbf{\begin{tabular}[c]{@{}l@{}}Aplicabilidad\\ Validación\end{tabular}} \\ \hline
IMP-DIM-01  & \begin{tabular}[c]{@{}l@{}}El dispositivo del nido no deberá exceder las\\ siguientes dimensiones\\ Largo \textless 26 cm\\ Ancho \textless 8,79 cm\\ Alto \textless 4,55 cm\end{tabular} & REQ-01         & P F - I D                                                                   \\ \hline
IMP-DIM-02  & \begin{tabular}[c]{@{}l@{}}La unidad de energía no deberá exceder las\\ siguientes dimensiones\\ Largo \textless \TBD \\ Ancho \textless \TBD \\ Alto \textless \TBD \end{tabular}              & REQ-01, REQ-09 & P F - I D                                                                   \\ \hline
IMP-DIM-03  & \begin{tabular}[c]{@{}l@{}}El equipo dentro del nido no deberá exceder\\ los \TBD gramos.\end{tabular}                                                                                     & REQ-01, REQ-09 & P F - I D                                                                   \\ \hline
IMP-DIM-04  & \begin{tabular}[c]{@{}l@{}}La unidad de energía no deberá exceder\\ los \TBD kilos.\end{tabular}                                                                                           & REQ-01, REQ-09 & P F - I D                                                                   \\ \hline
\end{tabular}
\caption{Especificaciones de dimensionales y de peso.}
\end{table}

\Subsubsection{Especificaciones de Implementación}

\begin{table}[H]
\centering
\begin{tabular}{|l|l|l|l|}
\hline
\textbf{ID} & \textbf{Descripción}                                                                                                                                                                                                                                     & \textbf{Origen} & \textbf{\begin{tabular}[c]{@{}l@{}}Aplicabilidad\\ Validación\end{tabular}} \\ \hline
IMP-OPE-01  & \begin{tabular}[c]{@{}l@{}}El sistema deberá poder operar normalmente cuando la \\ temperatura  ambiente sea -20°C \textless TAMB \textless 30°C.\end{tabular}                                                                                           & REQ-10          & P F - I D                                                                   \\ \hline
IMP-OPE-02  & \begin{tabular}[c]{@{}l@{}}Deberá poder operar normalmente cuando la humedad  sea: \\ 0\% \textless RH \textless 100\%, valores normales de humedad relativa ambiente.\end{tabular}                                                                      & REQ-10          & P F - I D                                                                   \\ \hline
IMP-OPE-03  & \begin{tabular}[c]{@{}l@{}}El dispositivo deberá poder operar normalmente cuando la presión \\ atmosférica sea: 84 kPa \textless PATM \textless 90 kPa. Esto equivale a 1500m \\ de altura para el mínimo de presión, y un máximo de 1100m.\end{tabular} & REQ-10          & P F - I D                                                                   \\ \hline
IMP-OPE-04  & El dispositivo deberá tener un grado de protección IPXX \TBD & REQ-10          & P(TBD) F - I D                                                              \\ \hline
\end{tabular}
\caption{Especificaciones de operación.}
\end{table}

\begin{table}[H]
\centering
\begin{tabular}{|l|l|l|l|}
\hline
\textbf{ID} & \textbf{Descripción}                                                                                                                                                   & \textbf{Origen}                       & \textbf{\begin{tabular}[c]{@{}l@{}}Aplicabilidad\\ Validación\end{tabular}} \\ \hline
IMP-AYT-01  & \begin{tabular}[c]{@{}l@{}}No se deberán sufrir daños cuando, estando\\ desenergizado, la temperatura ambiente\\ sea -20°C \textless TAMB \textless 40°C.\end{tabular} & REQ-17, REQ-10 & P F - I D                                                                   \\ \hline
IMP-AYT-02  & \begin{tabular}[c]{@{}l@{}}No se deberán sufrir daños cuando, estando\\ desenergizado, la humedad\\ sea 0\% \textless RH \textless 100\%.\end{tabular}                 & REQ-17, REQ-10 & P\TBD F - I D                                                              \\ \hline
IMP-AYT-03  & \begin{tabular}[c]{@{}l@{}}No se deberán sufrir daños cuando, estando\\ desenergizado, la presión atmosférica\\ sea 84 kPa \textless PATM  101kPa.\end{tabular}        & REQ-17, REQ-10 & P\TBD F - I D                                                              \\ \hline
IMP-AYT-04  & \begin{tabular}[c]{@{}l@{}}El equipo deberá tolerar vibraciones mecánicas\\ del siguiente modo \TBD \end{tabular}                                                      & REQ-17, REQ-10 & P\TBD F - I D                                                              \\ \hline
\end{tabular}
\caption{Especificaciones de almacenamiento y transporte.}
\end{table}

\begin{table}[H]
\centering
\begin{tabular}{|l|l|l|l|}
\hline
\textbf{ID} & \textbf{Descripción}                                                                                                                    & \textbf{Origen}               & \textbf{\begin{tabular}[c]{@{}l@{}}Aplicabilidad\\ Validación\end{tabular}} \\ \hline
IMP-COS-01  & \begin{tabular}[c]{@{}l@{}}La suma del costo de las partes que conforman al producto no \\ deberá ser superior a \TBD USD.\end{tabular}  & REQ-11 & F - D                                                                       \\ \hline
IMP-COS-02  & \begin{tabular}[c]{@{}l@{}}La suma del costo de las partes que conforman el prototipo no \\ deberá ser superior a \TBD USD.\end{tabular} & REQ-11 & P - D                                                                       \\ \hline
\end{tabular}
\caption{Especificaciones de costos.}
\end{table}

\begin{table}[H]
\centering
\begin{tabular}{|l|l|l|l|}
\hline
\textbf{ID} & \textbf{Descripción}                                                                                                                                        & \textbf{Origen}               & \textbf{\begin{tabular}[c]{@{}l@{}}Aplicabilidad\\ Validación\end{tabular}} \\ \hline
IMP-EMC-01  & \begin{tabular}[c]{@{}l@{}}El dispositivo deberá poder operar normalmente con inmunidad \\ al ruido electromagnético de acuerdo a la norma \TBD \end{tabular} & REQ-13 & F - D T                                                                     \\ \hline
\end{tabular}
\caption{Especificaciones de compatibilidad electromagnética.}
\end{table}

\Subsubsection{Especificaciones de Servicio (RAMS)}
\begin{table}[H]
\centering
\begin{tabular}{|l|l|l|l|}
\hline
\textbf{ID} & \textbf{Descripción}                                                                                                                                             & \textbf{Origen}                       & \textbf{\begin{tabular}[c]{@{}l@{}}Aplicabilidad\\ Validación\end{tabular}} \\ \hline
RAM-SEG-01  & \begin{tabular}[c]{@{}l@{}}La máxima temperatura que podrá tener la carcasa\\ será de \TBD °C\end{tabular}                                                        & REQ-12, REQ-16 & P F - I D T \TBD                                                            \\ \hline
RAM-SEG-02  & \begin{tabular}[c]{@{}l@{}}Si en algún lugar (accesible o no) hay tensiones\\ peligrosas, deberá haber un cartel que lo advierta.\end{tabular}                   & \TBC                                   & P(TBD) F - I                                                                \\ \hline
RAM-SEG-03  & \begin{tabular}[c]{@{}l@{}}El dispositivo contará con un sistema de autenticación\\ ante el pedido de transmisión de datos definido\\ por INT-COM2.\end{tabular} & REQ-07         & P F - I D T \TBD                                                            \\ \hline
RAM-SEG-04  & \TBD según la norma \TBD: seguridad ambiental.                                                                        & REQ-14         &                                                                             \\ \hline
RAM-SEG-05  & \TBD según la norma \TBD: seguridad eléctrica.                                                                        & REQ-12         &                                                                             \\ \hline
\end{tabular}
\caption{Especificaciones de seguridad.}
\end{table}

\begin{table}[H]
\centering
\begin{tabular}{|l|l|l|l|}
\hline
\textbf{ID} & \textbf{Descripción}                                                                                                                                                    & \textbf{Origen}           & \textbf{\begin{tabular}[c]{@{}l@{}}Aplicabilidad\\ Validación\end{tabular}} \\ \hline
RAM-MAN-01  & \begin{tabular}[c]{@{}l@{}}En caso de utilizar software o firmware, deberá ser posible para \\ técnicos calificados realizar actualizaciones del mismo.\end{tabular}    & {\textcolor{red}{(TBC)}} & {\textcolor{red}{(TBC)}}
  \\ \hline
RAM-MAN-02  & \begin{tabular}[c]{@{}l@{}}En caso de utilizar software o firmware, no deberá ser posible para el \\ usuario acceder al mismo.\end{tabular}                             & {\textcolor{red}{(TBC)}} & {\textcolor{red}{(TBC)}}                                                   \\ \hline
RAM-MAN-03  & \begin{tabular}[c]{@{}l@{}}El equipo deberá contener la siguiente documentación: \\ Manual de Usuario\\ Esquemáticos de circuitos\\ Esquemáticos de placas\end{tabular} & {\textcolor{red}{(TBC)}} & {\textcolor{red}{(TBC)}}                                                   \\ \hline
\end{tabular}
\caption{Especificaciones de mantenibilidad.}
\end{table}

\begin{table}[H]
\centering
\begin{tabular}{|l|l|l|l|}
\hline
\textbf{ID} & \textbf{Descripción}      & \textbf{Origen}           & \textbf{\begin{tabular}[c]{@{}l@{}}Aplicabilidad\\ Validación\end{tabular}} \\ \hline
RAM-DIS-01  & {\textcolor{red}{(TBC)}} & {\textcolor{red}{(TBC)}} & {\textcolor{red}{(TBC)}}                                                   \\ \hline
\end{tabular}
\caption{Especificaciones de disponibilidad.}
\end{table}

\begin{table}[H]
\centering
\begin{tabular}{|l|l|l|l|}
\hline
\textbf{ID} & \textbf{Descripción}                                      & \textbf{Origen} & \textbf{\begin{tabular}[c]{@{}l@{}}Aplicabilidad\\ Validación\end{tabular}} \\ \hline
RAM-CON-01  & El producto deberá tener una vida útil no menor a 2 años. & REQ-18          & P F - D                                                                     \\ \hline
\end{tabular}
\caption{Especificaciones de confiabilidad.}
\end{table}