Se tiene un único cliente principal el cual expresa requisitos de mínima y máxima. Los requisitos de mínima engloban una autonomía del producto de por lo menos 3 meses, el sistema no debe llamar la atención de humanos desde el nivel del piso, adquisición de temperatura y luz dentro del nido, robusto ante cambios de temperatura (-5 C$^{\circ}$ $\sim$ 20 C$^{\circ}$), capacidad de almacenamiento de datos por 7 dias, costo máximo por unidad de 300 USD, capacidad de transferencia de datos inalámbrica desde la base del arbol (4 $\sim$ 14 m) y determinar si es realizable un dispositivo de carga inalámbrica en las condiciones impuestas por el nido. Mientras que los requisitos de máxima son 
desarrollar un equipo de carga inalámbrica multi-axial en las condiciones impuestas por el nido, carga inalambrica en 6-7 horas, autonomía del producto de por lo menos 3 meses, el sistema no debe llamar la atención de humanos desde el nivel del piso, adquisición de temperatura y luz dentro del nido, robusto ante cambios de temperatura (-5 C$^{\circ}$ $\sim$ 20 C$^{\circ}$), capacidad de almacenamiento de datos por 15 dias, costo máximo por unidad de 300 USD, capacidad de transferencia de datos inalámbrica desde la base del   (4 $\sim$ 14 m), sistema con capacidad de traslado, adquisición de video y sonido, adquisición de horario de entrada y salida de pajaro del nido y su identificación.

\begin{table}[H]
\centering
\begin{tabular}{|l|l|l|l|}
\hline
\textbf{ID}         & \textbf{Descripción}                                                                                                                                                            & \textbf{Origen} & \begin{tabular}[c]{@{}l@{}}Aplicabilidad\\ Validación\end{tabular} \\ \hline
RAM-MAN-01 & \begin{tabular}[c]{@{}l@{}}En caso de utilizar SW o FW, deberá ser posible para técnicos\\ calificados realizar actualizaciones del mismo\end{tabular}                 & (TBC)  & (TBC)                                                              \\ \hline
RAM-MAN-02 & \begin{tabular}[c]{@{}l@{}}En caso de utilizar SW o FW, no deberá ser posible para el\\ usuario acceder al mismo.\end{tabular}                                         & (TBC)  & (TBC)                                                              \\ \hline
RAM-MAN-03 & \begin{tabular}[c]{@{}l@{}}El equipo deberá contener la siguiente documentación: Manual\\ de Usuario - Esquemáticos de circuitos - Esquemáticos de placas\end{tabular} & (TBC)  & (TBC)                                                              \\ \hline
\end{tabular}
\caption{Especificaciones de mantenibilidad.}
\end{table}

\begin{table}[H]
\centering
\begin{tabular}{|l|l|l|l|}
\hline
\textbf{ID}         & \textbf{Descripción}                                                                                                                                                                                                                                                                                                                                                                                                                                                                               & \textbf{Origen} & \begin{tabular}[c]{@{}l@{}}Aplicabilidad\\ Validación\end{tabular} \\ \hline
RAM-DIS-01 & \begin{tabular}[c]{@{}l@{}}El equipo deberá tener una disponibilidad no menor al TBD\% a lo largo del\\ total de la vida útil. Nótese que esto implica también el tiempo total que el\\ equipo puede estar en estado de falla/reparación. Con una vida útil de TBD\\ meses, un 99.9\% de disponibilidad implica que el equipo no puede no estar\\ disponible más de TBD minutos. Esto tiene impactos relevantes en las\\ facilidades de diagnóstico y reparación necesarias.\end{tabular} & (TBC)  & (TBC)                                                              \\ \hline
\end{tabular}
\caption{Especificaciones de disponibilidad.}
\end{table}

\begin{table}[H]
\centering
\begin{tabular}{|l|l|c|c|}
\hline
\multicolumn{1}{|c|}{\textbf{ID}}     & \multicolumn{1}{c|}{\textbf{Descripción}}                                     & \textbf{Origen}                             & \textbf{\begin{tabular}[c]{@{}c@{}}Aplicabilidad\\ Validación\end{tabular}} \\ \hline
                                      &                                                                               &                         & { (TBC)}                                                   \\ \cline{4-4} 
\multirow{-2}{*}{RAM-CON-01} & \multirow{-2}{*}{El producto deberá tener una vida útil no menor a TBD meses} & \multirow{-2}{*}{{(TBC)}} & { (TBC)}                                                   \\ \hline
\end{tabular}
\caption{Especificaciones de confiabilidad.}
\end{table}

\begin{table}[H]
\begin{tabular}{|l|l|l|l|}
\hline
\textbf{ID}         & \textbf{Descripción}                                                                                                                                                                                  & \textbf{Origen} & \begin{tabular}[c]{@{}l@{}}Aplicabilidad\\ Validación\end{tabular} \\ \hline
IMP-DIM-01 & \begin{tabular}[c]{@{}l@{}}El equipo dentro del nido no deberá exceder las siguientes dimensiones:\\ Largo \textless 26 cm \\ Ancho \textless 8,79 cm \\ Alto \textless 4,55 cm\end{tabular} & REQ-10 & P F - I D                                                          \\ \hline
IMP-DIM-02 & \begin{tabular}[c]{@{}l@{}}El equipo fuera del nido no deberá exceder las siguientes dimensiones:\\ Largo \textless TBD \\ Ancho \textless TBD \\ Alto \textless TBD\end{tabular}            & REQ-10 & P F - I D                                                          \\ \hline
IMP-DIM-03 & \begin{tabular}[c]{@{}l@{}}Los paneles solares no deberán exceder las siguientes dimensiones:\\ Largo \textless TBD \\ Ancho \textless TBD \\ Alto \textless TBD\end{tabular}                & REQ-10 & P F - I D                                                          \\ \hline
IMP-DIM-04 & El equipo dentro del nido no deberá exceder los TBD gramos                                                                                                                                   & REQ-10 & P F - I D                                                          \\ \hline
IMP-DIM-05 & El equipo fuera del nido no deberá exceder los TBD gramos                                                                                                                                    & REQ-10 & P F - I D                                                          \\ \hline
IMP-DIM-06 & Los paneles solares no deberan exceder los TBD gramos                                                                                                                                        & REQ-10 & P F - I D                                                          \\ \hline
\end{tabular}
\caption{Especificaciones de dimensionales y de peso.}
\end{table}

\begin{table}[H]
\begin{tabular}{llll}
\textbf{ID}         & \textbf{Descripción}                                                                                                      & \textbf{Origen} & \begin{tabular}[c]{@{}l@{}}Aplicabilidad\\ Validación\end{tabular} \\
IMP-EMC-01 & El dispositivo deberá poder operar normalmente con inmunidad al ruido electromagnético de acuerdo a la norma TBD & REQ-09 & F - D T                                                           
\end{tabular}
\caption{Especificaciones de compatibilidad electromagnética.}
\end{table}

%\Subsubsection{Relevamiento de Datos}
%\label{sec:RelevamientoDatos}
%La adquisición de datos para fijar los requerimientos del cliente fue realizada mediante sucesivas reuniones con el equipo de ornitólogas, las cuales informaron de las necesidades del producto para llevar a cabo su investigación, dado que son nuestro único cliente principal.

Además, se tuvieron en cuenta las diversas normas que rigen los equipos electrónicos vigentes en Argentina como se detalla en la Sección (\ref{sec:EspecificacionesDiseño}). 
%
%\Subsubsection{Casa de Calidad}
%\label{sec:CasaCalidad}
%\input{RequerimientosCliente/CasaCalidad/CasaCalidad.tex}
%
%\Subsubsection{Requerimientos Finales para Trazabilidad}
%\label{sec:RequerimientosTrazabilidad}
%\begin{table}[H]
\centering
\begin{tabular}{|l|l|l|}
\hline
\multicolumn{1}{|c|}{\textbf{ID}} & \multicolumn{1}{c|}{\textbf{Descripción}}                                                                                                                                                                                                          & \multicolumn{1}{c|}{\textbf{Origen}} \\ \hline
\textbf{REQ-01}                   & \begin{tabular}[c]{@{}l@{}}El producto estará colgado de un árbol a una altura de 4 a 14\\ metros y se instalará parcialmente dentro del nido del ave.\end{tabular}                                                                                & Cliente                              \\ \hline
\textbf{REQ-02}                   & \begin{tabular}[c]{@{}l@{}}El producto debe poder mantenerse energizado sin\\ intervención humana.\end{tabular}                                                                                                                                    & Cliente                              \\ \hline
\textbf{REQ-03}                   & \begin{tabular}[c]{@{}l@{}}El producto no debe requerir conexión a la red eléctrica \\ para su funcionamiento.\end{tabular}                                                                                                                        & Tácito                               \\ \hline
\textbf{REQ-04}                   & \begin{tabular}[c]{@{}l@{}}El producto debe ser capaz de adquirir los siguientes \\ datos dentro del nido: imágenes, temperatura, humedad\\ y nivel de luz.\end{tabular}                                                                           & Cliente                              \\ \hline
\textbf{REQ-05}                   & \begin{tabular}[c]{@{}l@{}}Un dispositivo ajeno al proyecto que irá sobre el ave debe poder\\ transmitirle datos al nido mediante protocolo Bluetooth.\end{tabular}                                                                                & Cliente                              \\ \hline
\textbf{REQ-06}                   & \begin{tabular}[c]{@{}l@{}}El producto debe poder almacenar los datos adquiridos \\ por el nido y el ave.\end{tabular}                                                                                                                             & Tácito                               \\ \hline
\textbf{REQ-07}                   & \begin{tabular}[c]{@{}l@{}}Una persona debe poder recibir los datos almacenados en el nido a la\\ distancia.\end{tabular}                                                                                                                          & Cliente                              \\ \hline
\textbf{REQ-08}                   & \begin{tabular}[c]{@{}l@{}}El producto no debe llamar la atención de humanos \\ desde el nivel del piso.\end{tabular}                                                                                                                              & Cliente                              \\ \hline
\textbf{REQ-09}                   & \begin{tabular}[c]{@{}l@{}}El producto debe soportar las condiciones meteorológicas del sur \\ Argentino, específicamente los alrededores de Bariloche, Rio Negro.\end{tabular}                                                                    & Tácito                               \\ \hline
\textbf{REQ-10}                   & El costo del producto debe ser menor o igual a \precio USD.                                                                                                                                                                                        & Tácito                               \\ \hline
\textbf{REQ-11}                   & \begin{tabular}[c]{@{}l@{}}Analizar la factibilidad tecnológica de la recarga inalámbrica de la batería\\ de la mochila del pájaro.\end{tabular}                                                                                                   & Cliente                              \\ \hline
\textbf{REQ-12}                   & \begin{tabular}[c]{@{}l@{}}El producto desarmado debe soportar las condiciones de translado\\  impuestas por los caminos rurales hasta llegar a la zona de instalación.\end{tabular}                                                               & Estado                               \\ \hline
\textbf{REQ-13}                   & La vida útil del producto deberá ser de por lo menos 2 años.                                                                                                                                                                                       & Estado                               \\ \hline
\textbf{REQ-14}                   & \begin{tabular}[c]{@{}l@{}}Se debe garantizar el abastecimiento energético de la mochila \\ del pájaro por una duración de tres a cuatro meses posterior a la\\ intervención en el nido y el pájaro para la instalación del proyecto.\end{tabular} & Cliente                              \\ \hline
\end{tabular}
\caption{Requerimientos de máxima.}
\end{table}
