La principal adversidad del desarrollo del proyecto es la más notable por varias razones. Se puso mucha confianza en que iba a ser factible realizar la carga inalámbrica de la UBM. En retrospectiva, se debería de haber comenzado el proyecto ya con un plan B planteado. Además, en caso de futuros trabajos, se considera que se debe empezar mucho antes con el proceso de investigación. 

En este proyecto se hizo evidente la falta de experiencia y conocimientos previos del manejo de antenas y propagación del equipo. Esto puede verse en el desarrollo de placas de alta frecuencia, por mencionar un ejemplo.

Por último, se proponen recomendaciones para futuros diseños además de la ya mencionada. Estas propuestas surgen a partir de lo experimentado a lo largo del proceso. Una mejora inmediata que se recomienda es la de no utilizar módulos prefabricados. En su lugar emplear circuitos que integren toda electrónica en una sola placa. Otra recomendación es la de usar un \textit{Flex-PCB} para la tecnología situada en la bóveda del nido, debido al poco espacio presente en esta zona.
