Hubo un fallo en el proyecto más notable que el resto. Esto fue idear al proyecto poniendo mucha confianza en que iba a ser factible realizar la carga inalámbrica de la UBM. En retrospectiva, se debería de haber comenzado mucho antes en el transcurso del proyecto con la investigación; y comenzar el proyecto ya con un plan B preparado. 

Por otra parte, hay varias recomendaciones para futuros diseños. Una mejora inmediata que se podría hacer es la de no utilizar módulos pre-fabricados, sino la de integrar esta electrónica en una sola placa. Otra recomendación sería la de usar un \textit{Flex-PCB} para la electrónica que va situada en la bóveda del nido debido al poco espacio presente en esta zona.