%Aplicación de distintas áreas de conocimiento
%Incursión en un proyecto extenso y de mayor complejidad
%Investigación

Comparando los inicios del proyecto con respecto a las instancias finales, se presentan sensaciones ambiguas. No todos los objetivos alcanzados arrojaron los resultados deseados. No obstante, las metas fueron completadas y cubiertas con éxito. 

Se logró obtener un alto nivel de confiabilidad en el producto superando con creces el periodo de tres meses por lo que un mismo dispositivo puede ser utilizado en varias investigaciones al cabo de los años. Además, el uso de los recursos naturales como abastecimiento energético permite que el producto sea utilizado hasta en las zonas más inhóspitas.

Gracias al uso de recursos como Ansible, que permite replicar de manera automática configuraciones en la unidad de procesamiento, este producto se vuelve fácil de replicar sin conocimientos técnicos avanzados tornándolo del tipo \textit{plug and play}. 

Se aplicaron distintas áreas de conocimiento, algunas de estas desarrolladas a lo largo de la carrera, mientras que otras no formaron parte de dicho proceso. Los desafíos que se presentaron muchas veces tuvieron una dificultad mayor a la esperada. A pesar de ello, estas adversidades no fueron impedimento para enfrentarlas y seguir adelante.

Pero la reflexión más importante se centra en la investigación acerca de la carga inalámbrica de la mochila. Como se mencionó, se esperaba que, de una forma u otra, sea posible lograr dicha transferencia de energía. Este desenlace no quita mérito al trabajo ni a su desarrollo. Lo que se destaca no es el resultado sino lo que proveyó la investigación. Se aprendió acerca del uso de antenas, transmisión de potencia por radiofrecuencia, \textit{energy harvesting}, manejo de datos (\textit{databases}) y a desarrollarse en el entorno de investigación, sobre todo a trabajar en el límite del desarrollo tecnológico del campo.
