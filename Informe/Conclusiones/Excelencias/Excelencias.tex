%Aplicación de distintas áreas de conocimiento
%Incursión en un proyecto extenso y de mayor complejidad
%Investigación

Comparando los inicios del proyecto con respecto a las instancias finales, se presentan sensaciones ambiguas. No todos los objetivos alcanzados arrojaron los resultados deseados. No obstante, las metas fueron completadas y cubiertas con éxito.

Se aplicaron distintas áreas de conocimiento, algunas de estas desarrolladas a lo largo de la carrera, mientras que otras no formaron parte de dicho proceso. Los desafíos que se presentaron muchas veces tuvieron una dificultad mayor a la esperada. A pesar de ello, estas adversidades no fueron impedimento para enfrentarlas y seguir adelante.

Pero la reflexión más importante se centra en la investigación acerca de la carga inalámbrica de la mochila. Como se mencionó, se esperaba que, de una forma u otra, sea posible lograr dicha transferencia de energía. Este desenlace no quita mérito al trabajo ni a su desarrollo. Lo que se destaca no es el resultado sino lo que proveyó la investigación. Se aprendió acerca del uso de antenas, transmisión de potencia por radiofrecuencia, \textit{energy harvesting}, manejo de datos (\textit{databases}) y a desarrollarse en el entorno de investigación, sobre todo a trabajar en el límite del desarrollo tecnológico del campo.
