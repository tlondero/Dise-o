%(Mercado, costos, ciclo de vida, %VAN, TIR)

\Subsubsection{Modelo de Negocios}

Este diseño se trata de un proyecto único, con posibilidad de realizar hasta \TBD unidades adicionales posteriores a su finalización.

El principal ingreso corresponderá al diseño del hardware y software para el control del módulo, para lo que se requiere de conocimiento técnico (siendo este el valor agregado del proyecto). Este último factor se lo contempla al finalizar el proyecto.

\Subsubsection{Gastos e ingresos}

Los gastos principales considerados son \TBD. %Para la compra de los componentes, se suman los costos estimados previamente, más el resto de los componentes misceláneos (resistores, capacitores, etc.) para el diseño de los circuitos involucrados.
Se estima de este modo un costo de componentes de \TBD USD, a contabilizar una única vez por unidad.

Se presenta el valor de los insumos de hardware:
\begin{table}[H]
\centering
\begin{tabular}{|c|c|}
\hline
\textbf{Item}                                                         & \textbf{Precio [USD]}				  \\ \hline
Sensor humedad-temperatura 											  & 4.9                                   \\ \hline
Sensor luminosidad                                                    & 1.54                                  \\ \hline
Cámara                                                                & 20                                    \\ \hline
SDI 32GBy                                                             & 32                                    \\ \hline
R-Pi Zero W                                                           & 25.5                                  \\ \hline
Batería                                                               & \TBD                   				  \\ \hline
Panel solar                                                           & \TBD                   				  \\ \hline
P2110                                                                 & 32                                    \\ \hline
Antena RX                                                             & \TBD                   				  \\ \hline
Antena TX                                                             & \TBD                   				  \\ \hline
MPPT                                                                  & 33                                    \\ \hline
Encapsulado                                                           & 16                                    \\ \hline
Montaje                                                               & 31                                    \\ \hline
\end{tabular}
\caption{Valores de insumos.}
\end{table}


En cuanto a los gastos en mano de obra, se estima un sueldo de $30 \ \frac{USD}{hora}$. Así multiplicandolo por la cantidad de horas de ingeniería esperada por persona se obtiene un total de: $$Costo \ mano \ de \ obra = 7100 \ hora \ \cdot 30 \ \frac{USD}{hora} = 213.000,00 \ USD$$

\observacion{\verObs}{Hablar de reserva de contingencia}

\observacion{\verObs}{Costo de desarrollo de N unidades}

\observacion{\verObs}{Modelo canvas como si lo fuesemos a vender?}

Como ingreso, se tiene en cuenta una suma de \TBD USD contemplando los gastos que no estén siendo tenidos en cuenta y el resto será la ganancia del proyecto. Se apunta a obtener una ganancia de \TBD.

%\Subsubsection{Flujo de fondos}
%
%Revisando los ítems anteriores, se arma el siguiente cuadro con el flujo de fondos para todo el proyecto.
