%\documentclass[a4paper]{article}
\usepackage[utf8]{inputenc}
\usepackage[spanish, es-tabla, es-noshorthands]{babel}
\usepackage[table,xcdraw]{xcolor}
\usepackage[a4paper, footnotesep=1.25cm, headheight=1.25cm, top=2.54cm, left=2.54cm, bottom=2.54cm, right=2.54cm]{geometry}
%\geometry{showframe}
%VERIFICAR EL HEAD Y EL FOOT EN
%https://ctan.dcc.uchile.cl/macros/latex/contrib/geometry/geometry.pdf

%\usepackage{lipsum}			%LOREM IPSUM

%\usepackage{wrapfig}		%Wrap figure in text
\usepackage[export]{adjustbox}	%Move images
\usepackage{changepage}			%Move tables

%Font
\usepackage{anyfontsize}	%Font size
% #1 = size, #2 = text
\newcommand{\setparagraphsize}[2]{{\fontsize{#1}{6}\selectfont#2 \par}}		%Cambia el size de todo el parrafo
\newcommand{\setlinesize}[2]{{\fontsize{#1}{6}\selectfont#2}}				%Cambia el font de una oración


%FONTS (IMPORTANTE): Compilar en XeLaTex o LuaLaTeX
\usepackage{fontspec}
%Si sigue sin andar comentar \usepackage[utf8]{inputenc}
%https://ctan.dcc.uchile.cl/macros/unicodetex/latex/fontspec/fontspec.pdf
%https://www.overleaf.com/learn/latex/XeLaTeX

\usepackage{tikz}
\usepackage{amsmath}
\usepackage{amsfonts}
\usepackage{amssymb}
\usepackage{float}
\usepackage{graphicx}
\usepackage{caption}
\usepackage{subcaption}
\usepackage{multicol}
\usepackage{multirow}
\setlength{\doublerulesep}{\arrayrulewidth}
\usepackage{booktabs}

\usepackage{hyperref}
\hypersetup{
    colorlinks=true,
    linkcolor=black,
    filecolor=magenta,      
    urlcolor=blue,
    citecolor=blue,    
}

\newcommand{\captionsection}{\setcounter{figure}{0} \renewcommand{\thetable}{\arabic{section}.\arabic{table}} \renewcommand{\thefigure}{\arabic{section}.\arabic{figure}}}

\newcommand{\captionsubsection}{\setcounter{figure}{0} \renewcommand{\thetable}{\arabic{section}.\arabic{subsection}.\arabic{table}} \renewcommand{\thefigure}{\arabic{section}.\arabic{subsection}.\arabic{figure}}}

\newcommand{\captionsubsubsection}{\setcounter{figure}{0} \renewcommand{\thetable}{\arabic{section}.\arabic{subsection}.\arabic{subsubsection}.\arabic{table}} \renewcommand{\thefigure}{\arabic{section}.\arabic{subsection}.\arabic{subsubsection}.\arabic{figure}}}

%PICTURES AND TABLE INDEX:
\newcommand{\Section}[1]{ \section{#1}\captionsection }
\newcommand{\Subsection}[1]{ \subsection{#1}\captionsubsection }
\newcommand{\Subsubsection}[1]{ \subsubsection{#1}\captionsubsubsection }

%NOTAS GRANDES
\newcommand{\note}[1]{
	\begin{center}
		\huge{ \textcolor{red}{#1} }
	\end{center}
}
%Notas pequeñas
\newcommand{\lnote}[1]{{\fontsize{20}{6}\selectfont\textcolor{green}{#1}}}

\newcommand{\quotes}[1]{``#1''}
\usepackage{array}
\newcolumntype{C}[1]{>{\centering\let\newline\\\arraybackslash\hspace{0pt}}m{#1}}
\usepackage[american]{circuitikz}
\usetikzlibrary{calc}
\usepackage{fancyhdr}
\usepackage{units} 

\graphicspath{{../Prefacio/}{../Acronimos/}{../Introduccion/}{../Objetivos/}{../Definicion/}{../Plan de validacion/}{../Resumen/}}

%COLORES
\definecolor{AzulFoot}{rgb}{0.682,0.809,0.926}	%RGB	%{174,206,235}
\definecolor{AzulInfo}{rgb}{0.180,0.455,0.710}	%RGB	%{46,116,181}
\definecolor{AzulTable}{rgb}{0.302,0.507,0.871}	%RGB	%{68,114,196}
\definecolor{PName}{rgb}{0.353,0.353,0.353}	%RGB	%{90,90,90}

\usepackage{xcolor}
\usepackage{sectsty}
\chapterfont{\color{AzulInfo}}  % sets colour of chapters
\sectionfont{\color{AzulInfo}}  % sets colour of sections
\subsectionfont{\color{AzulInfo}}
\subsubsectionfont{\color{AzulInfo}}

%Header y footer
\usepackage{etoolbox}

\pagestyle{fancy}
\fancyhf{}
\rfoot{\thepage}
\renewcommand{\footrulewidth}{4pt}
\renewcommand{\headrulewidth}{0pt}
\patchcmd{\footrule}{\hrule}{\color{AzulFoot}\hrule}{}{}
%
%\begin{document}

\Subsection{Factibilidad Tecnológica}

\Subsubsection{Esquema Modular}

A continuación, se presentan los distintos módulos que integran el proyecto. Estos fueron agrupados de forma tal que se puedan comunicar de manera ordenada y precisa entre sí.
\begin{figure}[H]
	\centering
	\includegraphics[width=0.9\linewidth]{ImagenesFactibilidad/EsquemaModular}
	\caption{Diagrama modular del sistema.}
	\label{fig:esquema_modular}
\end{figure}

Además, para un mejor entendimiento, se subdivido el módulo de \quotes{Sensores}, ya que de esta forma se interpreta mejor su funcionamiento.

Existen distintas alternativas al momento del desarrollar e implementar los nodos empleados en el trabajo. Estas variedades de implementación se analizan con más profundidad en las siguientes secciones.

\Subsubsection{Propuesta de Sensores}
Para las distintas mediciones se tuvieron en cuenta diversas tecnologías que existen. Se evaluaron parámetros que definen la performance, tales como la linealidad de salida, el costo, el rango de operación, la precisión, el tipo de salida, aplicación, entre otras tantas variables.

\Subsubsubsection{Temperatura}
En el caso de la medición de temperatura, se valoraron diversas tecnologías que existen, siendo por ejemplo la RTD cuyo funcionamiento se basa en el cambio de la resistencia en función de la temperatura bajo al ecuación $R(T)=R_0 + \alpha \cdot \Delta T$. También se consideró la tecnología TC, cuyo funcionamiento se basa en el efecto seebek. Finalmente, el uso de un IC, el cual se basa en propiedades de dispositivos semiconductores extrínsecos.

\begin{table}[H]
\centering
\begin{tabular}{|c|c|c|c|c|}
\hline
\textbf{\begin{tabular}[c]{@{}c@{}}Aspectos\\ comparativos\end{tabular}} & \textbf{\href{https://www.thermocoupleinfo.com/type-k-thermocouple.htm}{TC-K}}                                                                             & \textbf{\href{http://www.datasheet.es/PDF/900325/Pt100-pdf.html}{PT-100}}                                                                                                              & \textbf{\href{https://datasheets.maximintegrated.com/en/ds/DS18B20.pdf}{Ds18b20}}                                                 & \textbf{\href{https://www.sparkfun.com/datasheets/Sensors/Temperature/DHT22.pdf}{DHT-22}}  \\ \hline
\textbf{Costo [USD]}                                                           & 4.6	& 5.2	& 1.4	& 4.9	\\ \hline
\textbf{Tipo de salida}                                                  & Analógico                                                                                 & Analógico                                                                                                                    & Digital                                                          & Digital          \\ \hline

\textbf{\begin{tabular}[c]{@{}c@{}}Temperatura de\\ operación [°C]\end{tabular}}                                              & -40 $\sim$ 1200                                                                        & -50 $\sim$ 200  & -10 $\sim$ 85 & -40 $\sim$ 80 \\ \hline
\textbf{\begin{tabular}[c]{@{}c@{}}Interfaz de\\ conexionado\end{tabular}}                                                      & \begin{tabular}[c]{@{}c@{}}Se debe\\ proporcionar\\ un circuito\\ amplificador\end{tabular} & \begin{tabular}[c]{@{}c@{}}Se debe \\ proporcionar\\  un circuito\\  convertidor\\  de resistencia\\  a tensión\end{tabular} & -                                                                & -                \\ \hline
\textbf{Presición [°C]}                                                       & $\pm$ 1.5	& $\pm$ 0.1	& $\pm$ 0.5	& $\pm$ 0.5	\\ \hline

\textbf{Estabilidad}                                                     & Tienden a envejecer                                                                       & -                                                                                                                            & -                                                                & -                \\ \hline
\textbf{Autocalentamiento}                                               & -                                                                                         & \begin{tabular}[c]{@{}c@{}}Depende de\\  la corriente\\  de medición.\end{tabular}                                           & Bajo                                                             & Bajo             \\ \hline
\textbf{Imagen}                &  \includeintable{.1}{ImagenesFactibilidad/TC}                                                                                          &  \includeintable{.1}{ImagenesFactibilidad/PT100}                                                                                                                    & \includeintable{.1}{ImagenesFactibilidad/IC1}  &  \includeintable{.1}{ImagenesFactibilidad/DHT-22}                 \\ \hline
\end{tabular}
\caption{Comparación entre sensores de temperatura.}
\end{table}

\Subsubsubsection{Humedad}
Existen varias maneras de medir la magnitud física de la humedad, dentro de estas la mas común se basa en utilizar la dependencia que existe entre la humedad y la capacidad. Es por esto que se utilizan capacitores con un dieléctrico, el cual cambia constante con la humedad. Además existen sensores que se aprovechan de como cambia la resistencia en función de la temperatura, pero estas tecnologías son menos frecuentes.

\begin{table}[H]
\centering
\begin{tabular}{|c|c|c|c|c|}
\hline
\textbf{\begin{tabular}[c]{@{}c@{}}Aspectos\\ comparativos\end{tabular}} & \textbf{\href{https://www.mouser.com/datasheet/2/758/DHT11-Technical-Data-Sheet-Translated-Version-1143054.pdf}{DHT-11}}   & \textbf{\href{http://codigoelectronica.com/blog/am2301-datasheet}{AM-2301}}  & \textbf{\href{https://www.sparkfun.com/datasheets/Sensors/Temperature/DHT22.pdf}{DHT-22}}   & \textbf{\href{https://datasheetspdf.com/pdf/1298922/AOSONG/AM1001/1}{AM-1001}}  \\ \hline
\textbf{Costo [USD]}                                                           & 1.3           & 7 & 4.93 & 5.6	\\ \hline
\textbf{\begin{tabular}[c]{@{}c@{}}Rango de\\ operación [\%RH]\end{tabular}}                                              & 20 $\sim$ 90 & 0 $\sim$ 100 & 0 $\sim$ 100 & 20 $\sim$ 90 \\ \hline
\textbf{Presición [\%RH]}                                                       & $\pm$4	& $\pm$3	 & $\pm$2	& $\pm$5	\\ \hline
\textbf{Tipo de salida}                                                  & Digital           & Digital           & Digital           & Analógica         \\ \hline
\textbf{Imagen}                                                          & \includeintable{.1}{ImagenesFactibilidad/DHT-11}                 & \includeintable{.1}{ImagenesFactibilidad/AM-2301}                 & \includeintable{.1}{ImagenesFactibilidad/DHT-22}                 & \includeintable{.1}{ImagenesFactibilidad/AM-1001} \\ \hline
\end{tabular}
\caption{Comparación de sensores de humedad.}
\end{table}

\Subsubsubsection{Luminosidad}
En la medición del nivel de luminosidad se puede optar por diversos caminos. Existen sensores como el BH-1750 y OPT-100 que su funcionamiento se basa en un fotodiodo que conduce cierta corriente a partir de la luz que le impacta. Otros sensores, tales como el TEMT-600, emplean un fototransistor, cuya base se encuentra expuesta. En función de la intensidad lumínica en dicha zona, circulará cierta corriente por el colector. Finalmente existen fotoresistores, los cuales, tal como su nombre indica, cambian la resistencia en función del nivel de luz.

\begin{table}[H]
\centering
\begin{tabular}{|c|c|c|c|c|}
\hline
\textbf{\begin{tabular}[c]{@{}c@{}}Aspectos\\ comparativos\end{tabular}}              & \textbf{\href{https://www.mouser.com/datasheet/2/348/bh1750fvi-e-186247.pdf}{BH-1750}}  & \textbf{\href{https://www.vishay.com/docs/81579/temt6000.pdf}{TEMT-6000}}	& \textbf{\href{https://www.ti.com/lit/ds/symlink/opt101.pdf}{OPT-101}}                                              & \textbf{\href{https://www.alldatasheet.es/view.jsp?Searchword=GL55}{GL55-LM393}} \\ \hline
\textbf{Costo [USD]}	& 1.54 	& 2.27	& 2.26	& 1.26	\\ \hline
\textbf{\begin{tabular}[c]{@{}c@{}}Temperatura de\\ operación [°C]\end{tabular}} & -40 $\sim$ 85  & -40 $\sim$ 85 & 0 $\sim$ 70 & -30 $\sim$ 70 \\ \hline
\textbf{\begin{tabular}[c]{@{}c@{}}Potencia\\ disipada [mW]\end{tabular}}                                                            & - 	& 100 	& - 	& 75 	\\ \hline
\textbf{Tipo de salida}                                                               & I2C               & \begin{tabular}[c]{@{}c@{}}Analógica\\ (Corriente)\end{tabular}                 & \begin{tabular}[c]{@{}c@{}}Analógica\\ (Tensión)\end{tabular} &\begin{tabular}[c]{@{}c@{}}Analógica \\ Digital\end{tabular}  \\ \hline
\textbf{Aplicación}                                                                   & -                 & \begin{tabular}[c]{@{}c@{}}Necesita un\\amplificador\\ de corriente\end{tabular} & -                                                             & -                   \\ \hline
\begin{tabular}[c]{@{}c@{}}\textbf{Tensión de} \\ \textbf{alimentación [V]} \end{tabular}                                                      & 2.4$\sim$3.6 	& \textless \ 6 	& 2.7 $\sim$ 36.0 	& 3.3 $\sim$ 5.0 	\\ \hline
\textbf{\begin{tabular}[c]{@{}c@{}}Rango de\\ medición [nm]\end{tabular}}                                                            & 450 $\sim$ 650 & 400 $\sim$ 900 & 450 $\sim$ 1000 & 450 $\sim$ 750 \\ \hline
\textbf{Imagen}                                                                       & \includeintable{.1}{ImagenesFactibilidad/BH-1750} & \includeintable{.1}{ImagenesFactibilidad/TEMT-6000}                                                                               & \includeintable{.1}{ImagenesFactibilidad/OPT-101} & \includeintable{.1}{ImagenesFactibilidad/GL55-LM393}                   \\ \hline
\end{tabular}
\caption{Comparación de sensores de luminosidad.}
\end{table}


\Subsubsubsection{Imágenes}
Para la obtención de captura imágenes y teniendo en cuenta la tecnología utilizada para la unidad de procesamiento, se encontraron diversos módulos de cámara que se pueden usar:

\begin{table}[H]
\centering
\begin{tabular}{|c|c|c|c|c|}
\hline
\textbf{\begin{tabular}[c]{@{}c@{}}Aspectos\\ comparativos\end{tabular}} 					& \textbf{\href{https://cdn.sparkfun.com/datasheets/Dev/RaspberryPi/ov5647_full.pdf}{RPi-CMOD-V1}}      & \textbf{\href{https://www.arducam.com/sony/imx477/}{RPi-CMOD-V2}}     & \textbf{\href{http://www.opensourceinstruments.com/Electronics/Data/IMX219PQ.pdf}{RPi-HQC}}  		& \textbf{\href{https://www.digikey.com/catalog/es/partgroup/zero-spy-camera-for-raspberry-pi-zero-board/70309}{RPi-ZEROC}}     \\ \hline
\textbf{Costo [USD]}                                                     					& 25		                        																	& 25 	                           										& 50 																								& 20                          																									\\ \hline
\textbf{Tamaño [mm]}                                                 	 					& 25 x 24 x 9                      																		& 25 x 24 x 9                      										& 38 x 38 x 18.4     																				& 8.6 x 8.6 x 5.2            																									\\ \hline
\textbf{\begin{tabular}[c]{@{}c@{}}Resolución de la\\ cámara [MP]\end{tabular}}             & 5	                               																		& 8	                               										& 12.3   																							& 5                          																									\\ \hline
\textbf{Integración Linux}                                               					& V4L2 driver                     																		& V4L2 driver                      										& V4L2 driver																						& V4L2 driver                      																								\\ \hline
\textbf{C API}                                                           					& OpenMAX IL y otras              																		& OpenMAX IL y otras               										& -         																						& -                       																										\\ \hline
\textbf{Peso [g]}                                                        					& 3                               																		& 3.4                              										& 53                           																		& 1.1    																														\\ \hline
\textbf{Sensor}                                                          					& OmniVision OV5647                																		& Sony IMX219                      										& Sony IMX477  																						& OV5647                    																									\\ \hline
\textbf{\begin{tabular}[c]{@{}c@{}}Temperatura de\\ operación [°C]\end{tabular}}            & \multicolumn{1}{c|}{-25$\sim$80} 																		& \multicolumn{1}{c|}{-25$\sim$80} 										& \multicolumn{1}{c|}{-25$\sim$80} 																	& \multicolumn{1}{c|}{-25$\sim$80} 																								\\ \hline
\textbf{Imagen}                                                          					& \includeintable{.1}{ImagenesFactibilidad/RPICAMV1}                  									& \includeintable{.1}{ImagenesFactibilidad/RPICAMV2}                    & \includeintable{.1}{ImagenesFactibilidad/RPIHQCAM}         &
\includeintable{.1}{ImagenesFactibilidad/RPIZEROCAM}       \\ \hline
\end{tabular}
\caption{Comparación entre cámaras.}
\end{table}


\Subsubsubsection{Modulo RTC}
Es fundamental contar con un m\'etodo para conocer la fecha y hora de manera confiable. Tanto como para saber en que momento fueron tomadas las mediciones como as\'i tambi\'en saber en que momentos habilitar el hotspot wifi.
Teniendo esto en cuenta se proponen los siguientes m\'odulos:
\begin{table}[]
\centering
\begin{tabular}{|c|c|c|c|}
\hline
\textbf{Aspectos comparativos}                                                & Ds3231         & Ds1302         & Ds1307         \\ \hline
\textbf{Costo {[}USD{]}}                                                      & 2.33           & 0.81           & 1.37           \\ \hline
\textbf{\begin{tabular}[c]{@{}c@{}}Temperatura \\ de Operación\end{tabular}}  & -40 $\sim$85   & -40 $\sim$85   & -40 $\sim$85   \\ \hline
\textbf{Rango de tensiones}                                                   & 2.3 $\sim$5.5  & 2 $\sim$5.5    & 4.5$\sim$5.5   \\ \hline
\textbf{\begin{tabular}[c]{@{}c@{}}Protocolo de \\ Comunicación\end{tabular}} & I2C            & SPI            & I2C            \\ \hline
\textbf{Precisión}                                                            & 3.5ppm         & -              & -              \\ \hline
\textbf{Imagen}                                                               & \includeintable{.1}{ImagenesFactibilidad/DS3231} & \includeintable{.1}{ImagenesFactibilidad/ds1302} & \includeintable{.1}{ImagenesFactibilidad/ds1307} \\ \hline
\end{tabular}
\end{table}




\Subsubsection{Propuesta de Almacenamiento}
Para almacenar información, se considera la utilización memorias SD debido a su compacto tamaño comparado a la densidad de información que puede contener. Existe una gran variedad de memorias SD, permitiendo priorizar diversos aspectos a la hora de optar por una opción. La velocidad de lectura, la de escritura y el almacenamiento son algunos de estos aspectos, aunque en este proyecto también es importante considerar el rango de temperatura de operación.

\begin{table}[H]
\centering
\begin{tabular}{|c|c|c|c|}
\hline
\textbf{\begin{tabular}[c]{@{}c@{}}Aspectos\\ comparativos\end{tabular}}         & \textbf{\href{https://www.kingston.com/datasheets/sdcg3_es.pdf}{SDCG3}} & \textbf{\href{https://www.kingston.com/datasheets/mlpmr2_es.pdf}{SDCE}} & \textbf{\href{https://ar.mouser.com/datasheet/2/669/SanDisk_02052018_SDSDAF3_SDSDQAF3-1285144.pdf}{SDSDQAF3-XI}} \\ \hline
\textbf{Costo [USD]}                                                             & 20                                                        & 52 & 22                                                                                                 \\ \hline
\textbf{\begin{tabular}[c]{@{}c@{}}Temperatura de\\ operación [°C]\end{tabular}} & -25 $\sim$ 85                                                           & -25 $\sim$ 85                                                           & -40 $\sim$ 85                                                                                                    \\ \hline
\textbf{Almacenamiento [GB]}                                                     & 64 $\sim$ 512                                                           & 64 $\sim$ 256                                                           & 8 $\sim$ 128                                                                                                     \\ \hline
\textbf{\begin{tabular}[c]{@{}c@{}}Velocidad\\ R/W [MB/s]\end{tabular}}          & 170 / 90                                                                & 285 / 165                                                               & 50 / 80                                                                                                          \\ \hline
\textbf{Alimentación [V]}                                                        & 3.3                                                                     & 3.3                                                                     & 2.7 $\sim$ 3.6                                                                                                   \\ \hline
\textbf{Imagen}                                                                  & \includeintable{.1}{ImagenesFactibilidad/SDCG3}                         & \includeintable{.1}{ImagenesFactibilidad/SDCE}                          & \includeintable{.1}{ImagenesFactibilidad/SDSDQAF3}                                                               \\ \hline
\end{tabular}
\caption{Comparación entre memorias SD.}
\label{comp:sd}
\end{table}

%\Subsubsection{Propuesta de Comunicación}
%En cuanto a la comunicación se utilizará BLE para la conexión con el ave, y WiFi para la comunicación con un tercero.

\Subsubsection{Propuesta de Unidad de Procesamiento}
\label{sec:antenas}
La UP representa el cerebro del proyecto. Este se ocupa de procesar la información de los sensores, almacenarla en la SD e iniciar los procesos de comunicación. En otras palabras, el integrado se ocupa de conectar los distintos módulos entre sí y garantizar su adecuado funcionamiento.

\begin{table}[H]
\centering
\begin{tabular}{|c|c|c|c|}
\hline
\textbf{\begin{tabular}[c]{@{}c@{}}Aspectos\\ comparativos\end{tabular}}         & \textbf{\href{https://datasheets.raspberrypi.org/rpi4/raspberry-pi-4-product-brief.pdf}{RPi 4}} & \textbf{\href{https://www.raspberrypi.org/products/raspberry-pi-zero-w/}{RPi Zero W}}            		& \href{https://static.raspberrypi.org/files/product-briefs/Raspberry-Pi-Model-Bplus-Product-Brief.pdf}{RPi 3B} 	\\ \hline
\textbf{Costo [USD]}                                                             & 55                                                                              & 25.5 		& 100 	\\ \hline
\textbf{\begin{tabular}[c]{@{}c@{}}Temperatura de\\ operación [°C]\end{tabular}} & 0 $\sim$ 50                                                                                     & -20 $\sim$ 85 																							& -20 $\sim$ 85                                                                                                                                                                                                      	\\ \hline
\textbf{Memoria}                                                            	 & 1 [GB] $\sim$ 8 [GB]                                                                            & 512 [MB]                                                                                             	& 1 [GB] $\sim$ 8 [GB]                                                                                                                                                                                                  \\ \hline
\textbf{Conexiones}                                                              & \begin{tabular}[c]{@{}c@{}}Wireless LAN,\\ Bluetooth 5.0,\\ Ethernet, USB\end{tabular}          & \begin{tabular}[c]{@{}c@{}}Wireless LAN,\\ Bluetooth 4.1 (BLE),\\ Micro USB, mini HDMI\end{tabular} 	& \begin{tabular}[c]{@{}c@{}}Wireless LAN,\\ Bluetooth 5.0 (BLE),\\ Ethernet, USB,\\ antena externa\end{tabular}                                                                                                        \\ \hline
\textbf{Sonido y video}                                                          & \begin{tabular}[c]{@{}c@{}}Micro HDMI,\\ MIPI DSI y CSI\end{tabular}                            & \begin{tabular}[c]{@{}c@{}}Mini HDMI, HDMI,\\ CSI, PAL/NTSC pads\end{tabular}                   		& \begin{tabular}[c]{@{}c@{}}HDMI, MIPI DSI\\ y CSI, SDIO\end{tabular}                                                                                                                                               	\\ \hline
\textbf{Soporte SD}                                                              & \begin{tabular}[c]{@{}c@{}}Almacenamiento y\\ carga de SO\end{tabular}                          & Micro SD                                                                                        		& \begin{tabular}[c]{@{}c@{}}Entrada SD para tarjeta\\ o eMMC externo\end{tabular}                                                                                                                                   	\\ \hline
\textbf{Dimensiones [mm]}                                                        & 85.6 x 56.5                                                                                     & 65 x 30                                                                                         		& 40 x 55 																																																				\\ \hline
\textbf{Alimentación}                                                            & 5 V (3 A)                                                                                       & 5 (1.2 A)                                                                                       		& 5 V (1.4 A)                                                                                                                                                                                                        	\\ \hline
\textbf{Imagen}                                                                  & \includeintable{.1}{ImagenesFactibilidad/RPI4}                                   			   & \includeintable{.1}{ImagenesFactibilidad/RPIZero}                                						& \includeintable{.1}{ImagenesFactibilidad/RPICM}                                                                                                                                                                    	\\ \hline
\end{tabular}
\caption{Comparación entre placas \rspi.}
\end{table}

%\lnote{Creo que el procesador de todas va más o menos de -40 °C a 85 °C, pero las demás cosas lo limitan, por ejemplo el modulo de LAN creo que no va menos de 0 °C.}

%https://copperhilltech.com/content/The%20Operating%20Temperature%20For%20A%20Raspberry%20Pi%20%E2%80%93%20Technologist%20Tips.pdf

\Subsubsection{Propuesta de Batería}
\begin{table}[H]
\centering
\begin{tabular}{|c|c|c|c|c|}
\hline
\textbf{\begin{tabular}[c]{@{}c@{}}Aspectos\\ comparativos\end{tabular}}                          & \textbf{\begin{tabular}[c]{@{}c@{}}\href{https://www.kijo-battery.com/products/jdg-series-agm-gel-deep-cycle-battery.html}{Kijo Serie}\\ \href{https://www.kijo-battery.com/products/jdg-series-agm-gel-deep-cycle-battery.html}{JDG}\end{tabular}} & \textbf{\begin{tabular}[c]{@{}c@{}}\href{https://www.kijo-battery.com/products/jlg-series-pure-gel-deep-cycle-battery.html}{Kijo Serie}\\ \href{https://www.kijo-battery.com/products/jlg-series-pure-gel-deep-cycle-battery.html}{JLG}\end{tabular}} & \textbf{\begin{tabular}[c]{@{}c@{}}\href{http://www.fenk.com.ar/wp-content/uploads/2020/01/JS12-20-1.pdf}{Fenk}\\ \href{http://www.fenk.com.ar/wp-content/uploads/2020/01/JS12-20-1.pdf}{JS12-20}\end{tabular}} & \textbf{\begin{tabular}[c]{@{}c@{}}\href{http://www.fenk.com.ar/productos/energias-renovables/baterias-solares/}{Fenk Serie}\\ \href{http://www.fenk.com.ar/productos/energias-renovables/baterias-solares/}{JM12}\end{tabular}} \\ \hline
\textbf{Costo [ARS]}                                                                              & 30500 $\sim$ X                                                                                                                                                                                                                                      & -                                                                                                                                                                                                                                            			& 6365 $\sim$ 8744                                                                                                                                                                                                & 20332 $\sim$ 76110                                                                                                                                                                                                               \\ \hline
\textbf{\begin{tabular}[c]{@{}c@{}}Temperatura de\\ operación [°C]\end{tabular}}                  & -20 $\sim$ 50                                                                                                                                                                                                                                       & -20 $\sim$ 50                                                                                                                                                                                                                                         & -20 $\sim$ 50                                                                                                                                                                                                   & -20 $\sim$ 50                                                                                                                                                                                                                    \\ \hline
\textbf{\begin{tabular}[c]{@{}c@{}}Tensión\\ nominal [V]\end{tabular}}                            & 12                                                                                                                                                                                                                                                  & 12                                                                                                                                                                                                                                                    & 12                                                                                                                                                                                                              & 12                                                                                                                                                                                                                               \\ \hline
\textbf{Capacidad [Ah]}                                                                           & 33 $\sim$ 250                                                                                                                                                                                                                                       & 100 $\sim$ 200                                                                                                                                                                                                                                        & 13.2 $\sim$ 20.0                                                                                                                                                                                                & 32.7 $\sim$ 200.0                                                                                                                                                                                                                \\ \hline
\textbf{\begin{tabular}[c]{@{}c@{}}Dimensiones\\ (máximas) [mm]\end{tabular}}                     & 52 x 268 x 220                                                                                                                                                                                                                                      & 499 x 259 x 219                                                                                                                                                                                                                                       & 181 x 77 x 167                                                                                                                                                                                                  & 522 x 240 x 219                                                                                                                                                                                                                  \\ \hline
\textbf{Peso [kg]}                                                                                & 10 $\sim$ 61                                                                                                                                                                                                                                        & 30 $\sim$ 74                                                                                                                                                                                                                                          & 5.45                                                                                                                                                                                                            & 15.5 $\sim$ 57.0                                                                                                                                                                                                                 \\ \hline
\textbf{\begin{tabular}[c]{@{}c@{}}Porcentaje de\\ autodescarga\\ (mensual a 25 °C)\end{tabular}} & 3\%                                                                                                                                                                                                                                                 & 3\%                                                                                                                                                                                                                                                   & 3\%                                                                                                                                                                                                             & 3\%                                                                                                                                                                                                                              \\ \hline
\textbf{Imagen}                                                                                   & \includeintable{.1}{ImagenesFactibilidad/Kijo-JDG}                                                                                                                                                                                                  & \includeintable{.1}{ImagenesFactibilidad/Kijo-JLG}                                                                                                                                                                                                    & \includeintable{.1}{ImagenesFactibilidad/Fenk-JS12-20}                                                                                                                                                          & \includeintable{.1}{ImagenesFactibilidad/Fenk-JM12}                                                                                                                                                                              \\ \hline
\end{tabular}
\caption{Comparación entre baterías gel de carga profunda.}
\end{table}

\Subsubsection{Propuesta de Paneles solares}
Para poder abastecer a todos los módulos anteriormente mencionados, es necesario la existencia de un módulo que brinde dicha energía. Dadas la ubicación remota donde se encontrará el producto final, se opta por emplear un panel solar, capaz de obtener energía del entorno y no de la red eléctrica.

Lo que principalmente determinará la elección de este componente es el consumo de las demás partes.

\begin{table}[H]
\centering
\begin{tabular}{|c|c|c|c|c|c|}
\hline
\textbf{\begin{tabular}[c]{@{}c@{}}Aspectos\\ comparativos\end{tabular}}                   & \textbf{DSP-20P} & \textbf{DSP-30M} & \textbf{LN-50P} & \textbf{ESPMC210} & \textbf{LNSE-260P} \\ \hline
\textbf{Costo [ARS]}                                                       				   & 3400			  & 5000			 & 6000 		   & 15600 			   & 14500				\\ \hline
\textbf{\begin{tabular}[c]{@{}c@{}}Temperatura de\\ operación [°C]\end{tabular}}           & -45 $\sim$ 85    & -45 $\sim$ 85    & -45 $\sim$ 85   & -40 $\sim$ 85     & -45 $\sim$ 85      \\ \hline
\textbf{\begin{tabular}[c]{@{}c@{}}Potencia\\ máxima [W]\end{tabular}}                     & 20 $\pm$ 3\%     & 30 $\pm$ 3\%     & 50 $\pm$ 3\%    & 210               & 260                \\ \hline
\textbf{\begin{tabular}[c]{@{}c@{}}Tensión a\\ potencia\\ máxima [V]\end{tabular}}         & 17.6             & 18.0             & 18.0            & 18.85             & 30.4               \\ \hline
\textbf{\begin{tabular}[c]{@{}c@{}}Corriente\\ a potencia\\ máxima [A]\end{tabular}}       & 1.14             & 1.67             & 2.78            & 11.15             & 8.55               \\ \hline
\textbf{\begin{tabular}[c]{@{}c@{}}Tensión a\\ circuito abierto\\ máxima [V]\end{tabular}} & 22.0             & 21.5             & 22.3            & 23.2              & 37.4               \\ \hline
\textbf{\begin{tabular}[c]{@{}c@{}}Corriente a\\ corto circuito\\ máxima [A]\end{tabular}} & 1.39             & 1.86             & 3.01            & 11.8              & 9.11               \\ \hline
\end{tabular}
\caption{Comparación entre paneles solares.}
\end{table}

\Subsubsubsection{Carga MPPT batería principal}
En cuanto a la carga de la batería principal a partir de los paneles solares se tuvieron en cuenta los siguientes integrados

\begin{table}[H]
\centering
\begin{tabular}{|c|c|c|c|}
\hline
Aspecto comparativo             & \textbf{\href{https://ar.mouser.com/new/dfrobot/dfrobot-dfr0580-solar-power-manager/}{DFR0580}}             & \textbf{\href{http://www.solaryeolica.com.ar/contents/es/p1698\_REGULADORA-DE-CARGA-HASTA-10-AMPER.html}{P1698}}            & \textbf{\href{https://pdf1.alldatasheet.com/datasheet-pdf/view/1133225/CONSONANCE/CN3306.html}{CN3306}}            \\ \hline
Costo {[}USD{]}                 & 18                & 26 & 11                \\ \hline
Temperatura de operación        & -40°C $\sim$ 85°C & -35°C $\sim$ 60°C & -40°C $\sim$ 85°C \\ \hline
Tensión de operación {[}V{]}    & 6.6 $\sim$ 30     & 12 / 24     & 4.5 $\sim$ 32     \\ \hline
Tecnología de batería           & Lead-Acid                & Lead-Acid            & Multi-Chemistry   \\ \hline
Máxima corriente salida {[}A{]} & 4                 & 2/10 & 1.5               \\ \hline
MPPT                            & Si                & No                & Si                \\ \hline
Imagen                          & \includeintable{.1}{ImagenesFactibilidad/CN3767}                                                                                                                                                                                                           & \includeintable{.1}{ImagenesFactibilidad/P1698}                                                                                                                                                                                                                    & \includeintable{.1}{ImagenesFactibilidad/CN3306}                                                                                                                                                                                                                             \\ \hline
\end{tabular}
\end{table}



\Subsubsection{Primer Propuesta de Abastecimiento de Energía de UBM}
Si se parte de la ecuación del campo magnético en el eje azimutal de un dipolo de hertz, se tiene que
\begin{equation}
E_\theta = -\frac{\eta}{4\pi}I \cdot \Delta L \cdot k^2 \cdot sin\theta \cdot e^{-jkr} \left[ \frac{1}{jkr}+\left( \frac{1}{jkr}\right)^2 + \left(\frac{1}{jkr}\right)^3 \right]
\end{equation}
y los últimos tres términos se denominan, en orden de aparición, término de campo lejano, campo cercano y campo reactivo. Como las distancias a la cual se debe transmitir potencia en la solución es de entre 32 y 64cm y el término de campo lejano depende únicamente de la inversa de la distancia, este será el cual más efecto tenga sobre la transmisión de potencia. 

El campo lejano es utilizado para realizar todo tipo de telecomunicaciones hoy en día, por lo que nos centraremos en analizar la factibilidad del uso de antenas como nexo entre el nido, el campo electromagnético y la mochila del ave.

Para analizar la potencia recibida en la antena receptora, la cual estará montada en la mochila, se realiza el balance de potencias del circuito electromagnético.

\begin{equation}
P_r[dBm] = P_t[dBm] + Gt[dB] + G_r[dB] - L_{bf}[dB] - L_{cab}[dB] - L_{roe}[dB] - L_{r}[dB]
\end{equation}

donde $P_r$ es la potencia recibida en la antena receptora, $P_t$ la potencia emitida por la antena transmisora, $G_t$ la ganancia de la antena transmisora, $G_r$ la ganancia de la antena receptora, $L_{bf}$ las péridas por espacio libre, $L_{cab}$ las pérdidas en los cables de ambas antenas, $L_{roe}$ las pérdidas por retorno en ambas antenas, y $L_{r}$ las pérdidas por desacople entre las líneas de transmisión y las antenas.

Para el caso de la pérdida por espacio libre, esta se puede calcular como

\begin{equation}
L_{bf} = 32.5dB + 20log_{10}f[MHz]+20log_{10}R[km]
\end{equation}

mientras que el resto de los datos se puede obtener por medio de la datasheet o ensayos de las antenas, exceptuando la potencia a ser calculada y las pérdidas por desacople, las cuales dependen constructivamente del diseño de la líneas de transmisión que conectan con las antenas.

Para realizar las comparaciones entre antenas transmisoras, se tuvieron en cuenta las siguientes restricciones:
\begin{itemize}
\item \textbf{Dimensiones:} La antena transmisora deberá ser colocada en la bóveda del nido, dado que ese es el único lugar donde se puede colocar electrónica sin que esta sea perturbada por las aves y vice versa. El volumen de la bóveda se puede aproximar a un cilindro macizo chato de diámetro entre 7.9 y 9.7cm y aproximadamente 5cm de altura, por lo que las dimensiones de la antena emisora estarán acotadas por estos valores.
\item \textbf{Directividad:} Se quiere que la potencia enviada a la antena se transforme en radiación electromagnética que llegue a la mochila del ave, por lo que radiación que no sea dirigida directamente hacia el fondo del nido será potencia desperdiciada. Es por esto que se quiere una alta directividad en la antena emisora.
\item \textbf{Potencia Máxima:} Como las pérdidas en el circuito electromagnético son grandes, una muy baja parte de la potencia enviada a la antena transmisora formará parte de la potencia entregada a las baterías de la mochila, por lo que para recibir la potencia necesaria, se debe transmitir en el orden de los watts. Es por esto que la potencia máxima es una especificación relevante al momento de decidir entre soluciones.
\end{itemize}
mientras que para el caso de las antenas receptoras, se tuvieron en cuenta las siguientes restricciones:
\begin{itemize}
\item \textbf{Eficiencia:} Esta será la especificación más importante y es la que determinará la factibilidad de la solución. Se requiere una elevada eficiencia para lograr transmitir a las baterías la potencia recibida por el campo electromagnético.
\item \textbf{Lóbulo isotrópico:} Como se desconoce cuál será la posición del ave dentro del nido, se requiere que el lóbulo de radiación de la antena receptora sea lo más isotrópico posible, garantizando una recepción de potencia uniforme sin importar la posición del ave.
\item \textbf{Peso:} El ave no puede cargar con más de un cierto porcentaje de su propio peso, por lo que minimizar esta especificación es crucial.
\item \textbf{Dimensiones:} Es necesario no perturbar al ave con la mochila. Esto requiere que la antena receptora posea las mínimas dimensiones posibles. Sin embargo, como la banda a utilizar será la de $915MHz$ y un cuarto de onda en esta frecuencia es alrededor de $8cm$, existe una relación de compromiso entre las dimensiones de la antena receptora y la eficiencia de esta.
\end{itemize}

Finalmente, una restricción a tener en cuenta para ambas antenas será el costo.

\begin{table}[H]
\centering
\begin{tabular}{|c|c|c|c|}
\hline
\textbf{\begin{tabular}[c]{@{}c@{}}Aspectos\\ comparativos\end{tabular}}    & \textbf{\href{https://abracon.com/patchantenna/APAE915R2540ABDB1-T.pdf}{APAE915R2540ABDB1-T}} & \textbf{\href{https://www.digikey.com/en/products/detail/pulselarsen-antennas/W3215/9838686}{W3215}} & \textbf{\href{https://cdn.taoglas.com/datasheets/ISPC.91A.09.0092E.pdf}{ISPC.91A.09.0092E}} \\ \hline
\textbf{Costo [USD]}                                                        & 3.66                                                                                 & 12.47                                                                                       & 20.91                                                                              \\ \hline
\textbf{Dimensiones [mm]}                                                   & 25 x 25 x 4                                                                          & 40 x 40 x 6                                                                                 & 47 x 47 x 6.5                                                                      \\ \hline
\textbf{\begin{tabular}[c]{@{}c@{}}Frecuencia\\ Central [MHz]\end{tabular}} & 915                                                                                  & 915                                                                                         & 915                                                                                \\ \hline
\textbf{Impedancia [$\Omega$]}                                              & 50                                                                                   & 50                                                                                          & 50                                                                                 \\ \hline
\textbf{Polarización}                                                       & RHCP                                                                                 & Lineal vertical                                                                             & RHCP                                                                               \\ \hline
\textbf{Ganancia [dBi]}                                                     & 1.5                                                                                  & 4.5                                                                                         & 5 (30 x 30 ground plane)                                                           \\ \hline
\textbf{ROE}                                                                & 1.5                                                                                  & 1.23                                                                                        & 1.28                                                                               \\ \hline
\textbf{Imagen}                                                             & \includeintable{.1}{ImagenesFactibilidad/ANT1}                                       & \includeintable{.1}{ImagenesFactibilidad/ANT2}                                              & \includeintable{.1}{ImagenesFactibilidad/ANT3}                                     \\ \hline
\textbf{Lóbulo}                                                             & \includeintable{.1}{ImagenesFactibilidad/LOB1}                                       & \includeintable{.1}{ImagenesFactibilidad/LOB2}                                              & \includeintable{.1}{ImagenesFactibilidad/LOB3}                                     \\ \hline
\end{tabular}
\caption{Comparación entre antenas transmisoras (Parte 1).}
\end{table}

\begin{table}[H]
\centering
\begin{tabular}{|c|c|c|}
\hline
\textbf{\begin{tabular}[c]{@{}c@{}}Aspectos\\ comparativos\end{tabular}}    & \textbf{\href{https://abracon.com/patchantenna/APAES915R80C16-T.pdf}{APAES915R80C16-T}} & \textbf{\href{https://abracon.com/datasheets/ARRKP7059-S915B.pdf}{ARRKP7059-S915B}} \\ \hline
\textbf{Costo [USD]}                                                        & 34.28                                                                          & 50.73                                                                      \\ \hline
\textbf{Dimensiones [mm]}                                                   & 80 x 80 x 6                                                                    & 70 x 70 x 5.9                                                              \\ \hline
\textbf{\begin{tabular}[c]{@{}c@{}}Frecuencia\\ Central [MHz]\end{tabular}} & 915                                                                            & 915                                                                        \\ \hline
\textbf{Impedancia [$\Omega$]}   											& 50                                                                             & 50                                                                         \\ \hline
\textbf{Polarización}                                                       & RHCP                                                                           & RHCP                                                                       \\ \hline
\textbf{Ganancia [dBi]}                                                     & 2 (120 x 120 ground plane)                                                     & 2.8 (70 x 70 ground plane)                                                 \\ \hline
\textbf{ROE}                                                                & 1.3                                                                            & $\leq$ 2                                                                   \\ \hline
\textbf{Imagen}                                                             & \includeintable{.1}{ImagenesFactibilidad/ANT4}                                 & \includeintable{.1}{ImagenesFactibilidad/ANT5}                             \\ \hline
\textbf{Lóbulo}                                                             & \includeintable{.1}{ImagenesFactibilidad/LOB4}                                 & \includeintable{.1}{ImagenesFactibilidad/LOB5}                             \\ \hline
\end{tabular}
\caption{Comparación entre antenas transmisoras (Parte 2).}
\end{table}

\begin{table}[H]
\centering
\begin{tabular}{|c|c|c|c}
\hline
\textbf{\begin{tabular}[c]{@{}c@{}}Aspectos\\ comparativos\end{tabular}}    & \textbf{\href{https://linxtechnologies.com/wp/wp-content/uploads/ant-915-cpa-ds.pdf}{ANT-915-CPA}} & \textbf{\href{https://cdn.taoglas.com/datasheets/FXP290.07.0100A.pdf}{FXP290.07.0100A}} & \multicolumn{1}{c|}{\textbf{\href{https://media.digikey.com/pdf/Data\%20Sheets/Ignion\%20PDFs/NN01-105_Jan2021.pdf}{NN01-105}}} 	\\ \hline
\textbf{Costo [USD]}                                                       	& 3.9                                                                                                & 15.39                                                                                   & \multicolumn{1}{c|}{3.53}                                      		                                                            \\ \hline
\textbf{Dimensiones [mm]}                                                   & 25 x 25 x 4                                                                                        & 70 x 45 x 0.1                                                                           & \multicolumn{1}{c|}{18 x 7.3 x 0.8}                                                                                             	\\ \hline
\textbf{\begin{tabular}[c]{@{}c@{}}Frecuencia\\ Central [MHz]\end{tabular}} & 915                                                                                                & 915                                                                                     & \multicolumn{1}{c|}{915}                                                                                                      		\\ \hline
\textbf{Impedancia [$\Omega$]}                                              & 50                                                                                                 & 50                                                                                      & \multicolumn{1}{c|}{50} 		                                                                                                    \\ \hline
\textbf{Polarización}                                                       & RHCP                                                                                               & Lineal                                                                                  & \multicolumn{1}{c|}{Lineal}    		                                                                                            \\ \hline
\textbf{Ganancia [dBi]}                                                     & 1.5                                                                                                & 0.5                                                                                     & \multicolumn{1}{c|}{1.7}               		                                                                                    \\ \hline
\textbf{Eficiencia [\%]}                                                    & 38                                                                                                 & 43                                                                                      & \multicolumn{1}{c|}{85}                        	  	                                                                            \\ \hline
\textbf{ROE}                                                                & $\leq$ 1.2                                                                                           & 1.5                                                                                   & \multicolumn{1}{c|}{1.4}                               		                                                                    \\ \hline
\textbf{Peso [gr]}                                                          & 13.2                                                                                               & 1.5                                                                                     & \multicolumn{1}{c|}{0.2}                                                  		                                                    \\ \hline
\textbf{Imagen}                                                             & \includeintable{.1}{ImagenesFactibilidad/ANTR1}                                                    & \includeintable{.1}{ImagenesFactibilidad/ANTR2}                                         & \multicolumn{1}{c|}{\includeintable{.1}{ImagenesFactibilidad/ANTR3}}           		                                            \\ \hline
\textbf{Lóbulo}                                                             & \includeintable{.1}{ImagenesFactibilidad/LOBR1}                                                    & \includeintable{.1}{ImagenesFactibilidad/LOBR2}                                         & \multicolumn{1}{c|}{\quotes{Omnidireccional}}                                                 		                                    \\ \hline
\end{tabular}
\caption{Comparación entre antenas receptoras (Parte 1).}
\end{table}

\begin{table}[H]
\centering
\begin{tabular}{|c|c|c|c}
\hline
\textbf{\begin{tabular}[c]{@{}c@{}}Aspectos\\ comparativos\end{tabular}}    & \textbf{\href{https://www.yageo.com/upload/media/product/productsearch/datasheet/wireless/An_SMD_FR4_915M_1204_05.pdf}{ANT1204F005R0915A}} & \textbf{\href{https://www.te.com/commerce/DocumentDelivery/DDEController?Action=srchrtrv\&DocNm=1513156\&DocType=DS\&DocLang=English}{1513156-1}} 	& \multicolumn{1}{c|}{\textbf{\href{https://linxtechnologies.com/wp/wp-content/uploads/ant-915-usp410-ds.pdf}{ANT-915-USP410}}} \\ \hline
\textbf{Costo [USD]}                                                       	& 1.52                                                                                                                                       & 2.8                                                                                                                                            		& \multicolumn{1}{c|}{1.45}                                                                                                     \\ \hline
\textbf{Dimensiones [mm]}                                                   & 12.20 x 4 x 1.6                                                                                                                            & 38.1 x 6.6 x 1.57                                                                                                                              		& \multicolumn{1}{c|}{13.2 x 9.1 x 2.9}                                                                                         \\ \hline
\textbf{\begin{tabular}[c]{@{}c@{}}Frecuencia\\ Central [MHz]\end{tabular}} & 915                                                                                                                                        & 915                                                                                                                                            		& \multicolumn{1}{c|}{915}                                                                                                      \\ \hline
\textbf{Impedancia [$\Omega$]}                                              & 50                                                                                                                                         & 50                                                                                                                                            		& \multicolumn{1}{c|}{50}                                                                                                       \\ \hline
\textbf{Polarización}                                                       & Lineal                                                                                                                                     & Lineal                                                                                                                                         		& \multicolumn{1}{c|}{Lineal}                                                                                                   \\ \hline
\textbf{Ganancia [dBi]}                                                     & 1.59                                                                                                                                       & 1                                                                                                                                              		& \multicolumn{1}{c|}{0}                                                                                                        \\ \hline
\textbf{Eficiencia [\%]}                                                    & -                                                                                                                                          & 88                                                                                                                                             		& \multicolumn{1}{c|}{27}                                                                                                       \\ \hline
\textbf{ROE}                                                                & -                                                                                                                                          & 1.85                                                                                                                                           		& \multicolumn{1}{c|}{1.5}                                                                                                      \\ \hline
\textbf{Peso [gr]}                                                          & -                                                                                                                                          & < 0.9                                                                                                                                          		& \multicolumn{1}{c|}{0.6}                                                                                                      \\ \hline
\textbf{Imagen}                                                             & \includeintable{.1}{ImagenesFactibilidad/ANTR4}                                                                                            & \includeintable{.1}{ImagenesFactibilidad/ANTR5}                                                                                                		& \multicolumn{1}{c|}{\includeintable{.1}{ImagenesFactibilidad/ANTR6}}                                                          \\ \hline
\textbf{Lóbulo}                                                             & \includeintable{.1}{ImagenesFactibilidad/LOBR4}                                                                                            & \includeintable{.1}{ImagenesFactibilidad/LOBR5}                                                                                                		& \multicolumn{1}{c|}{\includeintable{.1}{ImagenesFactibilidad/LOBR6}}                                                          \\ \hline
\end{tabular}
\caption{Comparación entre antenas receptoras (Parte 2).}
\end{table}
\Subsubsection{Propuesta Final de Abastecimiento de Energía de UBM}
Ante la imposibilidad de realizar la carga inalámbrica se tuvo una reunión con el cliente para discutir el proceder. El cliente indicó que el dato principal que se requiere del vuelo del ave es la posición.
Para la obtención de datos de la trayectoria del ave se propuso el siguiente esquema:
\Subsubsubsection{Módulos del Sistema de Seguimiento}
La propuesta se basa en cuatro partes, la mochila del ave,
 las bases de seguimiento, una base principal de seguimiento y la comunicación de esta con la base del nido.
\begin{figure}[H]
	\centering
	\includegraphics[width=\linewidth,page=1]{ImagenesFactibilidad/beacon}
	\caption{Módulos del sistema de seguimieneto.}
	\label{fig:componentes beacon}
\end{figure}
cabe mencionar que la foto es a modo ilustrativo.
\Subsubsubsection{Mochila}
Para la propuesta de la mochila, se llego a la conclusión de que utilizar GPS era muy costoso en materia de energía, peso y dimensiones; por lo que se tomó un camino alternativo.
La tecnología recomendada es la de Beacon Bluetooth. Se toma como ejemplo el modelo de beacon EMBC22, cuyas dimensiones son útiles para nuestra aplicación dado que miden 30mm de diámetro y 10mm de altura. Esto cumple con los requerimientos de dimensiones y de peso, ya que con la batería incluida llega a un peso de 10 gramos. La característica mas notable del beacon es que permite mandar paquetes personalizables por Bluetooth 5.0 a bases receptoras para informar su presencia, a partir de las cuales se puede obtener la ubicación.
Adicionalmente, este beacon soporta un amplio rango de temperaturas y cuenta con un rango de protección IP-64. Se calcula una vida útil de la batería de 4 años\footnote{Utilizando la calculadora de vida útil del provedor del beacon}, un rango de detección de hasta 200 metros, y un microcontrolador con acelerómetro y firmware personalizable.
\begin{figure}[H]
	\centering
	\includegraphics[width=0.7\linewidth]{ImagenesFactibilidad/beaconpic}
	\caption{Beacon EMBC22.}
	\label{fig:beacon}
\end{figure}
\Subsubsubsection{Base de Seguimiento}
Las bases de seguimiento consisten en una red de módulos receptores esparcidos por el bosque a una distancia de aproximadamente 30 metros entre sí.
Además, se hicieron las siguientes recomendaciones:
\begin{itemize}
\item Comunicación Bluetooth 5.0 para el beacon acorde al estándar, con la finalidad de poder triangular su posición.
\item Utilización de red Lora o Bluetooth 5.0 para la comunicación entre bases de seguimiento de ser necesario y con la base principal de seguimiento para reporte de datos.
\item Independencia de la red eléctrica: Utilización de paneles solares y baterías.
\end{itemize}
\begin{figure}[H]
\begin{subfigure}{.5\textwidth}
  \centering
  \includegraphics[width=.99\linewidth,page=2]{/ImagenesFactibilidad/beacon}
  \caption{Ubicación de bases de seguimiento.}
  \label{fig:sfig1}
\end{subfigure}%
\begin{subfigure}{.5\textwidth}
  \centering
  \includegraphics[width=.99\linewidth,page=3]{/ImagenesFactibilidad/beacon}
  \caption{Triangulación de posición.}
  \label{fig:sfig2}
\end{subfigure}%
\caption{Boceto de funcionamiento de las bases de seguimiento.}
\label{fig:fig}
\end{figure}
\Subsubsubsection{Base Principal de Seguimiento}
La base principal de seguimiento nuclea toda la información de las bases de seguimiento. Además, se considera la existencia de bases repetidoras para la principal en el caso de que la distancia de la red receptora sea mayor que el alcance de la base principal de seguimiento.
Esta base también debe contar con independencia de la red eléctrica y debe poder comunicarse con la base del nido mediante Bluetooth. El propósito de esta comunicación es el enviar un set de datos del ave una vez por día.
\Subsubsubsection{Propuesta de Valor Aumentada}
Finamente cabe mencionara que realizar el diseño sugerido, aumenta con creces la escalabilidad del producto final, ya que con la red de receptores desplegada, basta con utilizar un beacon en cada animal que se desee seguir, y la red lo podrá hacer sin dificultad, y además teniendo la capacidad de distinguir todos los dispositivos entre si.


\Subsubsection{Elección de una Solución}
\Subsubsubsection{Sensores}

Para el sensor de temperatura la primer opción a descartar es aquella que no cumple con el rango de temperaturas a medir, por lo que el Ds18b20 queda descartado a pesar de su bajo costo. Luego de las opciones que quedan todas son de un costo similar, sin embargo hay que tener en cuenta que para la termocupla se debe proporcionar una manera de medir la temperatura de referencia, la cual puede ser tanto una RTD como un IC. Es por esto que el costo de la termocupla aumentaría con creces. Tanto la TC como la RTD necesitan un circuito convertidor para poder medir directamente el valor de la temperatura con un micro controlador, mientras que los IC ofrecen directamente una salida digital. La mejor precisión de la medición se da con una RTD, seguido por los IC y finalmente la TC.

Una desventaja de la TC es que tiende a envejecer rápidamente. Si bien el estudio no dura mas de 3 meses, el producto podrá ser reutilizado, siendo dicho envejecimiento un problema. El autocalientamiento también es contraproductivo en la medición de temperatura debido a que este puede alterar la misma si no es tenido en cuenta. Las TC no cuentan con este inconveniente debido a su principio de funcionamiento, mientras que con las otras opciones si lo es. Con la RTD este efecto depende directamente con la corriente que se suministra para la medición, y con los IC es un aspecto que es considerado por los diseñadores de los mismos.

Por estas razones los candidatos a terminan siendo DHT-22 y la PT-100. Un punto favorable para la DHT-22 es que no necesita un circuito extra y el problema del autocalentamiento ya fue pensado. Adicionalmente esta unidad cuenta con una medición de humedad la cual podría usarse como sensor de humedad, ya siendo el principal o usado como complemento.

En la elección para la medición de humedad, como primer criterio, se busca que pueda medir el rango entero de la humedad relativa y que cuente con una precisión considerable. Dadas estas consideraciones, se descarta el DHT-11 y AM-1001. Es así que de los dos restantes, se opta por el DHT-22 debido a que por un menor costo se obtienen mejores prestaciones. Teniendo en cuenta esto se utilizará tanto para la medición de temperatura y humedad el DHT-22.

En cuanto a la luminosidad, principalmente se deberá asegurar el funcionamiento en el rango de temperatura en el cual operará el dispositivo, por lo cual el OPT-101 queda descartado. Luego se tendrá en cuenta la potencia utilizada, el rango de medición de los sensores y el tipo de alimentación.

La comunicación puede ser analógica en corriente para el TEMT-6000, pero este necesitará un amplificador de corriente o un convertidor para esta corriente a un nivel medible. Existen también otros sensores que tienen una salida analógica de tensión como el GL55-LM393 con un rango entre 0 y VCC. Este también provee con una salida digital, pero esta funciona como un schmitt trigger. Finalmente el BJ-1750 cuanta con una salida digital con el protocolo de comunicación I2C.

Teniendo en cuenta esto se opta por utilizar el sensor \TBD.

\Subsubsubsection{Almacenamiento}

\Subsubsubsection{Comunicación}

\Subsubsubsection{Microprocesadores}

\Subsubsubsection{Bateria}

\Subsubsubsection{Cargador}

\Subsubsubsection{Alimentación}

\Subsubsection{DFMEA}
\begin{adjustbox}{angle=90, captionbelow={DFMEA (Parte 1).}, float={table}[H]}

%CAMBIAR TODOS LOS CCC POR CCCCCC
\setlength\arrayrulewidth{0.5pt}
\centering
\begin{tabular}{|ccccccccccccc|}
\hline
\multicolumn{13}{|c|}{ANÁLISIS DE RIESGOS}                                                                                                                                                                                                                                                                                                                                                                                                                                                                                                                                                                                                                                                                                                                                                                                                                                                                                                                                                                                                                                             \\
\multicolumn{1}{|l}{}                                              & \multicolumn{1}{l}{}                                                                                                      & \multicolumn{1}{l}{}                                                                                                      & \multicolumn{1}{l}{}                                                                                                                                    & \multicolumn{1}{l}{}                            & \multicolumn{1}{l}{}                            & \multicolumn{1}{l}{}                            & \multicolumn{1}{l}{}                            & \multicolumn{1}{l}{}                                                                                                                                                                                 & \multicolumn{1}{l}{}                            & \multicolumn{1}{l}{}                            & \multicolumn{1}{l}{}                            & \multicolumn{1}{l|}{}      \\ \cline{2-8}
\multicolumn{1}{|l|}{}                                             & \multicolumn{1}{c|}{\begin{tabular}[c]{@{}c@{}}Fecha de\\ elaboración:\end{tabular}}                                      & \multicolumn{1}{c|}{25/05/21}                                                                                             & \multicolumn{1}{c|}{\begin{tabular}[c]{@{}c@{}}Fecha de\\ revisión:\end{tabular}}                                                                       & \multicolumn{4}{c|}{12/06/21}                                                                                                                                                                         & \multicolumn{1}{l}{}                                                                                                                                                                                 & \multicolumn{1}{l}{}                            & \multicolumn{1}{l}{}                            & \multicolumn{1}{l}{}                            & \multicolumn{1}{l|}{}      \\ \cline{2-8}
\multicolumn{1}{|l}{}                                              & \multicolumn{1}{l}{}                                                                                                      & \multicolumn{1}{l}{}                                                                                                      & \multicolumn{1}{l}{}                                                                                                                                    & \multicolumn{1}{l}{}                            & \multicolumn{1}{l}{}                            & \multicolumn{1}{l}{}                            & \multicolumn{1}{l}{}                            & \multicolumn{1}{l}{}                                                                                                                                                                                 & \multicolumn{1}{l}{}                            & \multicolumn{1}{l}{}                            & \multicolumn{1}{l}{}                            & \multicolumn{1}{l|}{}      \\ \hline
\rowcolor[HTML]{CCCCCC} 
\multicolumn{1}{|c|}{\cellcolor[HTML]{CCCCCC}}                     & \multicolumn{1}{c|}{\cellcolor[HTML]{CCCCCC}}                                                                             & \multicolumn{1}{c|}{\cellcolor[HTML]{CCCCCC}}                                                                             & \multicolumn{1}{c|}{\cellcolor[HTML]{CCCCCC}}                                                                                                           & \multicolumn{4}{c|}{\cellcolor[HTML]{CCCCCC}Aceptabilidad}                                                                                                                                            & \multicolumn{1}{c|}{\cellcolor[HTML]{CCCCCC}}                                                                                                                                                        & \multicolumn{4}{c|}{\cellcolor[HTML]{CCCCCC}Aceptabilidad}                                                                                                                       \\ \cline{5-8} \cline{10-13} 
\rowcolor[HTML]{CCCCCC} 
\multicolumn{1}{|c|}{\multirow{-2}{*}{\cellcolor[HTML]{CCCCCC}N°}} & \multicolumn{1}{c|}{\multirow{-2}{*}{\cellcolor[HTML]{CCCCCC}\begin{tabular}[c]{@{}c@{}}Efectos de\\ falla\end{tabular}}} & \multicolumn{1}{c|}{\multirow{-2}{*}{\cellcolor[HTML]{CCCCCC}\begin{tabular}[c]{@{}c@{}}Modo de\\ la falla\end{tabular}}} & \multicolumn{1}{c|}{\multirow{-2}{*}{\cellcolor[HTML]{CCCCCC}\begin{tabular}[c]{@{}c@{}}Causas de\\ la falla\end{tabular}}}                             & \multicolumn{1}{c|}{\cellcolor[HTML]{CCCCCC}NS} & \multicolumn{1}{c|}{\cellcolor[HTML]{CCCCCC}PO} & \multicolumn{1}{c|}{\cellcolor[HTML]{CCCCCC}DE} & \multicolumn{1}{c|}{\cellcolor[HTML]{CCCCCC}IC} & \multicolumn{1}{c|}{\multirow{-2}{*}{\cellcolor[HTML]{CCCCCC}\begin{tabular}[c]{@{}c@{}}Acción de\\ reducción\end{tabular}}}                                                                         & \multicolumn{1}{c|}{\cellcolor[HTML]{CCCCCC}NS} & \multicolumn{1}{c|}{\cellcolor[HTML]{CCCCCC}PO} & \multicolumn{1}{c|}{\cellcolor[HTML]{CCCCCC}DE} & IC                         \\ \hline
\multicolumn{1}{|c|}{1}                                            & \multicolumn{1}{c|}{\begin{tabular}[c]{@{}c@{}}No se pueden\\ realizar mediciones\end{tabular}}                           & \multicolumn{1}{c|}{\begin{tabular}[c]{@{}c@{}}Los sensores dejan\\ de funcionar\end{tabular}}                            & \multicolumn{1}{c|}{\begin{tabular}[c]{@{}c@{}}Los sensores\\ son dañados\\ por el ave\end{tabular}}                                                    & \multicolumn{1}{c|}{4}                          & \multicolumn{1}{c|}{4}                          & \multicolumn{1}{c|}{2}                          & \multicolumn{1}{c|}{\cellcolor[HTML]{F1C232}32} & \multicolumn{1}{c|}{\begin{tabular}[c]{@{}c@{}}Ocultar los sensores\\ en la bóveda\end{tabular}}                                                                                                     & \multicolumn{1}{c|}{4}                          & \multicolumn{1}{c|}{3}                          & \multicolumn{1}{c|}{2}                          & \cellcolor[HTML]{6AA84F}24 \\ \hline
\multicolumn{1}{|c|}{2}                                            & \multicolumn{1}{c|}{\begin{tabular}[c]{@{}c@{}}No se pueden\\ realizar mediciones\end{tabular}}                           & \multicolumn{1}{c|}{\begin{tabular}[c]{@{}c@{}}Los sensores dejan\\ de funcionar\end{tabular}}                            & \multicolumn{1}{c|}{\begin{tabular}[c]{@{}c@{}}El conexionado es\\ dañado por el ave\end{tabular}}                                                      & \multicolumn{1}{c|}{4}                          & \multicolumn{1}{c|}{3}                          & \multicolumn{1}{c|}{3}                          & \multicolumn{1}{c|}{\cellcolor[HTML]{F1C232}36} & \multicolumn{1}{c|}{\begin{tabular}[c]{@{}c@{}}Hacer más\\ robusto el cableado\end{tabular}}                                                                                                         & \multicolumn{1}{c|}{4}                          & \multicolumn{1}{c|}{2}                          & \multicolumn{1}{c|}{3}                          & \cellcolor[HTML]{6AA84F}24 \\ \hline
\multicolumn{1}{|c|}{3}                                            & \multicolumn{1}{c|}{Falta de energía solar}                                                                               & \multicolumn{1}{c|}{\begin{tabular}[c]{@{}c@{}}Falla en los\\ paneles solares\end{tabular}}                               & \multicolumn{1}{c|}{\begin{tabular}[c]{@{}c@{}}Los paneles se\\ encuentran dañados\end{tabular}}                                                        & \multicolumn{1}{c|}{5}                          & \multicolumn{1}{c|}{3}                          & \multicolumn{1}{c|}{2}                          & \multicolumn{1}{c|}{\cellcolor[HTML]{F1C232}30} & \multicolumn{1}{c|}{\begin{tabular}[c]{@{}c@{}}Colocar protección\\ para los paneles\end{tabular}}                                                                                                   & \multicolumn{1}{c|}{4}                          & \multicolumn{1}{c|}{2}                          & \multicolumn{1}{c|}{2}                          & \cellcolor[HTML]{6AA84F}16 \\ \hline
\multicolumn{1}{|c|}{4}                                            & \multicolumn{1}{c|}{Falta de energía solar}                                                                               & \multicolumn{1}{c|}{\begin{tabular}[c]{@{}c@{}}La electrónica no\\ funciona correctametne\end{tabular}}                   & \multicolumn{1}{c|}{\begin{tabular}[c]{@{}c@{}}Fue colocado en un\\ lugar con obstrucciones\end{tabular}}                                               & \multicolumn{1}{c|}{5}                          & \multicolumn{1}{c|}{3}                          & \multicolumn{1}{c|}{3}                          & \multicolumn{1}{c|}{\cellcolor[HTML]{F1C232}45} & \multicolumn{1}{c|}{\begin{tabular}[c]{@{}c@{}}Instalar el panel solar\\ sobre un tronco, donde\\ no haya ramas u objetos\\ que puedan obstruir\end{tabular}}                                        & \multicolumn{1}{c|}{5}                          & \multicolumn{1}{c|}{2}                          & \multicolumn{1}{c|}{3}                          & \cellcolor[HTML]{F1C232}30 \\ \hline
\multicolumn{1}{|c|}{5}                                            & \multicolumn{1}{c|}{\begin{tabular}[c]{@{}c@{}}Falta de energía\\ en la batería\end{tabular}}                             & \multicolumn{1}{c|}{No hay alimentación}                                                                                  & \multicolumn{1}{c|}{\begin{tabular}[c]{@{}c@{}}\TBD dias con un nivel\\ de luz menor al necesario\\ para la carga de baterías\end{tabular}}             & \multicolumn{1}{c|}{5}                          & \multicolumn{1}{c|}{3}                          & \multicolumn{1}{c|}{3}                          & \multicolumn{1}{c|}{\cellcolor[HTML]{F1C232}45} & \multicolumn{1}{c|}{\begin{tabular}[c]{@{}c@{}}Contar con una batería\\ de emergencia para\\ sostener al sistema\\ \TBD días más\end{tabular}}                                                       & \multicolumn{1}{c|}{5}                          & \multicolumn{1}{c|}{1}                          & \multicolumn{1}{c|}{3}                          & \cellcolor[HTML]{6AA84F}15 \\ \hline
\multicolumn{1}{|c|}{6}                                            & \multicolumn{1}{c|}{\begin{tabular}[c]{@{}c@{}}Falta de energía\\ en la batería\end{tabular}}                             & \multicolumn{1}{c|}{La batería no funciona}                                                                               & \multicolumn{1}{c|}{\begin{tabular}[c]{@{}c@{}}Se inundó el\\ contenedor de\\ la batería\end{tabular}}                                                  & \multicolumn{1}{c|}{5}                          & \multicolumn{1}{c|}{3}                          & \multicolumn{1}{c|}{4}                          & \multicolumn{1}{c|}{\cellcolor[HTML]{CC0000}60} & \multicolumn{1}{c|}{\begin{tabular}[c]{@{}c@{}}Se utiliza una carcasa\\ para la batería con\\ proteccion \TBD que\\ asegure proteccion\\ contra agua\end{tabular}}                                   & \multicolumn{1}{c|}{2}                          & \multicolumn{1}{c|}{3}                          & \multicolumn{1}{c|}{4}                          & \cellcolor[HTML]{6AA84F}24 \\ \hline
\end{tabular}

\end{adjustbox}

\newpage

\begin{adjustbox}{angle=90, captionbelow={DFMEA (Parte 2).}, float={table}[H]}

%CAMBIAR TODOS LOS CCC POR CCCCCC
\setlength\arrayrulewidth{0.5pt}
\centering
\begin{tabular}{|ccccccccccccc|}
\hline
\multicolumn{13}{|c|}{ANÁLISIS DE RIESGOS}                                                                                                                                                                                                                                                                                                                                                                                                                                                                                                                                                                                                                                                                                                                                                                                                                                                                                                                                                                                                                                             \\
\multicolumn{1}{|l}{}                                              & \multicolumn{1}{l}{}                                                                                                      & \multicolumn{1}{l}{}                                                                                                      & \multicolumn{1}{l}{}                                                                                                                                    & \multicolumn{1}{l}{}                            & \multicolumn{1}{l}{}                            & \multicolumn{1}{l}{}                            & \multicolumn{1}{l}{}                            & \multicolumn{1}{l}{}                                                                                                                                                                                 & \multicolumn{1}{l}{}                            & \multicolumn{1}{l}{}                            & \multicolumn{1}{l}{}                            & \multicolumn{1}{l|}{}      \\ \cline{2-8}
\multicolumn{1}{|l|}{}                                             & \multicolumn{1}{c|}{\begin{tabular}[c]{@{}c@{}}Fecha de\\ elaboración:\end{tabular}}                                      & \multicolumn{1}{c|}{25/05/21}                                                                                             & \multicolumn{1}{c|}{\begin{tabular}[c]{@{}c@{}}Fecha de\\ revisión:\end{tabular}}                                                                       & \multicolumn{4}{c|}{12/06/21}                                                                                                                                                                         & \multicolumn{1}{l}{}                                                                                                                                                                                 & \multicolumn{1}{l}{}                            & \multicolumn{1}{l}{}                            & \multicolumn{1}{l}{}                            & \multicolumn{1}{l|}{}      \\ \cline{2-8}
\multicolumn{1}{|l}{}                                              & \multicolumn{1}{l}{}                                                                                                      & \multicolumn{1}{l}{}                                                                                                      & \multicolumn{1}{l}{}                                                                                                                                    & \multicolumn{1}{l}{}                            & \multicolumn{1}{l}{}                            & \multicolumn{1}{l}{}                            & \multicolumn{1}{l}{}                            & \multicolumn{1}{l}{}                                                                                                                                                                                 & \multicolumn{1}{l}{}                            & \multicolumn{1}{l}{}                            & \multicolumn{1}{l}{}                            & \multicolumn{1}{l|}{}      \\ \hline
\rowcolor[HTML]{CCCCCC} 
\multicolumn{1}{|c|}{\cellcolor[HTML]{CCCCCC}}                     & \multicolumn{1}{c|}{\cellcolor[HTML]{CCCCCC}}                                                                             & \multicolumn{1}{c|}{\cellcolor[HTML]{CCCCCC}}                                                                             & \multicolumn{1}{c|}{\cellcolor[HTML]{CCCCCC}}                                                                                                           & \multicolumn{4}{c|}{\cellcolor[HTML]{CCCCCC}Aceptabilidad}                                                                                                                                            & \multicolumn{1}{c|}{\cellcolor[HTML]{CCCCCC}}                                                                                                                                                        & \multicolumn{4}{c|}{\cellcolor[HTML]{CCCCCC}Aceptabilidad}                                                                                                                       \\ \cline{5-8} \cline{10-13} 
\rowcolor[HTML]{CCCCCC} 
\multicolumn{1}{|c|}{\multirow{-2}{*}{\cellcolor[HTML]{CCCCCC}N°}} & \multicolumn{1}{c|}{\multirow{-2}{*}{\cellcolor[HTML]{CCCCCC}\begin{tabular}[c]{@{}c@{}}Efectos de\\ falla\end{tabular}}} & \multicolumn{1}{c|}{\multirow{-2}{*}{\cellcolor[HTML]{CCCCCC}\begin{tabular}[c]{@{}c@{}}Modo de\\ la falla\end{tabular}}} & \multicolumn{1}{c|}{\multirow{-2}{*}{\cellcolor[HTML]{CCCCCC}\begin{tabular}[c]{@{}c@{}}Causas de\\ la falla\end{tabular}}}                             & \multicolumn{1}{c|}{\cellcolor[HTML]{CCCCCC}NS} & \multicolumn{1}{c|}{\cellcolor[HTML]{CCCCCC}PO} & \multicolumn{1}{c|}{\cellcolor[HTML]{CCCCCC}DE} & \multicolumn{1}{c|}{\cellcolor[HTML]{CCCCCC}IC} & \multicolumn{1}{c|}{\multirow{-2}{*}{\cellcolor[HTML]{CCCCCC}\begin{tabular}[c]{@{}c@{}}Acción de\\ reducción\end{tabular}}}                                                                         & \multicolumn{1}{c|}{\cellcolor[HTML]{CCCCCC}NS} & \multicolumn{1}{c|}{\cellcolor[HTML]{CCCCCC}PO} & \multicolumn{1}{c|}{\cellcolor[HTML]{CCCCCC}DE} & IC                         \\ \hline
\multicolumn{1}{|c|}{7}                                            & \multicolumn{1}{c|}{\begin{tabular}[c]{@{}c@{}}La electrónica\\ deja de funcionar\end{tabular}}                           & \multicolumn{1}{c|}{\begin{tabular}[c]{@{}c@{}}La unidad de\\ procesamiento deja\\ de funcionar\end{tabular}}             & \multicolumn{1}{c|}{\begin{tabular}[c]{@{}c@{}}La UP se\\ encuentra a una\\ temperatura baja\end{tabular}}                                              & \multicolumn{1}{c|}{5}                          & \multicolumn{1}{c|}{2}                          & \multicolumn{1}{c|}{2}                          & \multicolumn{1}{c|}{\cellcolor[HTML]{6AA84F}20} & \multicolumn{1}{c|}{\begin{tabular}[c]{@{}c@{}}Colocar la UP en\\ un encapsulado\end{tabular}}                                                                                                       & \multicolumn{1}{c|}{4}                          & \multicolumn{1}{c|}{1}                          & \multicolumn{1}{c|}{1}                          & \cellcolor[HTML]{6AA84F}4  \\ \hline
\multicolumn{1}{|c|}{8}                                            & \multicolumn{1}{c|}{\begin{tabular}[c]{@{}c@{}}Falla de\\ almacenamiento\end{tabular}}                                    & \multicolumn{1}{c|}{\begin{tabular}[c]{@{}c@{}}No se pueden\\ guardar más datos\end{tabular}}                             & \multicolumn{1}{c|}{\begin{tabular}[c]{@{}c@{}}La temperatura de\\ operación es menor\\ al mínimo aceptable\end{tabular}}                               & \multicolumn{1}{c|}{5}                          & \multicolumn{1}{c|}{4}                          & \multicolumn{1}{c|}{3}                          & \multicolumn{1}{c|}{\cellcolor[HTML]{CC0000}60} & \multicolumn{1}{c|}{\begin{tabular}[c]{@{}c@{}}Cambiar la\\ memoria por una de\\ nivel industrial\end{tabular}}                                                                                      & \multicolumn{1}{c|}{5}                          & \multicolumn{1}{c|}{1}                          & \multicolumn{1}{c|}{3}                          & \cellcolor[HTML]{6AA84F}15 \\ \hline
\multicolumn{1}{|c|}{9}                                            & \multicolumn{1}{c|}{\begin{tabular}[c]{@{}c@{}}Falla de\\ almacenamiento\end{tabular}}                                    & \multicolumn{1}{c|}{\begin{tabular}[c]{@{}c@{}}La memoria sufre una\\ perdida de información\end{tabular}}                & \multicolumn{1}{c|}{\begin{tabular}[c]{@{}c@{}}La memoria\\ es defecuosa\end{tabular}}                                                                  & \multicolumn{1}{c|}{5}                          & \multicolumn{1}{c|}{2}                          & \multicolumn{1}{c|}{3}                          & \multicolumn{1}{c|}{\cellcolor[HTML]{F1C232}30} & \multicolumn{1}{c|}{\begin{tabular}[c]{@{}c@{}}Colocar una memoria\\ de respaldo\end{tabular}}                                                                                                       & \multicolumn{1}{c|}{4}                          & \multicolumn{1}{c|}{1}                          & \multicolumn{1}{c|}{3}                          & \cellcolor[HTML]{6AA84F}12 \\ \hline
\multicolumn{1}{|c|}{10}                                           & \multicolumn{1}{c|}{\begin{tabular}[c]{@{}c@{}}Interrupción en\\ la transmisión\\ ave - nido\end{tabular}}                & \multicolumn{1}{c|}{\begin{tabular}[c]{@{}c@{}}Se pierde la\\ comunicación\\ con el ave\end{tabular}}                     & \multicolumn{1}{c|}{\begin{tabular}[c]{@{}c@{}}El ave se retira\\ del nido\end{tabular}}                                                                & \multicolumn{1}{c|}{5}                          & \multicolumn{1}{c|}{4}                          & \multicolumn{1}{c|}{1}                          & \multicolumn{1}{c|}{\cellcolor[HTML]{6AA84F}20} & \multicolumn{1}{c|}{\begin{tabular}[c]{@{}c@{}}Agregar indicadores\\ para retomar la\\ transmisión a\\ partir de ese punto\end{tabular}}                                                             & \multicolumn{1}{c|}{3}                          & \multicolumn{1}{c|}{4}                          & \multicolumn{1}{c|}{1}                          & \cellcolor[HTML]{6AA84F}12 \\ \hline
\multicolumn{1}{|c|}{11}                                           & \multicolumn{1}{c|}{\begin{tabular}[c]{@{}c@{}}Interrupción en\\ la transmisión\\ nido - persona\end{tabular}}            & \multicolumn{1}{c|}{\begin{tabular}[c]{@{}c@{}}Se pierde la\\ comunicación\\ con la persona\end{tabular}}                 & \multicolumn{1}{c|}{\begin{tabular}[c]{@{}c@{}}El dispositivo receptor\\ no se encuentra en\\ el rango de transmisión\end{tabular}}                     & \multicolumn{1}{c|}{5}                          & \multicolumn{1}{c|}{3}                          & \multicolumn{1}{c|}{2}                          & \multicolumn{1}{c|}{\cellcolor[HTML]{F1C232}30} & \multicolumn{1}{c|}{\begin{tabular}[c]{@{}c@{}}Informar la existencia\\ del error en el\\ dispositivo receptor\end{tabular}}                                                                         & \multicolumn{1}{c|}{3}                          & \multicolumn{1}{c|}{3}                          & \multicolumn{1}{c|}{2}                          & \cellcolor[HTML]{6AA84F}18 \\ \hline
\multicolumn{1}{|c|}{12}                                           & \multicolumn{1}{c|}{\begin{tabular}[c]{@{}c@{}}Interrupción en\\ la transmisión\\ nido - persona\end{tabular}}            & \multicolumn{1}{c|}{\begin{tabular}[c]{@{}c@{}}La transmision de\\ datos se ve\\ interrumpida\end{tabular}}               & \multicolumn{1}{c|}{\begin{tabular}[c]{@{}c@{}}La persona que recibe la\\ información está\\ posicionada demasiado\\ lejos del transmisor\end{tabular}} & \multicolumn{1}{c|}{4}                          & \multicolumn{1}{c|}{2}                          & \multicolumn{1}{c|}{2}                          & \multicolumn{1}{c|}{\cellcolor[HTML]{6AA84F}16} & \multicolumn{1}{c|}{\begin{tabular}[c]{@{}c@{}}La información es\\ borrada unicamente\\ cuando se recibe\\ un mensaje de OK\\ de la persona en la\\ base del arbol\\ a travéz del WIFI\end{tabular}} & \multicolumn{1}{c|}{2}                          & \multicolumn{1}{c|}{2}                          & \multicolumn{1}{c|}{2}                          & \cellcolor[HTML]{6AA84F}8  \\ \hline
\multicolumn{1}{|c|}{13}                                           & \multicolumn{1}{c|}{\begin{tabular}[c]{@{}c@{}}Falla en la transmisión\\ de energía al ave\end{tabular}}                  & \multicolumn{1}{c|}{\begin{tabular}[c]{@{}c@{}}El ave no trasmite\\ la información\end{tabular}}                          & \multicolumn{1}{c|}{\begin{tabular}[c]{@{}c@{}}La antena transmisora\\ fue dañada\end{tabular}}                                                         & \multicolumn{1}{c|}{5}                          & \multicolumn{1}{c|}{2}                          & \multicolumn{1}{c|}{2}                          & \multicolumn{1}{c|}{\cellcolor[HTML]{6AA84F}20} & \multicolumn{1}{c|}{\begin{tabular}[c]{@{}c@{}}Colocar un recubrimiento\\ protector sobre la antena\end{tabular}}                                                                                    & \multicolumn{1}{c|}{5}                          & \multicolumn{1}{c|}{1}                          & \multicolumn{1}{c|}{2}                          & \cellcolor[HTML]{6AA84F}10 \\ \hline
\end{tabular}

\end{adjustbox}

\newpage

\begin{multicols}{2}
\begin{table}[H]
\centering
\begin{tabular}{|c|c|c|c}
\cline{1-3}
Severidad                                                   & Probabilidad	& Detectabilidad 	& \multicolumn{1}{l}{}   \\ \hline
Insignificante                                              & Remota      	& Completa       	& \multicolumn{1}{c|}{1} \\ \hline
\begin{tabular}[c]{@{}c@{}}Poco\\ significante\end{tabular} & Poco remota	& Mayor          	& \multicolumn{1}{c|}{2} \\ \hline
Moderado                                                    & Media       	& Moderada       	& \multicolumn{1}{c|}{3} \\ \hline
Grave                                                       & Alta        	& Pequeña        	& \multicolumn{1}{c|}{4} \\ \hline
Muy grave                                                   & Muy alta    	& Mínima         	& \multicolumn{1}{c|}{5} \\ \hline
\end{tabular}
\caption{Criterio de IC.}
\end{table}

\begin{table}[H]
\setlength\arrayrulewidth{0.5pt}
\centering
\begin{tabular}{|c|c|}
\hline
\multicolumn{2}{|c|}{Nivel de IC}                                                                                      \\ \hline
\cellcolor[HTML]{6AA84F}Aceptable                                                                       & IC $\leq$ 27 \\ \hline
\cellcolor[HTML]{F1C232}\begin{tabular}[c]{@{}c@{}}Bajar hasta\\ razonablemente\\ práctico\end{tabular} & 27 < IC < 48 \\ \hline
\cellcolor[HTML]{CC0000}No aceptable                                                                       & 48 $\leq$ IC \\ \hline
\end{tabular}
\caption{Nivel de IC.}
\end{table}
\end{multicols}

\Subsection{Factibilidad de Tiempos}

\Subsubsection{Consideraciones}
Para el desarrollo de las siguientes secciones, se tiene en cuenta un equipo de trabajo de 4 personas, con días laborales de 8 horas. Además, se consideran 15 días de vacaciones en la primera quincena de enero.

\Subsubsection{Planificación}
Se procede a presentar un cuadro con las tareas a realizar. En la siguiente tabla se observa el tiempo mas probable, el optimista y el pesimista, para tener un análisis más real de la duración.

\begin{table}[H]
\centering
\begin{tabular}{|c|c|c|c|c|c|}
\hline
\textbf{N°} & \textbf{Nombre de tarea}                												& \textbf{\begin{tabular}[c]{@{}c@{}}Duracion\\ Optimista\end{tabular}} & \textbf{\begin{tabular}[c]{@{}c@{}}Duración\\ Media\end{tabular}} & \textbf{\begin{tabular}[c]{@{}c@{}}Duración\\ Pesimista\end{tabular}} & \textbf{Predecesora} \\ \hline
1           & Detectar Necesidad                      												& 1                           & 3           		& 4                           & -                                         \\ \hline
2           & Definir el alcance                      												& 2                           & 3                 & 4                           & 1                                         \\ \hline
3           & Antecedentes y Contexto                 												& 1                           & 2                 & 3                           & 1                                         \\ \hline
4           & Reuniones con cliente             	  												& 0.25                        & 0.25              & 0.25                        & 1                                         \\ \hline
5           & Definir objetivos de diseño             												& 1                           & 2                 & 5                           & 2,3                                       \\ \hline
6           & Definir requerimientos                  												& 4                           & 5                 & 8                           & 4,5                                       \\ \hline
7           & Definir Especificaciones                												& 3                           & 5                 & 8                           & 5                                         \\ \hline
8           & Planes de validacion                    												& 3                           & 5                 & 8                           & 11                                        \\ \hline
9           & DFMEA 1° reunion                        												& 0.25                        & 0.25              & 0.25                        & 8                                         \\ \hline
10          & \begin{tabular}[c]{@{}c@{}}Investigació antenas\\ y radiopropagación\end{tabular} 	& 44                          & 45                & 54                          & 6,7                                       \\ \hline
11          & \begin{tabular}[c]{@{}c@{}}Analisis de\\ factibilidad Tecnológica\end{tabular}    	& 40                          & 45                & 47                          & 6,7                                       \\ \hline
12          & \begin{tabular}[c]{@{}c@{}}Análisis de\\ presupuesto y costos\end{tabular}        	& 4                           & 5                 & 9                           & 6,7                                       \\ \hline
13          & \begin{tabular}[c]{@{}c@{}}Análisis de\\ factibilidad económica\end{tabular}      	& 3                           & 5                 & 8                           & 12                                        \\ \hline
14          & DFMEA 2° reunion                        												& 0.25                        & 0.25              & 0.25                        & 9,11,12                                   \\ \hline
15          & Cálculo y selección de HW               												& 12                          & 15                & 20                          & 11,12,13                                  \\ \hline
16          & \begin{tabular}[c]{@{}c@{}}Diagrama de HW y\\ plan de implementación\end{tabular} 	& 14                          & 15                & 18                          & 15                                        \\ \hline
17          & DFMEA 3° reunion                        												& 0.25                        & 0.25              & 0.25                        & 15                                        \\ \hline
18          & \begin{tabular}[c]{@{}c@{}}Diagrama de SW y\\ plan de implementación\end{tabular} 	& 11                          & 15                & 19                          & 15                                        \\ \hline
19          & Integración a nivel modulos             												& 23                          & 25                & 27                          & 16,18                                     \\ \hline
20          & Integración general                     												& 18                          & 20                & 21                          & 19                                        \\ \hline
21          & Integracion Prototipo                   												& 9                           & 12                & 13                          & 20                                        \\ \hline
22          & Vacaciones                              												& 15                          & 15                & 15                          & 21                                        \\ \hline
23          & Integración a Prototipo                 												& 9                           & 12                & 13                          & 22                                        \\ \hline
24          & Validación de prototipo                 												& 19                          & 20                & 21                          & 23                                        \\ \hline
25          & Estudio de confiabilidad                												& 6                           & 10                & 13                          & 24                                        \\ \hline
\end{tabular}
\caption{Actividades a realizar en el proyecto en \textit{días} de 8 horas cada uno.}
\label{tab:tareas}
\end{table}










\Subsubsection{Programación}


Se produjo el diagrama de Gantt acorde a la tabla recién mostrada, y se marcó el camino crítico en rojo.
\begin{figure}[H]
	\centering
	\includegraphics[width=1\linewidth]{ImagenesFactibilidad/project}
	\label{fig:gantt}
	\caption{Diagrama de Gantt del proyecto.}
\end{figure}

Luego se realizó una simulación de montecarlo utilizando la distribución $\beta$ para las variables aleatorias, obteniendo el siguiente gráfico donde se ve la probabilidad de terminar el proyecto en un intervalo de $1533 \ \sim \ 1957 $ horas con una probabilidad del 95$\%$. Teniendo en cuenta que por teorema central del límite la suma de las variables aleatorias $\beta$ convergen a una normal.
\begin{figure}[H]
	\centering
	\includegraphics[width=0.5\linewidth]{ImagenesFactibilidad/montecarlo}
	\label{fig:montecarlo_tiempos}
	\caption{Simulación de montecarlo.}
\end{figure}

Esta distribución es el lapso de tiempo entre que se comienza el proyecto y se termina, a travez del camino crítico. 
La cantidad de horas de ingeniería por ingeniero se muestran a continuación, teniendo en cuenta la paralelización de actividades y que se dispone de 4 trabajadores.
\begin{figure}[H]
	\centering
	\includegraphics[width=0.5\linewidth]{ImagenesFactibilidad/montecarlo_tiempo_largo}
	\label{fig:montecarlo_tiempos_ing}
	\caption{Simulación de montecarlo para tiempo de ingeniería.}
\end{figure}
Se puede observar que el timepo total de horas de ingeniería corresponde a un rango entre aproximadamente 6800 $\sim$ 7400 horas.
Con una media de aproximadamente 7100 horas.

\Subsection{Factibilidad Económica}
%(Mercado, costos, ciclo de vida, %VAN, TIR)
Para poder estudiar la viabilidad financiera de todo emprendimiento es necesario hacer un balance cuidadoso de los diferentes costos en los que se va a incurrir. Se debe analizar cuales serán la fuentes de ingresos que hagan del modelo de negocio planteado un negocio sostenible en el tiempo. 
En este caso en particular estamos analizando el caso de un proyecto único. Sin embargo, no se desea descartar la posibilidad de realizar más unidades en el futuro por fuera del marco del proyecto. Es por esto que presentamos aquí debajo el modelo de negocios planteado para referencia futura. 
\Subsubsection{Modelo de Negocios}

Este diseño se trata de un proyecto único, con posibilidad de realizar hasta \unidadespostfin unidades adicionales posteriores a su finalización. Para dar una visión global se planteo el siguiente modelo de Canvas:
 

\begin{figure}[H]
	\centering
	\includegraphics[scale=0.7]{../Factibilidad/ImagenesFactibilidad/ModeloDeCanvas}
	\caption{Modelo de negocio.}
	\label{fig:modelodecanvas}
\end{figure}



\Subsubsection{Investigación y Desarrollo}
Este proyecto contiene una gran componente de diseño. Según lo establecido en la programación del proyecto, el tiempo invertido de diseño esta estimado en 5472 horas. %todas las horas hasta "integracion a nivel modulos" sin incluir
Una vez completada la etapa de diseño el equipo se aboca a la implementación del primer prototipo, es decir a la integración del hardware a utilizar y el desarrollo del software que controla al nido que permite la interacción con el mismo. 

Realizando la estimación sobre un sueldo de $30 \ \frac{USD}{hora}$ y multiplicándolo por la cantidad de horas de ingeniería esperada por persona se obtiene un total de: $$Costo \ mano \ de \ obra = 7100 \ hora \ \cdot 30 \ \frac{USD}{hora} = 213.000,00 \ USD$$


\Subsubsection{Gastos fijos por unidad}

Los gastos principales considerados son USD$259.09$. %Para la compra de los componentes, se suman los costos estimados previamente, más el resto de los componentes misceláneos (resistores, capacitores, etc.) para el diseño de los circuitos involucrados.
%Se estima de este modo un costo de componentes de \TBD USD, a contabilizar una única vez por unidad.

Se presenta el valor de los insumos de hardware y de montaje:
\begin{table}[H]
\centering
\begin{tabular}{|c|c|}
\hline
\textbf{Item}                                                         & \textbf{Precio [USD]}				  \\ \hline
Sensor humedad-temperatura 											  & 4.9                                   \\ \hline
Sensor luminosidad                                                    & $1.54    $                              \\ \hline
M\'odulo RTC                                                     & $2.33$                                  \\ \hline
Cámara                                                                & $20$                                    \\ \hline
SDI 32GBy                                                             & $32$                                    \\ \hline
R-Pi 3B
 & $25.5$                                  \\ \hline
Batería                                                               & $86  $                 				  \\ \hline
Panel solar                                                           & $30$ 				  \\ \hline
MPPT                                                                  & $18$                                    \\ \hline
Encapsulado                                                           & $16$                                    \\ \hline
Módulo Bluetooth 5.0                                                           & $3.82 $                                   \\ \hline
Montaje                                                               & $31$                                    \\ \hline
Extras
& $15$									\\ \hline
\textbf{Total}
&\textbf{$259.09$} \\ \hline
\end{tabular}
\caption{Valores de insumos.}
\end{table}


\Subsubsection{Reserva de Contingencia}
Dado que este proyecto cuenta con un alto grado de investigación es necesario contar con un cierto margen de seguridad en caso de que ocurra un cambio de planes necesario para alcanzar el objetivo del proyecto. Es por eso que se decidió incluir en nuestro análisis un adicional del 5\% sobre el costo total del proyecto. Esta suma sera devuelta al cliente en caso de no necesitarla.

La reserva de contingencia se ubicara en  USD 11.000
\Subsubsection{Escenario de Escala}
Se contempla que en el caso de que la producción sea de \unidadespostfin unidades será posible conseguir los insumos necesarios para el ensamblado de las unidades a precio de venta al por mayor.


\Subsection{Factibilidad Legal y Responsabilidad Civil}
\label{sec:legal}
En cuanto a la emisión de ondas electromagnéticas, las normas y regulaciones que se deben considerar son:
\begin{itemize}
	\item Resolución 202/95 (Estándar Nacional de Seguridad para la Exposición a radiofrecuencias comprendidas entre 100 $KHz$ y 300 $GHz$).
	\item Resolución 530/2000 (Estándar Nacional de Seguridad de aplicación obligatoria a todos los sistemas de telecomunicaciones que irradian en determinadas frecuencias).
	\item Resolución 1994/2015 (Regulación SAR).
	\item \textit{Guidelines} de la FCC en exposición máxima recomendada. Una densidad de potencia de 580 $\mu W/{cm}^2$ @ 850 $MHz$.
	\item Ministerio de Salud y Acción Social admite como máxima una densidad de potencia de 450 $\mu W/{cm}^2$ @ 850 $MHz$.
	\item \textit{The Worldwide Approval Status fo}r $900 \ MHz$ \textit{and} $2.4 \ GHz$ \textit{Spread Spectrum Radio Products}. Establece un límite de 4 $W$ para la potencia de la antena transmisora.
\end{itemize}

En lo que a la preservación del medio ambiente respecta, la Ley 26.331 estipula los siguientes aspectos de interés:
\begin{itemize}
	\item Se debe considerar un Plan de Manejo Sostenible de Bosques Nativos.
	\item El Plan debe cumplir las condiciones mínimas de persistencia, producción sostenida y mantenimiento de los servicios ambientales de los bosques.
\end{itemize}

A esto se le suma lo estipulado en la Ley 13.273. En esta se encuentra que se constituye como contravenciones forestales destruir, remover o suprimir señales o indicadores colocados por la autoridad forestal, entre otras cosas.


%\end{document}