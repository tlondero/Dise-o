En cuanto a la emisión de ondas electromagnéticas, las normas y regulaciones que se deben considerar son:
\begin{itemize}
	\item Resolución 202/95 (Estándar Nacional de Seguridad para la Exposición a radiofrecuencias comprendidas entre 100 $KHz$ y 300 $GHz$).
	\item Resolución 530/2000 (Estándar Nacional de Seguridad de aplicación obligatoria a todos los sistemas de telecomunicaciones que irradian en determinadas frecuencias).
	\item Resolución 1994/2015 (Regulación SAR).
	\item \textit{Guidelines} de la FCC en exposición máxima recomendada. Una densidad de potencia de 580 $\mu W/{cm}^2$ @ 850 $MHz$.
	\item Ministerio de Salud y Acción Social admite como máxima una densidad de potencia de 450 $\mu W/{cm}^2$ @ 850 $MHz$.
	\item \textit{The Worldwide Approval Status fo}r $900 \ MHz$ \textit{and} $2.4 \ GHz$ \textit{Spread Spectrum Radio Products}. Establece un límite de 4 $W$ para la potencia de la antena transmisora.
\end{itemize}

En lo que a la preservación del medio ambiente respecta, la Ley 26.331 estipula los siguientes aspectos de interés:
\begin{itemize}
	\item Se debe considerar un Plan de Manejo Sostenible de Bosques Nativos.
	\item El Plan debe cumplir las condiciones mínimas de persistencia, producción sostenida y mantenimiento de los servicios ambientales de los bosques.
\end{itemize}

A esto se le suma lo estipulado en la Ley 13.273. En esta se encuentra que se constituye como contravenciones forestales destruir, remover o suprimir señales o indicadores colocados por la autoridad forestal, entre otras cosas.
