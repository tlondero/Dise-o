La UP representa el cerebro del proyecto. Este se ocupa de procesar la información de los sensores, almacenarla en la SD e iniciar los procesos de comunicación. En otras palabras, el integrado se ocupa de conectar los distintos módulos entre sí y garantizar su adecuado funcionamiento.

\begin{table}[H]
\centering
\begin{tabular}{|c|c|c|c|}
\hline
\textbf{\begin{tabular}[c]{@{}c@{}}Aspectos\\ comparativos\end{tabular}}         & \textbf{\href{https://datasheets.raspberrypi.org/rpi4/raspberry-pi-4-product-brief.pdf}{R-Pi 4}} & \textbf{\href{https://www.raspberrypi.org/products/raspberry-pi-zero-w/}{R-Pi Zero W}}            & \textbf{\begin{tabular}[c]{@{}c@{}}\href{https://datasheets.raspberrypi.org/cm4/cm4-product-brief.pdf}{R-Pi Compute}\\ \href{https://datasheets.raspberrypi.org/cm4/cm4-product-brief.pdf}{Module 4}\end{tabular}} \\ \hline
\textbf{Costo [ARS]}                                                                   & 10400 $\sim$ 17000                                                                               & 6000 $\sim$ 7500                                                                                & 12440 $\sim$ 14310                                                                                                                                                                                                 \\ \hline
\textbf{\begin{tabular}[c]{@{}c@{}}Temperatura de\\ operación [°C]\end{tabular}} & 0 $\sim$ 50                                                                                      & -20 $\sim$ 85 & -20 $\sim$ 85                                                                                                                                                                                                      \\ \hline
\textbf{Memoria}                                                            & 1 [GB] $\sim$ 8[GB]                                                                                        & 512 [MB]                                                                                             & 1 [GB] $\sim$ 8 [GB]                                                                                                                                                                                                         \\ \hline
\textbf{Conexiones}                                                              & \begin{tabular}[c]{@{}c@{}}Wireless LAN,\\ Bluetooth 5.0,\\ Ethernet, USB\end{tabular}               & \begin{tabular}[c]{@{}c@{}}Wireless LAN,\\ Bluetooth 4.1 (BLE),\\ Micro USB, mini HDMI\end{tabular} & \begin{tabular}[c]{@{}c@{}}Wireless LAN,\\ Bluetooth 5.0 (BLE),\\ Ethernet, USB,\\ antena externa\end{tabular}                                                                                                               \\ \hline
\textbf{Sonido y video}                                                          & \begin{tabular}[c]{@{}c@{}}Micro HDMI,\\ MIPI DSI y CSI\end{tabular}                             & \begin{tabular}[c]{@{}c@{}}Mini HDMI, HDMI,\\ CSI, PAL/NTSC pads\end{tabular}                   & \begin{tabular}[c]{@{}c@{}}HDMI, MIPI DSI\\ y CSI, SDIO\end{tabular}                                                                                                                                               \\ \hline
\textbf{Soporte SD}                                                              & \begin{tabular}[c]{@{}c@{}}Almacenamiento y\\ carga de SO\end{tabular}                           & Micro SD                                                                                        & \begin{tabular}[c]{@{}c@{}}Entrada SD para tarjeta\\ o eMMC externo\end{tabular}                                                                                                                                   \\ \hline
\textbf{Dimensiones [mm]}                                                        & 85.6 x 56.5                                                                                      & 65 x 30                                                                                         & 40 x 55 \\ \hline
\textbf{Alimentación}                                                            & 5 V (3 A)                                                                                        & 5 (1.2 A)                                                                                       & 5 V (1.4 A)                                                                                                                                                                                                        \\ \hline
\textbf{Imagen}                                                                  & \includeintable{.1}{ImagenesFactibilidad/RPI4}                                   & \includeintable{.1}{ImagenesFactibilidad/RPIZero}                                & \includeintable{.1}{ImagenesFactibilidad/RPICM}                                                                                                                                                                    \\ \hline
\end{tabular}
\caption{Comparación entre palcas Raspberry Pi.}
\end{table}

%\lnote{Creo que el procesador de todas va más o menos de -40 °C a 85 °C, pero las demás cosas lo limitan, por ejemplo el modulo de LAN creo que no va menos de 0 °C.}

%https://copperhilltech.com/content/The%20Operating%20Temperature%20For%20A%20Raspberry%20Pi%20%E2%80%93%20Technologist%20Tips.pdf