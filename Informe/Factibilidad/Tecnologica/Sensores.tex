Para las distintas mediciones se tuvieron en cuenta diversas tecnologías que existen. Se evaluaron parámetros que definen la performance, tales como la linealidad de salida, el costo, el rango de operación, la precisión, el tipo de salida, aplicación, entre otras tantas variables.

\Subsubsubsection{Temperatura}
En el caso de la medición de temperatura, se valoraron diversas tecnologías que existen, siendo por ejemplo la RTD cuyo funcionamiento se basa en el cambio de la resistencia en función de la temperatura bajo al ecuación $R(T)=R_0 + \alpha \cdot \Delta T$. También se consideró la tecnología TC, cuyo funcionamiento se basa en el efecto seebek. Finalmente, el uso de un IC, el cual se basa en propiedades de dispositivos semiconductores extrínsecos.

\begin{table}[H]
\centering
\begin{tabular}{|c|c|c|c|c|}
\hline
\textbf{\begin{tabular}[c]{@{}c@{}}Aspectos\\ comparativos\end{tabular}} & \textbf{\href{https://www.thermocoupleinfo.com/type-k-thermocouple.htm}{TC-K}}                                                                             & \textbf{\href{http://www.datasheet.es/PDF/900325/Pt100-pdf.html}{PT-100}}                                                                                                              & \textbf{\href{https://datasheets.maximintegrated.com/en/ds/DS18B20.pdf}{Ds18b20}}                                                 & \textbf{\href{https://www.sparkfun.com/datasheets/Sensors/Temperature/DHT22.pdf}{DHT-22}}  \\ \hline
\textbf{Costo [USD]}                                                           & 4.6	& 5.2	& 1.4	& 4.9	\\ \hline
\textbf{Tipo de salida}                                                  & Analógico                                                                                 & Analógico                                                                                                                    & Digital                                                          & Digital          \\ \hline

\textbf{\begin{tabular}[c]{@{}c@{}}Temperatura de\\ operación [°C]\end{tabular}}                                              & -40 $\sim$ 1200                                                                        & -50 $\sim$ 200  & -10 $\sim$ 85 & -40 $\sim$ 80 \\ \hline
\textbf{\begin{tabular}[c]{@{}c@{}}Interfaz de\\ conexionado\end{tabular}}                                                      & \begin{tabular}[c]{@{}c@{}}Se debe\\ proporcionar\\ un circuito\\ amplificador\end{tabular} & \begin{tabular}[c]{@{}c@{}}Se debe \\ proporcionar\\  un circuito\\  convertidor\\  de resistencia\\  a tensión\end{tabular} & -                                                                & -                \\ \hline
\textbf{Presición [°C]}                                                       & $\pm$ 1.5	& $\pm$ 0.1	& $\pm$ 0.5	& $\pm$ 0.5	\\ \hline

\textbf{Estabilidad}                                                     & Tienden a envejecer                                                                       & -                                                                                                                            & -                                                                & -                \\ \hline
\textbf{Autocalentamiento}                                               & -                                                                                         & \begin{tabular}[c]{@{}c@{}}Depende de\\  la corriente\\  de medición.\end{tabular}                                           & Bajo                                                             & Bajo             \\ \hline
\textbf{Imagen}                &  \includeintable{.1}{ImagenesFactibilidad/TC}                                                                                          &  \includeintable{.1}{ImagenesFactibilidad/PT100}                                                                                                                    & \includeintable{.1}{ImagenesFactibilidad/IC1}  &  \includeintable{.1}{ImagenesFactibilidad/DHT-22}                 \\ \hline
\end{tabular}
\caption{Comparación entre sensores de temperatura.}
\end{table}

\Subsubsubsection{Humedad}
Existen varias maneras de medir la magnitud física de la humedad, dentro de estas la mas común se basa en utilizar la dependencia que existe entre la humedad y la capacidad. Es por esto que se utilizan capacitores con un dieléctrico, el cual cambia constante con la humedad. Además existen sensores que se aprovechan de como cambia la resistencia en función de la temperatura, pero estas tecnologías son menos frecuentes.

\begin{table}[H]
\centering
\begin{tabular}{|c|c|c|c|c|}
\hline
\textbf{\begin{tabular}[c]{@{}c@{}}Aspectos\\ comparativos\end{tabular}} & \textbf{\href{https://www.mouser.com/datasheet/2/758/DHT11-Technical-Data-Sheet-Translated-Version-1143054.pdf}{DHT-11}}   & \textbf{\href{http://codigoelectronica.com/blog/am2301-datasheet}{AM-2301}}  & \textbf{\href{https://www.sparkfun.com/datasheets/Sensors/Temperature/DHT22.pdf}{DHT-22}}   & \textbf{\href{https://datasheetspdf.com/pdf/1298922/AOSONG/AM1001/1}{AM-1001}}  \\ \hline
\textbf{Costo [USD]}                                                           & 1.3           & 7 & 4.93 & 5.6	\\ \hline
\textbf{\begin{tabular}[c]{@{}c@{}}Rango de\\ operación [\%RH]\end{tabular}}                                              & 20 $\sim$ 90 & 0 $\sim$ 100 & 0 $\sim$ 100 & 20 $\sim$ 90 \\ \hline
\textbf{Presición [\%RH]}                                                       & $\pm$4	& $\pm$3	 & $\pm$2	& $\pm$5	\\ \hline
\textbf{Tipo de salida}                                                  & Digital           & Digital           & Digital           & Analógica         \\ \hline
\textbf{Imagen}                                                          & \includeintable{.1}{ImagenesFactibilidad/DHT-11}                 & \includeintable{.1}{ImagenesFactibilidad/AM-2301}                 & \includeintable{.1}{ImagenesFactibilidad/DHT-22}                 & \includeintable{.1}{ImagenesFactibilidad/AM-1001} \\ \hline
\end{tabular}
\caption{Comparación de sensores de humedad.}
\end{table}

\Subsubsubsection{Luminosidad}
En la medición del nivel de luminosidad se puede optar por diversos caminos. Existen sensores como el BH-1750 y OPT-100 que su funcionamiento se basa en un fotodiodo que conduce cierta corriente a partir de la luz que le impacta. Otros sensores, tales como el TEMT-600, emplean un fototransistor, cuya base se encuentra expuesta. En función de la intensidad lumínica en dicha zona, circulará cierta corriente por el colector. Finalmente existen fotoresistores, los cuales, tal como su nombre indica, cambian la resistencia en función del nivel de luz.

\begin{table}[H]
\centering
\begin{tabular}{|c|c|c|c|c|}
\hline
\textbf{\begin{tabular}[c]{@{}c@{}}Aspectos\\ comparativos\end{tabular}}              & \textbf{\href{https://www.mouser.com/datasheet/2/348/bh1750fvi-e-186247.pdf}{BH-1750}}  & \textbf{\href{https://www.vishay.com/docs/81579/temt6000.pdf}{TEMT-6000}}	& \textbf{\href{https://www.ti.com/lit/ds/symlink/opt101.pdf}{OPT-101}}                                              & \textbf{\href{https://www.alldatasheet.es/view.jsp?Searchword=GL55}{GL55-LM393}} \\ \hline
\textbf{Costo [USD]}	& 1.54 	& 2.27	& 2.26	& 1.26	\\ \hline
\textbf{\begin{tabular}[c]{@{}c@{}}Temperatura de\\ operación [°C]\end{tabular}} & -40 $\sim$ 85  & -40 $\sim$ 85 & 0 $\sim$ 70 & -30 $\sim$ 70 \\ \hline
\textbf{\begin{tabular}[c]{@{}c@{}}Potencia\\ disipada [mW]\end{tabular}}                                                            & - 	& 100 	& - 	& 75 	\\ \hline
\textbf{Tipo de salida}                                                               & I2C               & \begin{tabular}[c]{@{}c@{}}Analógica\\ (Corriente)\end{tabular}                 & \begin{tabular}[c]{@{}c@{}}Analógica\\ (Tensión)\end{tabular} &\begin{tabular}[c]{@{}c@{}}Analógica \\ Digital\end{tabular}  \\ \hline
\textbf{Aplicación}                                                                   & -                 & \begin{tabular}[c]{@{}c@{}}Necesita un\\amplificador\\ de corriente\end{tabular} & -                                                             & -                   \\ \hline
\begin{tabular}[c]{@{}c@{}}\textbf{Tensión de} \\ \textbf{alimentación [V]} \end{tabular}                                                      & 2.4$\sim$3.6 	& \textless \ 6 	& 2.7 $\sim$ 36.0 	& 3.3 $\sim$ 5.0 	\\ \hline
\textbf{\begin{tabular}[c]{@{}c@{}}Rango de\\ medición [nm]\end{tabular}}                                                            & 450 $\sim$ 650 & 400 $\sim$ 900 & 450 $\sim$ 1000 & 450 $\sim$ 750 \\ \hline
\textbf{Imagen}                                                                       & \includeintable{.1}{ImagenesFactibilidad/BH-1750} & \includeintable{.1}{ImagenesFactibilidad/TEMT-6000}                                                                               & \includeintable{.1}{ImagenesFactibilidad/OPT-101} & \includeintable{.1}{ImagenesFactibilidad/GL55-LM393}                   \\ \hline
\end{tabular}
\caption{Comparación de sensores de luminosidad.}
\end{table}


\Subsubsubsection{Imágenes}
Para la obtención de captura imágenes como de video, teniendo en cuenta la tecnología utilizada para la unidad de procesamiento, se encontraron diversos módulos de cámara que se pueden usar:

\begin{table}[H]
\centering
\begin{tabular}{|c|c|c|c|c|}
\hline
\textbf{\begin{tabular}[c]{@{}c@{}}Aspectos\\ comparativos\end{tabular}} 					& \textbf{\href{https://cdn.sparkfun.com/datasheets/Dev/RaspberryPi/ov5647_full.pdf}{RPi-CMOD-V1}}      & \textbf{\href{https://www.arducam.com/sony/imx477/}{RPi-CMOD-V2}}     & \textbf{\href{https://www.adafruit.com/product/3100?gclid=Cj0KCQjwkt6aBhDKARIsAAyeLJ0sQqYZitG6gblfi1iEIJIDYsrWSN4IqN2gniU_8WYVZfyj-V6t2ZgaAu8CEALw_wcB}{RPi NoIR V2}}  		& \textbf{\href{https://www.digikey.com/catalog/es/partgroup/zero-spy-camera-for-raspberry-pi-zero-board/70309}{RPi-ZEROC}}     \\ \hline
\textbf{Costo [USD]}                                                     					& 25		                        																	& 25 	                           										& 30 																								& 20                          																									\\ \hline
\textbf{Tamaño [mm]}                                                 	 					& 25 x 24 x 9                      																		& 25 x 24 x 9                      										& 38 x 38 x 18.4     																				& 8.6 x 8.6 x 5.2            																									\\ \hline
\textbf{\begin{tabular}[c]{@{}c@{}}Resolución de la\\ cámara [MP]\end{tabular}}             & 5	                               																		& 8	                               										& 8   																							& 5                          																									\\ \hline
\textbf{Integración Linux}                                               					& V4L2 driver                     																		& V4L2 driver                      										& V4L2 driver																						& V4L2 driver                      																								\\ \hline
\textbf{Capacidad nocturna}                                                           					& No              																		& No               										& Si         																						& No                       																										\\ \hline
\textbf{Peso [g]}                                                        					& 3                               																		& 3.4                              										& 12                           																		& 1.1    																														\\ \hline
\textbf{Sensor}                                                          					& OmniVision OV5647                																		& Sony IMX219                      										& Sony IMX219  																						& OV5647                    																									\\ \hline
\textbf{\begin{tabular}[c]{@{}c@{}}Temperatura de\\ operación [°C]\end{tabular}}            & \multicolumn{1}{c|}{-25$\sim$80} 																		& \multicolumn{1}{c|}{-25$\sim$80} 										& \multicolumn{1}{c|}{-25$\sim$80} 																	& \multicolumn{1}{c|}{-25$\sim$80} 																								\\ \hline
\textbf{Imagen}                                                          					& \includeintable{.1}{ImagenesFactibilidad/RPICAMV1}                  									& \includeintable{.1}{ImagenesFactibilidad/RPICAMV2}                    & \includeintable{.1}{ImagenesFactibilidad/RPInoir}         &
\includeintable{.1}{ImagenesFactibilidad/RPIZEROCAM}       \\ \hline
\end{tabular}
\caption{Comparación entre cámaras.}
\end{table}


\Subsubsubsection{Modulo RTC}
Es fundamental contar con un m\'etodo para conocer la fecha y hora de manera confiable. Tanto como para saber en que momento fueron tomadas las mediciones como as\'i tambi\'en saber en que momentos habilitar el hotspot wifi.
Teniendo esto en cuenta se proponen los siguientes m\'odulos:
\begin{table}[H]
\centering
\begin{tabular}{|c|c|c|c|}
\hline
\textbf{Aspectos comparativos}                                                & \textbf{\href{https://pdf1.alldatasheet.com/datasheet-pdf/view/112132/DALLAS/DS3231.html}{Ds3231}}          &\textbf{\href{https://pdf1.alldatasheet.es/datasheet-pdf/view/58478/DALLAS/DS1302.htmll}{Ds1302}}           & \textbf{\href{https://pdf1.alldatasheet.com/datasheet-pdf/view/58481/DALLAS/DS1307.html}{Ds1307}}                   \\ \hline
\textbf{Costo {[}USD{]}}                                                      & 2.33           & 0.81           & 1.37           \\ \hline
\textbf{\begin{tabular}[c]{@{}c@{}}Temperatura \\ de Operación\end{tabular}}  & -40 $\sim$85   & -40 $\sim$85   & -40 $\sim$85   \\ \hline
\textbf{Rango de tensiones}                                                   & 2.3 $\sim$5.5  & 2 $\sim$5.5    & 4.5$\sim$5.5   \\ \hline
\textbf{\begin{tabular}[c]{@{}c@{}}Protocolo de \\ Comunicación\end{tabular}} & I2C            & SPI            & I2C            \\ \hline
\textbf{Precisión}                                                            & 3.5ppm         & -              & -              \\ \hline
\textbf{Imagen}                                                               & \includeintable{.1}{ImagenesFactibilidad/DS3231} & \includeintable{.1}{ImagenesFactibilidad/ds1302} & \includeintable{.1}{ImagenesFactibilidad/ds1307} \\ \hline
\end{tabular}
\end{table}


