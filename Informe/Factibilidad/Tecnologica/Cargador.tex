La solución planteada para la transferencia de energía al dispositivo del ave será por medio de radiación electromagnética. \observacion{\verObs}{Aca habría que hablar de que vamos a usar un panel solar chico para complementar los de RF}

\Subsubsubsection{Amplificador de Potencia}

Para el amplificador de potencia se tuvieron en cuenta distintos módulos armados.

\begin{table}[H]
\centering
\begin{tabular}{|c|c|c|c|c|}
\hline
\textbf{\begin{tabular}[c]{@{}c@{}}Aspectos\\ comparativos\end{tabular}}         & \textbf{\begin{tabular}[c]{@{}c@{}}\href{https://www.ebay.com/itm/133414335638}{NWDZ RF PA}\\ \href{https://www.ebay.com/itm/133414335638}{V2.0}\end{tabular}} & \textbf{\begin{tabular}[c]{@{}c@{}}\href{https://www.ebay.com/itm/284470793842}{Unbranded Broadband}\\ \href{https://www.ebay.com/itm/284470793842}{RF Amplifier}\end{tabular}} & \textbf{\begin{tabular}[c]{@{}c@{}}\href{https://www.ebay.com/itm/124120749418}{DH-RF}\\ \href{https://www.ebay.com/itm/124120749418}{V2017}\end{tabular}} & \textbf{\begin{tabular}[c]{@{}c@{}}\href{http://www.fenk.com.ar/productos/energias-renovables/baterias-solares/}{JP-2Y560895}\\ \href{http://www.fenk.com.ar/productos/energias-renovables/baterias-solares/}{RF Amplifier}\end{tabular}} \\ \hline
\textbf{Costo [USD]}                                                             & 18.59                                                                                                                                                                                                                                               & 9.82                                                                                                                                                                                                                                                  & 23                                                                                                                                                                                                              & 69.31                                                                                                                                                                                                                            \\ \hline
\textbf{Alimentación [V]}                                                        & 12-15                                                                                                                                                                                                                                               & 12                                                                                                                                                                                                                                                    & 15                                                                                                                                                                                                              & 24-28                                                                                                                                                                                                                            \\ \hline
\textbf{Frecuencia [MHz]}                                                        & 2-700                                                                                                                                                                                                                                               & 1-930                                                                                                                                                                                                                                                 & 1-1000                                                                                                                                                                                                          & 915 +- 25                                                                                                                                                                                                                        \\ \hline
\textbf{\begin{tabular}[c]{@{}c@{}}Potencia salida\\ máxima [dBm]\end{tabular}}  & 34.8                                                                                                                                                                                                                                                & 29                                                                                                                                                                                                                                                    & 35                                                                                                                                                                                                              & 42                                                                                                                                                                                                                               \\ \hline
\textbf{\begin{tabular}[c]{@{}c@{}}Potencia entrada\\ máxima [dBm]\end{tabular}} & 10                                                                                                                                                                                                                                                  & 0                                                                                                                                                                                                                                                     & 0                                                                                                                                                                                                               & 23                                                                                                                                                                                                                               \\ \hline
\textbf{\begin{tabular}[c]{@{}c@{}}Ganancia de\\ potencia [dB]\end{tabular}}     & 35                                                                                                                                                                                                                                                  & 29                                                                                                                                                                                                                                                    & 32                                                                                                                                                                                                              & Ajustable                                                                                                                                                                                                                        \\ \hline
\textbf{Dimensiones [cm]}                                                        & 7x3                                                                                                                                                                                                                                                 & 5x5                                                                                                                                                                                                                                                   & 3.7x5.6                                                                                                                                                                                                         & 4x8                                                                                                                                                                                                                              \\ \hline
\textbf{Imagen}                                                                  & \includeintable{.1}{ImagenesIngenieria de Detalle/nwdz}                                                                                                                                                                                             & \includeintable{.1}{ImagenesIngenieria de Detalle/unbranded}                                                                                                                                                                                          & \includeintable{.1}{ImagenesIngenieria de Detalle/DH-RF-V2017}                                                                                                                                                  & \includeintable{.1}{ImagenesIngenieria de Detalle/s-l1600}                                                                                                                                                                       \\ \hline
\end{tabular}
\end{table}

\Subsubsubsection{Integrado de Energy Harvesting}

Se investigaron diversos métodos de transmisión de energía inalámbrica a través de campos electromagnéticos. Dentro de estos se en cuenta el campo cercano y el de radiofrecuencia. El primero se caracteriza por ser puramente imaginario y por poseer una caída proporcional al cuadrado de la distancia, mientras que el segundo (radiación) cae linealmente.

Debido a que la aplicación es de una distancia que cae en el rango del campo lejano, se optó por la radiofrecuencia. Es así que surgieron los integrados IC-P2110 y IC-P1110 de PowerCast, los cuales permiten la recolección de energía de radio frecuencia almacenando esta en capacitores o con la opción de directamente cargar una batería (P1110). A continuación se comparan los dos IC.

\begin{table}[H]
\centering
\begin{tabular}{|c|c|c|}
\hline
\textbf{\begin{tabular}[c]{@{}c@{}}Aspectos\\ comparativos\end{tabular}}                          	&  \textbf{\href{https://www.powercastco.com/documentation/p1110-evb-datasheet/}{IC-P1110}}  	&\textbf{\href{https://www.powercastco.com/wp-content/uploads/2016/12/P2110B-Datasheet-Rev-3.pdf}{IC-P2110}}                 \\ \hline
\textbf{Costo [USD]}                                                                              	& 48.33	                                                                                           				& 32                                                                                              		\\ \hline
\textbf{\begin{tabular}[c]{@{}c@{}}Funcionalidad\\ principal\end{tabular}}                          & \begin{tabular}[c]{@{}c@{}}Recolección y almacenamiento\\ de energía para uso variado\end{tabular} 			& \begin{tabular}[c]{@{}c@{}}Recolección de energía para carga\\ de baterías/Capacitores\end{tabular} 	\\ \hline
\textbf{\begin{tabular}[c]{@{}c@{}}Frecuencia de\\ trabajo [MHz]\end{tabular}}                    	& 910 $\sim$ 928                                                                                     			& 910 $\sim$ 928                                                                                        \\ \hline
\textbf{\begin{tabular}[c]{@{}c@{}}Eficiencia del PH\\ para RFin = 11 dBm\end{tabular}}            	& 60 \%                                                                                               			& 45 \%                                                                                                 \\ \hline
\textbf{\begin{tabular}[c]{@{}c@{}}Corriente de salida\\ para RFin = 11 dBm\end{tabular}}           & 3 mA                                                                                                			& -                                                                                                     \\ \hline
\textbf{\begin{tabular}[c]{@{}c@{}}Timepo de carga \\ inicial del capacitor [s]\end{tabular}} 	  	& -                                                                                                   			& < 5                                                                                           		\\ \hline
\textbf{\begin{tabular}[c]{@{}c@{}}Posee placa\\ de evaluación\end{tabular}}                  		& Sí                                                                                                  			& Sí                                                                                                    \\ \hline
\textbf{\begin{tabular}[c]{@{}c@{}}Impedancia de\\ entrada  [$\Omega$] \end{tabular}}             	& 50                                                                                                  			& 50                                                                                                    \\ \hline
\textbf{\begin{tabular}[c]{@{}c@{}}Temperatura de\\ operación [°C]\end{tabular}}                  	& -40 $\sim$ 85                                                                                       			& -40 $\sim$ 85                                                                                         \\ \hline
\end{tabular}
\caption{Comparación entre cargadores.}
\end{table}

