Se tiene en cuenta que la solución para la transferencia de energía para el dispositivo del ave será por medio de radiación electromangética. \observacion{\verObs}{Aca habría que hablar de que descartamos los piezoelectriocos, termoeléctricos y solares quedando como única opcion los campo}
\Subsubsubsection{Integrado de Energy Harvesting}
Se investigaron diversos métodos de obtención de energía inalambrica a travez de campos , estos siendo el campo cercano (puramente imaginario y cae con el cuadrado de la distancia) y el de radiofrecuencia (radiación, que cae linealmente) debido a que nuestra aplicacion es de una distancia considerable se optó por la radiofrecuencia, por lo que surgieron los IC p2110 y P1110 de PowerCast,  que permiten la recolección de energía de radio frecuencia almacenando esta en capacitores o con la opcion de directamente la carga de una batería (p1110) . A continuacion se comparan los dos IC.

\begin{table}[]
\centering
\begin{tabular}{|c|c|c|}
\hline
\textbf{\begin{tabular}[c]{@{}c@{}}Aspectos\\ comparativos\end{tabular}}                          & \textbf{L2110}                                                                                      & \textbf{L1110}                                                                                        \\ \hline
\textbf{Costo}                                                                                    & 48.33 USD                                                                                           & 32 USD                                                                                                \\ \hline
\textbf{Funcionalidad principal}                                                                  & \begin{tabular}[c]{@{}c@{}}Recolección y almacenamiento\\  de energía para uso variado\end{tabular} & \begin{tabular}[c]{@{}c@{}}Recollección de energía para carga\\  de baterías/Capacitores\end{tabular} \\ \hline
\textbf{Frecuencia de trabajo}                                                                    & 910$\sim$928MHz                                                                                     & 910$\sim$928MHz                                                                                       \\ \hline
\textbf{\begin{tabular}[c]{@{}c@{}}Eficiencia del PH\\ para RFin = 11dBm\end{tabular}}            & 60\%                                                                                                & 45\%                                                                                                  \\ \hline
\textbf{\begin{tabular}[c]{@{}c@{}}Corriente de salida\\ para RFin=11dBm\end{tabular}}            & 3mA                                                                                                 & -                                                                                                     \\ \hline
\textbf{\begin{tabular}[c]{@{}c@{}}Timepo de carga \\ inicial del capacitor {[}s{]}\end{tabular}} & -                                                                                                   & \textless 5                                                                                           \\ \hline
\textbf{\begin{tabular}[c]{@{}c@{}}Existen \\ placas de evaluacion\end{tabular}}                  & Si                                                                                                  & Si                                                                                                    \\ \hline
\textbf{Impedancia de entrada  {[}$\Omega${]}}                                                    & 50                                                                                                  & 50                                                                                                    \\ \hline
\textbf{Rango de temperaturas {[}°C{]}}                                                           & -40 $\sim$ 85                                                                                       & -40 $\sim$ 85                                                                                         \\ \hline
\end{tabular}
\end{table}