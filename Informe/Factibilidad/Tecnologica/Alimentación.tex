Para poder abastecer a todos los módulos anteriormente mencionados, es necesario la existencia de un módulo que brinde dicha energía. Dadas la ubicación remota donde se encontrará el producto final, se opta por emplear un panel solar, capaz de obtener energía del entorno y no de la red eléctrica.

Lo que principalmente determinará la elección de este componente es el consumo de las demás partes.

\begin{table}[H]
\centering
\begin{tabular}{|c|c|c|c|c|c|}
\hline
\textbf{\begin{tabular}[c]{@{}c@{}}Aspectos\\ comparativos\end{tabular}}                   & \textbf{DSP-20P} & \textbf{DSP-30M} & \textbf{LN-50P} & \textbf{ESPMC210} & \textbf{LNSE-260P} \\ \hline
\textbf{Costo [ARS]}                                                       				   & 3400			  & 5000			 & 6000 		   & 15600 			   & 14500				\\ \hline
\textbf{\begin{tabular}[c]{@{}c@{}}Temperatura de\\ operación [°C]\end{tabular}}           & -45 $\sim$ 85    & -45 $\sim$ 85    & -45 $\sim$ 85   & -40 $\sim$ 85     & -45 $\sim$ 85      \\ \hline
\textbf{\begin{tabular}[c]{@{}c@{}}Potencia\\ máxima [W]\end{tabular}}                     & 20 $\pm$ 3\%     & 30 $\pm$ 3\%     & 50 $\pm$ 3\%    & 210               & 260                \\ \hline
\textbf{\begin{tabular}[c]{@{}c@{}}Tensión a\\ potencia\\ máxima [V]\end{tabular}}         & 17.6             & 18.0             & 18.0            & 18.85             & 30.4               \\ \hline
\textbf{\begin{tabular}[c]{@{}c@{}}Corriente\\ a potencia\\ máxima [A]\end{tabular}}       & 1.14             & 1.67             & 2.78            & 11.15             & 8.55               \\ \hline
\textbf{\begin{tabular}[c]{@{}c@{}}Tensión a\\ circuito abierto\\ máxima [V]\end{tabular}} & 22.0             & 21.5             & 22.3            & 23.2              & 37.4               \\ \hline
\textbf{\begin{tabular}[c]{@{}c@{}}Corriente a\\ corto circuito\\ máxima [A]\end{tabular}} & 1.39             & 1.86             & 3.01            & 11.8              & 9.11               \\ \hline
\end{tabular}
\caption{Comparación entre paneles solares.}
\end{table}