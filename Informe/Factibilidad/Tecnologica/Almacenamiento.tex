Para almacenar información, se puede valer de memorias SD. Existe una gran variedad, permitiendo priorizar diversos aspectos a la hora de optar por una opción. La velocidad de lectura, la de escritura y el almacenamiento son algunos de estos aspectos, aunque en este proyecto también es importante considerar el rango de temperatura de operación.

\begin{table}[H]
\centering
\begin{tabular}{|c|c|c|c|}
\hline
\textbf{\begin{tabular}[c]{@{}c@{}}Aspectos\\ comparativos\end{tabular}}         & \textbf{\href{https://www.kingston.com/datasheets/sdcg3_es.pdf}{SDCG3}} & \textbf{\href{https://www.kingston.com/datasheets/mlpmr2_es.pdf}{SDCE}} & \textbf{\href{https://ar.mouser.com/datasheet/2/669/SanDisk_02052018_SDSDAF3_SDSDQAF3-1285144.pdf}{SDSDQAF3-XI}} \\ \hline
\textbf{Costo [USD]}                                                             & 20                                                        & 52 & 32                                                                                                 \\ \hline
\textbf{\begin{tabular}[c]{@{}c@{}}Temperatura de\\ operación [°C]\end{tabular}} & -25 $\sim$ 85                                                           & -25 $\sim$ 85                                                           & -40 $\sim$ 85                                                                                                    \\ \hline
\textbf{Almacenamiento [GB]}                                                     & 64 $\sim$ 512                                                           & 64 $\sim$ 256                                                           & 8 $\sim$ 128                                                                                                     \\ \hline
\textbf{\begin{tabular}[c]{@{}c@{}}Velocidad\\ R/W [MB/s]\end{tabular}}          & 170 / 90                                                                & 285 / 165                                                               & 50 / 80                                                                                                          \\ \hline
\textbf{Alimentación [V]}                                                        & 3.3                                                                     & 3.3                                                                     & 2.7 $\sim$ 3.6                                                                                                   \\ \hline
\textbf{Imagen}                                                                  & \includeintable{.1}{ImagenesFactibilidad/SDCG3}                         & \includeintable{.1}{ImagenesFactibilidad/SDCE}                          & \includeintable{.1}{ImagenesFactibilidad/SDSDQAF3}                                                               \\ \hline
\end{tabular}
\caption{Comparación entre memorias SD.}
\end{table}