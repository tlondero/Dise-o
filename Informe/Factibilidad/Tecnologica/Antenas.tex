

\Subsubsubsection{Planteamiento del Problema}

La morfología del nido plantea la necesidad de recargar la UBM a distancias de entre 32 y 64 cm. Esto es dictado por las mínimas y máximas dimensiones medidas del nido entre la bóveda, donde se puede colocar electrónica, y el fondo, donde duerme el ave \cite{ref:PaperValeriaOjeda}.

En el mejor de los casos, el pájaro carpintero permanece por ocho horas en el nido mientras duerme, y luego cuida a las crías turnándose con la hembra durante el día. Por lo tanto, este permanecería un total de aproximadamente doce horas en el nido.

En el peor de los casos, el ave únicamente permanece seis horas dentro del nido para dormir.

Como la UBM tiene un consumo diario de

\begin{equation}
2.5 \ V \cdot 1 \ mA \cdot 24 \ hs = 60 \ mWh
\end{equation}

por lo que, tomando ambos casos, se obtiene un requerimiento de transmisión de potencia de

\begin{equation}
	\begin{cases} 
       \frac{60 \ mWh}{12 \ hs} = 5 \ mW & \text{Mejor caso} \\
       \frac{60 \ mWh}{6 \ hs} = 10 \ mW & \text{Peor caso}    
    \end{cases}
\end{equation}

\Subsubsubsection{Carga por Acoplamiento Magnético}

La transmisión inalámbrica de potencia por acoplamiento magnético se basa en generar un campo magnético al hacer circular corriente por un arreglo de bobinas gracias a la Ley de Ampere.

\begin{equation}
\nabla \times \mathbf{B} = \mu_0\left( \mathbf{J} + \epsilon_0 \frac{\partial \mathbf{B}}{\partial t} \right)
\end{equation}

Este campo magnético es captado por una o más bobinas receptoras las cuales generan a causa de este una fuerza electromotriz según la Ley de Faraday. 

\begin{equation}
\nabla \times \mathbf{E} = -\frac{\partial \mathbf{B}}{\partial t}
\end{equation}

De esta manera, se genera un sistema que actúa como transformador, utilizando como reluctancia al aire que separa ambas bobinas, con un factor de acople $k$ definido como

\begin{equation}
k = \frac{M}{\sqrt{L_1L_2}}
\end{equation}

donde M es el coeficiente de mutua-inductancia entre las bobinas. 

La eficiencia en la transmisión de energía de este método es alta, pero depende en gran medida por el factor de acople entre ambas bobinas \cite{ref:wpteff}. Este factor de acople disminuye considerablemente con la distancia y la desalineación entre bobinas \cite{ref:wirelesschargingkeyelements} \cite{ref:couplingfactor_1}.

Como las distancias a las que la transmisión se debe efectuar son de entre 32 y 64 cm y no se puede garantizar alineación entre bobinas al estar sujeto al comportamiento impredecible del ave, se descarta este método como solución.

\Subsubsubsection{Carga por Radiofrecuencia}

Si se parte de la ecuación del campo magnético en el eje azimutal de un dipolo de hertz, se tiene que
\begin{equation}
E_\theta = -\frac{\eta}{4\pi}I \cdot \Delta L \cdot k^2 \cdot sin\theta \cdot e^{-jkr} \left[ \frac{1}{jkr}+\left( \frac{1}{jkr}\right)^2 + \left(\frac{1}{jkr}\right)^3 \right]
\end{equation}
donde los últimos tres términos se denominan, en orden de aparición, término de campo lejano, campo cercano radiativo, y campo cercano reactivo. 

Las fronteras entre estos campos no están estrictamente fijadas, ya que varían con el tipo y tamaño de antena. Para el caso de antenas eléctricamente cortas, es decir, más cortas que media longitud de onda, se adopta el siguiente criterio

\begin{equation}
\begin{cases} 
          0 < d <\frac{\lambda}{2\pi} & \text{Campo cercano reactivo o inductivo} \\
          \frac{\lambda}{2\pi} < d \overset{\approx}{<} \lambda & \text{Campo cercano radiativo o de Fresnel} \\
          \lambda \overset{\approx}{<} d \overset{\approx}{<} 2\lambda & \text{Zona de transición} \\
          2\lambda \overset{\approx}{<} d < \infty  & \text{Campo lejano o de Fraunhofer} 
       \end{cases}
\end{equation}

La zona de campo cercano puede dividirse entre la zona reactiva-inductiva y la zona radiativa o de Fresnel. 

En la zona reactiva la relación entre los campos eléctricos y magnéticos no es predecible. Además, como no solo hay ondas electromagnéticas siendo irradiadas en esta zona, sino que también hay una cierta cantidad de energía siendo almacenada en la cercanía de la antena, la verdadera densidad de potencia se torna difícil de encontrar.

En el caso de la zona radiativa o de Fresnel, toda la energía es radiada. Sin embargo, la relación entre el campo eléctrico y magnético sigue siendo impredecible.

A una distancia entre una y dos longitudes de onda, los efectos de campo cercano comienzan a cesar, mientras que los efectos de campo lejano comienzan a aparecer. Es en esta zona por lo tanto, que ambos efectos están presentes y tienen importancia \cite{ref:NearFieldVsFarField}. Los dispositivos RFID suelen operar en esta zona \cite{ref:NearFieldUHFRFID}.

Por otro lado, el campo lejano es el utilizado para realizar todo tipo de telecomunicaciones hoy en día. En esta zona, el campo eléctrico y campo magnético son ortogonales y la razón entre ambos es la impedancia del medio. Además, el vector de Poynting, definido como $\vec{S} = \vec{E}\times \vec{H}$, provee una medida de la energía electromagnética radiada\cite{ref:PhysicsOscillations}.

Para analizar la potencia recibida en la antena receptora, la cual estará montada en la mochila, se realiza el balance de potencias del circuito electromagnético, por lo que partiendo de la ecuación de transmisión de Friis, se tiene que

\begin{equation}
\frac{P_r}{P_t} = \left( \frac{A_rA_r}{d^2\lambda ^2} \right)
\end{equation}
reescribiendo esta fórmula para utilizar las ganancias de las antenas en vez de las áreas efectivas, e incluyendo otras pérdidas del circuito electromagnético, se arriba a

\begin{equation}
P_r[dBm] = P_t[dBm] + Gt[dB] + G_r[dB] - L_{bf}[dB] - L_{cab}[dB] - L_{roe}[dB] - L_{r}[dB]
\end{equation}
donde $P_r$ es la potencia recibida en la antena receptora, $P_t$ la potencia emitida por la antena transmisora, $G_t$ la ganancia de la antena transmisora, $G_r$ la ganancia de la antena receptora, $L_{bf}$ las pérdidas por espacio libre, $L_{cab}$ las pérdidas en los cables de ambas antenas, $L_{roe}$ las pérdidas por retorno en ambas antenas, y $L_{r}$ las pérdidas por desacople entre las líneas de transmisión y las antenas.

Para el caso de la pérdida por espacio libre, esta se puede calcular como

\begin{equation}
L_{bf} = 32.5dB + 20log_{10}f[MHz]+20log_{10}R[km]
\end{equation}
mientras que el resto de los datos se puede obtener por medio de las hojas de datos o ensayos de las antenas, exceptuando la potencia a ser calculada y las pérdidas por desacople, las cuales dependen constructivamente del diseño de la líneas de transmisión que conectan con las antenas.

\Subsubsubsection{Banda de Frecuencia Adoptada}

Se puede observar que las pérdidas de espacio libre aumentan mediante crece la frecuencia de la onda electromagnética emitida. Esto plantea una situación de compromiso. Si la frecuencia es muy alta, las pérdidas por espacio libre serán muy grandes. Mientras que si la frecuencia es muy baja, la longitud de onda será muy grande, por lo que se estaría trabajando en el campo reactivo. Esto no es deseado debido a la imposibilidad de determinar con precisión la densidad de potencia.

Se decidió utilizar la banda de 915 MHz por las siguientes razones:
\begin{itemize}
\item La zona de transición ocurre entre 32.8 y 65.6 cm que concuerda con las distancias mínimas y máximas entre emisor y receptor. Si bien no estaremos trabajando en campo lejano, esto no acarrea problemas, ya que hay basta cantidad de antecedentes del uso de los efectos de campo cercano en dispositivos comerciales \cite{ref:Humavox} \cite{ref:NearFieldUHFRFID}.
\item Esta frecuencia pertenece a la banda ISM, la cual está reservada para propósitos industriales, científicos o médicos, excluyendo las aplicaciones de telecomunicaciones \cite{ref:ITUISM}, atribuidas a la Región 2 definida por la ITU como América \cite{ref:ITUREGION}.
\end{itemize}
 
\Subsubsubsection{Condiciones de Borde}

Para realizar las comparaciones entre antenas transmisoras, se tuvieron en cuenta los siguientes criterios:
\begin{itemize}
\item \textbf{Dimensiones:} La antena transmisora deberá ser colocada en la bóveda del nido, dado que ese es el único lugar donde se puede colocar electrónica sin que esta sea perturbada por las aves y vice versa. El volumen de la bóveda se puede aproximar a el de un cilindro macizo chato de diámetro entre 7.9 y 9.7 cm y aproximadamente 5 cm de altura, por lo que las dimensiones de la antena emisora estarán acotadas por estos valores.
\item \textbf{Directividad:} Se quiere que la potencia enviada a la antena se transforme en radiación electromagnética que llegue a la mochila del ave, por lo que radiación que no sea dirigida directamente hacia el fondo del nido será potencia desperdiciada. Es por esto que se quiere una alta directividad en la antena emisora. También hay un límite máximo en la directividad de la antena. Sin embargo, esta limitación no se alcanzará, dado que la tecnología a utilizar será de antenas del tipo planas, más cortas eléctricamente que media longitud de onda por cuestiones de limitaciones en las dimensiones. 
\item \textbf{Potencia Máxima:} Como las pérdidas en el circuito electromagnético son grandes, una muy baja parte de la potencia enviada a la antena transmisora formará parte de la potencia entregada a las baterías de la mochila, por lo que para recibir la potencia necesaria, se debe transmitir en el orden de los watts. Es por esto que la potencia máxima es una especificación relevante al momento de decidir entre soluciones.
\end{itemize}

Mientras que para el caso de las antenas receptoras, se tuvieron en cuenta los siguientes criterios:
\begin{itemize}
\item \textbf{Eficiencia:} Esta será la especificación más importante y es la que determinará la factibilidad de la solución. Se buscará la mayor eficiencia posible para lograr transmitir a las baterías la potencia recibida por el campo electromagnético.
\item \textbf{Lóbulo isotrópico:} Como se desconoce cuál será la posición del ave dentro del nido, se requiere que el lóbulo de radiación de la antena receptora sea lo más isotrópico posible, garantizando una recepción de potencia uniforme sin importar la posición del ave. Se buscará una ganancia menor a 2 dBi.
\item \textbf{Peso:} El ave no puede cargar con más de un cierto porcentaje de su propio peso, por lo que minimizar esta especificación es crucial.
\item \textbf{Dimensiones:} Es necesario no perturbar al ave con la mochila. Esto requiere que la antena receptora posea las mínimas dimensiones posibles. Sin embargo, como la banda a utilizar será la de 915 MHz y un cuarto de onda en esta frecuencia es alrededor de 8 cm, existe una relación de compromiso entre las dimensiones de la antena receptora y la eficiencia de esta.
\end{itemize}

Finalmente, una restricción a tener en cuenta para ambas antenas será el costo.

\Subsubsubsection{Comparación entre Antenas}
\begin{table}[H]
\centering
\begin{tabular}{|c|c|c|c|}
\hline
\textbf{\begin{tabular}[c]{@{}c@{}}Aspectos\\ comparativos\end{tabular}}    & \textbf{\href{https://abracon.com/patchantenna/APAE915R2540ABDB1-T.pdf}{APAE915R2540ABDB1-T}} & \textbf{\href{https://www.digikey.com/en/products/detail/pulselarsen-antennas/W3215/9838686}{W3215}} & \textbf{\href{https://cdn.taoglas.com/datasheets/ISPC.91A.09.0092E.pdf}{ISPC.91A.09.0092E}} \\ \hline
\textbf{Costo [USD]}                                                        & 3.66                                                                                 & 12.47                                                                                       & 20.91                                                                              \\ \hline
\textbf{Dimensiones [mm]}                                                   & 25 x 25 x 4                                                                          & 40 x 40 x 6                                                                                 & 47 x 47 x 6.5                                                                      \\ \hline
\textbf{\begin{tabular}[c]{@{}c@{}}Frecuencia\\ Central [MHz]\end{tabular}} & 915                                                                                  & 915                                                                                         & 915                                                                                \\ \hline
\textbf{Impedancia [$\Omega$]}                                              & 50                                                                                   & 50                                                                                          & 50                                                                                 \\ \hline
\textbf{Polarización}                                                       & RHCP                                                                                 & Lineal vertical                                                                             & RHCP                                                                               \\ \hline
\textbf{Ganancia [dBi]}                                                     & 1.5                                                                                  & 4.5                                                                                         & 5 (30 x 30 ground plane)                                                           \\ \hline
\textbf{ROE}                                                                & 1.5                                                                                  & 1.23                                                                                        & 1.28                                                                               \\ \hline
\textbf{Imagen}                                                             & \includeintable{.1}{ImagenesFactibilidad/ANT1}                                       & \includeintable{.1}{ImagenesFactibilidad/ANT2}                                              & \includeintable{.1}{ImagenesFactibilidad/ANT3}                                     \\ \hline
\textbf{Lóbulo}                                                             & \includeintable{.1}{ImagenesFactibilidad/LOB1}                                       & \includeintable{.1}{ImagenesFactibilidad/LOB2}                                              & \includeintable{.1}{ImagenesFactibilidad/LOB3}                                     \\ \hline
\end{tabular}
\caption{Comparación entre antenas transmisoras (Parte 1).}
\end{table}

\begin{table}[H]
\centering
\begin{tabular}{|c|c|c|}
\hline
\textbf{\begin{tabular}[c]{@{}c@{}}Aspectos\\ comparativos\end{tabular}}    & \textbf{\href{https://abracon.com/patchantenna/APAES915R80C16-T.pdf}{APAES915R80C16-T}} & \textbf{\href{https://abracon.com/datasheets/ARRKP7059-S915B.pdf}{ARRKP7059-S915B}} \\ \hline
\textbf{Costo [USD]}                                                        & 34.28                                                                          & 50.73                                                                      \\ \hline
\textbf{Dimensiones [mm]}                                                   & 80 x 80 x 6                                                                    & 70 x 70 x 5.9                                                              \\ \hline
\textbf{\begin{tabular}[c]{@{}c@{}}Frecuencia\\ Central [MHz]\end{tabular}} & 915                                                                            & 915                                                                        \\ \hline
\textbf{Impedancia [$\Omega$]}   											& 50                                                                             & 50                                                                         \\ \hline
\textbf{Polarización}                                                       & RHCP                                                                           & RHCP                                                                       \\ \hline
\textbf{Ganancia [dBi]}                                                     & 2 (120 x 120 ground plane)                                                     & 2.8 (70 x 70 ground plane)                                                 \\ \hline
\textbf{ROE}                                                                & 1.3                                                                            & $\leq$ 2                                                                   \\ \hline
\textbf{Imagen}                                                             & \includeintable{.1}{ImagenesFactibilidad/ANT4}                                 & \includeintable{.1}{ImagenesFactibilidad/ANT5}                             \\ \hline
\textbf{Lóbulo}                                                             & \includeintable{.1}{ImagenesFactibilidad/LOB4}                                 & \includeintable{.1}{ImagenesFactibilidad/LOB5}                             \\ \hline
\end{tabular}
\caption{Comparación entre antenas transmisoras (Parte 2).}
\end{table}

\begin{table}[H]
\centering
\begin{tabular}{|c|c|c|c}
\hline
\textbf{\begin{tabular}[c]{@{}c@{}}Aspectos\\ comparativos\end{tabular}}    & \textbf{\href{https://linxtechnologies.com/wp/wp-content/uploads/ant-915-cpa-ds.pdf}{ANT-915-CPA}} & \textbf{\href{https://cdn.taoglas.com/datasheets/FXP290.07.0100A.pdf}{FXP290.07.0100A}} & \multicolumn{1}{c|}{\textbf{\href{https://media.digikey.com/pdf/Data\%20Sheets/Ignion\%20PDFs/NN01-105_Jan2021.pdf}{NN01-105}}} 	\\ \hline
\textbf{Costo [USD]}                                                       	& 3.9                                                                                                & 15.39                                                                                   & \multicolumn{1}{c|}{3.53}                                      		                                                            \\ \hline
\textbf{Dimensiones [mm]}                                                   & 25 x 25 x 4                                                                                        & 70 x 45 x 0.1                                                                           & \multicolumn{1}{c|}{18 x 7.3 x 0.8}                                                                                             	\\ \hline
\textbf{\begin{tabular}[c]{@{}c@{}}Frecuencia\\ Central [MHz]\end{tabular}} & 915                                                                                                & 915                                                                                     & \multicolumn{1}{c|}{915}                                                                                                      		\\ \hline
\textbf{Impedancia [$\Omega$]}                                              & 50                                                                                                 & 50                                                                                      & \multicolumn{1}{c|}{50} 		                                                                                                    \\ \hline
\textbf{Polarización}                                                       & RHCP                                                                                               & Lineal                                                                                  & \multicolumn{1}{c|}{Lineal}    		                                                                                            \\ \hline
\textbf{Ganancia [dBi]}                                                     & 1.5                                                                                                & 0.5                                                                                     & \multicolumn{1}{c|}{1.7}               		                                                                                    \\ \hline
\textbf{Eficiencia [\%]}                                                    & 38                                                                                                 & 43                                                                                      & \multicolumn{1}{c|}{85}                        	  	                                                                            \\ \hline
\textbf{ROE}                                                                & $\leq$ 1.2                                                                                           & 1.5                                                                                   & \multicolumn{1}{c|}{1.4}                               		                                                                    \\ \hline
\textbf{Peso [gr]}                                                          & 13.2                                                                                               & 1.5                                                                                     & \multicolumn{1}{c|}{0.2}                                                  		                                                    \\ \hline
\textbf{Imagen}                                                             & \includeintable{.1}{ImagenesFactibilidad/ANTR1}                                                    & \includeintable{.1}{ImagenesFactibilidad/ANTR2}                                         & \multicolumn{1}{c|}{\includeintable{.1}{ImagenesFactibilidad/ANTR3}}           		                                            \\ \hline
\textbf{Lóbulo}                                                             & \includeintable{.1}{ImagenesFactibilidad/LOBR1}                                                    & \includeintable{.1}{ImagenesFactibilidad/LOBR2}                                         & \multicolumn{1}{c|}{\quotes{Omnidireccional}}                                                 		                                    \\ \hline
\end{tabular}
\caption{Comparación entre antenas receptoras (Parte 1).}
\end{table}

\begin{table}[H]
\centering
\begin{tabular}{|c|c|c|c}
\hline
\textbf{\begin{tabular}[c]{@{}c@{}}Aspectos\\ comparativos\end{tabular}}    & \textbf{\href{https://www.yageo.com/upload/media/product/productsearch/datasheet/wireless/An_SMD_FR4_915M_1204_05.pdf}{ANT1204F005R0915A}} & \textbf{\href{https://www.te.com/commerce/DocumentDelivery/DDEController?Action=srchrtrv\&DocNm=1513156\&DocType=DS\&DocLang=English}{1513156-1}} 	& \multicolumn{1}{c|}{\textbf{\href{https://linxtechnologies.com/wp/wp-content/uploads/ant-915-usp410-ds.pdf}{ANT-915-USP410}}} \\ \hline
\textbf{Costo [USD]}                                                       	& 1.52                                                                                                                                       & 2.8                                                                                                                                            		& \multicolumn{1}{c|}{1.45}                                                                                                     \\ \hline
\textbf{Dimensiones [mm]}                                                   & 12.20 x 4 x 1.6                                                                                                                            & 38.1 x 6.6 x 1.57                                                                                                                              		& \multicolumn{1}{c|}{13.2 x 9.1 x 2.9}                                                                                         \\ \hline
\textbf{\begin{tabular}[c]{@{}c@{}}Frecuencia\\ Central [MHz]\end{tabular}} & 915                                                                                                                                        & 915                                                                                                                                            		& \multicolumn{1}{c|}{915}                                                                                                      \\ \hline
\textbf{Impedancia [$\Omega$]}                                              & 50                                                                                                                                         & 50                                                                                                                                            		& \multicolumn{1}{c|}{50}                                                                                                       \\ \hline
\textbf{Polarización}                                                       & Lineal                                                                                                                                     & Lineal                                                                                                                                         		& \multicolumn{1}{c|}{Lineal}                                                                                                   \\ \hline
\textbf{Ganancia [dBi]}                                                     & 1.59                                                                                                                                       & 1                                                                                                                                              		& \multicolumn{1}{c|}{0}                                                                                                        \\ \hline
\textbf{Eficiencia [\%]}                                                    & -                                                                                                                                          & 88                                                                                                                                             		& \multicolumn{1}{c|}{27}                                                                                                       \\ \hline
\textbf{ROE}                                                                & -                                                                                                                                          & 1.85                                                                                                                                           		& \multicolumn{1}{c|}{1.5}                                                                                                      \\ \hline
\textbf{Peso [gr]}                                                          & -                                                                                                                                          & < 0.9                                                                                                                                          		& \multicolumn{1}{c|}{0.6}                                                                                                      \\ \hline
\textbf{Imagen}                                                             & \includeintable{.1}{ImagenesFactibilidad/ANTR4}                                                                                            & \includeintable{.1}{ImagenesFactibilidad/ANTR5}                                                                                                		& \multicolumn{1}{c|}{\includeintable{.1}{ImagenesFactibilidad/ANTR6}}                                                          \\ \hline
\textbf{Lóbulo}                                                             & \includeintable{.1}{ImagenesFactibilidad/LOBR4}                                                                                            & \includeintable{.1}{ImagenesFactibilidad/LOBR5}                                                                                                		& \multicolumn{1}{c|}{\includeintable{.1}{ImagenesFactibilidad/LOBR6}}                                                          \\ \hline
\end{tabular}
\caption{Comparación entre antenas receptoras (Parte 2).}
\end{table}
