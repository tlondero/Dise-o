%\documentclass[a4paper]{article}
\usepackage[utf8]{inputenc}
\usepackage[spanish, es-tabla, es-noshorthands]{babel}
\usepackage[table,xcdraw]{xcolor}
\usepackage[a4paper, footnotesep=1.25cm, headheight=1.25cm, top=2.54cm, left=2.54cm, bottom=2.54cm, right=2.54cm]{geometry}
%\geometry{showframe}
%VERIFICAR EL HEAD Y EL FOOT EN
%https://ctan.dcc.uchile.cl/macros/latex/contrib/geometry/geometry.pdf

%\usepackage{lipsum}			%LOREM IPSUM

%\usepackage{wrapfig}		%Wrap figure in text
\usepackage[export]{adjustbox}	%Move images
\usepackage{changepage}			%Move tables

%Font
\usepackage{anyfontsize}	%Font size
% #1 = size, #2 = text
\newcommand{\setparagraphsize}[2]{{\fontsize{#1}{6}\selectfont#2 \par}}		%Cambia el size de todo el parrafo
\newcommand{\setlinesize}[2]{{\fontsize{#1}{6}\selectfont#2}}				%Cambia el font de una oración


%FONTS (IMPORTANTE): Compilar en XeLaTex o LuaLaTeX
\usepackage{fontspec}
%Si sigue sin andar comentar \usepackage[utf8]{inputenc}
%https://ctan.dcc.uchile.cl/macros/unicodetex/latex/fontspec/fontspec.pdf
%https://www.overleaf.com/learn/latex/XeLaTeX

\usepackage{tikz}
\usepackage{amsmath}
\usepackage{amsfonts}
\usepackage{amssymb}
\usepackage{float}
\usepackage{graphicx}
\usepackage{caption}
\usepackage{subcaption}
\usepackage{multicol}
\usepackage{multirow}
\setlength{\doublerulesep}{\arrayrulewidth}
\usepackage{booktabs}

\usepackage{hyperref}
\hypersetup{
    colorlinks=true,
    linkcolor=black,
    filecolor=magenta,      
    urlcolor=blue,
    citecolor=blue,    
}

\newcommand{\captionsection}{\setcounter{figure}{0} \renewcommand{\thetable}{\arabic{section}.\arabic{table}} \renewcommand{\thefigure}{\arabic{section}.\arabic{figure}}}

\newcommand{\captionsubsection}{\setcounter{figure}{0} \renewcommand{\thetable}{\arabic{section}.\arabic{subsection}.\arabic{table}} \renewcommand{\thefigure}{\arabic{section}.\arabic{subsection}.\arabic{figure}}}

\newcommand{\captionsubsubsection}{\setcounter{figure}{0} \renewcommand{\thetable}{\arabic{section}.\arabic{subsection}.\arabic{subsubsection}.\arabic{table}} \renewcommand{\thefigure}{\arabic{section}.\arabic{subsection}.\arabic{subsubsection}.\arabic{figure}}}

%PICTURES AND TABLE INDEX:
\newcommand{\Section}[1]{ \section{#1}\captionsection }
\newcommand{\Subsection}[1]{ \subsection{#1}\captionsubsection }
\newcommand{\Subsubsection}[1]{ \subsubsection{#1}\captionsubsubsection }

%NOTAS GRANDES
\newcommand{\note}[1]{
	\begin{center}
		\huge{ \textcolor{red}{#1} }
	\end{center}
}
%Notas pequeñas
\newcommand{\lnote}[1]{{\fontsize{20}{6}\selectfont\textcolor{green}{#1}}}

\newcommand{\quotes}[1]{``#1''}
\usepackage{array}
\newcolumntype{C}[1]{>{\centering\let\newline\\\arraybackslash\hspace{0pt}}m{#1}}
\usepackage[american]{circuitikz}
\usetikzlibrary{calc}
\usepackage{fancyhdr}
\usepackage{units} 

\graphicspath{{../Prefacio/}{../Acronimos/}{../Introduccion/}{../Objetivos/}{../Definicion/}{../Plan de validacion/}{../Resumen/}}

%COLORES
\definecolor{AzulFoot}{rgb}{0.682,0.809,0.926}	%RGB	%{174,206,235}
\definecolor{AzulInfo}{rgb}{0.180,0.455,0.710}	%RGB	%{46,116,181}
\definecolor{AzulTable}{rgb}{0.302,0.507,0.871}	%RGB	%{68,114,196}
\definecolor{PName}{rgb}{0.353,0.353,0.353}	%RGB	%{90,90,90}

\usepackage{xcolor}
\usepackage{sectsty}
\chapterfont{\color{AzulInfo}}  % sets colour of chapters
\sectionfont{\color{AzulInfo}}  % sets colour of sections
\subsectionfont{\color{AzulInfo}}
\subsubsectionfont{\color{AzulInfo}}

%Header y footer
\usepackage{etoolbox}

\pagestyle{fancy}
\fancyhf{}
\rfoot{\thepage}
\renewcommand{\footrulewidth}{4pt}
\renewcommand{\headrulewidth}{0pt}
\patchcmd{\footrule}{\hrule}{\color{AzulFoot}\hrule}{}{}

%\begin{document}

%\lnote{Cosas que faltan:}
%\begin{itemize}
%	\item \lnote{Hablar del panel conectado a la batería y que esto alimenta a todo.}
%	\item \lnote{Hablar del cargador para la mochila.}
%\end{itemize}

\Subsection{Antecedentes}

Cuando se estudian aves, por lo general, los investigadores optan por colocar pequeños dispositivos transmisores sobre el cuerpo de las mismas. Sin embargo, las soluciones disponibles en el mercado tienen restricciones de energía y peso lo cual resultan incompatibles con las expectativas del grupo INBIOMA.

Actualmente las unidades de recolección de información toman datos sobre la posición, temperatura, el estado vital del espécimen, entre otras. Estos dispositivos comerciales requieren de una antena para la transmisión de datos mediante redes celulares, las cuales no siempre están presentes en las zonas de interés y ademas generan costos de comunicación. Las antenas que se emplean, cuyo largo es comparable con el largo del ave, no presentan dificultad alguna para aves que duermen y anidan en dormideros o nidos abiertos (al aire libre). Por el contrario, para el caso de las aves que viven en el interior de los árboles, tal como los pájaros carpinteros, el uso de dichos dispositivos es un inconveniente. Estas pueden poner en peligro a las demás aves que habitan dentro del nido y dificultarle la movilidad, haciéndolas más vulnerables ante depredadores.

Por otro lado, los productos existentes que están pensados para especies de menor tamaño, no contemplan la naturaleza territorial y violenta del Campephilus Magellanicus, nombre científico del pájaro carpintero gigante.

También existen productos para aves de mayor tamaño, el problema en estos radica en la incapacidad del sujeto de estudio de transportar el peso de la electrónica asociada a estos productos.

Por último consideramos las opciones que se pueden conseguir en el mercado no profesional, destinadas para el uso hogareño: pequeños nidos de fácil instalación que poseen sensores variados. Nuevamente, ese tipo de productos no contemplan el comportamiento del ave en cuestión, ya que dicha especie fabrica su propio nido en lugar de tomar alguno ya construido, descartando estos productos como solución. Estos refugios tampoco están equipados con todos sensores que permitan medir todos los factores de interés. Si bien la mayoría cuenta con cámaras internas o sensores de temperatura, no todos poseen medidores de humedad o de luminosidad del entorno. 

\observacion{\verObs}{Más allá de la redacción, no quedan claras las ideas ni para qué están en este informe. Hay un uso \quotes{hogareño} de sensores para aves? Qué sensores? Qué deberían tener par \quotes{contemplar el comportamiento del ave en cuestión}? Cuáles son los factores de interés?}

\Subsection{Contexto del proyecto}

\observacion{\verObs}{Se tiene que dejar bien en claro que hay un panel solar y un cargador o algo así.}

El CIDEI (Centro de Investigación y Desarrollo en Electrónica Industrial del ITBA) está trabajando junto al INBIOMA (Investigaciones en Biodiversidad y Medioambiente radicado en la Universidad Nacional del Comahue) para participar de un estudio en conjunto que busca aprender de algunos aspectos de la vida del Carpintero Gigante, ave que sirve de vector de referencia para analizar el estado de otros elementos de la vida silvestre en el área \cite{ref:PaperValeriaOjeda}.

El CIDEI-ITBA tiene la tarea de desarrollar la tecnología para la obtención de las variables físicas, tanto del vuelo y comportamiento de las aves, como de su entorno (nido y alrededores). El estudio de los patrones de alimentación y movimiento del ave en cuestión pueden alertar sobre diversos factores que están cambiando en el ambiente. En la actualidad no existe en el mercado un dispositivo que permita cumplir con los requerimientos para el relevo de los datos necesarios para la investigación, por lo que se trabajará junto al grupo de biólogas en su desarrollo.

Las unidades de adquisición de datos móviles que se encuentran actualmente en el mercado no son compatibles con lo que se requiere para estudiar al pájaro carpintero, ya que o son dispositivos que van montados sobre el ave, o son equipos de tipo hobbista. En otras palabras, no se encuentra disponible una solución integral que permita obtener mediciones y extraer contenido visual dentro y fuera de los nidos.

\observacion{\verObs}{Redactar mejor. Tiran varias ideas una detrás de otra sin tener un orden o secuencia lógica.} 

%\end{document}