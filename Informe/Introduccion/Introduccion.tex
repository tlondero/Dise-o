%\documentclass[a4paper]{article}
\usepackage[utf8]{inputenc}
\usepackage[spanish, es-tabla, es-noshorthands]{babel}
\usepackage[table,xcdraw]{xcolor}
\usepackage[a4paper, footnotesep=1.25cm, headheight=1.25cm, top=2.54cm, left=2.54cm, bottom=2.54cm, right=2.54cm]{geometry}
%\geometry{showframe}
%VERIFICAR EL HEAD Y EL FOOT EN
%https://ctan.dcc.uchile.cl/macros/latex/contrib/geometry/geometry.pdf

%\usepackage{lipsum}			%LOREM IPSUM

%\usepackage{wrapfig}		%Wrap figure in text
\usepackage[export]{adjustbox}	%Move images
\usepackage{changepage}			%Move tables

%Font
\usepackage{anyfontsize}	%Font size
% #1 = size, #2 = text
\newcommand{\setparagraphsize}[2]{{\fontsize{#1}{6}\selectfont#2 \par}}		%Cambia el size de todo el parrafo
\newcommand{\setlinesize}[2]{{\fontsize{#1}{6}\selectfont#2}}				%Cambia el font de una oración


%FONTS (IMPORTANTE): Compilar en XeLaTex o LuaLaTeX
\usepackage{fontspec}
%Si sigue sin andar comentar \usepackage[utf8]{inputenc}
%https://ctan.dcc.uchile.cl/macros/unicodetex/latex/fontspec/fontspec.pdf
%https://www.overleaf.com/learn/latex/XeLaTeX

\usepackage{tikz}
\usepackage{amsmath}
\usepackage{amsfonts}
\usepackage{amssymb}
\usepackage{float}
\usepackage{graphicx}
\usepackage{caption}
\usepackage{subcaption}
\usepackage{multicol}
\usepackage{multirow}
\setlength{\doublerulesep}{\arrayrulewidth}
\usepackage{booktabs}

\usepackage{hyperref}
\hypersetup{
    colorlinks=true,
    linkcolor=black,
    filecolor=magenta,      
    urlcolor=blue,
    citecolor=blue,    
}

\newcommand{\captionsection}{\setcounter{figure}{0} \renewcommand{\thetable}{\arabic{section}.\arabic{table}} \renewcommand{\thefigure}{\arabic{section}.\arabic{figure}}}

\newcommand{\captionsubsection}{\setcounter{figure}{0} \renewcommand{\thetable}{\arabic{section}.\arabic{subsection}.\arabic{table}} \renewcommand{\thefigure}{\arabic{section}.\arabic{subsection}.\arabic{figure}}}

\newcommand{\captionsubsubsection}{\setcounter{figure}{0} \renewcommand{\thetable}{\arabic{section}.\arabic{subsection}.\arabic{subsubsection}.\arabic{table}} \renewcommand{\thefigure}{\arabic{section}.\arabic{subsection}.\arabic{subsubsection}.\arabic{figure}}}

%PICTURES AND TABLE INDEX:
\newcommand{\Section}[1]{ \section{#1}\captionsection }
\newcommand{\Subsection}[1]{ \subsection{#1}\captionsubsection }
\newcommand{\Subsubsection}[1]{ \subsubsection{#1}\captionsubsubsection }

%NOTAS GRANDES
\newcommand{\note}[1]{
	\begin{center}
		\huge{ \textcolor{red}{#1} }
	\end{center}
}
%Notas pequeñas
\newcommand{\lnote}[1]{{\fontsize{20}{6}\selectfont\textcolor{green}{#1}}}

\newcommand{\quotes}[1]{``#1''}
\usepackage{array}
\newcolumntype{C}[1]{>{\centering\let\newline\\\arraybackslash\hspace{0pt}}m{#1}}
\usepackage[american]{circuitikz}
\usetikzlibrary{calc}
\usepackage{fancyhdr}
\usepackage{units} 

\graphicspath{{../Prefacio/}{../Acronimos/}{../Introduccion/}{../Objetivos/}{../Definicion/}{../Plan de validacion/}{../Resumen/}}

%COLORES
\definecolor{AzulFoot}{rgb}{0.682,0.809,0.926}	%RGB	%{174,206,235}
\definecolor{AzulInfo}{rgb}{0.180,0.455,0.710}	%RGB	%{46,116,181}
\definecolor{AzulTable}{rgb}{0.302,0.507,0.871}	%RGB	%{68,114,196}
\definecolor{PName}{rgb}{0.353,0.353,0.353}	%RGB	%{90,90,90}

\usepackage{xcolor}
\usepackage{sectsty}
\chapterfont{\color{AzulInfo}}  % sets colour of chapters
\sectionfont{\color{AzulInfo}}  % sets colour of sections
\subsectionfont{\color{AzulInfo}}
\subsubsectionfont{\color{AzulInfo}}

%Header y footer
\usepackage{etoolbox}

\pagestyle{fancy}
\fancyhf{}
\rfoot{\thepage}
\renewcommand{\footrulewidth}{4pt}
\renewcommand{\headrulewidth}{0pt}
\patchcmd{\footrule}{\hrule}{\color{AzulFoot}\hrule}{}{}

%\begin{document}

\Subsection{Antecedentes}

El estudio del comportamiento de las aves es importante para entender cómo sus actividades impactan en su entorno. Pueden aportar información directa e indirecta sobre el mismo. Actualmente se desconoce la frecuencia con las que estas aves pasan dentro y fuera del nido. Esta métrica ayudaría a enriquecer la información actual sobre 

\Subsection{Contexto del Proyecto}

\begin{table}[H]
\centering
\begin{tabular}{|l|l|l|l|}
\hline
\textbf{ID} & Descripción & Origen & \begin{tabular}[c]{@{}l@{}}Aplicabilidad\\ Validación\end{tabular} \\ \hline
IMP-OPE-01 & \begin{tabular}[c]{@{}l@{}}El dispositivo deberá poder operar normalmente cuando la\\ temperatura ambiente sea -5°C \textless TAMB \textless 30°C. Si bien la\\ temperatura más baja observada en el territorio argentino es\\ de -25°C, se acepta que son circunstancias excepcionales.\\ Si bien la temperatura máxima histórica de la zona observada \\ fue de 34.6°C la media es menor.\end{tabular} & REQ-05 & \begin{tabular}[c]{@{}l@{}}P F\\ ID\end{tabular} \\ \hline
IMP-OPE-02 & \begin{tabular}[c]{@{}l@{}}El dispositivo deberá poder operar normalmente cuando la \\ humedad sea: 0\% \textless RH \textless 100\%, valores normales de humedad\\ relativa ambiente. Si bien puede ser superior al 100\%, esto no \\ es común y sólo puede suceder en aire extremadamente limpio,\\ donde la condensación no se adhiere a partículas de polvo.\end{tabular} & REQ-05 & \begin{tabular}[c]{@{}l@{}}P F\\ ID\end{tabular} \\ \hline
IMP-OPE-03 & \begin{tabular}[c]{@{}l@{}}El dispositivo deberá poder operar normalmente cuando la \\ presión atmosférica sea: 84 kPa \textless PATM \textless 90 kPa. Esto \\ equivale a 1500m de altura para el mínimo de presión, \\ y un máximo de 1100m.\end{tabular} & REQ-05 & \begin{tabular}[c]{@{}l@{}}P F\\ ID\end{tabular} \\ \hline
IMP-OPE-03 & El dispositivo deberá tener un grado de protección IPXX (TBD) & REQ-05 & \begin{tabular}[c]{@{}l@{}}P(TBD) F\\ ID\end{tabular} \\ \hline
\end{tabular}
\end{table}

%\end{document}