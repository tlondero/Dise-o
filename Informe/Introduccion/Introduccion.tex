%\documentclass[a4paper]{article}
\usepackage[utf8]{inputenc}
\usepackage[spanish, es-tabla, es-noshorthands]{babel}
\usepackage[table,xcdraw]{xcolor}
\usepackage[a4paper, footnotesep=1.25cm, headheight=1.25cm, top=2.54cm, left=2.54cm, bottom=2.54cm, right=2.54cm]{geometry}
%\geometry{showframe}
%VERIFICAR EL HEAD Y EL FOOT EN
%https://ctan.dcc.uchile.cl/macros/latex/contrib/geometry/geometry.pdf

%\usepackage{lipsum}			%LOREM IPSUM

%\usepackage{wrapfig}		%Wrap figure in text
\usepackage[export]{adjustbox}	%Move images
\usepackage{changepage}			%Move tables

%Font
\usepackage{anyfontsize}	%Font size
% #1 = size, #2 = text
\newcommand{\setparagraphsize}[2]{{\fontsize{#1}{6}\selectfont#2 \par}}		%Cambia el size de todo el parrafo
\newcommand{\setlinesize}[2]{{\fontsize{#1}{6}\selectfont#2}}				%Cambia el font de una oración


%FONTS (IMPORTANTE): Compilar en XeLaTex o LuaLaTeX
\usepackage{fontspec}
%Si sigue sin andar comentar \usepackage[utf8]{inputenc}
%https://ctan.dcc.uchile.cl/macros/unicodetex/latex/fontspec/fontspec.pdf
%https://www.overleaf.com/learn/latex/XeLaTeX

\usepackage{tikz}
\usepackage{amsmath}
\usepackage{amsfonts}
\usepackage{amssymb}
\usepackage{float}
\usepackage{graphicx}
\usepackage{caption}
\usepackage{subcaption}
\usepackage{multicol}
\usepackage{multirow}
\setlength{\doublerulesep}{\arrayrulewidth}
\usepackage{booktabs}

\usepackage{hyperref}
\hypersetup{
    colorlinks=true,
    linkcolor=black,
    filecolor=magenta,      
    urlcolor=blue,
    citecolor=blue,    
}

\newcommand{\captionsection}{\setcounter{figure}{0} \renewcommand{\thetable}{\arabic{section}.\arabic{table}} \renewcommand{\thefigure}{\arabic{section}.\arabic{figure}}}

\newcommand{\captionsubsection}{\setcounter{figure}{0} \renewcommand{\thetable}{\arabic{section}.\arabic{subsection}.\arabic{table}} \renewcommand{\thefigure}{\arabic{section}.\arabic{subsection}.\arabic{figure}}}

\newcommand{\captionsubsubsection}{\setcounter{figure}{0} \renewcommand{\thetable}{\arabic{section}.\arabic{subsection}.\arabic{subsubsection}.\arabic{table}} \renewcommand{\thefigure}{\arabic{section}.\arabic{subsection}.\arabic{subsubsection}.\arabic{figure}}}

%PICTURES AND TABLE INDEX:
\newcommand{\Section}[1]{ \section{#1}\captionsection }
\newcommand{\Subsection}[1]{ \subsection{#1}\captionsubsection }
\newcommand{\Subsubsection}[1]{ \subsubsection{#1}\captionsubsubsection }

%NOTAS GRANDES
\newcommand{\note}[1]{
	\begin{center}
		\huge{ \textcolor{red}{#1} }
	\end{center}
}
%Notas pequeñas
\newcommand{\lnote}[1]{{\fontsize{20}{6}\selectfont\textcolor{green}{#1}}}

\newcommand{\quotes}[1]{``#1''}
\usepackage{array}
\newcolumntype{C}[1]{>{\centering\let\newline\\\arraybackslash\hspace{0pt}}m{#1}}
\usepackage[american]{circuitikz}
\usetikzlibrary{calc}
\usepackage{fancyhdr}
\usepackage{units} 

\graphicspath{{../Prefacio/}{../Acronimos/}{../Introduccion/}{../Objetivos/}{../Definicion/}{../Plan de validacion/}{../Resumen/}}

%COLORES
\definecolor{AzulFoot}{rgb}{0.682,0.809,0.926}	%RGB	%{174,206,235}
\definecolor{AzulInfo}{rgb}{0.180,0.455,0.710}	%RGB	%{46,116,181}
\definecolor{AzulTable}{rgb}{0.302,0.507,0.871}	%RGB	%{68,114,196}
\definecolor{PName}{rgb}{0.353,0.353,0.353}	%RGB	%{90,90,90}

\usepackage{xcolor}
\usepackage{sectsty}
\chapterfont{\color{AzulInfo}}  % sets colour of chapters
\sectionfont{\color{AzulInfo}}  % sets colour of sections
\subsectionfont{\color{AzulInfo}}
\subsubsectionfont{\color{AzulInfo}}

%Header y footer
\usepackage{etoolbox}

\pagestyle{fancy}
\fancyhf{}
\rfoot{\thepage}
\renewcommand{\footrulewidth}{4pt}
\renewcommand{\headrulewidth}{0pt}
\patchcmd{\footrule}{\hrule}{\color{AzulFoot}\hrule}{}{}

%\begin{document}

\Subsection{Antecedentes}

Cuando se trata de investigar aves, por lo general, los investigadores optan por colocar pequeños dispositivos transmisores sobre el cuerpo de las mismas. Sin embargo, aquellas soluciones disponibles en el mercado tienen restricciones de energía y peso lo cual resulta incompatible con las expectativas del grupo INBIOMA, más acerca de ellos en la siguiente sección.

Actualmente se utilizan unidades de adquisición de datos sobre las aves para recoger información sobre posición, temperatura, etc. Estos dispositivos requieren de una antena para la transmisión de datos mediante tecnología celular. La antena no presenta dificultad alguna para aves que duermen y anidan en el exterior. Por el contrario, para el caso de las aves que viven en el interior de los árboles, tal como los pájaros carpinteros, sí es un inconveniente. Además, este tipo de tecnología está diseñada para operar en zonas donde existe cobertura telefónica. Esta razón no solo es una limitante sino que también generan costos de comunicación. Por otro lado, los productos existentes que están pensados para especies de menor tamaño, no contemplan la naturaleza territorial y violenta del Campephilus Magellanicus.
Tambien existen productos para aves de mayor tamaño, el problema en estos radica en la incapacidad del sujeto de estudio de transportar el peso de la electronica asociada a estos productos.
Por último consideramos las opciones que se pueden conseguir en el mercado no profesional, destinadas para el uso hogareño: pequeños nidos de fácil instalación que poseen ciertos sensores. Nuevamente, ese tipo de productos no contemplan el comportamiento del ave en cuestión, ya que dicha especie fabrica su propio nido en lugar de tomar alguno ya construido. Estos refugios tampoco están equipados con sensores que permitan medir los factores de interés. 


\Subsection{Contexto del proyecto}
El CIDEI (Centro de Investigación y Desarrollo en Electrónica Industrial del ITBA) está trabajando junto al INBIOMA (Investigaciones en Biodiversidad y Medioambiente radicado en la Universidad Nacional del Comahue) para participar de un estudio en conjunto que busca aprender de algunos aspectos de la vida del Carpintero Gigante, ave que sirve de vector de referencia para analizar el estado de otros elementos de la vida silvestre en el área. El CIDEI-ITBA tiene la tarea de desarrollar la tecnología para la obtención de las variables físicas, tanto del vuelo y comportamiento de las aves, como de su entorno (nido y alrededores). Actualmente no se encuentra en el mercado un dispositivo que permita cumplir con los objetivos de investigación planteados, por lo que se trabajará junto al grupo de biólogas en su desarrollo.

El pájaro carpintero gigante, Campephilus Magellanicus, es un vector de referencia para analizar el estado de otros elementos de la vida silvestre en el bosque andino-patagónico \cite{ref:PaperValeriaOjeda}. El estudio de los patrones de alimentación y movimiento de este pueden alertar sobre diversos factores que están cambiando en el ambiente.

Hoy en día se necesita una solución que permita utilizar baterías más pequeñas montadas sobre las aves. Se debe reducir el consumo de energía y aumentar la capacidad de transmisión de información. Es necesario también poder operar en áreas donde las comunicaciones celulares no están disponibles.

En el mercado actual solo se comercializan unidades de adquisición de datos móviles que van montadas sobre las especies de estudio y equipos de tipo hobbista. Sin embargo es de interés poder obtener mediciones y extraer contenido visual dentro y fuera de los nidos, aun cuando estos se encuentren en alturas de difícil acceso para una persona. Actualmente no se encuentra disponible una solución integral que permita satisfacer esta necesidad de poder captar esta información y distribuirla hacia los investigadores.

%\end{document}
