\textbf{Banco de pruebas 1:}
\begin{itemize}
	\item El dispositivo contará con una manera de desacoplar la alimentación principal y permitir la alimentación de los módulos a través de una fuente regulable de \TBD V $\pm$ \TBD mV que pueda suministrar por lo menos \TBD mA.
	\item Se tendrá un software que permita activar  la comunicación COM2. Transmitir y recibir data conocida, tanto en un sentido como en el otro.
	\item \TBC
\end{itemize}

\textbf{Banco de pruebas 2:}
\begin{itemize}
	\item El dispositivo contará con una manera de desacoplar la alimentación principal y permitir la alimentación de los módulos a través de una fuente regulable de \TBD V $\pm$ \TBD mV que pueda suministrar por lo menos \TBD mA.
	\item Se acercará un dispositivo que emula la mochila para realizar el disparo. Se podrá transmitir y recibir data conocida, tanto en un sentido como en el otro \TBC.
	\item \TBC
\end{itemize}

\textbf{Banco de pruebas 3:}
\begin{itemize}
	\item El dispositivo contará con una manera de desacoplar la alimentación principal y permitir la alimentación de los módulos a través de una fuente regulable de \TBD V $\pm$ \TBD mV que pueda suministrar por lo menos \TBD mA.
	\item Se tendrá un osciloscopio para medir el nivel de carga de la batería al igual que medirla potencia suministrada, para obtener la eficiencia, al igual que cronometrar el tiempo de carga.
	\item \TBC
\end{itemize}

\textbf{Banco de pruebas 4:}
\begin{itemize}
	\item El dispositivo contará con una manera de desacoplar la alimentación principal y permitir la alimentación de los módulos a través de una fuente regulable de \TBD V $\pm$ \TBD mV que pueda suministrar por lo menos \TBD mA.
	\item Se tendrán sensores calibrados para las magnitudes físicas a medir para comparar la precisión de estos.
	\item Se contará con una modalidad en el software de debug que permita conmutar un pin para poder medir el tiempo entre medidas de los diversos sensores.
	\item \TBC
\end{itemize}

\textbf{Banco de pruebas 5:}
\begin{itemize}
	\item \TBC
\end{itemize}

\textbf{Banco de pruebas 6:}
\begin{itemize}
	\item Se podrá regular la carga con la que se quitará energía del sistema.
	\item Se podrá desacoplar la alimentación para simular una perdida de energía
	\item Se podrá alimentar el sistema con una tension mínima menor a la optima en un rango de tensiones determinado para comprobar su correcto funcionamiento
	\item \TBC
\end{itemize}

\textbf{Banco de pruebas 7:}
\begin{itemize}
	\item Con el producto finalizado se procederá a medir sus dimensiones físicas.
	\item Al igual que su peso con un calibre/metro y una balanza respectivamente.
	\item \TBC
\end{itemize}