\begin{table}[H]
\begin{tabular}{|l|l|l|}
\hline
\multicolumn{1}{|c|}{\textbf{ID}}                                       & \multicolumn{1}{c|}{\multirow{2}{*}{\textbf{Procedimiento}}}                                                                                                                                                                                                                                                                                                                                                                                                                                                                                                                                                                                                                                                                                                                                                                                                                                                                                                                              & \multicolumn{1}{c|}{\multirow{2}{*}{\textbf{Criterio}}}                                                                                                                                                   \\ \cline{1-1}
\multicolumn{1}{|c|}{\textbf{Aplicabilidad}}                            & \multicolumn{1}{c|}{}                                                                                                                                                                                                                                                                                                                                                                                                                                                                                                                                                                                                                                                                                                                                                                                                                                                                                                                                                                     & \multicolumn{1}{c|}{}                                                                                                                                                                                     \\ \hline
\begin{tabular}[c]{@{}l@{}}T-INT-FUN-10\\ Prototipo, Final\end{tabular} & \begin{tabular}[c]{@{}l@{}}Se utiliza el banco de pruebas \#3:\\ \\ \tabitem Se debe posicionar el panel solar de manera\\ tal que a este lo alumbre la luz del sol por completo.\\ \tabitem Conectar las baterías al módulo DFR0580\\ en la entrada BAT IN.\\ \tabitem Conectar el panel solar al módulo DFR0580\\ en la entrada SOLAR IN.\\ \tabitem Verificar según las instrucciones del módulo\\ DFR0580 el correcto funcionamiento del regulador\\ \tabitem Conectar la \rpi al módulo DFR0580 en la\\ entrada USB1.\\ \tabitem Utilizar la computadora para \\ conectarse a la red WiFi \quotes{SIMFS\_HOTSPOT}.\\ \tabitem Abrir un navegador de internet y conectarse a\\ la página \quotes{\ip}.\\ \tabitem Corroborar funcionamiento acorde a los tests \\ anteriores.\end{tabular}                                                                                                                                                                                   & \begin{tabular}[c]{@{}l@{}}El módulo DFR0580 debe reportar\\ correcto funcionamiento y se \\ debe corroborar el correcto\\ funcionamiento del sistema acorde\\ a los tests anteriores.\end{tabular} \\ \hline
\begin{tabular}[c]{@{}l@{}}T-INT-FUN-11\\ Prototipo, Final\end{tabular} & \begin{tabular}[c]{@{}l@{}}Se utiliza el banco de pruebas \#3:\\ \\ \tabitem Se debe conectar el Simulador de Arreglos \\ de Paneles Solares.\\ \tabitem Conectarlo al módulo DFR0580.\\ \tabitem Cuando aparezca la red WiFi\\ \quotes{SIMFS\_HOTSPOT} conectarse.\\ \tabitem Abrir un navegador de internet y conectarse a\\ la página \quotes{\ip}.\\ \tabitem Corroborar funcionamiento acorde a los tests \\ anteriores.\\ \tabitem Bajar gradualmente la tensión provista por el \\ simulador hasta el desabastecimiento, por lo que se\\ apagará el sistema.\\ \tabitem Esperar 1 minuto.\\ \tabitem Subir gradualmente la tensión provista por el \\ simulador hasta una tensión nominal de 21 V.\\ \tabitem Cuando aparezca la red WiFi\\ \quotes{SIMFS\_HOTSPOT} conectarse.\\ \tabitem Abrir un navegador de internet y conectarse a\\ la página \quotes{\ip}.\\ \tabitem Corroborar funcionamiento acorde a los tests \\ anteriores.\end{tabular} & \begin{tabular}[c]{@{}l@{}}El funcionamiento al momento de \\ estar energizado no se ve afectado \\ luego de la pérdida de energía.\end{tabular}                                                          \\ \hline
\end{tabular}
\caption{Tabla de test de funcionalidad (Parte 3).}
\label{tab:test_fun_3}
\end{table}