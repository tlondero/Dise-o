\begin{table}[H]
\begin{tabular}{|lll|}
\hline
\multicolumn{1}{|c|}{\textbf{ID}}                                                                                               & \multicolumn{1}{c|}{\multirow{2}{*}{\textbf{Procedimiento}}}                                                                                                                                                                                                                                                                                                                                                                                                                                                                                                                                                                                              & \multicolumn{1}{c|}{\multirow{2}{*}{\textbf{Criterio}}}                                                                                                                                                                   \\ \cline{1-1}
\multicolumn{1}{|c|}{\textbf{Aplicabilidad}}                                                                                    & \multicolumn{1}{c|}{}                                                                                                                                                                                                                                                                                                                                                                                                                                                                                                                                                                                                                                     & \multicolumn{1}{c|}{}                                                                                                                                                                                                     \\ \hline
\multicolumn{3}{|l|}{\textbf{\begin{tabular}[c]{@{}l@{}}Precondiciones:\\ Para realizar la prueba T-INT-FUN-11 las anteriores deben haberse realizado satisfactoriamente.\end{tabular}}}                                                                                                                                                                                                                                                                                                                                                                                                                                                                                                                                                                                                                                                                                                                                                                                                                                \\ \hline
\multicolumn{3}{|l|}{\textbf{\begin{tabular}[c]{@{}l@{}}Procedimiento general:\\ Conectar la Raspberry Pi a la alimentación de tensión mediante el cable de alimentación mini USB.\end{tabular}}}                                                                                                                                                                                                                                                                                                                                                                                                                                                                                                                                                                                                                                                                                                                                                                                                                       \\ \hline
\multicolumn{1}{|l|}{\begin{tabular}[c]{@{}l@{}}T-INT-FUN-01\\ Prototipo, Final\end{tabular}}                                   & \multicolumn{1}{l|}{\begin{tabular}[c]{@{}l@{}}Se utiliza el banco de pruebas \#2:\\ \tabitem Utilizar la computadora\\ para conectarse  a la red WiFi \quotes{SIMFS\_HOTSPOT}.\\ \tabitem Abrir un navegador de internet y conectarse a \\ la página \quotes{192.168.0.1:1880}.\\ \tabitem Prender el sensor de temperatura calibrado y\\ colocarlo próximo al sensor del producto.\end{tabular}}                                                                                                                                                                                                                                                        & \begin{tabular}[c]{@{}l@{}}La lectura en el sensor calibrada\\ coincida con el provisto por la \\ página con una desviación de \\ como máximo  $ \pm 0.5^\circ C $.\end{tabular}                                          \\ \hline
\multicolumn{1}{|l|}{\begin{tabular}[c]{@{}l@{}}T-INT-FUN-02\\ \\ Prototipo, Final\end{tabular}}                                & \multicolumn{1}{l|}{\begin{tabular}[c]{@{}l@{}}Se utiliza el banco de pruebas \#2:\\ \tabitem Utilizar la computadora para conectarse\\ para conectarse  a la red WiFi  \quotes{SIMFS\_HOTSPOT}\\ \tabitem Abrir un navegador de internet y conectarse a\\ la página \quotes{192.168.0.1:1880}.\\ \tabitem Prender el sensor de humedad calibrado y\\ colocarlo próximo al sensor del producto.\end{tabular}}                                                                                                                                                                                                                                             & \begin{tabular}[c]{@{}l@{}}La lectura en el sensor calibrada \\ coincida con el provisto por la \\ página con una desviación de \\ como máximo  $ \pm  2\% $.\end{tabular}                                                \\ \hline
\multicolumn{1}{|l|}{\begin{tabular}[c]{@{}l@{}}T-INT-FUN-03\\ Prototipo, Final\end{tabular}}                                   & \multicolumn{1}{l|}{\begin{tabular}[c]{@{}l@{}}Se utiliza el banco de pruebas \#2:\\ \tabitem Utilizar la computadora para conectarse\\ para conectarse  a la red WiFi  \quotes{SIMFS\_HOTSPOT}.\\ \tabitem Abrir un navegador de internet y conectarse a\\ la página \quotes{192.168.0.1:1880}.\\ \tabitem Prender el sensor de luminosidad calibrado y\\ colocarlo próximo al sensor del producto.\end{tabular}}                                                                                                                                                                                                                                        & \begin{tabular}[c]{@{}l@{}}La lectura en el sensor calibrada\\ coincida con el provisto por la pá-\\ gina con una desviación de como\\ máximo  $ \pm 0.5 \ lux $.\end{tabular}                                              \\ \hline
\multicolumn{1}{|l|}{\begin{tabular}[c]{@{}l@{}}T-INT-FUN-04\\ Prototipo, Final\\ T-INT-FUN-12\\ Prototipo, Final\end{tabular}} & \multicolumn{1}{l|}{\begin{tabular}[c]{@{}l@{}}Se utiliza el banco de pruebas \#2:\\ \tabitem Utilizar la computadora para conectarse\\ para conectarse a la red WiFi  \quotes{SIMFS\_HOTSPOT}.\\ \tabitem Abrir un navegador de internet y conectarse a\\ la página \quotes{192.168.0.1:1880}.\\ \tabitem Tocar el boton de la página web llamado\\ \quotes{Ver el nido en vivo}.\end{tabular}}                                                                                                                                                                                                                                                          & \begin{tabular}[c]{@{}l@{}}Lo visto en la página web coincide\\ con la realidad.\end{tabular}                                                                                                                             \\ \hline
\multicolumn{1}{|l|}{\begin{tabular}[c]{@{}l@{}}T-INT-FUN-05\\ Prototipo, Final\end{tabular}}                                   & \multicolumn{1}{l|}{\begin{tabular}[c]{@{}l@{}}Se utiliza el banco de pruebas \#2:\\ \tabitem Utilizar la computadora para conectarse\\ para conectarse a la red WiFi \quotes{SIMFS\_HOTSPOT}.\\ \tabitem Abrir un navegador de internet y conectarse a\\ la página \quotes{192.168.0.1:1880}.\\ \tabitem Abrir una pestaña mas y conectarse a\\ la página \quotes{192.168.0.1:1880/admin}.\\ \tabitem Hacer click en el inject de la zona del flow\\ marcado como \quotes{Photo Filter Debug}.\\ \tabitem Volver a la primera página abierta e ir al\\ item del  menú lateral descargas, y seleccionar\\ el archivo con el prefijo Dev.\end{tabular}} & \begin{tabular}[c]{@{}l@{}}Lo visto en la imagen descargada\\ coincide con la realidad.\end{tabular}                                                                                                                      \\ \hline
\multicolumn{1}{|l|}{\begin{tabular}[c]{@{}l@{}}T-INT-FUN-06\\ Prototipo, Final\end{tabular}}                                   & \multicolumn{1}{l|}{\begin{tabular}[c]{@{}l@{}}Se utiliza el banco de pruebas \#4:\\ \tabitem Utilizar la computadora para conectarse\\  para conectarse  a la red WiFi  \quotes{SIMFS\_HOTSPOT}\\ \tabitem Abrir un navegador de internet y conectarse a\\ la página \quotes{192.168.0.1:1880}\\ \tabitem  Energizar el módulo BLE previamente\\ programado.\\ \tabitem  Acercar el dispositvo al nido.\end{tabular}}                                                                                                                                                                                                                                    & \begin{tabular}[c]{@{}l@{}}Lo visto en el sector de la página \\ web que indica la presencia, sea\\ \quotes{El pájaro no está en casa} o \\ \quotes{¡El pájaro está en el nido} \\ coincida con la realidad.\end{tabular} \\ \hline
\end{tabular}
\caption{Tabla de teste de funcionalidad (Parte 1).}
\label{tab:test_fun_1}
\end{table}