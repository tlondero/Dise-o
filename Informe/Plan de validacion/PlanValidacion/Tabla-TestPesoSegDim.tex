\begin{table}[H]
\begin{tabular}{|c|l|l|}
\hline
\textbf{ID}            & \multicolumn{1}{c|}{\multirow{2}{*}{\textbf{Procedimiento}}}                                                                                                                                                                                                                                                                                                                                                            & \multicolumn{1}{c|}{\multirow{2}{*}{\textbf{Criterio}}}                                                                                                                                                                                                                                                      \\ \cline{1-1}
\textbf{Aplicabilidad} & \multicolumn{1}{c|}{}                                                                                                                                                                                                                                                                                                                                                                                                   & \multicolumn{1}{c|}{}                                                                                                                                                                                                                                                                                        \\ \hline
T-IMP-DIM-01           & \begin{tabular}[c]{@{}l@{}}Se utiliza el banco de pruebas \#5:\\ \tabitem Se utiliza la cinta métrica para medir las \\ dimensiones de tanto el dispositivo del nido\\ como de la unidad de energía.\end{tabular}                                                                                                                                                                                                       & \begin{tabular}[c]{@{}l@{}}El dispositivo del nido no debe\\ exceder las siguientes dimensiones:\\ Largo < $26 \ cm$\\ Ancho < $8,79  \ cm$\\ Alto < $4,55 \  cm$ \\ La unidad de energía no debe\\ exceder las siguientes dimensiones:\\ Largo < $1 \ m$\\ Ancho < $50 \ cm$\\ Alto <  $1 \ m$\end{tabular} \\ \hline
T-IMP-DIM-02           & \begin{tabular}[c]{@{}l@{}}Se utiliza el banco de pruebas \#5:\\ \tabitem Se utiliza la balanza para medir el peso \\ tanto del dispositivo del nido como\\ de la unidad de energía.\end{tabular}                                                                                                                                                                                                                       & \begin{tabular}[c]{@{}l@{}}El equipo dentro del nido no debe\\ exceder los $500 \ g$. La unidad de\\ energía no deberá exceder\\ los $20 \ kg$.\end{tabular}                                                                                                                                                 \\ \hline
T-RAM-SEG-01           & \begin{tabular}[c]{@{}l@{}}Se utiliza el banco de pruebas \#1:\\ \tabitem Se debe energizar la \rspi y\\ conectarse a ella utilizando una computadora\\ en la red \quotes{SIMFS\_HOTSPOT}. \\ \tabitem Abrir un navegador de internet y conectarse\\ a la página \quotes{\ip}.\\ \tabitem Abrir un navegador de internet y conectarse\\ a la página \quotes{\ipadmin}.\end{tabular} & \begin{tabular}[c]{@{}l@{}}Cuando se conecta a ambos sitios\\ web se pide autenticación de usuario\\ y contraseña.\end{tabular}                                                                                                                                                                              \\ \hline
\end{tabular}
\caption{Tabla de tests de peso, dimensión y seguridad.}
\label{tab:test_peso}
\end{table}