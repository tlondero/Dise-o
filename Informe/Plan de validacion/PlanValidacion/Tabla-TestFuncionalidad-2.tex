\begin{table}[H]
\begin{tabular}{|lll|}
\hline
\multicolumn{1}{|c|}{\textbf{ID}}                                                                              & \multicolumn{1}{c|}{\multirow{2}{*}{\textbf{Procedimiento}}}                                                                                                                                                                                                                                                     & \multicolumn{1}{c|}{\multirow{2}{*}{\textbf{Criterio}}}                                                                                                                                                                \\ \cline{1-1}
\multicolumn{1}{|c|}{\textbf{Aplicabilidad}}                                                                   & \multicolumn{1}{c|}{}                                                                                                                                                                                                                                                                                            & \multicolumn{1}{c|}{}                                                                                                                                                                                                  \\ \hline
\multicolumn{3}{|l|}{\textbf{\begin{tabular}[c]{@{}l@{}}Precondiciones:\\ Para realizar la prueba T-INT-FUN-11 las anteriores deben haberse realizado satisfactoriamente.\end{tabular}}}                                                                                                                                                                                                                                                                                                                                                                                                                                                                   \\ \hline
\multicolumn{3}{|l|}{\textbf{\begin{tabular}[c]{@{}l@{}}Procedimiento general:\\ Conectar la Raspberry Pi a la alimentación de tensión mediante el cable de alimentación mini USB.\end{tabular}}}                                                                                                                                                                                                                                                                                                                                                                                                                                                          \\ \hline
\multicolumn{1}{|l|}{\multirow{6}{*}{\begin{tabular}[c]{@{}l@{}}T-INT-FUN-07\\ Prototipo, Final\end{tabular}}} & \multicolumn{1}{l|}{\begin{tabular}[c]{@{}l@{}}Se utiliza el banco de pruebas \#1:\\ \tabitem Utilizar la computadora para conectarse para \\ conectarse  a la red WiFi  \quotes{SIMFS\_HOTSPOT}-\\ \tabitem Desde la terminal de la computadora se\\ insertar las siguientes lineas:\end{tabular}}              & \multirow{6}{*}{\begin{tabular}[c]{@{}l@{}}La diferencia horaria entre las \\ mediciones debe corresponder con\\ lo especificado en la pestaña de \\ \quotes{Configuración}.\end{tabular}}                              \\
\multicolumn{1}{|l|}{}                                                                                         & 
\begin{lstlisting}[language=Python]
ssh pi@[IP Raspberry Pi]
cd simfs/Python_Scripts/
python3 getTime.py
\end{lstlisting}
&                                                                                                                                                                                                                       \multicolumn{1}{|l|}{} \\
\multicolumn{1}{|l|}{}                                                                                         & \multicolumn{1}{l|}{\begin{tabular}[c]{@{}l@{}}\tabitem Anotar el valor que se encuentra en la terminal.\\ \tabitem Desde la terminal de la computadora se deben\\ insertar las siguientes lineas:\end{tabular}}                                                                                                 &                                                                                                                                                                                                                        \\
\multicolumn{1}{|l|}{}                                                                                         & \begin{lstlisting}[language=Python]
sudo shutdown -h now
\end{lstlisting} & \multicolumn{1}{|l|}{}                                                                                                                                                                                                                        \\
\multicolumn{1}{|l|}{}                                                                                         & \multicolumn{1}{l|}{\begin{tabular}[c]{@{}l@{}}\tabitem Esperar 1 minuto\\ \tabitem Renergizar la Rapsberry Pi\\ \tabitem Desde la terminal de la computadora se deben\\ insertar las siguientes lineas:\end{tabular}}                                                                                           &                                                                                                                                                                                                                        \\
\multicolumn{1}{|l|}{}                                                                                         & \begin{lstlisting}[language=Python]
ssh pi@[IP Raspberry Pi]
cd simfs/Python_Scripts/
python3 getTime.py
\end{lstlisting}                                                                                                                                                                  & \multicolumn{1}{|l|}{} \\ \hline
\multicolumn{1}{|l|}{\multirow{2}{*}{\begin{tabular}[c]{@{}l@{}}T-INT-FUN-08\\ Prototipo, Final\end{tabular}}} & \multicolumn{1}{l|}{\begin{tabular}[c]{@{}l@{}}Se utiliza el banco de pruebas \#1:\\ \tabitem Utilizar la computadora para conectarse para \\ conectarse a la red WiFi \quotes{SIMFS\_HOTSPOT}.\\ \tabitem Desde la terminal de la computadora se deben\\  insertar las siguientes lineas:\end{tabular}}         & \multirow{2}{*}{\begin{tabular}[c]{@{}l@{}}La hora impresa en la terminal \\ coincida con la hora correspondie-\\ nte al lugar de instalación del pro-\\ ducto.\end{tabular}}                                          \\
\multicolumn{1}{|l|}{}                                                                                         & \begin{lstlisting}[language=Python]
ssh pi@[IP Raspberry Pi]
cd simfs/Python_Scripts/
python3 getTime.py
\end{lstlisting}                                                                                                                 &                                                                                                                                                                                                                       \multicolumn{1}{|l|}{} \\ \hline
\multicolumn{1}{|l|}{\multirow{6}{*}{\begin{tabular}[c]{@{}l@{}}T-INT-FUN-09\\ Prototipo, Final\end{tabular}}} & \multicolumn{1}{l|}{\begin{tabular}[c]{@{}l@{}}Se utiliza el banco de pruebas \#1:\\ \tabitem Se debe utilizar la computadora para\\ conectarse para conectarse a la red WiFi\\ \quotes{SIMFS\_HOTSPOT}.\\ \tabitem Desde la terminal de la computadora se\\ deben insertar las siguientes lineas:\end{tabular}} & \multirow{6}{*}{\begin{tabular}[c]{@{}l@{}}La hora impresa en la terminal \\ coincida con la hora correspondie-\\ nte al lugar de instalación del pro-\\ ducto incluso luego de la pérdida\\ de energía.\end{tabular}} \\
\multicolumn{1}{|l|}{}                                                                                         & \begin{lstlisting}[language=Python]
ssh pi@[IP Raspberry Pi]
cd simfs/Python_Scripts/
python3 getTime.py
\end{lstlisting}                                                                                                  &                                                                                                                                                                                                                        \multicolumn{1}{|l|}{} \\
\multicolumn{1}{|l|}{}                                                                                         & \multicolumn{1}{l|}{\begin{tabular}[c]{@{}l@{}}\tabitem Anotar el valor que se encuentra en la terminal.\\ \tabitem Desde la terminal de la computadora se\\ deben insertar las siguientes lineas:\end{tabular}}                                                                                                 &                                                                                                                                                                                                                        \\
\multicolumn{1}{|l|}{}                                                                                         & \begin{lstlisting}[language=Python]
sudo shutdown -h now
\end{lstlisting}                                                                                                                                                               & \multicolumn{1}{|l|}{}                                                                                                                                                                                                                        \\
\multicolumn{1}{|l|}{}                                                                                         & \multicolumn{1}{l|}{\begin{tabular}[c]{@{}l@{}}\tabitem Esperar 1 minuto\\ \tabitem Renergizar la Rapsberry Pi\\ \tabitem Desde la terminal de la computadora se\\ deben insertar las siguientes lineas:\end{tabular}}                                                                                           &                                                                                                                                                                                                                        \\
\multicolumn{1}{|l|}{}                                                                                         & \begin{lstlisting}[language=Python]
ssh pi@[IP Raspberry Pi]
cd simfs/Python_Scripts/
python3 getTime.py
\end{lstlisting}                                                                                                    &                                                                                                                                                                                                                        \multicolumn{1}{|l|}{} \\ \hline
\end{tabular}
\caption{Tabla de teste de funcionalidad (Parte 2).}
\label{tab:test_fun_2}
\end{table}